在这一小节, 我们给出超实数在博弈论中的简单应用.

\subsection{游戏局面与超实数的对应}
我们考虑满足如下条件的博弈问题: 一个双人游戏, 给定初始状态, 每人轮流进行操作; 在游戏状态满足给定条件时游戏结束, 且有唯一的胜利者; 在有限步操作后游戏必然结束 (例如, 严格禁止全局同形再现规则下的围棋即满足此条件).

我们接下来将以一种广义超实数来表示一个游戏状态. 这里的广义超实数与超实数定义类似, 即仍形如 $(X_L:X_R)$, 但不要求 $X_L\leq X_R$. 例如 $(0:0)$ 是广义超实数, 却不是超实数.

称两人为L,R.如果形成轮到L操作的局面时触发结束条件, 且L失败, 则记 $X_L=\emptyset$, 反之亦然. 由此可以归纳地定义每个局面所对应的广义超实数 $(X_L:X_R)$. 例如轮到谁操作谁输的局面即为 $(\emptyset:\emptyset)=0$; 轮到R操作时R失败, 但轮到L操作时L可以将此局面变为 $0$ 的游戏局面即为 $(0:\emptyset)=1$; 如果此局面任意人操作以后均变为 $0$, 则为 $(0:0)$, 一个不是超实数的广义超实数. 我们用记号 $*$ 表示 $(0:0)$.

反之, 给定一个广义超实数 $(X_L:X_R)$, 可以构造如下的游戏: 轮到L操作时, 其可以将此超实数变为 $X_L$ 中的一个元素, 如果轮到L操作时 $X_L=\emptyset$, 则 $L$ 失败, 反之亦然. 注意到集合只能嵌套有限次, 游戏必然在有限步操作后结束. 由此我们构造了游戏局面和广义超实数的一一对应, 接下来我们使用广义超实数来研究游戏.

广义超实数可以如同超实数般定义运算 (加法, 相反数与乘法) , 序关系与等价关系. 其构成一个偏序交换幺环, 却不是域 ($*^2=*$) , 也没有全序 ($*$ 与 $0$无法比较大小) .

在这种对应下, 广义超实数的加减法运算和序关系均有对应的意义.
广义超实数的加减法在游戏中具有如下的意义: 
\begin{theorem}
  在上述对应之下, 广义超实数的加法即对应游戏之并, 即下面的游戏: 对多个广义超实数 $(X_{L,1}:X_{R_1}),\cdots,(X_{L_n}:X_{R_n})$, 某个人可以选择对任何一个数进行操作, 游戏的结束条件即为: 如轮到某人操作时其无法对任何一个数操作 (即所有数均被变为了$(\emptyset:x_R)$的形式), 则其失败. 广义超实数的相反数对应反过来的游戏, 即一个人进行的操作为原来游戏中另一个人进行的操作.
\end{theorem}
超实数的序关系则对应游戏的胜负关系. 设某游戏局面对应某广义超实数 $x$, 则: 
\begin{theorem}
  $x=0$当且仅当对应的游戏局面为先操作的人失败. $x>0$当且仅当对应的游戏局面为L胜利 (无论谁先操作), $x<0$当且仅当对应的游戏局面为R胜利. $x$与$0$无法比较大小当且仅当对应的游戏局面为先操作的人胜利.
\end{theorem}
\begin{remark}
  超实数的乘法在游戏理论中没有好的对应.
\end{remark}
广义超实数中序关系十分微妙. 这由下面的例子可以看出: 
\begin{example}
  对所有的超实数 $x>0$, 有 $x>*$ 成立. $(0:*)>0$ 成立, 但 $*$ 与 $(0:*)$ 却无法比较大小. $(1:-1)$ 是广义超实数, 但对所有满足 $-1\leq x\leq 1$ 的超实数 $x$, 其与 $x$ 均无法比较大小.
\end{example}
为解决此问题, 我们定义广义超实数的均值. 我们接下来仅考虑可以在有限天被定义的广义超实数 (游戏局面对应的广义超实数均满足这个条件).
\begin{definition}
  对广义超实数 $x$, 存在实数 $x_1,x_2$ 使得 $x_1>x$, $x_2<x$. 此时定义 $x$ 的\textbf{上界}为
  \[
    \sup(x):=\inf\{x_1\in \mathbb{R}\mid x_1\geq x\}.
  \]
  同样的方法可以定义其\textbf{下界}$\inf(x)$.
\end{definition}
\begin{definition}
  对广义超实数 $x$, 在实数意义下的极限
  \[
    \lim_{n\to \infty} \frac{\sup(nx)}{n}
  \]
  存在, 此值被定义为 $x$ 的\textbf{均值}, 记作 $\mean(x)$ (将 $\sup$ 换为 $\inf$ 会得到相同的结果).
\end{definition}
可以看出, 如果均值越大, 则此局面越可能对 L 有利.
\begin{example}
  如果一个广义超实数 $x$ 是实数, 其均值就是其正常意义下的值. 如 $x$ 形如 $(X_L:X_R)$ 使 $X_L>X_R$ 均为实数, 其均值为$(X_L+X_R)/2$. 对$X_{L,L}>X_{L,R}>X_R$ 为实数, $((X_{L,L}:X_{L,R}):X_R)$ 的均值为 $\min\{(X_{L,L}+X_{L,R}+2X_R)/4,X_{L,R}\}$.
\end{example}
\subsection{游戏策略}
对于一般的游戏局面, 我们并无法使用广义超实数对其进行好的分析. 有时一个游戏局面可以表示为若干较为简单的子局面之和 (和的定义见上述), 则可以使用广义超实数给出一个较好的策略. 这类局面的一个有代表性的例子是围棋的官子理论.

我们首先定义一步操作的价值.
\begin{definition}
  设局面 $x=(X_L:X_R)$. 假设现在轮到L操作, 定义L操作 $x$ 的\textbf{价值}
  \[
    \alpha_{L}=\sup\{\mean(y)\mid y\in X_L\}-\mean(x),
  \]
  即L操作后与操作前得到的局面的均值之差.
\end{definition}
对一个局面 $x=(X_L:X_R)$, 
使得
\[
  x=x_1+\cdots+x_n,\quad x_i=(X_{L_i}:X_{R_i}).
\]
则对 $L$ 来说, 更可能有利的策略是对操作价值 $\alpha_{i,L}$ 最大的 $x_i$ 进行操作.
\begin{example}
  $x_i$ 如上述. 如 $x_i=(X_{L_i}:X_{R_i})$, 其中 $X_{L,i}>X_{R,i}$ 均为实数, 则对L而言最有利的操作是操作满足 $X_{L,i}-X_{R,i}$ 最大的 $x_i$. 这与使用广义超实数理论所得出的结论一致.
\end{example}
然而这样的策略未必是最佳的. 这从下面的例子可以看出.
\begin{example}
  设局面 $x=x_1+x_2$, 其中$x_1=(3:0)$, $x_2=(0:(-2:-10))$. 则 $\alpha_{1,L}=3/2$ 而 $\alpha_{2,L}=2$. 然而, L操作$x_1$会赢, 而操作$x_2$会输. 又如如局面$x=x_1+x_2+x_3$, 其中$x_1=(0:*)$, $x_2=x_3=*$. 则所有操作的价值均为$0$, 然而L只有操作$x_1$才能赢.
\end{example}
这说明了使用广义超实数分析游戏的局限. 只有在一些简单的情况之下, 可以给出相同均值的广义超实数的序关系 (例如$(0:*)>0$), 从而给出精确的游戏策略, 参见\cite{BW97}, 一般情况只能进行近似估计.