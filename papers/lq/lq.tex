\documentclass[twoside]{article}

\input{../common/preamble}
\newcommand{\Hom}{\operatorname{Hom}}
\newcommand{\Sp}{\mathbf{sp}}
\newcommand{\colim}{\mathrm{colim}}
\newcommand{\im}{\operatorname{im}}
\newcommand{\vp}{\varphi}
\newcommand{\supp}{\mathrm{supp}}
\newcommand{\tr}{\operatorname{tr}}
\newcommand{\YP}{]Y[_\CP}
\newcommand{\XP}{]X[_\CP}
\newcommand{\ZP}{]Z[_\CP}
\newcommand{\HSn}{H^n_{cris}(X/S)}
\newcommand{\Hkn}{H^n_{cris}(X/W(k))}
\newcommand{\HSd}{H^\cdot_{cris}(X/S)}
\newcommand{\Hkd}{H^\cdot_{cris}(X/W(k))}
\newcommand{\HCd}{H^\cdot_{conv}(X)}
\newcommand{\HCn}{H^n_{conv}(X)}
\newcommand{\HRd}{H^\cdot_{rig}(X)}
\newcommand{\HRn}{H^n_{rig}(X)}
\newcommand{\DRX}{\Omega_{\XP/K}^\cdot}
\newcommand{\DRY}{\Omega_{\YP/K}^\cdot}
\newcommand{\Spec}{\mathrm{Spec}}
\newcommand{\Spf}{\mathrm{Spf}}
\newcommand{\Spp}{\mathrm{Sp}}
\newcommand{\Frac}{\mathrm{Frac}}
\newcommand{\DRXn}{\Omega_{\XP/K}^n}
\newcommand{\BA}{{\mathbb {A}}}
\newcommand{\BB}{{\mathbb {B}}}
\newcommand{\BC}{{\mathbb {C}}} 
\newcommand{\BD}{{\mathbb {D}}}
\newcommand{\BE}{{\mathbb {E}}} 
\newcommand{\BF}{{\mathbb {F}}}
\newcommand{\BG}{{\mathbb {G}}} 
\newcommand{\BH}{{\mathbb {H}}}
\newcommand{\BI}{{\mathbb {I}}} 
\newcommand{\BJ}{{\mathbb {J}}}
\newcommand{\BK}{{\mathbb {K}}} 
\newcommand{\BL}{{\mathbb {L}}}
\newcommand{\BM}{{\mathbb {M}}} 
\newcommand{\BN}{{\mathbb {N}}}
\newcommand{\BO}{{\mathbb {O}}} 
\newcommand{\BP}{{\mathbb {P}}}
\newcommand{\BQ}{{\mathbb {Q}}} 
\newcommand{\BR}{{\mathbb {R}}}
\newcommand{\BS}{{\mathbb {S}}} 
\newcommand{\BT}{{\mathbb {T}}}
\newcommand{\BU}{{\mathbb {U}}} 
\newcommand{\BV}{{\mathbb {V}}}
\newcommand{\BW}{{\mathbb {W}}} 
\newcommand{\BX}{{\mathbb {X}}}
\newcommand{\BY}{{\mathbb {Y}}} 
\newcommand{\BZ}{{\mathbb {Z}}}
\newcommand{\CA}{{\mathcal {A}}} 
\newcommand{\CB}{{\mathcal {B}}}
\newcommand{\CC}{{\mathcal {C}}} 
\newcommand{\CD}{{\mathcal {D}}}
\newcommand{\CE}{{\mathcal {E}}} 
\newcommand{\CF}{{\mathcal {F}}}
\newcommand{\CG}{{\mathcal {G}}} 
\newcommand{\CH}{{\mathcal {H}}}
\newcommand{\CI}{{\mathcal {I}}} 
\newcommand{\CJ}{{\mathcal {J}}}
\newcommand{\CK}{{\mathcal {K}}} 
\newcommand{\CL}{{\mathcal {L}}}
\newcommand{\CM}{{\mathcal {M}}} 
\newcommand{\CN}{{\mathcal {N}}}
\newcommand{\CO}{{\mathcal {O}}} 
\newcommand{\CP}{{\mathcal {P}}}
\newcommand{\CQ}{{\mathcal {Q}}} 
\newcommand{\CR}{{\mathcal {R}}}
\newcommand{\CS}{{\mathcal {S}}} 
\newcommand{\CT}{{\mathcal {T}}}
\newcommand{\CU}{{\mathcal {U}}} 
\newcommand{\CV}{{\mathcal {V}}}
\newcommand{\CW}{{\mathcal {W}}} 
\newcommand{\CX}{{\mathcal {X}}}
\newcommand{\CY}{{\mathcal {Y}}} 
\newcommand{\CZ}{{\mathcal {Z}}}

\addbibresource{lq.bib}

\begin{document}

\title{Applications of the Eells--Kuiper Invariant to Exotic $7$-Spheres}
\author{Lan Qing\footnote{蓝青,清华大学数学系数 83 班.}}

\begin{abstract}
	We introduce the Eells--Kuiper invariant, and apply the Eells--Kuiper invariant to find the $28$ differentiable structures on $S^7$. We also apply it to circle bundles over homotopy $\mathbb{C}P^3$, to show that any homotopy $7$-sphere that admits smooth regular $S^1$-actions
	is realized as the total space of a principal $S^1$-bundle over some homotopy $\mathbb{C}P^3$, with primitive Euler class. 
\end{abstract}

\tableofcontents

\section{Introduction}

In 1956, Milnor discovered that there exists a differentiable structure on $S^7$ that is different from the usual one in \cite{milnor7sphere}, where he introduced a differential  invariant and applied it to a collection of $S^3$-bundles over $S^4$. From the theory of h-cobordism discussed in \cite{milnor3} and \cite{smale}, there are exactly $28$ differentiable structures on $S^7$. In 1962, Eells and Kuiper introduced $\mu$ invariant which takes different values for different differentiable structures on $S^7$, completely classifying the $28$ differentiable structures on $S^7$. 
In \cite{kervairemilnor}, the group $\Theta_n$ of homotopy spheres is discussed, where for  $n \geq  5$, $\Theta_n$ describes exotic spheres of dimension $n$. 

In this thesis, we introduce the invariants of Milnor and Eells--Kuiper, and apply the Eells--Kuiper invariant to the collection of $S^3$-bundles over $S^4$ defined in \cite{milnor7sphere} to find $16$ differentiable structures on $S^7$, and then using connected sum to find the $28$ differentiable structures on $S^7$. 

We also apply a generalized Eells--Kuiper invariant to circle bundles over homotopy $\mathbb{C}P^3$, to show that any homotopy $7$-sphere that admits smooth regular $S^1$-actions
is realized as the total space of a principal $S^1$-bundle over some homotopy $\mathbb{C}P^3$, with primitive Euler class. 

Throughout this thesis we will use $\mathbb{Q}$ as the coefficient ring, unless otherwise stated. 






\section{The invariants of Milnor}

\subsection{The \texorpdfstring{$\lambda$}{lambda} invariant}

In \cite{milnor7sphere}, Milnor introduced $\lambda$ invariant for the discovery of exotic $7$-spheres. For an oriented closed $7$-manifold $M$ such that $H^3(M) = H^4(M) = 0$, since the seventh oriented cobordism ring $\Omega_7^{SO} = 0$, $M$ bounds a compact oriented $8$-manifold $B$. The assumption on the cohomology of $M$ implies that we have an isomorphism
\[
j \colon H^4(B,M) \to  H^4(B). 
\]

Let $\nu$ be the orientation class in $H_8(B,M)$, and let $\tau(B)$ be the signature of $B$. Then we have a well-defined differential invariant 
\[
\lambda(M) \colon  = 2\langle (j^{-1}p_1(B))^2, \nu\rangle -\tau(B)  \bmod  7, 
\]
which is indeed independent of $B$ chosen. For two such choices $B_1, B_2$, we reverse the orientation of $B_2$ and glue them to obtain a closed manifold $C$, and by Hirzebruch signature theorem we have 
\[
0 \equiv  45\tau(C) + \langle p_1(C)^2, [C]\rangle \equiv  4(-\tau(C) + 2\langle p_1(C)^2, [C]\rangle)  \bmod  7. 
\]
Thus 
\[
-\tau(B_1) + 2\langle j^{-1}_1p_1(B_1)^2, \nu_1 \rangle \equiv  -\tau(B_2) + 2\langle j^{-1}_2p_1(B_2)^2, \nu_2\rangle  \bmod  7. 
\]

Using this invariant, Milnor finds an $S^3$-bundle over $S^4$ which is homeomorphic to the $7$-sphere but has nonzero $\lambda$. Thus it cannot bound $D^8$ whose fourth Betti number is zero, and consequently it is an exotic sphere. 

Later, this invariant is generalized in \cite{milnorspheres}. For a closed smooth oriented $(4k-1)$-manifold $M$ such that over $\mathbb{Q}$, 
\[
H^{2k}(M) = H^{4i}(M) = 0, \forall 0 < i < k, 
\]
by Poincar\'e duality and universal coefficient theorem we have
\[
H^{2k-1}(M) = H^{4i-1}(M) = 0, \forall 0 < i < k. 
\]

If $M$ bounds a compact smooth oriented $4k$-manifold $W$, we still have isomorphisms
\[
j \colon H^{4i}(W,M) \to  H^{4i}(W). 
\]

Recall that $ \{ L_k(p_1, \dots  ,p_k) \} $ is the multiplicative sequence associated to the formal power series ${\sqrt{t}}/{\operatorname{tanh}(\sqrt{t})}$, where $L_k$ is a homogeneous polynomial of degree $4k$ and $deg(p_i) = 4i$. Note that $L_k(p_1, \dots  ,p_k) = L_k(p_1, \dots  ,p_{k-1},0) + s_kp_k$ where $s_{k} = 2^{2 k}(2^{2 k-1}-1) B_{k} /(2 k) !$, and $B_{k}$ denotes the Bernoulli numbers. Let $\nu = [W,M]$ be the fundamental class. 

\begin{define}
	The $\lambda$ invariant in $\mathbb{Q}/\mathbb{Z}$ is defined to be 
	\[
	\lambda(M) \colon  = (\tau(W)-\langle L_k(j^{-1}p_1(W), \dots  ,j^{-1}p_{k-1}(W),0),\nu\rangle)/s_k  \bmod  1. 
	\]
\end{define}

Observe that if $M$ is empty, by Hirzebruch signature theorem we have 
\[
\tau(W) = \langle L_k(p_1(W), \dots  ,p_k(W)),[W]\rangle
\]
and then 
\[
\lambda(\emptyset) = \langle s_kp_k(W),[W]\rangle/s_k = \langle p_k(W),[W]\rangle \equiv  0  \bmod  1
\]
becomes trivial. In essentially the same way as above we can show that the $\lambda$ invariant is well-defined, since two definitions will differ by some top Pontrjagin number which is an integer. 

In the special case where $k = 2$, 
\begin{gather*}
\lambda(M) = \frac{45}{7}({\tau(W)-\langle(7j^{-1}p_2-j^{-1}p_1^2)/45,[W,M]\rangle}) \\
\equiv  \frac{3}{7}\tau(W) + \frac{1}{7}\langle j^{-1}p_1^2,[W,M]\rangle  \bmod  1.  
\end{gather*}
Trivially for integers $a,b$, $2(a + 3b) \equiv 2a-b  \bmod  7$ and $4(2a-b) \equiv  a + 3b \bmod  7$. Consequently this $\lambda$ invariant and the previous one defined for $7$-manifolds indeed determine each other.  


\subsection{The \texorpdfstring{$\lambda'$}{lambda'} invariant}

We say that a manifold $W$ is \textbf{almost parallelizable} if for some finite subset $F$, $W-F$ is parallelizable, i.e.\ $M-F$ has trivial tangent bundle. 

Let $\tau_k$ be the greatest common divisor of all $\tau(W)$ where $W$ ranges over all almost parallelizable $4k$-manifolds without boundary. 

Suppose $M$ is a smooth oriented homology $(4k-1)$-sphere over $\mathbb{Z}$ and $M$ bounds a compact smooth oriented parallelizable manifold $W$. Thus $\tau(W)  \bmod  \tau_k$ is a well-defined invariant of $M$. But in fact, under the assumption above $\tau(W)$ and $\tau_k$ are always divisible by $8$. 

\begin{define}
	The $\lambda'$ invariant of $M$ is defined to be 
	\[
	\lambda'(M) \equiv  \tau(W)/8  \bmod  \tau_k/8. 
	\]
\end{define}




\section{The Eells--Kuiper \texorpdfstring{$\mu$}{mu} invariant}


\subsection{The domain of \texorpdfstring{$\mu$}{mu}}
We would like to define another invariant for certain manifolds. An oriented manifold is said to be \textbf{spin} if its second Stiefel-Whitney class is zero. 

Let $M$ be a closed smooth oriented $(4k-1)$-manifold that bounds a compact smooth oriented spin $4k$-manifold $W$, such that 

\textbf{(a)} over $\mathbb{Q}$, 
\[
H^{2k}(M) = H^{4i}(M) = 0, \forall 0 < i < k, 
\]

\textbf{(b)} the inclusion induces an epimorphism
\[
i^* \colon H^1(W;\mathbb{Z}_2) \to  H^1(M;\mathbb{Z}_2). 
\]


Again condition (a) implies $H^{2k-1}(M) = H^{4i-1}(M) = 0, \forall 0 < i < k$, so $j \colon H^{4i}(W,M) \to  H^{4i}(W)$ and $j \colon H^{2k}(W,M) \to  H^{2k}(W)$ are still isomorphisms. Indeed, condition (a) can be replaced by the slightly weaker condition requiring them to be isomorphisms, so that one can pull back Pontrjagin classes of $W$. We call this condition \textbf{(a')}. 

One reason to introduce condition (b) is that the definition seems to be dependent on the spin structure on $M$. If $(W, \sigma)$ is a spin manifold with boundary $M$ with the induced spin structure $\sigma_M$, then for any spin structure $\sigma_M'$ of $M$, condition (b) allows to replace the spin structure of $W$ so that the induced spin structure of $M$ becomes $\sigma_M'$. Later we will see that the formula for $\mu$ invariant will be independent of the spin structure on $W$, so $\mu$ invariant is indeed independent of the spin structure of $M$. 

An example satisfying these conditions is given by $S^3$-bundles $M$ over $S^4$ with nonvanishing Euler class, which are the boundary of the corresponding disk bundles $W$. These disk bundles, having the homotopy type of $S^4$, is automatically spin. From the Serre spectral sequence it follows that the first cohomology of the sphere bundle is $0$, so condition (b) is satisfied. Also, it follows from Serre spectral sequence that if the Euler class does not vanish, then over $\mathbb{Q}$ the $E_\infty$ page consists of only two nonzero terms $E_\infty^{0,0}$ and $E_\infty^{4,3}$, so condition (a) is also satisfied. The following diagram shows the $E_4$ page over $\mathbb{Z}$ where the arrow is given by cup product with Euler class (up to a sign) . 

\[
\xymatrix{
	\mathbb{Z}\ar[rrrrddd]&0&0&0&\mathbb{Z}\\
	0&0&0&0&0\\
	0&0&0&0&0\\
	\mathbb{Z}&0&0&0&\mathbb{Z}
}
\]

The following diagram is the $E_\infty$ page over $\mathbb{Q}$, assuming the Euler class does not vanish. 

\[
\xymatrix{
	0&0&0&0&\mathbb{Q}\\
	0&0&0&0&0\\
	0&0&0&0&0\\
	\mathbb{Q}&0&0&0&0
}
\]





However, if the Euler class vanishes, the $E_\infty$ page is
\[
\xymatrix{
	\mathbb{Q}&0&0&0&\mathbb{Q}\\
	0&0&0&0&0\\
	0&0&0&0&0\\
	\mathbb{Q}&0&0&0&\mathbb{Q}
}
\]
and consequently $H^3(M;\mathbb{Q}) = H^4(M;\mathbb{Q}) = \mathbb{Q}$ is not zero, and condition (a) is not satisfied. 



In fact, in this case the weaker condition (a') is not satisfied either. We have
\[
\xymatrix{
	H^0(S^4)\ar[d]_{  \cup   Thom \ class}\ar[rd]^{  \cup   e = 0}& \\
	H^4(W,M)\ar[r]_j&H^4(W) = \mathbb{Q}\\
}
\]
where the vertical map is the Thom isomorphism. Thus $j \colon H^{4}(W,M) \to  H^{4}(W)$ is not an isomorphism. 

\subsection{The definition of \texorpdfstring{$\mu$}{mu}}

Let $ \{ \widehat{A}_k = \widehat{A}_k(p_1, \dots  ,p_k) \} $ be the multiplicative sequence associated to $\frac{1}{2}z^{1/2}/\operatorname{sinh}(\frac{1}{2}z^{1/2})$. The following theorem of Hirzebruch will be used. For a proof, see \cite{borelhirzebruch}. 

\begin{thm}
	Let $X$ be a closed, smooth, oriented spin manifold of dimension $4k$. Then the \textbf{$\widehat{A}$-genus} $\widehat{A}[X] = \langle\hat{A}_k(p_1(X), \dots  ,p_k(X)),[X]\rangle$ is an integer. If $k$ is odd, $\widehat{A}[X]$ is an even integer. 
\end{thm}


For manifolds $M$ in our domain, Pontrjagin classes $p_1, \dots  ,p_{k-1}$ of $W$ can be pulled back to $H^*(W,M)$. From
\[
\tau(X) = L_k(p_1, \dots  ,p_{k-1},0)[X] + L_k(0,0, \dots  ,1)p_k[X]
\]
\[
\widehat{A}_k[X] = \widehat{A}_k(p_1, \dots  ,p_{k-1},0)[X] + \widehat{A}_k(0,0, \dots  ,1)p_k[X]
\]
we eliminate $p_k[X]$ and get
\[
\widehat{A}_k[X] = N_k(p_1, \dots  ,p_{k-1})[X] + t_k\tau(X)
\]
where 
\[
t_k = \frac{\widehat{A}_k(0,0, \dots  ,1)}{L_k(0,0, \dots  ,1)}
\]
\[
N_k(p_1, \dots  ,p_{k-1})[X] = \widehat{A}_k(p_1, \dots  ,p_{k-1},0)[X]-t_kL_k(p_1, \dots  ,p_{k-1},0)[X]. 
\]

Let $a_k = 1$ if $k$ is even, and $a_k = 2$ if $k$ is odd. 

\begin{define}
	The $\mu$ invariant of $M^{4k-1}$ in $\mathbb{Q}/\mathbb{Z}$ is 
	\[
	\mu(M) = (\langle N_k(j^{-1}p_1(W), \dots  ,j^{-1}p_{k-1}(W)),[W,M]\rangle + t_k\tau(W))/a_k  \bmod  1. 
	\]
\end{define}

\begin{thm}
	The $\mu$ invariant is well-defined independent of $W$ chosen. 
\end{thm}
\proof{This is essentially the same as in the previous section. Let $(W_1, M_1)$ and $(W_2, M_2)$ be two such choices for the space $M$. Reverse the orientation of the latter and glue the two manifolds along the boundary to get $X$. Let $r$ be $2k$ or $4i(0  < i   <   k)$. We have a commutative diagram
	\[
	\xymatrix{
		H^r(W_1,M) \oplus  H^r(W_2,M)\ar[d]^{j_1 \oplus  j_2}&H^r(X,M)\ar[l]^-h\ar[d]^j\\
		H^r(W_1) \oplus  H^r(W_2)&H^r(X)\ar[l]^-k
	}
	\]
	where $h$ is an isomorphism by Mayer-Vietories sequence, and $j_1 \oplus  j_2$ is an isomorphism by assumption. 
	
	Since 
	\[
	H^{r}(M) \stackrel{\text { zero }}{\leftarrow} H^{r}(W_{1}) \stackrel{\text { isom. }}{\leftarrow} H^{r}(W_{1}, M) \stackrel{\text { zero }}{\leftarrow} H^{r-1}(M), 
	\]
	we find $H^r(X) \to  H^r(W_1)\stackrel{\text { zero }}{ \to }H^{r}(M)$ is the zero map. Thus
	\[
	H^r(M)\stackrel{\text { zero }}{\leftarrow} H^r(X)\stackrel{\text { j }}{\leftarrow} H^r(X,M)
	\]
	implies that $j$ is epic. 
	
	It follows from easy diagram chasing that $j$ is an isomorphism and thus $k$ is an isomorphism. 
	
	If $\alpha = j h^{-1}(\alpha_{1}  \oplus  \alpha_{2})  \in   H^{2k}(X),$ then
	\begin{gather*}
	\left\langle\nu, \alpha^{2}\right\rangle = \left\langle\nu, j h^{-1}(\alpha_{1}^{2}  \oplus  \alpha_{2}^{2})\right\rangle \\
	= \left\langle\nu_{1}  \oplus (-\nu_{2}), \alpha_{1}^{2}  \oplus  \alpha_{2}^{2}\right\rangle = \left\langle\nu_{1}, \alpha_{1}^{2}\right\rangle-\left\langle\nu_{2}, \alpha_{2}^{2}\right\rangle
	\end{gather*}
	where $\nu = [X], \nu_1 = [W_1,M], \nu_2 = [W_2,M]$ are orientation classes. Thus the intersection form of $X$ splits into the direct sum of those of $W_1$ and $W_2$, and 
	\[
	\tau(X) = \tau(W_1)-\tau(W_2). 
	\]
	
	Clearly, the pullback of the tangent bundle of $X$ by inclusion $i_l \colon W_l\to X$ is the tangent bundle of $W_l$, $l = 1,2$. 
	By naturality, $H^*(X) \to  H^*(W_1) \oplus  H^*(W_2) \to  H^*(W_l), p_s(X)\mapsto p_s(W_l)$ for any Pontrjagin class $p_s$. 
	So $k(p_s(X)) = p_s(W_1) \oplus  p_s(W_2)$. 
	As above, for any monomial $f$, $f(p(X))$ becomes the direct sum of $f(p(W_1))$ and $f(p(W_2))$, so
	\begin{align*}
	f(p_{1}, \ldots, p_{k-1}, 0)[X] = \langle f(j_1^{-1}p_{1}(W_1), \ldots, j_1^{-1}p_{k-1}(W_1), 0),[W_{1},M]\rangle  \\
	-\langle f(j_2^{-1}p_{1}(W_2), \ldots, j_2^{-1}p_{k-1}(W_2), 0),[W_{2},M]\rangle. 
	\end{align*}
	
	For simplicity, write 
	\[
	    f(p_{1}, \ldots, p_{k-1}, 0)[W] =
	    \langle f(j^{-1}p_{1}(W), \ldots, j^{-1}p_{k-1}(W), 0),[W,M]\rangle. 
	\]
	The equality above becomes
	\[
	f(p_{1}, \ldots, p_{k-1}, 0)[X] = f(p_{1}, \ldots, p_{k-1}, 0)[W_{1}]-f(p_{1}, \ldots, p_{k-1}, 0)[W_{2}]. 
	\]
	
	Combining results above, with $f$ replaced by $N_k$, 
	\[
	\mu(M_1)-\mu(M_2) = \widehat{A}[X]/a_k 
	\]
	is an integer (modulo $1$) assuming $X$ is spin. 
	
	Now we show that $X$ is spin. From the exact sequence
	\begin{align*}
	\dots   \leftarrow H^{2}(W_{1} ; \mathbb{Z}_{2})  \oplus  H^{2}(W_{2} ; \mathbb{Z}_{2})& \stackrel{k_{1}^{*}  \oplus  k_{2}^{*}}{\leftarrow} {H^{2}}(X ; \mathbb{Z}_{2}) \stackrel{\Delta}{\leftarrow}{H}^{1}(M ; \mathbb{Z}_{2}) \\
	\stackrel{i_{1}^{*} - i_{2}^{*}}{\leftarrow}H^{1}(W_{1} ; \mathbb{Z}_{2}) & \oplus  H^{1}(W_{2} ; \mathbb{Z}_{2})\leftarrow  \dots  
	\end{align*}
	we see that $i_{1}^{*} - i_{2}^{*}$ is surjective by condition (b), so $\Delta$ is zero. Therefore $k_{1}^{*}  \oplus  k_{2}^{*}$ is injective. Since by naturality $k_{1}^{*}  \oplus  k_{2}^{*}(w_2(X)) = w_2(W_1) \oplus  w_2(W_2) = 0 \oplus  0 = 0$, $X$ is spin. \qedhere}

Here are some facts about $\mu$ invariant. 

\begin{prop}
	
	1) $\mu(-M) = -\mu(M)$. 
	
	2) If $M_1$, $M_2$ are h-cobordant, then $\mu(M_1) = \mu(M_2)$. 
	
	3) If $M_1$, $M_2$ are in the domain of $\mu$, then so is their connected sum $M_1 \#  M_2$, and $\mu(M_1 \#  M_2) = \mu(M_1) + \mu(M_2)$. 
\end{prop}

Given two spin $n$-manifolds $X_1$, $X_2$, we can form a spin structure on $X_1 \#  X_2$ such that $X_1 \#  X_2$ and $X_1\coprod X_2$ are spin cobordant. See Milnor's paper in \cite{milnorspin} or Remark 2.17 in \cite{lawsonspin}. 







\subsection{Rohlin invariant}

This subsection is added to show that in $3$-dimensional case, in the common domain of the Rohlin invariant and the Eells--Kuiper invariant, these invariants are equivalent. This will not be used in the applications. 

\subsubsection{Definition of the Rohlin invariant}

The following facts can be found in \cite{saveliev}. 

\begin{thm}
	Any closed orientable $3$-dimensional manifold can be obtained from $S^3$ by Dehn surgery along an even link.
\end{thm}

The procedure of Dehn surgery along links in $S^3$ can be identified with the behavior of the boundary when one glue handles $D^2 \times  D^2$ to the boundary of $D^4$. 

Let $\mathscr{L} = L_{1}   \cup   \cdots   \cup   L_{n}$ be an oriented framed link in $S^{3},$ the $i$-th component being framed by $e_{i}  \in   \mathbb{Z} .$ The symmetric matrix $A = (a_{i j}), i, j = 1, \ldots, n,$ with the entries
\[
a_{i j} = \left \{ 
\begin{array}{ll}
e_{i}, & \text { if } i = j \\
\operatorname{lk}(L_{i}, L_{j}), & \text { if } i          \neq   j
\end{array}\right. 
\]
is called the \textbf{linking matrix} of $\mathscr{L}$. 

\begin{thm}
	Let $M$ be a $4$-manifold with boundary obtained by integral surgery on a framed link $\mathscr{L}$ in $S^3$. Then the intersection form $Q_M$ is isomorphic to the linking matrix of $\mathscr{L}$. 
\end{thm}

Let $Q$ be a $\mathbb{Z}$-valued unimodular symmetric $\mathbb{Z}$-bilinear form defined on a finitely generated free abelian group $L$. It is \textbf{even} if $Q(x,x)$ is even, $\forall x\in L$. There is a basic fact from linear algebra that the signature of such an even form is divisible by $8$, see \cite{serre} or \cite{milnorsymmetric}. 




Let $\Sigma$ be a compact oriented integral homology $3$-sphere. From  theorems above we deduce that $\Sigma$ bounds  a smooth simply-connected oriented $4$-manifold $W$ with even intersection form. Thus $\tau(W) \equiv  0\bmod 8$. 

\begin{define}
	Let $\Sigma$ be a compact oriented integral homology $3$-sphere, and let $W$ be a simply-connected, compact, smooth, oriented manifold with boundary $\Sigma$. Assume that the intersection form $Q_W$ is even. The Rohlin invariant of $\Sigma$ in $\mathbb{Z}/2\mathbb{Z}\subset\mathbb{Q}/\mathbb{Z} $ is defined to be 
	\[
	\nu(\Sigma) = \frac{\tau(W)}{16} \bmod 1. 
	\]
\end{define}
This is well-defined because of the theorem of Rohlin:  
\begin{thm}
	If $M$ is a simply-connected, closed, smooth, oriented $4$-manifold with even intersection form then $\tau(M) \equiv 0 \bmod 16$. 
\end{thm}

\begin{prop}
	$\nu(-M) = -\nu(M)$, and $\nu(M_1 \#  M_2) = \nu(M_1) + \nu(M_2)$. 
\end{prop}



There is another definition of Rohlin invariant, using the fact that homology $3$-spheres have a unique spin structure, and the third spin cobordism ring is zero.  Define the Rohlin invariant of a homology $3$-sphere to be  
\[
\nu(\Sigma) = \frac{\tau(W)}{16} \bmod 1, 
\]
where $W$ is any smooth compact spin $4$-manifold with spin boundary the homology sphere. 

To show that this is independent of $W$, there is a corresponding version of Rohlin's theorem: 
\begin{thm}
	The signature of a smooth closed spin $4$-manifold $X$ is divisible by $16$. 
\end{thm}

This extends the previous definition. In fact, if the manifold $W$ is obtained from $D^4$ by gluing $2$-handles according to an even surgery on a link in $S^3$, then the canonical spin structure on $D^4$ extends to $W$. See Proposition 5.7.1 in \cite{gompf} for a proof. 


\subsubsection{Comparison with \texorpdfstring{$\mu$}{mu}}

Let $\Sigma$ be a compact oriented integral homology $3$-sphere. From universal coefficient theorem, since for any $k$,  $H_k(S^3;\mathbb{Z}) = H_k(\Sigma;\mathbb{Z})$ is free over $\mathbb{Z}$, all $\operatorname{Tor}$ and $\operatorname{Ext}$ terms vanish and thus $\Sigma$ is a compact oriented homology $3$-sphere over any $\mathbb{Z}$-algebra and in particular over $\mathbb{Q}$ and $\mathbb{Z}_2$. 


To show that $\Sigma$ is in the domain of $\mu$, note that the third spin cobordism ring is zero, so $\Sigma$ always spin bounds a spin manifold. Since $H^2(\Sigma;\mathbb{Q}) = 0$, condition (a) is satisfied. Since $H^1(\Sigma;\mathbb{Z}_2) = 0$, condition (b) is satisfied. 

From direct calculation one shows that for any $3$-manifold $M$ in the domain of $\mu$, 
\[
\mu(M) = -\frac{\tau(W)}{16} \bmod 1. 
\]

Consequently, in $3$-dimensional case, in the common domain of the Rohlin invariant and the Eells--Kuiper invariant, these invariants are equivalent. 














\section{Applications to \texorpdfstring{$S^3$}{S3} bundles over \texorpdfstring{$S^4$}{S4}}

\subsection{Calculation of characteristic classes of \texorpdfstring{$S^3$}{S3} bundles over \texorpdfstring{$S^4$}{S4}}

Define a collection of $S^3$-bundle over $S^4$ as follows. For each $(h,j) \in   \mathbb{Z} \oplus \mathbb{Z}$, consider an element $\phi_{hj} \in   \pi_3(SO_4) = \mathbb{Z} \oplus \mathbb{Z}$ as the clutching function of the sphere bundle $S^3\to SE_{hj} \to  S^4$, defined by 
\[
x\in S^3\subset \mathbb{H}, v\in\mathbb{H} \cong \mathbb{R}^4, \phi_{hj}(x)v \colon  = x^hvx^j\in\mathbb{H} \cong \mathbb{R}^4,  
\]
where $\mathbb{H}$ denotes the quaternions. 

Let $SE_{hj}$,  $DE_{hj}$, $E_{hj}$ be the sphere bundle, disk bundle and vector bundle associated to $\phi_{hj}$ respectively. We need to calculate the Euler class and the first Pontrjagin class of these bundles, in order to apply $\mu$ invariant. 

\begin{thm}
	$SE_{hj}$ is homeomorphic to the standard sphere $S^7$ iff $h + j = \pm 1$. 
\end{thm}

\proof{ We give a sketch of proof. By a simple application of  the long exact sequence, $SE_{hj}$ is simply connected. From Poincar\'e conjecture or h-cobordism theorem, $SE_{hj}$ is homotopy equivalent to $S^7$ iff $SE_{hj}$ is homeomorphic to $S^7$. Consequently, we only need to find out for which $h,j$ is $SE_{hj}$ an integral homology sphere. 
	
	Applying Mayer-Vietories sequence to the restrictions of the sphere bundle to the northern hemisphere $D^4_ + $ and southern hemisphere $D^4_-$, it follows that $SE_{hj}$ an integral homology sphere iff
	\[
	H_3(\partial D^4 \times  S^3;\mathbb{Z}) \to  H_3(D^4_ +   \times   S^3;\mathbb{Z}) \oplus  H_3(D^4_-  \times   S^3;\mathbb{Z})
	\]
	is an isomorphism, where both sides are isomorphic to $\mathbb{Z} \oplus \mathbb{Z}$. Choosing a basis for these $\mathbb{Z}$-modules and using the definition of the clutching map, this map is given by the matrix
	\[
	\begin{pmatrix}
	0&1\\
	h + j&1
	\end{pmatrix}
	\]
	which is invertible iff $h + j = \pm 1$. \qedhere}

\begin{lem}
	$\mathbb{Z} \oplus \mathbb{Z} \to \pi_3(SO_4) \cong \pi_4(BSO_4), (i,j)\mapsto \phi_{ij} $ is a group homomorphism. 
\end{lem}
\proof{Note that $\phi_{i + j,k + l}(q) = \phi_{i,k}(q)\phi_{j,l}(q)$ for each $q\in S^3$, where the multiplication on right hand side is the multiplication in $SO_4$. But in a Lie group $SO_4$, $q\mapsto\phi_{i,k}(q)\phi_{j,l}(q)$ is homotopic to the sum in $\pi_3(SO_4)$ of $q\mapsto\phi_{i,k}(q)$ and $q\mapsto\phi_{j,l}(q)$. \qedhere}

\begin{lem}
	$\pi_3(SO_4)  \to  H^4(S^4;\mathbb{Z}), f\in \pi_3(SO_4)\mapsto $ the Euler class of the bundle determined by $f$, is a group homomorphism. The same is true for the first Pontrjagin class. 
\end{lem}
\proof{Let $E_f$ denote the bundle determined by $f$. 
Assume $f,g\in \pi_3(SO_4)$ are basepoint preserving maps. 
Their sum is a map 
$f + g \colon S^3\to S^3\vee S^3\stackrel{f\vee g}{ \to }SO_4$. $f\vee g$ 
determines a bundle on the reduced suspension 
$\Sigma(S^3\vee S^3) = S^4\vee S^4$, 
and the map $h \colon S^4\to S^4\vee S^4$ 
pulls back this bundle to get the bundle $E_{f + g}$. 
	
Consequently, $h^*(e(E_f) \oplus  e(E_g)) = e(E_{f + g})$. 
But $h^*(a \oplus  b) = a + b$. The result follows. 
\qedhere}

From lemmas above, 
we deduce that after picking a generator $u\in H^4(S^4;\mathbb{Z})$, 
\[
e(E_{ij}) = (\alpha i + \beta j)u  \in   H^4(S^4;\mathbb{Z})
\]
\[
p_1(E_{ij}) = (\gamma i + \delta j)u  \in   H^4(S^4;\mathbb{Z})
\]
for some $\alpha, \beta, \gamma, \delta$. 

\begin{lem}
	$E_{-j,-i}$ is obtained from $E_{i,j}$ by reversing the orientation of fiber. 
\end{lem}
\proof{We define an orientation reversing bundle isomorphism $\psi \colon E_{i,j} \to  E_{-j,-i}$ as follows. For each $v$ in a fiber $\mathbb{H}\subset E_{i,j}$, define $\psi(v) = \bar{v} \in   \mathbb{H}$. The conjugation in $\mathbb{H}$ is orientation reversing. To show that this is well-defined near the equator, simply note that $\overline{q^ivq^j} = q^{-j}\bar{v}q^{-i}$ for $q\in S^3$. \qedhere}

Since when the orientation of the bundle is reversed, the Euler class changes sign while the first Pontrjagin class does not change, we have
\[
e(E_{ij}) = \alpha (i +  j)u  \in   H^4(S^4;\mathbb{Z})
\]
\[
p_1(E_{ij}) = \gamma (i- j)u  \in   H^4(S^4;\mathbb{Z})
\]
for some $\alpha, \gamma$. 

To find out the coefficients, it suffices to calculate for a concrete $E_{ij}$, say $E_{01}$. 

Note that the clutching function of $E_{01}$ is complex linear, which implies that it can be identified with a complex vector bundle. The space $SE_{01}$ is homeomorphic to $S^7$, so from the fourth page of the Serre spectual sequence
\[
\xymatrix{
	\mathbb{Z}\ar[rrrrddd]&0&0&0&\mathbb{Z}\\
	0&0&0&0&0\\
	0&0&0&0&0\\
	\mathbb{Z}&0&0&0&\mathbb{Z}
}
\]
we see that the Euler class must give rise to an isomorphism. Thus 
\[
e(E_{ij}) = \pm(i +  j) \in   H^4(S^4;\mathbb{Z}) \cong \mathbb{Z}
\]
where the sign depends on our choice of generator of $\mathbb{Z}$. 

Since $E_{01}$ is a complex bundle, 
\[
c(E_{01} \otimes \mathbb{C}) = c(E_{01} \oplus  \overline{E_{01}}) = (1 + c_2(E_{01}))^2 = 1 + 2c_2(E_{01}) = 1 + 2e(E_{01}), 
\]
\[
p_1(E_{01}) = -c_2(E_{01} \otimes \mathbb{C}) = -2e(E_{01}) = \pm 2 u. 
\]
Consequently, 
\[
p_1(E_{ij}) = \pm 2(i- j)  \in   H^4(S^4;\mathbb{Z}) \cong \mathbb{Z}. 
\]
If we set $e(E_{01}) = 1$, then the sign convention is 
\[
e(E_{ij}) = i +  j\in H^4(S^4;\mathbb{Z}) \cong \mathbb{Z}
\]
\[
p_1(E_{ij}) = 2(i- j)  \in   H^4(S^4;\mathbb{Z}) \cong \mathbb{Z}. 
\]






\subsection{Differentiable structures on \texorpdfstring{$S^7$}{S7}}


If $k = 2$, the formula for the Eells--Kuiper invariant becomes
\[
    \mu(M^{7})  \equiv \frac{(p_{1}^{2}[W]-4 \tau[W]) }{ (2^{7} \times 7)}  \bmod 1. 
\]

Since $SE_{hj}$ is homeomorphic to the standard sphere $S^7$ iff the Euler class of the bundle  $e(SE_{hj}) = \pm 1$, discussions in previous sections show that the Eells--Kuiper $\mu$ invariant applies to these topological spheres. Since by reversing the orientation of fiber we obtain  $\overline{E_{hj}} = E_{-j,-h}$, it suffices to consider the case $h + j = 1$. 

Assume $h + j = 1, h-j = p = 2h-1$, and let $M_p = SE_{hj}$,  $B_p = DE_{hj}$, $E_{hj}$ be the sphere bundle, disk bundle and vector bundle associated to $\phi_{hj}$ respectively. 

Since $H^4(B_p;\mathbb{Z}) = H^4(B_p,M_p;\mathbb{Z}) = \mathbb{Z}$, the orientation may be chosen such that
\[
    \langle(j^{-1}\pi^*u)^2,[B_p,M_p]\rangle = 1\text{ and }\tau(B_p) = 1, 
\]
where $u$ is a generator of $H^4(S^4;\mathbb{Z})$, $\pi^* \colon H^4(S^4) \to  H^4(DE_{hj})$ is an isomorphism,  $\pi^*u$ is a generator of $H^4(DE_{hj};\mathbb{Z})$, and $j$ is the isomorphism defined before. This can be done because of the following lemma. 


\begin{lem}
    Let $M^{4k}$ be a compact oriented connected $4k$-manifold with connected boundary $\partial M$, such that $H^{2k-1}(M) = 0$. Assume all cohomology groups are free $\mathbb{Z}$-modules. Then the intersection form is unimodular iff $H_{2k}(\partial M) = 0$. 
\end{lem}
\proof{Assume first that the intersection form
\[
    Q \colon H^{2k}(M,\partial M) \otimes  H^{2k}(M,\partial M) \to  \mathbb{Z}, 
\]
which is dual to
\[
    Q \colon H_{2k}(M) \otimes  H_{2k}(M) \to  \mathbb{Z}, 
\]
is unimodular. Let $i_* \colon H_{2k}(\partial M) \to  H_{2k
}(M)$. Since $\forall $ fixed $a\in \operatorname{im} i_*,\forall b\in H_{2k}(M)$, $Q(a,b) = 0$, it follows that $\operatorname{im} i_* = 0$. By Poincar\'e duality and universal coefficient theorem, $H_{2k + 1}(M,\partial M) = H^{2k-1}(M) = \operatorname{Hom}(H_{2k-1}(M),\mathbb{Z})$, so $H_{2k-1}(M) = 0$ implies $H_{2k + 1}(M,\partial M) = 0$. From the long exact sequence
\begin{gather*}
    \cdots\to H_{2k + 1}(M,\partial M)  \to  H_{2k}(\partial M)\stackrel{i_*}{ \to } H_{2k}(M) \\
	\to H_{2k}(M,\partial M) \to  H_{2k-1}(\partial M) \to \cdots
\end{gather*}
we see that $\operatorname{im} i_* = 0$ implies $H_{2k}(\partial M) = 0$.  

Conversely, if $H_{2k}(\partial M) = 0$, 
then $H_{2k-1}(\partial M) = 0$. 
Thus $H_{2k}(M) \cong  H_{2k}(M,\partial M)$ and 
$H^{2k}(M) \cong  H^{2k}(M,\partial M)$. 
From Poincar\'e duality we have that 
\[
    H^{2k}(M) \otimes H^{2k}(M,\partial M) \to  \mathbb{Z}, 
	(a,b)\mapsto \langle a \cup  b,[M,\partial M]\rangle
\]
is nondegenerate. The result follows. 
\qedhere}



To calculate the first Pontrjagin class, we use the fact that $p_1(E_{hj}) = \pm 2(h-j)u\in H^4(S^4;\mathbb{Z})$ where $u$ is a generator, and find over $\mathbb{Q}$ that 
\[
    p_1(TDE_{hj}) = p_1(\pi^*E_{hj} \oplus \pi^*TS^4) = \pi^*p_1(E_{hj}) = \pm 2(h-j)\pi^*u. 
\]
Consequently 
\[
    p_1^2[B_{2h-1}] = 4(h-j)^2 = 4(2h-1)^2. 
\]
\[
    \mu(M_{2h-1}) \equiv \frac{(4(2h-1)^2-4)}{(2^{7} \times 7)} \equiv  \frac{h(h-1)}{56} \bmod 1. 
\]

\begin{lem}
    The possible values of ${h(h-1)/2} \in   \mathbb{Z}/28\mathbb{Z}$ are 
\[
    0,1,3,6,7,8,10,13,14,15,17,20,21,22,24,27. 
\]
\end{lem}
\proof{Note that $h(h-1) \equiv  m(m-1) \bmod56$ iff $(m + h-1)(m-h) \equiv 0 \bmod 56$ iff
\[
    (m + h-1)(m-h) \equiv 0 \bmod 8 \text{ and } (m + h-1)(m-h) \equiv 0 \bmod 7. 
\]
Thus it suffices to calculate for the $16$ values of $h$ such that 
\[
    h \equiv  1,2,3,4 \bmod 7  \text{ and } h \equiv  1,2,3,4 \bmod 8. 
\]
This lemma follows from direct calculation. \qedhere}

Now we have found $16$ exotic spheres in the collection of sphere bundles over spheres. Actually we can construct more. 
\begin{thm}
    There are at least $28$ different differentiable structures on the $7$-dimensional topological sphere. 
\end{thm}
\proof{Let $h = 2$. Since  $\mu(M_3) \equiv 1/28 \bmod 1$, $\forall 1\leq m \leq 28$, the $\mu$ invariant of the connected sum of $m$ copies of $M_3$ is $m/28 \bmod 1$. \qedhere}

Indeed, there are exactly $28$ different differentiable structures on the $7$-dimensional topological sphere. Thus two such manifolds are diffeomorphic iff their $\mu$ are the same. 

\begin{cor}
    The group $\Theta_7$ formed by these spheres is a cyclic group. 
\end{cor}

\begin{cor}
    The $16$ spheres of the form $M_p$ admit infinitely many essentially different differentiable fibrations by $S^3$. 
\end{cor}
\proof{For each of the values in the lemma above, there are infinitely many values of $h$ corresponding to it. \qedhere}


\begin{prop}
    For any $7$-dimensional manifold $M$ in the domain of $\mu$, the underlying topological manifold has at least $28$ different differentiable structures. 
\end{prop}
\proof{Consider $M \#  (M_3 \#   \dots   \#  M_3)$ where there are $m$ copies of $M_3$. The underlying topological manifolds are the same, but their $\mu$ invariants vary among $28$ different values in $\mathbb{Q}/\mathbb{Z}$. \qedhere}

\begin{prop}
    There are at least $14$ different differentiable structures on the $7$-dimensional real projective space $\mathbb{R}P^7$. 
\end{prop}
\proof{Consider $\mathbb{R}P^7 \#  (M_3 \#   \dots   \#  M_3)$ where there are $m$ copies of $M_3$. Its universal cover is the connected sum of $2m$ copies of $M_3$, with $\mu$ invariant $2m/28 = m/14\bmod 1$. \qedhere}

It is proved in Milnor's paper in \cite{milnorspin} that there are at least $28$ different differentiable structures on  $\mathbb{R}P^7$. 




\section{Applications to \texorpdfstring{$S^1$}{S1} bundles over homotopy \texorpdfstring{$\mathbb{C}P^3$}{CP3}}

Consider certain circle bundles over homotopy $\mathbb{C}P^3$ whose total space is homeomorphic to $S^7$. The purpose of this section is to use Eells--Kuiper invariant to find out the diffeomorphism classes of these bundles over $\mathbb{C}P^3$, or more generally over homotopy $\mathbb{C}P^3$. 

\subsection{Characteristic classes of bundles over \texorpdfstring{$\mathbb{C}P^3$}{CP3}}

Circle bundles over homotopy $\mathbb{C}P^3$ are completely determined by their Chern classes, which can be identified with an integer. Recall for example that $\mathcal{O}(-1)$ is the tautological line bundle, and $\mathcal{O}(1)$ is the hyperplane bundle. 

Denote the total space of the line bundle $\mathcal{O}(d)$ by $E_d$, and denote the corresponding disk bundle and sphere bundle by $DE_d$ and $SE_d$ respectively. 
\begin{thm}
	$SE_d$ is homeomorphic to $S^7$ iff $d = \pm 1$. 
\end{thm}
\proof{From the cell structure we see that $\mathbb{C}P^3$ is simply connected. The second page $E_2$ of the Serre spectual sequence associated to $SE_d$ is
	\[
	\xymatrix{
		\mathbb{Z}\ar[rrd]^d&0&\mathbb{Z}\ar[rrd]^d&0&\mathbb{Z}\ar[rrd]^d&0&\mathbb{Z}\\
		\mathbb{Z}&0&\mathbb{Z}&0&\mathbb{Z}&0&\mathbb{Z}
	}
	\]
	So $SE_d$ is a homology sphere iff $d = \pm 1$. 
	
	If $SE_d$ is a homology sphere, the first nontrivial homotopy group is $\pi_7(SE_d) = \mathbb{Z}$. Take a generator of $\pi_7(SE_d) = \mathbb{Z}$ and we see that this map induces isomorphisms in homology groups between simply connected manifolds (proved in the next lemma), so this map gives rise to a homotopy equivalence. Thus $SE_d$ is a homotopy $7$-sphere, which implies that $SE_d$ is homeomorphic to $S^7$. 
	\qedhere}

\begin{lem}
	\[
	\pi_1(SE_d) = \left \{ 
	\begin{array}{ll}
	\mathbb{Z}, & \text { if } d = 0 \\
	0, & \text { if } d = \pm 1\\
	\mathbb{Z}/|d|\mathbb{Z}, 
	& \text{otherwise.}\\
	\end{array}\right. 
	\]
\end{lem}

\proof{From the long exact sequence
	\[
	\cdots\to\pi_2(\mathbb{C}P^3) \to  \pi_1(S^1) \to  \pi_1(SE_d) \to \pi_1(\mathbb{C}P^3) \to \cdots
	\]
	where $\pi_2(\mathbb{C}P^3) = \pi_1(S^1) = \mathbb{Z}$, $\pi_1(\mathbb{C}P^3) = 0$, we see that $\pi_1(SE_d)$ is a quotient of $\mathbb{Z}$ and hence is abelian, so $\pi_1(SE_d) = H_1(SE_d;\mathbb{Z}) = H^6(SE_d;\mathbb{Z})$. The result follows from the Serre spectual sequence associated to $SE_d$. 
	\qedhere}



Let $\pi$ be the bundle projection.  Let $v\in H^2(\mathbb{C}P^3;\mathbb{Z})$ be the first Chern class of the hyperplane bundle. Then 
\[
H^*(\mathbb{C}P^3;\mathbb{Z}) = \mathbb{Z}[v]/(v^4). 
\]

\begin{lem}
	$w_2(TDE_{d}) \equiv  d\cdot\pi^*v \bmod 2$. Consequently, $DE_{d}$ is spin iff $d$ is even, and in particular $DE_{-1}$ is not spin. 
\end{lem}
\proof{
\begin{gather*}
w_2(TDE_{d}) = \pi^*(w_2(E_{d} \oplus  T\mathbb{C}P^3)) 
= \pi^*(w_2(E_{d}) + 0 + w_2( T\mathbb{C}P^3)) \\
\equiv \pi^*(dv + 4v) \equiv  d\cdot\pi^*v \bmod 2.
\end{gather*}
Here we use the fact that when considering the real bundle underlying a complex rank $n$ vector bundle $E$, 
$w_2(E) \equiv  c_1(E)\bmod 2$. 
\qedhere}

It is possible to apply the Eells--Kuiper invariant to find out the diffeomorphism classes of the homotopy spheres $SE_1$ and $SE_{-1}$, which are spin. Also since $E_1$ is just the complex conjugate of $E_{-1}$, they are the same as real vector bundles and thus $SE_1$ and $SE_{-1}$ are diffeomorphic, via an orientation reversing diffeomorphism. It suffices to calculate the Eells--Kuiper invariant for $SE_{-1}$. However, $DE_{-1}$ is not spin. We cannot apply the formula in previous sections directly to the pair $(DE_{-1},SE_{-1})$. 


















\subsection{A generalization of \texorpdfstring{$\mu$}{mu} for nonspin case}

We need to calculate $\mu$ for a nonspin coboundary when considering disk bundles with boundary circle bundles over homotopy $\mathbb{C}P^3$. This follows from \cite{kreckstolz2}, \cite{kreckstolz}. 


Let $M$ be a $7$-dimensional nonspin closed manifold with 
$H^4(M; \mathbb{Q})  =  0$, 
together with a class $u  \in   H^2(M; \mathbb{Z})$ whose $\bmod  \ 2$ 
reduction is $w_2(M)$. Assume there exist an $8$-dimensional manifold
$W$ and elements $z,c \in H^2(W; \mathbb{Z})$ 
restricting to $u,0$ respectively, 
such that $w_2(W) = z + c \bmod 2$. 

Similarly, if $M$ is a $7$-dimensional spin closed manifold with 
$H^4(M; \mathbb{Q})  =  0$, 
take $u = 0$, $z = 0$ and $c  \in   H^2(W; \mathbb{Z})$ 
restricting to $0$, such that $w_2(W) = c \bmod 2$. 

Define invariants $s_i(M, u) = S_i(W,z,c) \in   \mathbb{Q}/\mathbb{Z}, i = 1,2,3$ as follows: 
\[
\begin{aligned}
S_{1}(W, z, c)& = \left\langle e^{(c + z) / 2} \widehat{A}(W),[W, \partial W]\right\rangle \\
S_{2}(W, z, c)& = \left\langle\operatorname{ch}(\lambda(z)-1) e^{(c + z) / 2} \widehat{A}(W),[W, \partial W]\right\rangle \\
S_{3}(W, z, c)& = \left\langle\operatorname{ch}(\lambda^{2}(z)-1) e^{(c + z) / 2} \widehat{A}(W),[W, \partial W]\right\rangle.
\end{aligned}
\]

Here $\lambda(z)$ is the complex line bundle over $W$ with first Chern class $z$, $\operatorname{ch}$ is the Chern character, and $\widehat{A}(W)$ is the $\widehat{A}$-polynomial of $W$. Since we are requiring $H^4(M; \mathbb{Q})  =  0$, $p_1$ and $z^2$ can be considered as elements of $H^4(W,M; \mathbb{Q})  =  0$ and thus $p_1^2$, $p_1z^2$ and $z^4$ can be evaluated on the fundamental class $[W, \partial W] = [W, M]$. However, $p_2$ appears in $e^{(c + z) / 2} \widehat{A}(W)$, which cannot be considered as a relative class in general. Our strategy is the same as for $\mu$ invariant, using the signature of $W$ to eliminate $p_2$ from the expression, i.e.\ replacing some constant multiple of $L(W)$ (evaluated on the fundamental class) by some constant multiple of $\tau(W)$, so that $p_2$ term disappears. 

These characteristic numbers for closed manifolds are integers. For a proof, see Theorem 26.1.1. in \cite{hirzebruchtopological}. It follows that $s_i(M, u) = S_i(W,z,c) \in   \mathbb{Q}/\mathbb{Z}, i = 1,2,3$ are well-defined. 

Note that, if $W$ is spin and $M$ has the induced spin structure, we may take $z = 0$, $c = 0$ and thus 
\[
s_1(M, 0) = \langle \widehat{A}(W),[W, \partial W]\rangle = \mu(M). 
\]
So $s_1$ is a generalization of $\mu$. 

If $c = 0$, explicit formulas are:
for spin case
\[
S_{1}(W, z,0) = -\frac{1}{2^{5}  \cdot 7}  \tau(W) + \frac{1}{2^{7}  \cdot 7} p_{1}^{2},  
\]
for nonspin case
\[
S_{1}(W, z,0) = -\frac{1}{2^{5}  \cdot 7} \tau(W) + \frac{1}{2^{7}  \cdot 7} p_{1}^{2} 
-\frac{1}{2^{6}  \cdot 3} z^{2} p_{1} + \frac{1}{2^{7}  \cdot 3} z^{4} , 
\]
where, by abuse of notation, 
everything is considered as a relative cohomology class and is evaluated on $[W, \partial W] = [W, M]$. If $M$ is spin but $W$ is not, then
\[
S_{1}(W, z,c) = -\frac{1}{2^{5}  \cdot 7} \tau(W) + \frac{1}{2^{7}  \cdot 7} p_{1}^{2} 
-\frac{1}{2^{6}  \cdot 3} (z + c)^{2} p_{1} + \frac{1}{2^{7}  \cdot 3} (z + c)^{4} . 
\]

\subsection{Calculation for \texorpdfstring{$SE_{-1}$}{SE{-1}}}
Take $u = 0$, $z = 0$, $c = \pi^*v$. Since $H^4(DE_{-1};\mathbb{Z}) \cong  H^4(DE_{-1},SE_{-1};\mathbb{Z}) = \mathbb{Z}$, the orientation may be chosen so that
\[
\langle(j^{-1}\pi^*v)^4,[DE_{-1},SE_{-1}]\rangle = 1\text{ and }\tau(DE_{-1}) = 1, 
\]
where $\pi^*v^2\in H^4(DE_{-1};\mathbb{Z})$ is a generator, and $v\in H^2(\mathbb{C}P^3;\mathbb{Z})$ denotes the first Chern class of the hyperplane bundle. This can be done because in this case the intersection form is unimodular, which follows from Lemma 4. A different choice of orientation leads to negative of the final result. 

To calculate the $s_1$ invariant, we need to calculate the first Pontrjagin class from Chern classes: 
\[
p_1(TDE_{-1}) = \pi^*(p_1(E_{-1}) + p_1(T\mathbb{C}P^3)) = \pi^*(v^2 + 4v^2) = 5\pi^*v^2. \]

Substituting into the last formula in the previous section, we find
\[
\mu(SE_{-1}) = s_1(SE_{-1},0) = \frac{1}{32}(-\frac{1}{7} + \frac{25}{28}-\frac{5}{6} + \frac{1}{12}) = 0. 
\]
This implies
\begin{thm}
	$SE_{-1}$ and $SE_{1}$ are diffeomorphic to $S^7$ with the standard differentiable structure. 
\end{thm}




\subsection{Calculation for spin manifolds homotopy equivalent to \texorpdfstring{$\mathbb{C}P^3$}{CP3}}

In \cite{wall}, a classification of certain $6$-manifolds is given by Theorem 3 and Theorem 5. For spin manifolds homotopy equivalent to $\mathbb{C}P^3$, this is restated as follows:
\begin{thm}
	Diffeomorphism classes of closed, smooth, simply-connected spin $6$-manifolds $M$ homotopy equivalent to $\mathbb{C}P^3$ correspond bijectively to isomorphism classes of the homomorphism
	\[
	p_1 \colon H^2(M;\mathbb{Z}) \cong  H_4(M;\mathbb{Z}) \cong  \mathbb{Z} \to  \mathbb{Z}
	\]
	subject to 
	\[
	\forall x\in H^2(M;\mathbb{Z}), p_1(x) \equiv  4\mu(x,x,x) \bmod 24. 
	\]
\end{thm}

Equivalently, these manifolds are characterized by their first Pontrjagin classes satisfying $ p_1(TM) = (24k  +  4)v^2,k\in \mathbb{Z}$, where $v\in H^2(M;\mathbb{Z})$ corresponds to $v\in H^2(\mathbb{C}P^3;\mathbb{Z})$ defined above. Write $X_k$ for this manifold. Note that $X_0$ is just $\mathbb{C}P^3$. 

Again circle bundles over $X_k$ are completely determined by their Chern classes, which can be identified with an integer. 
Write $E_d$ for the complex line bundle over $X_k$ with first Chern class $dv$, and define $SE_d, DE_d$ in the same way. One verifies that only $SE_{1}$ and $SE_{-1}$ are homeomorphic to $S^7$ as in the previous sections, and it suffices to consider $SE_{-1}$. 

$X_k$ is spin, so $w_2(TDE_d) \equiv  d\cdot \pi^* v \bmod 2$ and it is zero iff $d$ is divisible by $2$, and in particular $DE_{-1}$ is not spin. $SE_{-1}$ is spin, being a homotopy sphere. 

Take $u = 0$, $z = 0$, $c = \pi^*v$. 
Since 
\[
H^4(DE_{-1};\mathbb{Z}) \cong  H^4(DE_{-1},SE_{-1};\mathbb{Z}) = \mathbb{Z},
\] 
the orientation may be chosen so that
\[
\langle(j^{-1}\pi^*v)^4,[DE_{-1},SE_{-1}]\rangle = 1\text{ and }\tau(DE_{-1}) = 1, 
\]
where $\pi^*v^2\in H^4(DE_{-1};\mathbb{Z})$ is a generator. This can be done by the same reason. 

The first Pontrjagin class is 
\begin{gather*}
p_1(TDE_{-1}) = \pi^*(p_1(E_{-1}) + p_1(TX_k)) = \\
\pi^*(v^2 + (24k  +  4)v^2) = (24k  +  5)\pi^*v^2. 
\end{gather*}

The $s_1$ invariant is 

\begin{gather*}
\mu(SE_{-1}) = s_1(SE_{-1},0) = -\frac{1}{2^{5}  \cdot 7}  + \frac{ (24k  +  5)^2}{2^{7}  \cdot 7} 
-\frac{(24k  +  5)}{2^{6}  \cdot 3} + \frac{1}{2^{7}  \cdot 3}  \\
= \frac{9}{14}k^2 + \frac{1}{7}k\bmod 1 . 
\end{gather*}

Equivalently, we may consider
\[
\mu(SE_{-1}) = 18k^2 + 4k\bmod 28. 
\]

For $k$ ranging from $0$ to $27$, the values are
\begin{gather*}
0,22,24,6,24,22,0,14,8,10,20,10,8,14,0 \\
,22,24,6,24,22,0,14,8,10,20,10,8,14. 
\end{gather*}

So the possible values of $\mu(SE_{-1})$ 
(with the orientation chosen in the proof above) are 
\[
0,6,8,10,14,20,22,24. 
\]

Note that $\mu(SE_{-1})$ and $\mu(SE_{1})$ are diffeomorphic via an orientation reversing diffeomorphism. The possible values of $\mu(SE_{-1})$ and $\mu(SE_{1})$ are 
\[
0,4,6,8,10,14,18,20,22,24. 
\]

In \cite{yijiang}, Corollary 4.1 states that
\begin{prop}
	Among the 28 homotopy $7$-spheres $\Sigma_r,0 \leq r \leq 27$,  the following ones admit smooth regular circle actions
	\[
	\Sigma_r,r = 0,4,6,8,10,14,18,20,22,24. 
	\]
\end{prop}

Thus we arrive at the final result
\begin{thm}
	Among the 28 homotopy $7$-spheres $\Sigma_r,0 \leq r \leq 27$,  the following ones admit smooth regular circle actions
	\[
	\Sigma_r,r = 0,4,6,8,10,14,18,20,22,24. 
	\]
	Each of these $\Sigma_r$ can be realized as the total space of  a principal $S^1$-bundle over some homotopy $\mathbb{C}P^3$ with primitive Euler class, and hence admits smooth regular $S^1$-actions. 
\end{thm}










































\section{Further Discussions}

For more information about $S^1$-actions on homotopy spheres, see \cite{hsiang}, \cite{my} and  \cite{rs}.  


The topology of sphere bundles over spheres has been discussed in \cite{james1}, \cite{james2}. A complete classification of
$S^3$-bundles over $S^4$ is done in \cite{crowley}. 

In \cite{kervairemilnor}, the group $\Theta_n$ of homotopy  spheres is discussed. The h-cobordism classes of homotopy $n$-spheres form an abelian group $\Theta_n$ under connected sum operation. Actually, for $n \geq  5$ homotopy $n$-spheres are indeed homeomorphic to $S^n$, and two homotopy $n$-spheres are h-cobordant if and only if they are diffeomorphic. Consequently $\Theta_n$ describes exotic spheres of dimension $n$, for $n \geq  5$. 

A generalized version of the Eells--Kuiper invariant is the Kreck--Stolz $s$-invariant. See \cite{moduli}. 





\nocite{*}

\printbibliography

\end{document}