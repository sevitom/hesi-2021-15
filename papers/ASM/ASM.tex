\documentclass[twoside]{article}

\usepackage{geometry}
\geometry{
    paperwidth = 155mm,
    paperheight = 235mm,
    outer = 20mm,
    inner = 20mm,
    top = 25mm,
    bottom = 20mm
}

% fonts & unicode
\usepackage[PunctStyle=kaiming]{xeCJK}
\usepackage{amsmath}
\usepackage{unicode-math}

\setCJKmainfont{NotoSerifCJKsc-Regular.otf}[
    Path            = ../../fonts/,
    BoldFont        = NotoSansCJKsc-Medium.otf,
    ItalicFont      = fzktk.ttf,
    Scale           = .97,
    ItalicFeatures  = {Scale = 1}
]

\setCJKsansfont{NotoSansCJKsc-DemiLight.otf}[
    Path            = ../../fonts/,
    BoldFont        = NotoSansCJKsc-Bold.otf,
    Scale           = .97
]

\setCJKmonofont{NotoSansCJKsc-DemiLight.otf}[
    Path            = ../../fonts/,
    BoldFont        = NotoSansCJKsc-Bold.otf,
    Scale           = .9
]

\newCJKfontfamily{\KaiTi}{fzktk.ttf}[
    Path            = ../../fonts/,
    BoldFont        = NotoSansCJKsc-Medium.otf,
    BoldFeatures    = {Scale = .97},
    ItalicFont      = NotoSerifCJKsc-Regular.otf,
    ItalicFeatures  = {Scale = .97}
]

\setmainfont{XITS}[
    Path            = ../../fonts/,
    Extension       = .otf,
    UprightFont     = *-Regular,
    BoldFont        = *-Bold,
    ItalicFont      = *-Italic,
    BoldItalicFont  = *-BoldItalic
]

\setsansfont{Lato}[
    Path            = ../../fonts/,
    Scale           = MatchUppercase,
    Extension       = .ttf,
    UprightFont     = *-Regular,
    BoldFont        = *-Bold,
    ItalicFont      = *-Italic,
    BoldItalicFont  = *-BoldItalic
]

\setmonofont{FiraMono}[
    Path            = ../../fonts/,
    Scale           = .9,
    Extension       = .otf,
    UprightFont     = *-Regular,
    BoldFont        = *-Bold
]

\setmathfont{XITSMath-Regular.otf}[
    Path            = ../../fonts/,
    BoldFont        = XITSMath-Bold.otf
]

\setmathfont{latinmodern-math.otf}[
    Path            = ../../fonts/,
    range           = {frak, bffrak},
    BoldFont        = latinmodern-math.otf
]

\setmathfont{LatoMath.otf}[
    Path            = ../../fonts/,
    Scale           = .95,
    BoldFont        = LatoMath.otf,
    version         = sf
]

\setmathfont{LatoMath.otf}[
    Path            = ../../fonts/,
    Scale           = .95,
    BoldFont        = LatoMath.otf,
    range           = {bb, sfup -> up, sfit -> it, bfsfup -> bfup, bfsfit -> bfit}
]

\setmathfont{STIX2Math.otf}[
    Path            = ../../fonts/,
    BoldFont        = STIX2Math-Bold.otf,
    range           = {\int, \sum, \prod, \coprod, \bigoplus, \bigotimes, \bigcup, \bigcap, \bigvee, \bigwedge}
]

\Umathcode`/  =  "0 "0 "2215    % / -> U+2215 division slash

% headers and footers
\usepackage{fancyhdr}
\fancyhf{}
\fancyhead[CE]{\sf\mathversion{sf}\theheadertitle}
\fancyhead[CO]{\sf\mathversion{sf}\nouppercase{\leftmark}}
\fancyhead[LE,RO]{\textbf{\textsf{\thepage}}}
\headsep=8mm
\headheight=6mm

\AtBeginDocument{
    \renewcommand{\thepage}{\roman{page}}
    \pagestyle{fancy}\thispagestyle{empty}
}

% spacing
\AtBeginDocument{
    \hfuzz=2pt
    \emergencystretch 2em
    \setlength{\belowdisplayshortskip}{\belowdisplayskip}
}

% parts
\usepackage{tikz}

\renewcommand{\titlepage}[2]{%
    \clearpage%
    \thispagestyle{empty}%
    \vspace*{20mm}%
    \centerline{\begin{tikzpicture}
        \node [scale = 3] at (0, 0) {\sffamily 荷\hspace{.5em}思};
        \node [scale = 1.8] at (0, -94.5mm) {\sffamily #1};
        \node [scale = 1.2] at (0, -105mm) {\sffamily 第 #2 期};
        \draw (-16mm, -8mm) -- (16mm, -8mm);
        \draw (-14mm, -100mm) -- (14mm, -100mm);
        \draw (-8mm, -110mm) -- (8mm, -110mm);
    \end{tikzpicture}}%
    \clearpage%
}

\newcommand{\committee}{%
    \clearpage%
    \thispagestyle{empty}%
    \vspace*{120mm}%
}

\newcommand{\committeeitem}[2]{%
    \par%
    {%
        \leftskip=3em%
        \rightskip=8em%
        \parindent=-3em%
        {\bfseries\sffamily#1}\quad%
        {\sffamily#2}%
        \par\vspace{6pt}%
    }%
}

\newcommand{\toctitle}{%
    \clearpage%
    \thispagestyle{empty}%
    \vspace*{15mm}%
    \noindent{\huge\sffamily 目录}%
    \par\vspace{10mm}%
}

\newcommand{\tocsection}[1]{%
    \par\vspace{6mm}%
    \noindent{\large\sffamily #1}%
    \par\vspace{4mm}%
}

\newcommand{\tocitem}[3]{%
    \par%
    {%
        \leftskip=3em%
        \parindent=-3em%
        \makebox[2em][r]{\textbf{\textsf{#3}}}%
        \quad#1%
        \hfill\mbox{}\hfill\phantom{#2}\hfill\makebox[0em][r]{#2}%
        \par\vspace{8pt}%
    }%
}

\usepackage{indentfirst}
\parindent=2em

% names in chinese
\def\abstractname{摘\quad 要}
\def\contentsname{目录}
\def\proofname{证明}

% bibliography
\DefineBibliographyStrings{english}{
    references = {参考文献},
}

% spacing
\usepackage{setspace}
\setdisplayskipstretch{} %https://tex.stackexchange.com/q/529214

\AtBeginDocument{
    \spacing{1.25}
}
\AtBeginBibliography{
    \spacing{1}
}

% lengths
\AtBeginDocument{
    \setlength{\abovedisplayskip}{8pt plus 4pt minus 4pt}
    \setlength{\belowdisplayskip}{8pt plus 4pt minus 4pt}
    \setlength{\belowdisplayshortskip}{8pt plus 4pt minus 4pt}
}

% theorems & proofs
\newtheoremstyle{cjk-theorem}{}{}{\KaiTi}{}{\bfseries}{.}{.5em}{}
\newtheoremstyle{cjk-definition}{}{}{}{}{\bfseries}{.}{.5em}{}
\newtheoremstyle{cjk-remark}{}{}{}{}{\KaiTi}{.}{.5em}{}

\theoremstyle{cjk-theorem}
\renewtheorem{theorem}{定理}[section]
\renewtheorem{lemma}[theorem]{引理}
\renewtheorem{corollary}[theorem]{推论}
\renewtheorem{proposition}[theorem]{命题}

\theoremstyle{cjk-definition}
\renewtheorem{definition}[theorem]{定义}
\renewtheorem{remark}[theorem]{注}
\renewtheorem{example}[theorem]{例}
\theoremstyle{cjk-theorem}

\numberwithin{equation}{theorem}

\makeatletter
\renewenvironment{proof}[1][\proofname]{\par
    \pushQED{\qed}%
    \normalfont \topsep6\p@\@plus6\p@\relax
    \trivlist
    \item\relax{\bfseries#1}\hspace{1em}\ignorespaces
}{\popQED\endtrivlist\@endpefalse}
\makeatother

% customized numbering for Step
\usepackage{cleveref}
\newtheorem{mstep}{Step}
\newtheorem*{convention}{convention}
\newenvironment{step}[1]{%
	\renewcommand\themstep{#1}%
	\mstep
}{\endmstep}
\crefname{step}{Step}{Steps}

% math letters
\newcommand{\Ab}{{\mathbb{A}}}
\newcommand{\Bb}{{\mathbb{B}}}
\newcommand{\Cb}{{\mathbb{C}}}
\newcommand{\Db}{{\mathbb{D}}}
\newcommand{\Eb}{{\mathbb{E}}}
\newcommand{\Fb}{{\mathbb{F}}}
\newcommand{\Gb}{{\mathbb{G}}}
\newcommand{\Hb}{{\mathbb{H}}}
\newcommand{\Ib}{{\mathbb{I}}}
\newcommand{\Jb}{{\mathbb{J}}}
\newcommand{\Kb}{{\mathbb{K}}}
\newcommand{\Lb}{{\mathbb{L}}}
\newcommand{\Mb}{{\mathbb{M}}}
\newcommand{\Nb}{{\mathbb{N}}}
\newcommand{\Ob}{{\mathbb{O}}}
\newcommand{\Pb}{{\mathbb{P}}}
\newcommand{\Qb}{{\mathbb{Q}}}
\newcommand{\Rb}{{\mathbb{R}}}
\newcommand{\Sb}{{\mathbb{S}}}
\newcommand{\Tb}{{\mathbb{T}}}
\newcommand{\Ub}{{\mathbb{U}}}
\newcommand{\Vb}{{\mathbb{V}}}
\newcommand{\Wb}{{\mathbb{W}}}
\newcommand{\Xb}{{\mathbb{X}}}
\newcommand{\Yb}{{\mathbb{Y}}}
\newcommand{\Zb}{{\mathbb{Z}}}
\newcommand{\Ac}{{\mathcal{A}}}
\newcommand{\Bc}{{\mathcal{B}}}
\newcommand{\Cc}{{\mathcal{C}}}
\newcommand{\Dc}{{\mathcal{D}}}
\newcommand{\Ec}{{\mathcal{E}}}
\newcommand{\Fc}{{\mathcal{F}}}
\newcommand{\Gc}{{\mathcal{G}}}
\newcommand{\Hc}{{\mathcal{H}}}
\newcommand{\Ic}{{\mathcal{I}}}
\newcommand{\Jc}{{\mathcal{J}}}
\newcommand{\Kc}{{\mathcal{K}}}
\newcommand{\Lc}{{\mathcal{L}}}
\newcommand{\Mc}{{\mathcal{M}}}
\newcommand{\Nc}{{\mathcal{N}}}
\newcommand{\Oc}{{\mathcal{O}}}
\newcommand{\Pc}{{\mathcal{P}}}
\newcommand{\Qc}{{\mathcal{Q}}}
\newcommand{\Rc}{{\mathcal{R}}}
\newcommand{\Sc}{{\mathcal{S}}}
\newcommand{\Tc}{{\mathcal{T}}}
\newcommand{\Uc}{{\mathcal{U}}}
\newcommand{\Vc}{{\mathcal{V}}}
\newcommand{\Wc}{{\mathcal{W}}}
\newcommand{\Xc}{{\mathcal{X}}}
\newcommand{\Yc}{{\mathcal{Y}}}
\newcommand{\Zc}{{\mathcal{Z}}}

% general notation
\newcommand{\id}{{\mathrm{id}}}
\newcommand{\im}{\operatorname{im}}
\newcommand{\coker}{\operatorname{coker}}
\newcommand{\coim}{\operatorname{coim}}
\newcommand{\codim}{\operatorname{codim}}
\newcommand{\Hom}{\operatorname{Hom}}
\newcommand{\End}{\operatorname{End}}
\newcommand{\gfr}{{\mathfrak{g}}}
\newcommand{\GL}{{\mathrm{GL}}}
\newcommand{\SU}{{\mathrm{SU}}}
\newcommand{\ad}{\operatorname{ad}}
\newcommand{\Ad}{\operatorname{Ad}}

% shorthands
\newcommand{\la}{\langle}
\newcommand{\ra}{\rangle}
\newcommand{\fr}{\forall\,}
\newcommand{\exi}{\exists\,}
\newcommand{\pa}{\partial}
\newcommand{\pab}{{\bar{\partial}}}
\newcommand{\ph}{\phantom{{}={}}}
\newcommand{\wt}[1]{\widetilde{#1}}
\newcommand{\wh}[1]{\widehat{#1}}
\newcommand{\ol}[1]{\overline{#1}}
\newcommand{\pfrac}[2]{\frac{\partial #1}{\partial #2}}

% special notation
\newcommand{\Crit}{\operatorname{Crit}}
\newcommand{\ind}{\operatorname{ind}}
\newcommand{\Sym}{\mathrm{Sym}}
\newcommand{\reg}{{\mathrm{reg}}}
\newcommand{\Morse}{{\mathrm{M}}}

\addbibresource{ASM.bib}
\nocite{*}

\begin{document}

\title{变号矩阵与可积系统\\\large{——从一道新领军测试题到现代数学}}
\author{徐凯\footnote{原清华大学数学系数 41 班.}}
\headertitle{变号矩阵与可积系统}

\begin{abstract}
    在这篇短文中,
    我们从变号矩阵的计数问题出发,
    将其表述为六顶点模型中态和的计算,
    并通过 Yang--Baxter 方程和可积性给出问题的显式解.
\end{abstract}

清华大学2021年新领军综合测试中有如下一道试题: 

\begin{problem}
    要求矩阵满足条件:  (1) 只有三种元素$-1,0,1$;  (2) 每一行每一列不全为0;  (3) 每行每列在去掉所有 $0$ 之后, 剩下的元素均形如 $1,-1,1,\dotsc,-1,1$. 已知这样的三阶矩阵有7个, 求四阶的有多少. 
\end{problem}

这道题目虽然表述完全初等, 但背后蕴藏非常丰富的结构, 是中学阶段的同学接触现代数学之深邃高明十分难得的机会, 故撰此文聊作剖析, 以为引玉之砖. 本文主要证明来自G.Kuperberg. 


\section{变号矩阵到六顶点模型}
我们首先重新表述变号矩阵的计数问题,将其化归为统计力学中的六顶点模型. 

在六顶点模型中, 我们关心平面上每边带定向, 并且每个顶点恰好有两条边进入两条边离开的正方形图, 每一个这样的图称为一个态(state). 给定一个态, 在每个顶点附近有如下六种可能的定向: 
\[
    \sixvertex{$a$}{$b$}{$c$}{$d$}{$e$}{$f$}
\]

我们为每一种可能$i$赋一个权重$w(i)$, 每个顶点的权重相乘定义为图的权重, 而 (满足给定条件的) 所有态的权重之和称为态和(state sum). 六顶点模型中也可设置边界条件, 即引入只连一条 (给定) 定向边的顶点. 特别的, 我们可以考虑$n\times n$方形网格, 再加左右两侧向内上下两侧向外的边界条件, 这个模型称为方冰 (square ice). 一个方冰态可以用如下对应转化为一个变号矩阵: (注意这里的$-1,0,1$和权重无关)
$$
\sixvertex{$1$}{$-1$}{$0$}{$0$}{$0$}{$0$}
$$
并且可以验证这是一个一一对应. 因此, 变号矩阵的计数问题即等价于六顶点模型所有权重为1的态和. 故我们只需求解六顶点模型即可. 六顶点模型可以严格求解, 其中的核心结构为Yang--Baxter方程 (以下简称YBE). 

\section{YBE的场论来源}
这一节中我们简短介绍YBE在场论中的起源. 可积场论可以视为可积格点模型 (lattice model) 的连续极限. 譬如六顶点模型取时间方向的连续极限得到Heisenberg自旋链(spin chain), 时空均取连续极限得到正弦Gordon模型, 正弦Gordon模型即为最简单的可积场论之一. 本节内容与后文内容并无逻辑关联, 仅通过一个自然的来源以启发YBE这一概念, 故相对简略, 不感兴趣的读者可以跳过本节. 

 (相对论场论中) 可积性是一个仅存在于二维的独特现象, 事实上, Coleman--Mandula论证了在更高维度, 倘若存在自旋至少为2的守恒荷, 则我们总可以利用相应的对称性移动粒子到一般位置, 使得他们的轨迹永不相交, 不会发生散射因而$S$矩阵(即粒子散射过程中初始状态到最终状态的线性变换)一定为$1$.

但平面上两条一般位置的直线总会相交, 故以上论证不能成立. 然而我们总能通过移动至一般位置避免三条直线交于一点, 这时所有散射都为两两弹性碰撞, 所以全部散射振幅都由弹性碰撞的所决定. 另外, 给定三条定向直线$l,m,n$, 倘若$m,n$交点在$l$左侧, 那么我们总是可以利用前文的对称性将其移动到右侧. 这时得到的构型与先前并不拓扑等价, 二者散射振幅相等需要弹性碰撞的散射振幅满足一个三次方程, 即为YBE (这时动量是其中的一个谱参数). 

\section{六顶点模型的解}
取一个未定元$q$, 记$[x]=\frac{q^{x/2}-q^{-x/2}}{q^{1/2}-q^{-1/2}}$, 对一个带$x$标记的顶点
\[
    \begin{tikzpicture}[
        scale = .6,
        thick
    ]
        \begin{scope}[shift={(0,0)}]
            \draw (-1, 0)--( 0, 0);
            \draw ( 1, 0)--( 0, 0);
            \draw ( 0, 0)--( 0, 1);
            \draw ( 0, 0)--( 0,-1);
            \node at (-.4, -.4) {$x$};
        \end{scope}
    \end{tikzpicture}    
\]
我们为六种定向赋权如下: 
$$
\sixvertex{$-q^{-x/2}$}{$-q^{x/2}$}{$[x-1]$}{$[x-1]$}{$[x]$}{$[x]$}
$$

这可以视为指定了一个二维场论的弹性碰撞(大小为$2^2\times 2^2$的) $S$矩阵(在可积格点模型中通常称为$R$矩阵), 而YBE正是这个可积场论的相容性条件: 
\begin{theorem}[Baxter]  若 $x
= y + z$,  则 $R$-矩阵 $R(x)$, $R(y)$, 和 $R(z)$ 满足如下两图对应矩阵相等
\[
    \begin{tikzpicture}[
        scale = .7,
        thick
    ]
        \begin{scope}[shift={(0,0)}]
            \draw (-85:1.5) .. controls (-60:1) and (60:1) .. (85:1.5);
            \draw (35:1.5) .. controls (60:1) and (180:1) .. (205:1.5);
            \draw (155:1.5) .. controls (180:1) and (-60:1) .. (-35:1.5);
            \node at (-60:1.2) {$x$};
            \node at (60:1.2) {$y$};
            \node at (180:1.2) {$z$};
        \end{scope}
        \begin{scope}[shift={(4,0)}]
            \draw (95:1.5) .. controls (120:1) and (-120:1) .. (-95:1.5);
            \draw (-145:1.5) .. controls (-120:1) and (0:1) .. (25:1.5);
            \draw (-25:1.5) .. controls (0:1) and (120:1) .. (145:1.5);
            \node at (120:1.2) {$x$};
            \node at (-120:1.2) {$y$};
            \node at (0:1.2) {$z$};
        \end{scope}
    \end{tikzpicture}    
\]
\end{theorem}
我们首先解释何为两图对应的矩阵: 一言以蔽之, 他们是这个散射图对应的$S$矩阵. 每个图都有六条外边, 并有自然的一一对应. 他们共有64种定向, 每种定向都可以作为边界条件得到一个态和, 这正是这些矩阵的项. 

我们将权重的定义延拓到更一般的光滑曲线系统上.
对一种选定的定向方式,
约定每个切线水平向左的点, 若上凸则赋权$-q^{1/2}$,下凸则$-q^{-1/2}$,
切线水平向右的点则赋权$1$,
然后乘起来得到一个单项式.
把每种定向方式所得到的单项式加起来, 得到的多项式就是态和.
我们有如下的态和表达式:
\[
    \begin{tikzpicture}[
        scale = .5,
        thick
    ]
        \begin{scope}[shift={(0,0)}]
            \draw (-1, -1.2) -- (-1, 0) arc (180:0:.5) arc (180:360:.5) -- (1, 1.2);
        \end{scope}
        \node at (2, 0) {$=$};
        \begin{scope}[shift={(4,0)}]
            \draw (-1, 1.2) -- (-1, 0) arc (180:360:.5) arc (180:0:.5) -- (1, -1.2);
        \end{scope}
        \node at (6, 0) {$=$};
        \draw (7, 1.2) -- (7, -1.2);

        \draw (11, 0) circle (1);
        \node[inner sep = 0] at (16, 0) {$= -q^{1/2} - q^{-1/2} = -[2]$};
    \end{tikzpicture}  
\]
% \begin{equation}
% \pspicture(-1,0)(1,1)
% \qline(-.7,-1)(-.7,0)
% \psarc(-.35,0){.35}{0}{180}\psarc(.35,0){.35}{180}{0}
% \qline(.7,0)(.7,1)
% \endpspicture
% =
% \pspicture(-1.2,0)(1.2,1)
% \qline(.7,-1)(.7,0)
% \psarc(.35,0){.35}{0}{180}\psarc(-.35,0){.35}{180}{0}
% \qline(-.7,0)(-.7,1)
% \endpspicture
% =
% \pspicture(-.5,0)(.5,.7)
% \qline(0,-.7)(0,.7)
% \endpspicture
% \hspace{2cm}
% \pspicture(-.7,0)(.7,.5)
% \pscircle(0,0){.5}
% \endpspicture
% = -q^{1/2} - q^{-1/2} = -[2]
% \label{etl}
% \end{equation}
% \\
可以看出曲线的贡献只取决于它的同伦类, 并且$R(x)$可以表示为
\[
    \begin{tikzpicture}[
        scale = .5,
        thick
    ]
        \begin{scope}[shift={(0,0)}]
            \draw (-1, -1) -- (1, 1) (-1, 1) -- (1, -1);
        \end{scope}
        \node at (3.3, 0) {$= \quad [x]$};
        \begin{scope}[shift={(6,0)}]
            \draw (-1, 1) .. controls (0, 0) and (0, 0) .. (1, 1);
            \draw (-1, -1) .. controls (0, 0) and (0, 0) .. (1, -1);
        \end{scope}
        \node at (9.8, 0) {${} + \quad [x - 1]$};
        \begin{scope}[shift={(13,0)}]
            \draw (-1, 1) .. controls (0, 0) and (0, 0) .. (-1, -1);
            \draw (1, 1) .. controls (0, 0) and (0, 0) .. (1, -1);
        \end{scope}
    \end{tikzpicture}  
\]
% \\
% $$
% \pspicture(-.7,-0.3)(.7,.7)
% \psline(.7;45)(.7;225)\psline(.7;135)(.7;315)
% \rput[t](0,-.2828){$x$}\endpspicture
% = 
% [x]
% \pspicture(-.9,-0.3)(.9,.7)
% \pccurve[angleA= 45,angleB=135,ncurv=1,nodesep=0](.7;225)(.7;315)
% \pccurve[angleA=225,angleB=315,ncurv=1,nodesep=0](.7; 45)(.7;135)
% \endpspicture
% +[x-1]
% \pspicture(-.9,-0.3)(.9,.7)
% \pccurve[angleA=225,angleB=135,ncurv=1,nodesep=0](.7; 45)(.7;315)
% \pccurve[angleA= 45,angleB=315,ncurv=1,nodesep=0](.7;225)(.7;135)
% \endpspicture
% $$

因此一个态和可以表示为对曲线的求和, 其中每个闭圈贡献一个因子 $-[2]$ (这种计算规则起源于Temperley--Lieb范畴, 与 $\mathfrak{sl}_2$的量子群密切相关). 由此直接计算便可验证YBE. 
以下我们记
\[
    \begin{tikzpicture}[
        scale = .6,
        thick
    ]
        \begin{scope}[shift={(0,0)}]
            \draw (-1, 0)--( 0, 0);
            \draw ( 1, 0)--( 0, 0);
            \draw ( 0, 0)--( 0, 1);
            \draw ( 0, 0)--( 0,-1);
            \node at (-1.4, 0) {$x$};
            \node at (0, -1.4) {$y$};
        \end{scope}
        \node at (2, 0) {$=$};
        \begin{scope}[shift={(4,0)}]
            \draw (-1, 0)--( 0, 0);
            \draw ( 1, 0)--( 0, 0);
            \draw ( 0, 0)--( 0, 1);
            \draw ( 0, 0)--( 0,-1);
            \node at (-.8, -.4) {$x - y$};
        \end{scope}
    \end{tikzpicture}    
\]
% $$
% \pspicture(-.9,0)(.9,.9)
% \psline(0,0)(0,.7)    \psline(0,0)(.7,0)
% \psline(-0,0)(-.7,0)  \psline(0,-0)(0,-.7)
% \rput[r](-.9,0){$x$}
% \rput[t](0,-.9){$y$}
% \endpspicture
% =
% \pspicture(-1.6,0)(.7,.9)
% \psline(0,0)(0,.7)    \psline(0,0)(.7,0)
% \psline(-0,0)(-.7,0)  \psline(0,-0)(0,-.7)
% \rput[rt](-.2,-.2){$x-y$}
% \endpspicture
% $$

对于$X=(x_i)$, $Y=(y_i)$, 记$Z(n,X,Y)$为如下的态和
\[
    \begin{tikzpicture}[
        scale = .8,
        thick,
        decoration={
            markings,
            mark=at position 0.65 with {\arrow{stealth}}
        }
    ]
        \draw[postaction={decorate}] (-1, 0)--( 0, 0);
        \draw[postaction={decorate}] (-1,-1)--( 0,-1);
        \draw[postaction={decorate}] (-1,-3)--( 0,-3);
        \draw[postaction={decorate}] ( 4, 0)--( 3, 0);
        \draw[postaction={decorate}] ( 4,-1)--( 3,-1);
        \draw[postaction={decorate}] ( 4,-3)--( 3,-3);
        \draw[postaction={decorate}] ( 0, 0)--( 0, 1);
        \draw[postaction={decorate}] ( 1, 0)--( 1, 1);
        \draw[postaction={decorate}] ( 3, 0)--( 3, 1);
        \draw[postaction={decorate}] ( 0,-3)--( 0,-4);
        \draw[postaction={decorate}] ( 1,-3)--( 1,-4);
        \draw[postaction={decorate}] ( 3,-3)--( 3,-4);
        \draw (0, 0) -- (0,-1.5) (0,-2.5) -- (0,-3);
        \draw (1, 0) -- (1,-1.5) (1,-2.5) -- (1,-3);
        \draw (3, 0) -- (3,-1.5) (3,-2.5) -- (3,-3);
        \draw (0, 0) -- (1.5, 0) (2.5, 0) -- (3, 0);
        \draw (0,-1) -- (1.5,-1) (2.5,-1) -- (3,-1);
        \draw (0,-3) -- (1.5,-3) (2.5,-3) -- (3,-3);
        \node at ( 2,-.5) {$\cdots$};
        \node at ( 2, -3) {$\cdots$};
        \node at (.5, -2) {$\vdots$};
        \node at ( 3, -2) {$\vdots$};
        \node at ( 2, -2) {$\ddots$};
        \node at (0, -4.4) {$y_0$};
        \node at (1, -4.4) {$y_1$};
        \node at (3, -4.4) {$y_{n-1}$};
        \node at (-1.4, 0) {$x_0$};
        \node at (-1.4,-1) {$x_1$};
        \node at (-1.6,-3) {$x_{n-1}$};
    \end{tikzpicture}
\]

% $$
% \pspicture(-1.5,-1.5)(4,4)
% \pnode(-1, 4){aa}\pnode(0, 4){ab}\pnode(1, 4){ac}\pnode(1.5, 4){ad}
% \pnode(2.5, 4){ax}\pnode(3, 4){ay}\pnode(4, 4){az}
% \pnode(-1, 3){ba}\pnode(0, 3){bb}\pnode(1, 3){bc}\pnode(1.5, 3){bd}
% \pnode(2.5, 3){bx}\pnode(3, 3){by}\pnode(4, 3){bz}
% \pnode(-1, 2){ca}\pnode(0, 2){cb}\pnode(1, 2){cc}\pnode(1.5, 2){cd}
% \pnode(2.5, 2){cx}\pnode(3, 2){cy}\pnode(4, 2){cz}
% \pnode(-1,1.5){da}\pnode(0,1.5){db}\pnode(1,1.5){dc}\pnode(1.5,1.5){dd}
% \pnode(2.5,1.5){dx}\pnode(3,1.5){dy}\pnode(4,1.5){dz}
% \pnode(-1,.5){xa}\pnode(0,.5){xb}\pnode(1,.5){xc}\pnode(1.5,.5){xd}
% \pnode(2.5,.5){xx}\pnode(3,.5){xy}\pnode(4,.5){xz}
% \pnode(-1, 0){ya}\pnode(0, 0){yb}\pnode(1, 0){yc}\pnode(1.5, 0){yd}
% \pnode(2.5, 0){yx}\pnode(3, 0){yy}\pnode(4, 0){yz}
% \pnode(-1,-1){za}\pnode(0,-1){zb}\pnode(1,-1){zc}\pnode(1.5,-1){zd}
% \pnode(2.5,-1){zx}\pnode(3,-1){zy}\pnode(4,-1){zz}
% \ncline[nodesepA=.2]{ba}{bb}\mto \ncline{bb}{bc}\ncline{bc}{bd}
% \ncline{bx}{by}\ncline[nodesepB=.2]{by}{bz}\mfro
% \ncline[nodesepA=.2]{ca}{cb}\mto \ncline{cb}{cc}\ncline{cc}{cd}
% \ncline{cx}{cy}\ncline[nodesepB=.2]{cy}{cz}\mfro
% \ncline[nodesepA=.2]{ya}{yb}\mto \ncline{yb}{yc}\ncline{yc}{yd}
% \ncline{yx}{yy}\ncline[nodesepB=.2]{yy}{yz}\mfro
% \ncline[nodesepA=.2]{ab}{bb}\mfro\ncline{bb}{cb}\ncline{cb}{db}
% \ncline{xb}{yb}\ncline[nodesepB=.2]{yb}{zb}\mto 
% \ncline[nodesepA=.2]{ac}{bc}\mfro\ncline{bc}{cc}\ncline{cc}{dc}
% \ncline{xc}{yc}\ncline[nodesepB=.2]{yc}{zc}\mto 
% \ncline[nodesepA=.2]{ay}{by}\mfro\ncline{by}{cy}\ncline{cy}{dy}
% \ncline{xy}{yy}\ncline[nodesepB=.2]{yy}{zy}\mto 
% \rput[r](ba){$x_0$}\rput[r](ca){$x_1$}\rput[r](ya){$x_{n-1}$}
% \rput[t](zb){$y_0$}\rput[t](zc){$y_1$}\rput[t](zy){$y_{n-1}$}
% \rput(.5,1){$\vdots$}\rput(2,1){$\ddots$}\rput(2,2.5){$\cdots$}
% \rput(2,0){$\cdots$}\rput(3,1){$\vdots$}
% \endpspicture
% $$

$Z$关于$x_i$和$y_i$分别都是对称的: 这可以由YBE直接证明, 也可以用第二节中YBE的场论推导类似可得. 这对$Z$的结构施加了很强的限制, 通过仔细分析它的零点与极点, 我们可以显式解得
\begin{theorem}[Izergin, Korepin] 
    \label{thkorepin}
    态和 $Z(n;X,Y)$ 有如下表达
    $$Z(n;X,Y) = {(-1)^n\left(\prod_{i=0}^{n-1} q^{(y_i-x_i)/2}\right)
    \prod_{0 \le i,j < n} [x_i-y_j][x_i-y_j-1] \over \left(\prod_{0 \le j<i<n}
    [x_i-x_j]\right)\left(\prod_{0 \le i < j < n} [y_i-y_j]\right)}\det M,$$
    其中
    $$M_{i,j} = {1 \over [x_i - y_j][x_i - y_j - 1]}.$$
\end{theorem}

由第一节的论证我们知道$n\times n$变号矩阵的数量恰好为方冰态数量, 而方冰态中顶点$a$ 比$b$多$n$个, $c,d$数目相等, $e,f$也相等, 所以可得变号矩阵数目为
$$ \biggl(\frac12\biggr)^{n-n^2}(-1)^n q^{n/4} Z_{1/2}(n)|_{x=1},$$
其中$Z_{1/2}(n)=Z(n;1/2,...,1/2,0,...,0)$. 


这是定理2中表达式的 (可去) 奇点, 使用L'H\^opital法则我们可以直接计算得如下表达式(计算细节请参见\cite{10.1155/S1073792896000128}): 
\begin{theorem}
$n\times n$变号矩阵数目为
\[ \frac{1! \, 4! \, 7! \cdots (3n-2)!}{n! \, (n+1)! \, (n+2)! \cdots (2n-1)!}. \]
\end{theorem}
由此我们就解决了变号矩阵的计数问题. 

\section{总结}
可积系统对现代数学的深刻影响远不止于此: YBE的解通常称为$R$矩阵, 是量子群与辫张量范畴的核心信息, 对其与场论的关系的研究启发了大量重要的数学. 譬如著名的Kazhdan--Lusztig等价联系起量子群的$R$矩阵与共形场论中Knizhnik--Zamolodchikov方程解的单性(monodromy). 又如Maulik--Okounkov通过$R$矩阵构造了瞬子模空间上同调上的量子群作用, 给出了量子场论中AGT对应的数学验证. 希望本文对中学阶段的同学们有所启发, 借此机会了解现代数学中众多深刻精微的洞见. 

\printbibliography

\end{document}