\subsection{Divided powers and crystalline topoi}
A few basic facts about divided power rings are enumerated here for further applications. 
For details, see \cite{Sta}, \cite{BO}, \cite{Be2}.
\begin{definition}
    A divided power ring is a triple $(A, I, \gamma)$, 
    where $A$ is a ring, $I$ an ideal of $A$ and $\gamma_n, n \in \BN$ 
    a collection of maps $I \to A$ 
    such that those desired properties of ``$\frac {x^n} {n!}$'' hold 
    ( $x, y \in I,a \in A$ in the following):
    \begin{itemize} 
        \item 
            $\gamma_0(I) = 1,\quad \gamma_1(x) = x, 
            \quad \gamma_n(I) \subset I,\forall n > 0$.
        \item 
            $\gamma_n(x) \gamma_m(x) = 
            \frac {(n+m)!} {n!m!} \gamma_{n+m}(x)$.
        \item 
            $\gamma_n(x+y) = 
            \sum_{i = 0}^n \gamma_i (x) \gamma_{n-i}(y)$.
        \item 
            $\gamma_n(ax) = a^n \gamma(x)$.
        \item 
            $\gamma_n(\gamma_m(x)) = 
            \frac{(nm)!} {n!(m!)^n} \gamma_{nm}(x)$.
    \end{itemize}
    A morphism between two divided power rings, 
    $(A, I, \gamma)$, $(B, J, \delta)$, 
    is a ring homomorphism $f\colon (A,I) \to (B,J)$ such that 
    $f \circ \gamma_n = \delta_n \circ f$. 
    Let $(A, I, \gamma)$ be a divided power ring, 
    a divided power algebra over $A$ is a divided power ring $(B, J, \delta)$ with 
    a morphism $(A, I, \gamma)\to (B, J, \delta)$.
\end{definition}

All divided power rings form a category with initial object $(\BZ, 0, 0)$.
\begin{definition}
    Let $(A, I, \gamma)$ be a divided power ring, 
    $B$ a ring and $J$ an ideal of $B$. 
    The divided power envelope of $B$, $(D_{B, \gamma}(J), \bar{J}, \gamma)$ 
    is defined by the representable functor:
    \[
        (C, K, \delta) \mapsto Hom_A ((B, J), (C, K)).
    \]
    in the category of divided power algebras over $A$.
\end{definition}

\begin{example}
    Let $(A, I, \gamma) = (\BZ, 0, 0)$, $(B, J)=(\BZ[x], (x))$. 
    Then
    \[
        D_{B, \gamma}(J) = \BZ \langle x \rangle = 
        \{\sum_{\text{finite}} a_n \frac{x^n}{n!}, a_n \in \mathbb{Z}\}.
    \]
\end{example}

\begin{proposition}
    If $(A, I, \gamma)$ is a divided power ring with $m A = 0$, 
    then $J$ is nilpotent for any divided power algebra 
    $(B, J, \delta)$ over $(A, I, \gamma)$.
\end{proposition}

\begin{proof}
    $x^m = m! \gamma_m(x) = 0$.
\end{proof}

\begin{definition}
    Let $(A, I, \gamma),(B, J, \delta)$ be divided power rings, 
    $A \to B$ a ring morphism. 
    $\delta$ is called compatible with $\gamma$ 
    if $\delta$ extends to a divided power structure on $\bar{\delta}$ on $J + IB$ 
    such that $(A, I, \delta)\to (B, J, \gamma)$ 
    is a morphism between divided power rings. 
    In particular, if $(B, J, \delta)=(B, 0, 0)$, 
    we say that the divided power structure of $A$ extends to $B$.
\end{definition}
In particular, if $B$ is an $A / I$-algebra, 
the divided power structure of $A$ extends to $B$.
\begin{definition}
    A PD-derivation $d \colon B \to M$, 
    where $B$ is a divided power ring $(B, J, \delta)$ over a ring $A$, 
    is a usual derivation with $d \gamma_n(x) = \gamma_{n-1}(x) dx$. 
    Then there is a universal module $\tilde{\Omega}_{B/A}^1$ 
    and the PD-de Rham complex $\tilde{\Omega}_{B/A}^\bullet$ as before. 
    We shall denote it simply by $\Omega_{B/A}^\bullet$, 
    which will not lead confusion.
\end{definition}
The crystalline cohomology is defined via the crystalline sites and topoi. 
In the above discussion, 
we may replace ``rings" by ``ringed topoi" 
to define similar notions for divided power schemes. 
For example:
\begin{definition}
    A divided power scheme is a triple $(S, \CI, \gamma)$ 
    with $\CI$ a quasi-coherent ideal of $\CO_S$ 
    and $\gamma$ compatible divided power structures on each open set.
\end{definition}

Other notions, such as divided power envelopes, extensions, 
divided power de Rham complexes can be similarly defined.
\begin{definition}
    An arrow $U \to T$ is a divided power thickening in which 
    $T$ is a divided power scheme with ideal $\CJ$ and 
    $U$ the closed subscheme defined by $\CJ$ and 
    $\CJ$ is nilpotent (This is always the case if $m \CO_S=0$ for some $m$.)
\end{definition}

Like \'etale cohomology, crystalline cohomology works well 
only in ``finite" case (or ``profinite" case). 
That is, we will assume $m \CO_S=0$ for some integer $m$ without mentioned.
\begin{definition}
    Let $(S, \CI, \gamma)$ be a divided power scheme, 
    $S_0 = S / \CI$ and $X$ an $S_0$-scheme such that 
    the divided power structure extends. 
    The big crystalline site $\CRIS(X/S)$ contains objects defined by the diagram:
    \[
        \begin{tikzcd}
            U \ar[r] \ar[d] & T \ar[d] \\
            X \ar[r] & S,
        \end{tikzcd}
    \]
    with $U \to T$ a divided power thickening and 
    the divided power structures of $T$ and $S$ are compatible and 
    morphisms defined by 
    morphisms of divided power thickenings with compatibility conditions. 
    A collection of morphisms $\{(U_i \to T_i) \to (U \to T)\}$ is a covering 
    if $\{T_i\}$ forms a open covering of $T$ and for each $i$, 
    the commutative diagram below is cartesian.
    \[
        \begin{tikzcd}
            U_1 \ar[r] \ar[d] & T_1 \ar[d] \\
            U_2 \ar[r] & T_2
        \end{tikzcd}
    \]
    The small crystalline site $\Cris(X/S)$ contains those $U \to T$ 
    such that $U$ is an open subscheme of $X$ 
    with morphisms and coverings defined the same.
\end{definition}

It can be checked that $\CRIS(X/S)$ and $\Cris(X/S)$ are actually sites. 
Then the big crystalline topos $(X/S)_{\CRIS}$ is the topos 
corresponding to the big crystalline site 
and similarly for the small crystalline topos $(X/S)_{\Cris}$.
From the definition, one notices that 
a sheaf on the crystalline site is a compatible collection of Zariski sheaves.

The inclusion of sites $\Cris(X/S) \to \CRIS(X/S)$ is continuous and cocontinuous. 
Hence we may define functors $i_*, i_! \colon \Cris(X/S) \to \CRIS(X/S)$ 
and $i^{-1} \colon \CRIS(X/S) \to \Cris(X/S)$ respectively. 
Moreover, for a morphism $f\colon X/S \to Y/S'$, 
the natural map $\CRIS(X/S)\to \CRIS(Y/S')$ defines $f^{-1}, f_*$ respectively.

Then we are able to define the push forward 
and pull back of crystalline topoi. 
For a morphism $f \colon X/S \to Y/S'$, we have arrows:
\[
    \begin{tikzcd}
        (X/S)_{\Cris} \ar[r, "{i_*}", bend left] & 
        (X/S)_{\CRIS} \ar[l,"{i^{-1}}", bend left] \ar[r, "{f_*}", bend left] & 
        (Y/S')_{\CRIS} \ar[l, "{f^{-1}}", bend left] \ar[r, "{i^{-1}}", bend left] & 
        (Y/S')_{\Cris} \ar[l, "{i_!}", bend left].
    \end{tikzcd}
\]
Then $i^{-1} \circ f_* \circ i_*$ and $i^{-1} \circ f^{-1} \circ i_!$ 
are morphisms of topoi $(X/S)_{\Cris}$ and $(Y/S)_{\Cris}$, 
also denoted by $f_*$ and $f^{-1}$. 
From now on, we will consider the small crystalline site only.

The structure sheaf $\CO_{X/S}$ associates 
each divided power thickening $U \to T$ with the ring $\CO(T)$. 
For morphism $X/S \to Y/S'$, 
there exists a canonical morphism $\CO_{Y/S'} \to \CO_{X/S}$. 
Then we may define module push forward and pull back, 
denoted $f_*$ and $f^*$ respectively. 

The crystalline site and Zariski sites are linked. 
There is a morphism of sites 
\[
    u_{X/S} \colon \Cris(X/S) \to X_{\Zar}, \quad (U \to T) \mapsto U.
\]
Moreover, for an object $U \to T$ in the crystalline site, 
those objects lying over it form a subsite, 
which is equivalent to the Zariski site of $T$. 
For a sheaf $\CF \in (X/S)_{\Cris}$, 
the corresponding Zariski sheaf is denoted by $\CF_T$.

As in the Zariski case, 
quasi-coherent modules are the mainly objects for study. 
Those are, ``crystals" for some intuitive reasons.
\begin{definition}
    A crystal on $\Cris(X/S)$ is a sheaf of $\CO_{X/S}$-modules 
    such that for any morphism $u\colon T' \to T$, 
    the map induced by $u$, $u^*\CF_T \to \CF_{T'}$ is an isomorphism 
    and for each $T$, $\CF_T$ is a quasi-coherent $\CO_T$ module. 
\end{definition}

From definitions, 
a crystal is indeed a quasi-coherent sheaf of $\CO_{X/S}$-modules.
\begin{example}
    The sheaf $\CO_X$ defined by $\CO_X(U \to T) = \CO(U)$ is a crystal. 
    There is a canonical morphism: $\CO_{X/S} \to \CO_X$, which is surjective.
\end{example}

\begin{remark}
    In the definition of the crystalline site, 
    we require $U$ to become an Zariski open set in the object $U \to T$. 
    We can substitute the condition by requiring 
    $U \to X$ to be an \'etale map and all the theory still preserved. 
\end{remark}

\subsection{Calculus with divided powers}
Firstly we recall the characteristic $0$ case. 
Here $X$ is a smooth scheme over a field of characteristic $0$, $K$. 
Those abelian categories are equivalent:
\begin{itemize}
    \item 
        Crystals on $\Cris(X/K)$,
    \item 
        Integrable connections of quasi-coherent sheaves on $X$,
    \item 
        Stratifications of quasi-coherent modules on $X$,
    \item 
        Quasi-coherent $\CD$ -modules.
\end{itemize}
Now we return to the usual case, that is $m \CO_S = 0$. 
Similar results hold. We shall follow \cite{Sta},

Let $i \colon X \to Y$ be an immersion over $S$ 
with the ideal of definition $\CJ$ and $Y/S$ smooth. 
Let $D_{X,\gamma}(Y)$ be the spectrum of 
the quasi-coherent algebra $D_{\CO_Y,\gamma}(\CJ)$. 
Then $\Hom(-,D_{X,\gamma}(Y)) \simeq i^*\Hom(-,Y)$ by definition. 
$D_{X,\gamma}(Y^{n+1})$ is denoted by $D(n)$. 

\begin{theorem}
    The following categories are equivalent:
    \begin{itemize}
        \item 
            Crystals on $\Cris(X/S)$,
        \item 
            $D_{X,\gamma}(Y)$-modules with HPD-stratification as $\CO_Y$-modules,
        \item 
            $D_{X,\gamma}(Y)$-modules with integrable, 
            quasi-nilpotent connection as $\CO_Y$ -modules,
        \item 
            $D_{X,\gamma}(Y)$-modules $\CE$ 
            which are also  ``divided power" $\CD$-modules.
    \end{itemize}
    We shall mention the first and the third categories in the article only. 
    ``Quasi-nilpotent" means that 
    the ``taking derivative" action is locally nilpotent.
\end{theorem}

\begin{proof}
    There is an exact sequence:
    \[
        0 \to \Omega_{Y/S} \to \CO_{D(1)} \to \CO_{D(0)} \to 0.
    \]
    The sequence is split by two sections $p_1^*, p_2^*$, 
    where $p_1, p_2 \colon Y \times Y \to Y$. 
    $D(0)$ and $D(1)$ are objects in the crystalline site. 
    Then for a crystal $\CE$, we have an exact sequence:
    \[
        0\to \CE_{D(0)} \otimes \Omega_{Y/S} 
        \to \CE_{D(1)} \to \CE_{D(0)} \to 0.
    \]
    The two projections $p_1, p_2$ induce 
    $p_1^*, p_2^* \colon \CE_{D(0)} \to \CE_{D(1)}$. 
    $p_1^* - p_2^*$ gives such a connection. 
    One checks that the compatibility condition for a crystal 
    translates to the integrability condition. 
    
    Conversely, given a module with integrable connection, $\CE$,
    by smoothness, for any object $U \to T$ in the crystalline site, 
    locally there is a map $T \to Y$ lifting the immersion $U \to Y$. 
    Then locally there is a map $T \to D(0)$ 
    by universal property of divided power envelopes. 
    One defines $\CE_T$ by pulling $\CE$ back via the map above. 
    From the connection and the integrability, 
    one checks that those $\CE_T$ can form a crystal.
\end{proof}

We prove the fundamental comparison theorem in the section 
and derive some corollaries. 
We shall use the notations in the above sections.
\begin{lemma}[Poincar\'e lemma]
    \label{Poincare}
    Let $(A, I, \gamma)$ be a divided power ring and 
    $P = A \langle t_1, \cdots, t_n \rangle$ 
    with the natural divided power structure, 
    then there is an exact sequence:
    \[
        0 \to P \to \Omega_{P/A}^1 \to \cdots \Omega_{P/A}^n \to 0.
    \]
\end{lemma}

\begin{proof}
    One checks it directly by noticing
    \[
        \Omega_{P/A}^1 \simeq \bigoplus_{i = 1}^n Pdt_i.
        \qedhere
    \]
\end{proof}

\begin{lemma}
    \label{comparison}
    Suppose there is a commutative diagram 
    \[
        \begin{tikzcd}
            & Y'\ar[d,"p"] \\
            X \ar[r] \ar[ur] \ar[dr] & Y \ar[d] \\
            & S,
        \end{tikzcd}
    \]
    with $X, Y, S$ as above and $Y' = \BA^n_Y$. 
    For a crystal $\CE \in (X/S)_{\Cris}$, 
    there is a quasi-isomorphism of complexes:
    \[
        Rp_*(\CE_{D_X(Y')} \otimes \Omega_{Y'/S}^\bullet) 
        \simeq \CE_{D_X(Y)} \otimes \Omega_{Y/S}^\bullet.
    \] 
\end{lemma}
\begin{proof}
    We have an exact sequence
    \[
        0 \to \Omega_{Y'/Y} \to 
        \Omega_{Y'/S} \to p^*\Omega_{Y/S} \to 0.
    \] 
    We have the Gauss--Manin filtration of $\Omega_{Y'/S}$, 
    given by 
    \[
        Fil^k\Omega_{Y'/S}^j = 
        \im(\Omega_{Y'/Y}^k \otimes \Omega_{Y'/S}^{j-k} \to \Omega_{Y'/S}^j)
    \]
    with filtrants 
    \[
        gr^k \Omega_{Y'/S}^j \simeq 
        \Omega_{Y'/Y}^k \otimes p^* \Omega_{Y/S}^{j-k}.
    \] 
    It suffices to check for each filtrants.

    Moreover,
    \[
        \CE_{D_X(Y')} \simeq 
        \CE_{D_X(Y)} \otimes_{\CO_{D_X(Y)}} \CO_{D_X(Y')}
    \]
    by the crystal condition. 
    The result then follows from the Poincar\'e lemma \ref{Poincare}.
\end{proof}

\begin{lemma}
    \label{Cech}
    For $i \colon X\to Y$ a closed immersion with $Y/S$ smooth, 
    $\CE$ a quasi-coherent $D_{X,\gamma}(Y)$-module with a connection. 
    $Ru_{X/S*}\CE$ is represented by the cosimplicial module 
    (or a complex by Dold--Kan correspondence)
    \[
        \CE_{D(0)} \to \CE_{D(1)} \to \CE_{D(2)} \to \cdots.
    \]
\end{lemma}

\begin{proof}
    $u_{X/S}$ is acyclic (indeed, an isomorphism) 
    on the subsite of objects over $D(0)$. 
    Moreover, $D(0)$ covers the ``final object" $*$ 
    by the lifting property of smooth maps.
    One applies the \v Cech resolution for $D(0) \to *$ to conclude 
    by noticing $D(n) = D(0)^{n+1}$ in the category $\Cris(X/S)$. 
\end{proof}

\begin{lemma}
    For $A^\bullet$ a cosimplicial ring, 
    the cosimplicial module $M^\bullet$ defined by
    \[
        M^n = \bigoplus_{i = 1}^n A^n e_i
    \]
    with obvious arrows is contractible.
\end{lemma}

\begin{proof}
    The homotopy
    \[
        h \colon M^\bullet \to \Hom(\Delta[1], M^\bullet)
    \]
    is given by 
    \[
        h_n (e_i) (\alpha_j^n) = \left\{
            \begin{array}{rr}
                e_i & i<j \\
                0   & i \geq j,
            \end{array}
        \right.
    \]
    where $\alpha_j^n$ is the element in $\Delta[1]_n$ given by 
    \[
        \alpha_j^n \colon \{0,1,\cdots,n\} \to \{0,1\}, 
        \quad \alpha_j^n(i) = \left\{
            \begin{array}{rr}
                0 & i<j, \\
                1 & i \geq j.
            \end{array}
        \right.
    \]
\end{proof}

\begin{lemma}
    \label{Poincare ii}
    Notations as above. 
    When $Y = \BA^n_S$, the cosimplicial module 
    \[
        \Omega_{D(0)/S}^1 \to \Omega_{D(1)/S}^1 
        \to \Omega_{D(2)/S}^1 \to \cdots
    \]
    is contractible.
\end{lemma}

\begin{proof}
    One has
    \[
        \Omega_{D(0)/S}^1 \simeq 
        \bigoplus_{i=1}^d \CO_{D(0)} dt_d.
    \]
    One applies $d$ -fold of the $h$ above to conclude.
\end{proof}

\begin{theorem}
    Situations as above. 
    $\CE$ corresponds to a crystal on $(X/S)_{\Cris}$, 
    also denoted by $\CE$. 
    Then there is a canonical isomorphism:
    \[
        Ru_{X/S*} \CE \simeq \CE \otimes_Y \Omega_{Y/S}^\bullet 
        \simeq \CE \otimes_D \Omega_{D/S}^\bullet.
    \]
    In particular,
    \[
        R \Gamma ((X/S)_{\Cris},\CE) \simeq 
        R \Gamma (X_{\Zar}, \CE \otimes \Omega_{Y/S}^\bullet). 
    \]
\end{theorem}

\begin{proof}
    There is a commutative diagram:
    \[
        \begin{tikzcd}
            \CE_{D(0)} \ar[r]\ar[d] & 
            \CE_{D(0)} \otimes \Omega_{D(0)/S}^1 \ar[r] \ar[d] & \cdots \\
            \CE_{D(1)} \ar[r] \ar[d] & 
            \CE_{D(1)} \otimes \Omega_{D(1)/S}^1 \ar[r] \ar[d] & \cdots \\ 
            \vdots & \vdots,
        \end{tikzcd}
    \]
    where the rows are de Rham complexes 
    and the horizontal arrows are given by 
    $p_1^* - p_2^* + \cdots + (-1)^n p_{n + 1}^*$.

    The problem is \'etale local. 
    Thus we may assume that $Y = \BA^n_S$. 
    The left vertical arrow of the diagram represents the 
    $Ru_{X/S*} \CE$ by \ref{Cech}. 
    Other vertical arrows have zero cohomology for they are contractible. 
    One applies the spectral sequence to conclude 
    the cohomology of the total complex 
    is the cohomology of $Ru_{X/S*} \CE$.

    We apply the spectral sequence to the other direction. 
    Applying \ref{comparison} for $p_i \colon Y^{n+1} \to Y^n$, 
    one deduces that the $E_1$ package of the spectral sequence becomes
    \[
        \begin{tikzcd}
            \CH^0 \ar[r] \ar[d, "0"] & 
            \CH^1 \ar[r] \ar[d, "0"] & \cdots \\
            \CH^0 \ar[r] \ar[d, "\sim" sloped] & 
            \CH^1 \ar[r] \ar[d, "\sim" sloped] & \cdots \\ 
            \vdots & \vdots.
        \end{tikzcd}
    \]
    Here $\CH^i$ means the cohomology of the first row. 
    Thus, all arrows except the first row are cancelled 
    in the $E_2$ package of the spectral sequence. 
    Thus the cohomology of the total complex is 
    the cohomology of the first arrow in the original commutative diagram, 
    i.e., the de Rham complex. 
    The theorem thus follows by comparing the spectral sequence of two directions.
\end{proof}

\subsection{Verification of the desired properties}
The theorem above provides a way to express the crystalline cohomology 
explicitly by de Rham cohomology. 
We shall use this to prove those results 
that a good cohomology theory should satisfy. 
The main reference here is \cite{BO} and \cite{Be2}.

\begin{theorem}[Vanishing theorem]
    Let $f \colon X \to S$ be quasi-compact 
    and quasi-separated and $X/S_0$ smooth. 
    Denote $f_{X/S} = f \circ u_{X/S}$. 
    Then there exists an integer $r$ such that 
    $R^i f_{X/S*} \CE$ vanishes for any $i>r$ and any crystal $\CE$.
\end{theorem}

\begin{proof}
    Using the \v Cech-to-derived spectral sequence, 
    we reduce to the local case. 
    That is, $X$ admits a lifting $Y$, such that $Y/S$ smooth. 
    Then there is an isomorphism:
    \[
        Rf_{X/S*} \CE \simeq Rf_* (\CE \otimes_Y \Omega_{Y/S}^\bullet).
    \]
    The result follows from the vanishing result of 
    cohomology of (Zariski) quasi-coherent sheaves.
\end{proof}
\begin{theorem}[Base change thoerem]
    Suppose there is a base change diagram for 
    $f \colon X \to S$ and $f' \colon X' \to S'$,
    \[
        \begin{tikzcd}
            X' \ar[r,"{u'}"] \ar[d] & X \ar[d] \\
            S_0' \ar[r] \ar[d] & S_0 \ar[d] \\
            S' \ar[r,"u"] & S.
        \end{tikzcd}
    \]
    Then there is an isomorphism:
    \[
        Lu^* Rf_{X/S*} \CE \simeq Rf'_{X'/S'*} Lu'^*\CE,
    \]
    for any flat crystal $\CE$ on $(X/S)_{\Cris}$.
\end{theorem}

\begin{proof}
    For the case that there is a smooth lifting $Y/S$ of $X$, 
    it follows from the base change theorem for Zariski sheaves. 
    The general case follows from cohomological descent.
\end{proof}

\begin{theorem}[Finiteness theorem]
    Let $X/S_0$ be proper and smooth and $S$ be noetherian. 
    Then for any coherent crystal $\CF$ flat over $\CO_{X/S}$, 
    the complex $Rf_{X/S*}\CF$ is quasi-isomorphic to a perfect complex, 
    i.e. a complex whose terms are locally free sheaves of finite rank 
    and the number of non-zero terms of which is finite.
\end{theorem}

\begin{proof}
    By \v Cech resolution and the computation in the local case 
    $Rf_{X/S*} \CF$ has finite tor-dimension. 
    Hence it suffices prove that $Rf_{X/S*} \CF$ has coherent cohomology. 
    For $S$ is noetherian, $\CI^n = 0$ for sufficiently large $n$. 
    By d\'evissage arguments, 
    it suffices to prove for coherent $\CO_{X/S_0}$-modules. 
    Then 
    \[
        Rf_{X/S*} \CF \simeq Rf_{X/S_0*} \CF 
        \simeq Rf_* \CF \otimes_X \Omega_{X/S}^\bullet.
    \]
    Then the result follows from the finiteness result 
    of cohomology of coherent sheaves in Zariski topology.
\end{proof}

We may then define the ``completed" version. 
That is, $S$ is a $p$ -adically complete formal scheme with 
$\CI = (p)$, $S_n = S / \CI^n$. 
The completed crystalline site $\Cris(X/\hat{S})$ is 
the combination of all $\Cris(X/S_n)$ and crystals similarly defined. 
One defines the completed crystalline topos then.
\begin{proposition}
    For a crystal $\CE$ on $(X/S)_{\Cris}$ 
    with restriction $\CE_n$ on $(X/S_n)_{\Cris}$, 
    there is an isomorphism:
    \[
        Ru_{X/S*} \CE = R \lim_n Ru_{X/S_n*} \CE_n.
    \]
\end{proposition}

The proof is given in \cite{BO}.
Then completions of the comparison results hold.
For example,
\[
    Ru_{X/S*} \CE \simeq \CE \otimes \Omega_{\hat{D}/S}^\bullet,
\]
where $\hat{D}$ is the $p$-adic completion of $D_X(Y)$ 
and the latter $\CE$ is a $\hat{D}$ -module.
In particular, there is an isomorphism:
\[
    R \Gamma((X/S)_{\Cris},\CE) \simeq
    R \Gamma(X_{\Zar}, \hat{\CE} \otimes \Omega_{\hat{D}/S}^\bullet).
\]
Then the vanishing theorem, the base change theorem and the finiteness theorem 
also hold in the completed case for coherent crystals, 
since $R \lim$ works well then.

If moreover $X/S_0$ is proper, $\CE$ is coherent 
and $X$ lifts to $Y$, then $Y$ is also proper. 
By Grothendieck's existence theorem, 
there exists a coherent $\CO_Y$-module, 
also denoted by $\CE$ such that 
\[
    R\Gamma((X/S)_{\Cris},\CE) \simeq 
    R\Gamma(X_{\Zar},\CE\otimes \Omega_{Y/S}^\bullet).
\]
In particular,
\[
    R\Gamma((X/S)_{\Cris}, \CO_{X/S}) \simeq 
    R \Gamma (X_{\Zar}, \Omega_{Y/S}^\bullet).
\]
We will denote the left hand side by $\HSd$. 
We restrict to the case $S = \Spec (\W(k))$ and 
$S_0 = \Spec (k)$ for some perfect field $k$ of characteristic $p$, 
which is the crystalline cohomology for characteristic $p$ varieties.

We firstly discuss the case that $X/k$ is proper and $X$ admits a lifting. 
Then there exists a cartesian diagram 
(if there exists an embedding $\W(k) \to \BC$, which is the case for $k$ finite):
\[
    \begin{tikzcd}
        X \ar[r] \ar[d] & \CX \ar[d] & X^{\an} \ar[l] \ar[d] \\
        \Spec (k) \ar[r] & \Spec (\W(k)) & \Spec (\BC) \ar[l],
    \end{tikzcd}
\]
with $\CX$ a proper smooth scheme over $\W(k)$. 
Then we have the following comparison theorem 
between crystalline cohomology and Betti cohomology.
\begin{theorem}
    There is a canonical isomorphism 
    \[
        \Hkn \otimes \BC \simeq H^n(X^{\an}, \BC),
    \]
    where the right hand side is the Betti cohomology.
\end{theorem}

\begin{proof}
    There exists an canonical isomorphism:
    \[ 
        \Hkn \simeq \BH^n (\Omega_{\CX/\W(k)}^\bullet).
    \]
    There exists a spectral sequence 
    \[
        E_1^{pq} = \BH^q(\Omega_{\CX/\W(k)}^p) 
        \Rightarrow \BH^{p+q} (\Omega_{\CX/\W(k)}^\bullet).
    \]
    All the modules in the left hand side is locally free. 
    After tensoring $\BC$, it becomes the spectral sequence
    \[
        E_1^{pq} = \BH^q(\Omega_{X^{\an}/\BC}^p) 
        \Rightarrow \BH^{p+q} (\Omega_{X^{\an}/\BC}^\bullet).
    \]
    By (complex) Hodge theory, the spectral sequence degenerates 
    and the right hand side is isomorphic to $H^{p+q} (X^{\an},\BC)$. 
    Hence the original spectral sequence degenerates and the result follows.
\end{proof}
For example, $\BP^n$, and hypersurfaces in $\BP^n$ admits a lifting. 
Moreover, by obstruction theory, such a lifting exists if and only if 
the obstruction class $\xi \in H^2(X, \CT_X)$ vanishes. 
In particular, this is the case when $X$ is a curve. 

\begin{theorem}[K\"unneth formula]
    For a crystal $\CE_1$ on $X_1$ and a crystal $\CE_2$ on $X_2$, 
    $\CE_1 \boxtimes \CE_2$ defines a crystal on $X_1\times X_2$ and 
    \[
        R \Gamma ((X_1\times X_2/S)_{\cris}, \CE_1 \boxtimes \CE_2) \simeq 
        R \Gamma((X_1/S)_{\Cris}, \CE_1) \otimes^{L} R \Gamma ((X_2/S)_{\Cris}, \CE_2).
    \]
\end{theorem}
\begin{proof}
    The map is natural and we may check it locally. 
    Then there are closed immersions $X_1\to Y_1$ 
    and $X_2\to Y_2$ with $Y_1,Y_2/S$ smooth. 
    The result follows from the fact that 
    the tensor product of the de Rham complex of $\CE_1$ and $\CE_2$ 
    is isomorphic to the de Rham complex of $\CE_1\boxtimes \CE_2$.
\end{proof}
\begin{theorem}[Poincar\'e duality]
    For $X/k$ proper smooth of dimension $d$, 
    there is a trace map $\tr\colon H^{2d}_{\cris}(X)\simeq \W(k)$, 
    such that for all coherent crystal $\CE$ over $X/\W(k)$, the pairing
    \[
        R\Gamma((X/\W(k))_{\Cris}, \CE) \times 
        R \Gamma((X/\W(k))_{\Cris}, \CE^\vee) \to \W(k)[2d]
    \]
    is perfect.
\end{theorem}
\begin{proof}
    We define the trace map firstly. 
    Locally, $X$ lifts to a smooth scheme $\tilde{X}$ over $\W(k)$ 
    with $p$-adic completion by $\CX$ (though it is no longer compact)
    . Take a compactification of $\tilde{X}$, $\bar{X}$. 
    $\bar{X} \to \Spec (\W(k))$ has a dualizing complex $\omega$. 
    Then
    \[
        \omega|_{\tilde{X}} \simeq \Omega_{\tilde{X}/\W(k)}^d.
    \]
    The trace map is defined as (by the usual procedure, 
    one checks the map is independent of $\bar{X}$)
    \[
        H^{2d}_{\cris}(X) \simeq 
        H^d(\CX,\Omega_{\CX}^d) \to H^d(\bar{X},\omega) \simeq \W(k).
    \]
    By derived Nakayama lemma and the finiteness theorem, 
    it suffices to prove after tensoring $k$. By base change theorem, 
    it suffices to prove similar results on $(X/k)_{\Cris}$. 
    Then the derived global sections is represented by de Rham complexes 
    and the result follows from the Serre duality.
\end{proof}

\begin{remark}
    The cycle class map can be defined: 
    defining it locally by de Rham cohomology 
    and then applying cohomological descent.
\end{remark}

\begin{remark}
    Crystalline cohomology does not work well for non-proper or non-smooth schemes. 
    See \ref{counter ex} and tag 07LI in \cite{Sta}.
\end{remark}

\begin{remark}
    For varieties with normal crossing singularities and 
    becoming an open set in a proper variety 
    with complement a normal crossing divisor, 
    there is a canonical log structure on it. 
    One defines their log-crystalline sites 
    (similar to the crystalline site, but requiring all objects having a log structure) 
    and the corresponding cohomology, called the log crystalline cohomology. 
    
    The log crystalline cohomology has similar comparison results to de Rham cohomology 
    and one can apply this to show that it is a good cohomology theory.

    In fact, it can be proved that 
    there is a comparison between log crystalline cohomology and rigid cohomology. 
    However, the log crystalline cohomology is indispensable for there is an operator $N$, 
    the log monodromy operator on it. 
    Those operators are required in $p$-adic Hodge theory.
\end{remark}

\subsection{Frobenius action}
For this section, let $S = \Spec (\W(k))$ and $S_0 = \Spec (k)$, 
where $k$ is a characteristic $p$ perfect field. 
$X$ be a smooth scheme over $k$. 
The (absolute) Frobenius actions $F_X$ and $F_S$ induce a commutative diagram, 
where the square is cartesian:
\[
    \begin{tikzcd}
        X \ar[ddr, bend right] \ar[dr, "{F_{X/S}}"] \ar[drr, "F_X", bend left] \\
        & X'\ar[r, "{F_S'}"] \ar[d] & X \ar[d]\\
        & S \ar[r, "{F_S}"] & S.
    \end{tikzcd}
\]
Then a morphism $\CO_{X/S} \to RF_{X*} \CO_{X/S}$ 
and morphism $\CO_{X'/S} \to RF_{X/S*} \CO_{X/S}$, 
the first induces the Frobenius action on $\Hkd$ and the cohomology of $X$ 
becomes an $F$ -crystal after modulo torsion 
and the latter induces the $\W(k)$ -linearization of such action.

As we have mentioned, the eigenvalues of the Frobenius action 
provide informations of the zeta function. 
We analyse the Frobenius in this section and the goal is to prove the Katz conjecture. 
The main reference here is \cite{BO}.

We firstly analyse the Frobenius action when there exists a lifting $(Y,F_Y)$ of $(X,F_X)$. 
Then the above diagram lifts to a similar diagram of $Y$. 
As the map $\Omega_{X'/k}^1 \to F_{X/k*} \Omega_{X/k}^1$ is zero, 
the map $\Omega_{Y'/S}^1 \to F_{Y/S*} \Omega_{Y/S}^1$ 
factors through $p F_{Y/S*} \Omega_{Y/S}^1$. 
Thus the image of the Frobenius action 
lies in the largest subcomplex of $p^iF_{Y/S*} \Omega_{Y/S}^i$. 

\begin{proposition}
    The image of the map above is exactly 
    the largest subcomplex of $p^i F_{Y/S*} \Omega_{Y/S}^i$. 
    Moreover, the map is an isomorphism to its image.
\end{proposition}

To prove this, we shall prove a general result. 
We need some terminology:
\begin{definition}
    A gauge is a map $\epsilon \colon \BZ \to \BN$ 
    such that $\epsilon(i + 1) \leq \epsilon(i) \leq \epsilon(i + 1) + 1$. 
    A cogauge is a map: $\eta \colon \BZ \to \BN$ 
    such that $\epsilon(i) \leq \epsilon(i + 1) \leq \epsilon(i) + 1$. 
    For a map $\zeta \colon \BZ \to \BN$ and a complex $K$, 
    $K_\zeta$ is the largest complex that contains in $p^{\zeta(i)} K^i$. 
    For two gauges $\epsilon$ and $\epsilon'$, 
    $\epsilon'$ is called a simple augmentation of $\epsilon$ at $i$ 
    if $\epsilon(i) = \epsilon'(i) - 1$ and $\epsilon(j) = \epsilon'(j)$ for $j \neq i$. 
\end{definition}

\begin{proposition}
    We have a quasi-isomorphism:
    \[
        (\Omega_{Y'/S}^\bullet)_\epsilon 
        \simeq (F_{Y/S*} \Omega_{Y/S}^\bullet)_{\epsilon + 1}.
    \]
\end{proposition}

To prove this, we firstly states the Cartier theorem:
\begin{proposition}
    There is an isomorphism:
    \[
        \begin{tikzcd}
            \Omega_{X/k}^i \ar[r, "{\sim}"swap, "{c^{-1}}"] 
            & \CH^i (\Omega_{X/k}^\bullet).
        \end{tikzcd}
    \]
    with $c^{-1}(x) = x$, $c^{-1}(dx) = x^{p-1} dx$ 
    and $c^{-1}$ is compatible with the wedge product.
\end{proposition}

\begin{proof}
    Note that both sides are compatible with \'etale base change.
    Hence it suffices to construct a canonical isomorphism when 
    $X = \Spec (k[t_1, \cdots, t_n])$. 
    Then the result follows from a direct calculation.
\end{proof}

Now we are able to prove the proposition above:
\begin{proof}
    For the complex is left-bounded, 
    we restrict the domain of $\epsilon$ to $\BN$. 
    We proceed by induction: 
    we can find a series of simple augmentations $\{\epsilon_i\}_{i=1}^{\infty}$ 
    such that $\epsilon_1=\epsilon$ and $\epsilon_m(n)\to \infty$ when $m\to \infty$, 
    which induces a filtration of $(\Omega_{Y/S}^\bullet)_\epsilon$. 
    Then it suffices to prove the quasi-isomorphism for each filtrant. 
    We then have a commutative diagram ($\eta_l=\epsilon_l+1$) 
    and $\epsilon_{l+1}$ is the simple augmentation of $\epsilon_l$ at $j$.
    \[
        \begin{tikzcd}
            (\Omega_{Y'/S}^\bullet)_{\epsilon_l} / (\Omega_{Y'/S}^\bullet)_{\epsilon_{l + 1}} 
            \ar[r] \ar[d, "\sim"sloped]
            & (F_{Y/S} \Omega_{Y/S}^\bullet)_{\epsilon_l} / 
            (F_{Y/S} \Omega_{Y/S}^\bullet)_{\epsilon_{l+1}} 
            \dar[] {\bullet p^{-\epsilon_l(j)}}[sloped, swap] {\sim} \\
            \Omega_{X'/k}[-j] \ar[r, "c^{-1}", "\sim"swap] & 
            \CH^j (F_{X/k*} \Omega_{Y/S_0}) [-j].
        \end{tikzcd}
    \]
Hence the arrow above is an isomorphism as other arrows are.
\end{proof}

To globalize the result, we can define $(-)_\eta$ is the derived category.
\begin{definition}
    For a cogauge $\eta$, the functor $L \eta$ 
    in the derived category of abelian sheaves $\CD(\CA b(X))$ 
    is defined as follows: for a complex $K$, 
    take a flat (as a $\BZ$ -module) resolution and then 
    apply $(-)_\eta$ on it. 
    For two cogauges $\eta, \eta'$, with $\eta' \leq \eta$, 
    the functor $L\eta'/\eta$ is defined by the distinguished triangle:
    \[
        L\eta \to L \eta' \to L \eta'/\eta.
    \]
\end{definition}

It can be proved that $L\eta$ is well-defined.
\begin{theorem}
    There is an isomorphism, where $\eta(i)=i$:
    \[
        Ru_{X'/S} \CO_{X'/S} \simeq F_{X/S*}L \eta Ru_{X/S} \CO_{X/S}.
    \]
\end{theorem}

\begin{proof}
    Choose such a lifting locally and 
    glue the above local isomorphisms together by cohomological descent. 
\end{proof}

\begin{definition}
    For an $F$-crystal $M$ 
    (i.e. a finite rank free $\W(k)$ -module with a semilinear Frobenius action), 
    choose a basis of it and then the Frobenius action is represented by a matrix. 
    The slopes of the action are the absolute values of the eigenvalues of the matrix 
    (the slopes are independent of the choice of bases). 
    Let $\alpha_1 \leq \cdots \leq \alpha_m$ be the slopes counted by multiplicity. 
    Then the Newton polygon of $X$ is defined by the polygon with domain $[0, m]$ 
    with slope $\alpha_i$ in the interval $[i - 1, i]$. 
    If $M' = \im(F) = \bigoplus_{i = 0}^l p^i M_i$, 
    the Hodge number of $M$ is $(e_0, \cdots, e_l)$, 
    the ranks of $M_0, \cdots, M_l$.
\end{definition}

\begin{definition}
    The Hodge polygon with parameter $(a_0, \cdots)$ is the polygon 
    with slope $i$ in the interval $[a_0 + \cdots + a_{i-1}, a_0 + \cdots + a_i]$.
\end{definition}

\begin{lemma}
    The following basic result holds:
    \begin{itemize}
        \item 
            The Newton polygon of an $F$ -crystal lies on the Hodge polygon of it.
        \item 
            The Hodge polygon with parameter $(a_0, \cdots)$ lies on the Hodge polygon 
            with parameter $(b_0, \cdots)$ if and only if for any $i>1$, 
            $i a_0 + (i - 1) a_1 + \cdots + a_{i-1} 
            \leq i b_0 + (i - 1) b_1 + \cdots b_{i - 1}$ 
            holds for any $i$.
        \item 
            $\ell(M'/M' \cap p^i M) = i e_0 + \cdots + e_{i - 1}$, 
            where $\ell$ means length.
    \end{itemize}
\end{lemma}

We are then able to prove the Katz conjecture:
\begin{theorem}[Katz, Mazur-Ogus]
The Newton polygon of $M = \Hkn$ lies on the Hodge polygon 
with parameters $a_q = h^{n-q}(X, \Omega_{X/k}^q)$. 
\end{theorem}

\begin{proof}
    From the above facts, we are reduced to prove the following:
    \[
        \ell(M'/M' \cap p^i M) \leq i h^0 + (i - 1) h^1 + \cdots+ h^{i - 1}. 
    \]
    There exists a commutative diagram ($L\eta Ru_{X/S}\CO_{X/S})$ is simply denoted $L\eta$)
    \[
        \begin{tikzcd}
            \BH^n(X, L(\zeta_i + i)) \ar[r] \ar[d] & 
            \BH^n(X, L\zeta_0) \ar[r, "q"] \ar[d,"\vp"] & 
            \BH^n(X, L\zeta_0/(\zeta_i + i)) \\
            \BH^n(X, Li) \ar[r, "\psi"] & \Hkn,
        \end{tikzcd}
    \]
    where $\zeta_i(n) = \max \{0, n - i\}$.
    After modulo torsion, the image of $\vp$ is $M'$ and the image of $\psi$ is $p^i M$. 
    Then $M'/M' \cap p^i M$ is a quotient of $\im q$. 
    Moreover, there is a distinguished triangle:
    \[
        \BH^n(X, L(\zeta_{j - 1} + j - 1)/(\zeta_j + j)) \to 
        \BH^n(X, L\zeta_0/(\zeta_j + j)) \to 
        \BH^n(X, L\zeta_0/(\zeta_{j-1} + j - 1))
    \]
    which induces a long exact sequence of cohomology. Moreover, 
    \[
        \BH^n(X, L(\zeta_{j - 1} + j - 1)/(\zeta_j + j)) 
        \simeq \BH^n(X, \tau_{\leq j - 1} \Omega_{X/k}^\bullet),
    \]
    where $\tau$ means the truncation.
    Then there is a spectral sequence, 
    where the isomorphism is given by the Cartier isomorphism:
    \[
        E_2^{pq}(q \leq j - 1) = H^p(X, \Omega_{X/k}^q) 
        \simeq H^p(X, \CH^q(\Omega_{X/k}^\bullet)) 
        \Rightarrow \BH^{p+q}(X, \tau_{\leq j - 1} \Omega_{X/k}^\bullet).
    \]
    Thus, $\dim \BH^{p+q}(X, \tau_{\leq j - 1} \Omega_{X/k}^\bullet) \leq h^0 + \cdots + h^{j - 1}$. 
    Combining all those $j$ gives the result.
\end{proof}