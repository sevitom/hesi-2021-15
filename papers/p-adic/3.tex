\subsection{Basic definitions and comparison results}
In this section, let $k$ be a perfect field with characteristic $p$, 
$\CO = W(k)$ the Witt ring and $K = Frac(W(k))$. 
$X, Y$ etc. are varieties over $k$ and 
$\CP$ a formal scheme over $\CO$. 
$\CP_K, \CP_k$ the generic fibre and special fibre of $\CP$.
We firstly define the convergent cohomology for a basic approach.
\begin{definition}
    Let $X \to \CP$ be an immersion, 
    $\Sp \colon \CP_K \to \CP_k$ the specialization map. 
    The tube of $X$ is the inverse image $\Sp^{-1}(X) \subset \CP$, denoted $]X[_\CP$.
\end{definition}

\begin{definition}
    If there exists an immersion $X \to \CP$ such that 
    $\XP$ is smooth, the convergent cohomology of $X$ is defined as
    \[
        \HCd = R\Gamma(\DRX),
    \]
    where $\DRX$ is the de Rham complex.
\end{definition}

\begin{proposition}[Weak fibration theorem \cite{St}]
    Suppose there is a commutative diagram 
    \[
        \begin{tikzcd}
            & \CP'\ar[d, "p"]\\
            X \ar[r] \ar[ur] \ar[dr] & \CP \ar[d] \\
            & S,
        \end{tikzcd}
    \]
    where $\CP' \to \CP$ is smooth in a neighbourhood of $X$. 
    Then $]X[_{\CP'} = ]X[_{\CP} \times B^\circ(0,1)^d$ locally on $X$, 
    for $d$ the relative dimension of $\CP' \to \CP$.
\end{proposition}

\begin{proposition}
    The cohomology $\HCd$ is independent of the choice of $\CP$, 
    which means the convergent cohomology is well-defined. 
\end{proposition}

\begin{proof}
    For two immersions $X \to \CP_1$ and $X\to \CP_2$ with $\CP_1$ and $\CP_2$ smooth, 
    $X \to \CP = \CP_1 \times \CP_2$ maps to both of them by projections $p_1$ and $p_2$. 
    By the method of Gauss--Manin filtration and 
    Poincar\'e lemma mentioned in \ref{comparison}, 
    one proves that
    \[
        Rp_{1*} \DRX \simeq \Omega_{\XP/K}^\cdot ]X[_{\CP_1}.
    \] 
    Taking global sections gets the result.
\end{proof}

\begin{remark}
    Such an immersion exists locally: for affine $X$, $X$ can be embedded into $\BA_k^n$. 
    Then into $\widehat{\BA^n_{\CO}}$. 
    We may use the cohomological descent method 
    to define the convergent cohomology in general. 
\end{remark}

The convergent cohomology behaves badly for the non-proper varieties.
\begin{example}
    \label{counter ex}
    Let $X = \BA_k^1$ be the affine line, $\CP = \Spf \BZ_p \langle T \rangle$, 
    where $\BZ_p \langle T \rangle$ consists of all the convergent series. 
    Then $\CP_K = \Spp K \langle T \rangle$ and $\XP = \CP_K$. 
    By the "Theorem B" in rigid geometry, all the sheaves $\DRXn$ are acyclic. 
    Then the convergent cohomology is the cohomology of the complex
    \[
        \begin{tikzcd}
            K \langle T \rangle \ar[r,"d"] & K \langle T \rangle dT.
        \end{tikzcd}
    \]
    However, $H^1$ is not finite dimensional 
    for there are infinitely many linearly independent convergent series 
    that no longer converge after integration.
\end{example}

We then introduce the rigid cohomology to resolve the problem mentioned above.
\begin{definition}
    A frame of $X$ contains arrows,
    \[
        \begin{tikzcd}
            X \ar[r,"j"] & Y \ar[r,"i"] & \CP
        \end{tikzcd}
    \]
    such that $j$ is an open immersion and $i$ a closed immersion. 
    A proper smooth frame is a frame such that $Y$ is proper and $\YP$ is smooth. 
\end{definition}

\begin{definition}
    Let $\CF$ be an abelian sheaf on $]Y[_\CP$, 
    the overconvergent sheaf $j_X^\dagger \CF$ is 
    \[
        j_X^\dagger \CF = \colim j_{V*} j_{V}^* \CF,
    \]
    where $V$ runs through all the strict neighbourhood of $\XP$ in $\YP$ 
    and $j_V$ the inclusion $V \to \YP$. 
\end{definition}

\begin{definition}
    If there exists a proper smooth frame of $X$, the rigid cohomology $\HRd$ is 
    \[
        \HRd = R\Gamma(j_X^ \dagger \DRY),
    \]
    where $\DRY$ is the de Rham complex.
\end{definition}

\begin{proposition}
    The rigid cohomology of $X$ is independent of the choice of the frame if such frame exists.
\end{proposition}

\begin{proof}
    It is similar to the case of convergent cohomology. 
    See chapter 6 in \cite{St} for details.
\end{proof}

\begin{remark}
    A proper smooth frame of $X$ exists locally. 
    That is, for affine $X$, $X$ can be embedded into $\BA_k^n$ and then $\BP_k^n$. 
    Then we may take $Y$ be $\bar{X} \subset \BP_k^n$ and $\CP$ be $\widehat{\BP_\CO^n}$. 
    Then we may use cohomological descent to define rigid cohomology in general (\cite{Tsu}).
\end{remark}

The rigid cohomology resolves the problem above.
\begin{example}
    Let $X = \BA_k^1$, $Y = \BP_k^1$, $\CP = \widehat{\BP_{\CO}^1}$. 
    Then $\XP = \Spp K \langle T \rangle$. $\YP = \BP^{1,an}_K$. 
    Then $\{V^\lambda = \Spp K \langle \lambda T \rangle \mid |\lambda| < 1 \}$ 
    forms a cofinal system of strict neighbourhoods of $\XP$ in $\YP$. 
    Then the cohomology becomes
    \[
        \begin{tikzcd}
            K \langle T \rangle^\dagger \ar[r] & K \langle T \rangle^\dagger dT,
        \end{tikzcd}
    \]
    where $K \langle T \rangle^ \dagger$ contains all the overconvergent series 
    (i.e. converges in some radius $\mu > 1$). 
    Then $H^1 = 0$ as the integration of overconvergent series remains overconvergent.

    In general, for an affine smooth variety $X = \Spec A$, 
    $A$ can be lifted to a finitely presented formally smooth algebra 
    $\CA = \CO \langle T_1, \cdots, T_n \rangle/(f_1, \cdots, f_m)$. 
    Then the rigid cohomology of $X$ becomes the cohomology of the de Rham complex 
    $\Omega_{\CA_K^\dagger/K}^\cdot$, 
    where $\CA_K^\dagger \simeq K \langle T_1, \cdots,T_n \rangle^\dagger/(f_1, \cdots, f_m)$. 
    That is, the Monsky-Washnitzer cohomology.
\end{example}

\begin{remark}
    The way considering overconvergent series can be justified by the language of adic spaces. 
    Indeed, $\XP$ is not closed in $\YP$ when viewing them as adic spaces 
    and we should not expect that a non-closed thing can have a good cohomology theory. 
    Thus we should consider the closure of $\XP$ in $\YP$ and work out its cohomology. 
    Indeed, the strict neighbourhoods are neighbourhoods of the closure of $\XP$, $\overline{\XP}$. 
    Thus, for $i \colon \overline{\XP} \to \YP$,
    \[
        \HRd = R\Gamma_{\YP} i_* i^* \DRY.
    \]
\end{remark}

In proper case, the convergent and rigid cohomology concide.
\begin{proposition}
    For proper $X$, there is a canonical isomorphism in derived category
    \[
        \HCd \simeq \HRd.
    \]
\end{proposition}

\begin{proof}
    The proposition is trivial for projective $X$, 
    for there exists a closed immersion $X \to \BP_k^n$, 
    then an immersion $X \to \widehat{\BP_\CO^n}$ and a frame $X \to X \to \widehat{\BP_\CO^n}$. 
    In general, one checks that both convergent cohomology and rigid cohomology functors 
    are right Kan extensions from the category of projective varieties 
    to the category of proper varieties 
    (for a proper variety, it has a simplicial resolution by projective varieties). 
    See \cite{Tsu} for details. 
\end{proof}

We are now able to prove the comparison theorem 
between crystalline cohomology and convergent/rigid cohomology.
\begin{theorem}
    There exists an canonical isomorphism for smooth $X$:
    \[
        \Hkd \otimes K \simeq \HCd.
    \]
\end{theorem}

\begin{proof}
    By cohomological descent, it suffices to prove that it holds locally. 
    For affine $X$, there exists a formally smooth lift $\CP=\CX$. Then $\XP=\CX_K$ and 
    \[
        \Hkd \simeq \Omega_{\CX/\CO}^\cdot, \quad \HCd \simeq \Omega_{\CX_K/K}^\cdot.
    \]
    Hence the result follows from the fact that 
    \[
        \Omega_{\CX/\CO}^\cdot \otimes K \simeq \Omega_{\CX_K/K}^\cdot.
    \]
\end{proof}

The comparison between crystalline and rigid cohomology 
is a direct consequence of the above two facts
\begin{corollary}[\cite{Be1}]
    For proper smooth $X$, there exists an isomorphism 
    \[
        \Hkd \otimes K \simeq \HRd.
    \]
\end{corollary}

As crystalline cohomology can be defined for crystals, 
rigid cohomology can be defined for "overconvergent isocrystals". 
As the categorical equivalence mentioned above, 
the following abelian categories are equivalent 
(chapter 7 in \cite{St}) (let $X \to Y \to \CP$ be a proper smooth frame):
\begin{itemize}
    \item 
        Coherent $j^\dagger \CO_{\YP/K}$ -modules with integrable connection.
    \item 
        Coherent $j^\dagger \CO_{\YP/K}$ -modules with stratification,
    \item 
        $j^\dagger \CO_{\YP/K}$-modules 
        which are coherent as $j^\dagger \CO_{\YP/K}$ -modules,
    \item 
        Overonvergent isocrystals, denoted by $Isoc^\dagger(X \to Y)$. 
        That is, associating each morphism of frames
        \[
            \begin{tikzcd}
                X' \ar[r] \ar[d] & Y' \ar[r] \ar[d] & \CP' \ar[d] \\
                X \ar[r] & Y \ar[r]& \CP
            \end{tikzcd}
        \]
    a coherent $j^\dagger \CO_{\YP/K}$-module $\CE_\CP'$, 
    satisfying the crystal relations as above.
\end{itemize}

Moreover, those categories are "stacks" in some sense. 
That is, taking the fifth for example, 
there is an equivalence ($\CU = \{Y_i \to Y\}$ be an open covering):
\[
    Isoc^\dagger (X \to Y) \to \lim_{\CU} Isoc^\dagger (X. \to Y.)
\]
where $X.,Y.$ means the simplicial scheme with respect to the open covering $\CU$. 
Then we can define those concepts for the case $X$ 
does not admit a proper smooth frame by descent.

Then for an overconvergent isocrystal $\CE$ 
(with the corresponding coherent sheaf with integrable connection also denoted $\CE$), 
we may define the rigid cohomology with coefficients
\[
    H^n_{rig}(X,\CE) := \BH^n(\CE \otimes \Omega_{\YP/K}^\cdot).
\]
Then there is a comparison theorem between crystalline cohomology and rigid cohomology.
\begin{theorem}
    There is a functor for smooth $X$:
    \[
        \{\textit{coherent crystals on } (X/W(k))_{Cris}\} \to Isoc(X)
    \]
    and a functor for proper smooth $X$
    \[
        \{\textit{coherent crystals on }(X/W(k))_{Cris}\} \to Isoc^\dagger(X),
    \]
    ($Isoc(X)$ means convergent isocrystals), 
    denoted $\CE \mapsto \CE_K$ such that for smooth $X$,
    \[
        R^i\Gamma((X/S)_{Cris}, \CE) \otimes K \simeq H^i_{conv}(X, \CE_K),
    \]
    and for proper smooth $X$,
    \[
        R^i\Gamma((X/S)_{Cris}, \CE) \otimes K \simeq H^i_{rig}(X, \CE_K).
    \]
\end{theorem}

\begin{proof}
    For the first functor, when $X$ admits a smooth lifting $\CX$ (this is the local case), 
    $\CE$ corresonds a $\CO_{\CX}$-module with an integrable connection, denoted $\CE$ also. 
    $CE_K := \CE \otimes K$ is then a coherent sheaf 
    over $\CX_K$ with induced integrable connection. 
    The comparison then holds. 
    We may use descent theory to work out in general case.

    For the second functor, note that for projective $X$, 
    there is a frame $X \to X \to \CP$ and thus $Isoc^\dagger(X) \simeq Isoc(X)$. 
    Resolving general proper smooth varieties 
    by projective smooth verieties proves the general case. 
\end{proof}

\begin{remark}
    In practice, we shall use overconvergent $F$-isocrystals only. 
    That is, overconvergent isocrystals with a compatible Frobenius action.
\end{remark}

One can define the cohomology with compact support and cohomology 
supported on a closed set in rigid cohomology theory.
\begin{definition}
    Let $X \to Y \to \CP$ be a proper smooth frame. 
    View $\YP$ as an adic space and then $\XP$ is an open subspace of it. 
    Denote $i\colon \overline{\XP} \to \YP$ and $j$ its complement.
    \[
        H^\cdot_{c,rig}(X) = R\Gamma_{\YP} (i_*i^! \DRY) 
        \simeq R\Gamma_{\YP} (\DRY \to j_*j^*\DRY).
    \]
\end{definition}

\begin{definition}
    Let $X \to Y \to \CP$ be a proper smooth frame and $Z$ a closed subvariety of $X$. 
    Denote $j \colon \ZP \to \YP$ and $i$ its complement.
    \[
        H^\cdot_Z(X) = R\Gamma_{\YP} (j_! j^* \DRY) 
        \simeq R\Gamma(\DRY \to i_*i^* \DRY).
    \]
\end{definition}

Then we have exact sequences for $X=U\cup Z$, 
with $U$ open and $Z$ closed 
(we shall omit the subscript ``rig'' if there is no confusion):
\[
    \cdots \to H^*_Z(X) \to H^*(X) \to H^*(U) \to H^{*+1}_Z(X) \to \cdots
\]
and 
\[
    \cdots \to H^*_c(U) \to H^*_c(X) \to H^*_c(Z) \to H^{*+1}_c(U) \cdots,
\]
and some excision theorems. 
For example, for $Z \subset Y \subset X$ being closed varieties, 
there is an exact sequence
\[
    \cdots \to H^*_Y(X) \to H^*_Z(X) \to H^*_Z(Y) \to H^{*+1}_Y(X) \to \cdots.
\]

\subsection{Verification of the desired properties}
We verify the assuptions in the Weil cohomology theory in remains of this section. 
\begin{proposition}[Gysin isomorphism, \cite{St}]
    For a smooth variety $X$ and its smooth, closed subvariety $Z$, 
    if $X$ is liftable, there is an isomorphism
    \[
        H^*_Z(X) \simeq H^{*-2c}(Z),
    \]
    called the Gysin isomorphism, where $c$ is the codimension of $Z$ in $X$.
\end{proposition}

\begin{proof}
    For $X$ is liftable, $Z$ is also. 
    One represents those rigid cohomology by de Rham cohomology 
    and then define the Gysin map. 
    It can be checked that it is an isomorphism.
\end{proof}

\begin{theorem}[Finiteness theorem, \cite{Be3}]
    For any variety $X / k$ and overconvergent $F$-isocrystal $\CE$ on it, 
    the cohomology $H^*_{rig}(X, \CE)$ is finite dimensional.
\end{theorem}

\begin{proof}
    We shall prove the case that $\CE$ is trivial only. 
    Firstly consider the case that $X$ is smooth. 
    We apply induction. Consider the following two propositions:
    \begin{itemize}
        \item [(a)\textsubscript{$n$}] 
            $H^*(X)$ is finite dimensional for all smooth varieties $X$ 
            with dimension no greater than $n$.
        \item [(b)\textsubscript{$n$}] 
            $H^*_Z(X)$ is finite dimensional for all varieties $Z$ 
            with dimension no greater than $n$ and $X$ being smooth.
    \end{itemize}
    The proposition (a)\textsubscript{$0$} is clear. 
    (b)\textsubscript{$0$} follows from the Gysin ismorphism. 
    (One can choose an affine open set $U$ containing $Z$, 
    $H^*_Z(U) \simeq H^*_Z(X)$ by excision. $U$ is liftable.)

    The implication (b)\textsubscript{$n-1$} $\Rightarrow$ (a)\textsubscript{$n$}: 
    for such $X$, by de Jong's alternation theorem (theorem 4.1 in \cite{dJ}), 
    there is a projective smooth variety $X'$ and its open set $U$, 
    such that $p \colon U \to X$ is proper and generically \'etale. 
    There is then a dense open set $U_1$ of $X$ such that 
    $p \colon p^{-1}(U_1) \to U_1$ is \'etale (hence finite). 
    By finiteness theorem for crystalline cohomology, 
    the cohomology of $X'$ is of finite dimension. 
    By the long exact sequence above and (b)\textsubscript{$n-1$}, 
    the cohomology of $p^{-1}(U_1)$ is also. 
    There is a trace map for finite morphisms 
    (it can be defined locally via de Rham cohomology, and then globally via gluing), 
    becoming a section (up to a scalar) of the pull back map. 
    Thus the cohomology of $U_1$ is finite dimensional. 
    By (b)\textsubscript{$n-1$} and the long exact sequence again, 
    the cohomology of $X$ is finite dimensional.

    The implication (b)\textsubscript{$n-1$} and 
    (a)\textsubscript{$n$} $\Rightarrow$ (b)\textsubscript{$n$}: 
    one applies excision to reduce to the case that $X, Z$ are smooth and $X$ is liftable. 
    Then apply (a)\textsubscript{$n$} and the Gysin isomorphism to conclude.

    For general $X$, it is concluded by choosing a simplicial resolution of $X$ 
    by smooth varieties and applying cohomological descent. 
    See \cite{Tsu} for details. 
\end{proof}

Using similar methods, one proves that 
the cohomology with compact support is of finite dimension.
\begin{theorem}[K\"unneth formula, \cite{Be4}]
    For $X,Y$ arbitrary varieties, $\CE_1$ an overconvergent isocrystal on $X$ 
    and $\CE_2$ an overconvergent isocrystal on $Y$, 
    there is an isomorphism 
    (then $\CE_1 \boxtimes \CE_2$ is an overconvergent isocrystal on $X \times Y$):
    \[
        H^\cdot(X, \CE_1) \otimes H^\cdot(Y, \CE_2) \simeq H^\cdot(X \times Y, \CE_1 \boxtimes \CE_2).
    \]
    Similarly for cohomology with compact supports.
\end{theorem}

\begin{proof}
    It is almost the same as the case for crystalline cohomology. 
    One can check it by first working locally 
    via the de Rham complex and then noticing that the map is canonical. 
\end{proof}

\begin{theorem}[Poincar\'e duality, \cite{Be4}]
    For smooth $X$ of dimension $d$, 
    there is a trace map $\tr \colon H^{2d}(X) \to K$ such that 
    for an overconvergent $F$-isocrystal $\CE$, the pairing
    \[
        H^*(X, \CE) \otimes H^{2d-*}_c(X, \CE^\vee) \to H^{2d}(X) \to K
    \]
    is perfect.
\end{theorem}

\begin{proof}
    The trace map is defined almost the same as the case for crystalline cohomology. 
    We shall prove the case that $\CE$ is trivial. We consider two propositions:
    \begin{itemize}
        \item [(a)\textsubscript{$n$}] 
            Poincar\'e duality holds 
            for all smooth varieties of dimension no greater than $n$.
        \item [(b)\textsubscript{$n$}] 
            The pairing ($d$ is the dimension of $Z$)
            \[
                H^*_Z(X) \times H^{2d-*}_c(Z) \to K
            \] 
            (the trace map can be similarly defined) is perfect 
            for $Z$ (may not be smooth) of dimension no greater than $n$ and $X$ smooth.
    \end{itemize}
    The case (a)\textsubscript{$0$} is trivial. 
    (b)\textsubscript{$0$} is from the Gysin isomorphism. 
    (b)\textsubscript{$n-1$} $\Rightarrow$ (a)\textsubscript{$n$} is done 
    by de Jong's alternation theorem 
    and the Poincar\'e duality for crystalline cohomology 
    (note that the trace map for finite morphisms are compatible with all those constructions). 
    (b)\textsubscript{$n-1$} and 
    (a)\textsubscript{$n$} $\Rightarrow$ (b)\textsubscript{$n$} is done 
    by excision and Gysin isomorphism. 
    The whole process is similar to the proof of finiteness and we shall omit it.
\end{proof}

\begin{remark}
    The case for general overconvergent $F$-isocrystals $\CE$ 
    can be proved by ``$p$-adic local monodromy theorem'', 
    which claims that after a suitable base change, 
    the isocrystal is unipotent, i.e., 
    has a filtration with each subquotient becoming trivial. 
    See \cite{Ke1} for details.
\end{remark}

\begin{remark}
    The cycle class map can be defined 
    firstly by de Rham cohomology locally 
    and then applying cohomological descent.
\end{remark}

All requirements for a good cohomology theory are therefore proved. 
We end this article by remarking an example.
\begin{example}
    If a family of varieties over $k$ comes from reduction of a proper smooth family, 
    the cohomology of all varieties in the family has the same dimension. 
    The family of hypersurfaces in the projective space is the case. 
    For example, all plane cubic curves have 
    $h^0 = 1$, $h^1 = 2$, $h^2=1$ and $h^k = 0$ for $k > 2$. 
    Here $h^k := \dim H^k$.
\end{example}

\begin{proof}
    The first statement is from the fact that 
    Betti numbers are invariant over a family and 
    the comparison theorem between de Rham cohomology and Betti cohomology. 

    The second statement is from the fact that 
    for any homogenous polynomial $f\in k[x_0, \dots, x_n]$, 
    there is a element $\tilde{f} \in W(k)[x_0, \dots, x_n]$ 
    defining a smooth hypersurface of $\BP_K^n$, 
    as the set of singular hypersurfaces is Zariski closed in the parameter space 
    and all $\tilde{f}$ is Zariski dense in the the parameter space.

    The last statement follows from computations of Betti numbers of cubic curves. 
\end{proof}

The example illustrates that for singular varieties, 
rigid cohomology may not be the ``intuitive'' one. 
For example, the first Betti number of a cuspidal cubic curve over $\BC$ is $0$ 
and the first rigid cohomology of a cuspidal cubic curve over $k$ is of dimension 2. 