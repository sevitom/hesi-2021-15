Let us fix the notations: $p$ a prime, $\ell$ a prime different from $p$, 
$k$ a field of characteristic $p$ and $K$ a field of characteristic $0$.

The motivation of a ``good" cohomology theory 
(more precisely, Weil cohomology theory, \cite{Sta}) 
for algebraic varieties is from Weil conjecture, 
which is about the zeta function of algebraic variety $X$ of characteristic $p$,
\[
    \zeta(X, t)=\sum_{n = 1} ^ {\infty} |X(\mathbb{F} _ {q^n})| \frac {t^n} {n}.
\]
\begin{theorem}[Weil conjecture]
    The following holds:
    \begin{itemize}
        \item (Rationality) 
            \[
                \zeta(X, t) = \prod_ {i = 0} ^ {2d} P_i(t) ^ {(-1) ^ {i + 1}},
            \]
            with $P_i(t)\in \BQ[t]$ and $d$ the dimension of $X$.
        \item (Functional equation) 
            For proper smooth $X$, 
            \[
                \zeta(X, \frac {1} {q^d t})=\pm q^{\frac {dE} {2}} t^E \zeta(X, t), 
                \quad E = \sum_{i = 0}^{2d} \deg P_i(t).
            \]
        \item (Purity) 
            The inverse of roots of $P_i(t)$ are algebraic integers 
            with absolute value $q^{\frac{k}{2}}$, 
            where $k$ is an integer between $0$ and $2i$. 
            When $X$ is proper and smooth, $k=i$. 
            (It is an analog of purity of mixed Hodge structures.)
    \end{itemize}
\end{theorem}
To prove the theorem one needs a Weil cohomology theory, 
that is, contravariant functors $H^\bullet(X)$, $H^\bullet_c(X)$, $H^\bullet_Z(X)$ 
(the usual cohomology, cohomology with compact support and cohomology supported on a closed set) 
from varieties over a field $k$ of characteristic $p$ 
to vector spaces over some characteristic $0$ field, satisfying:
\begin{itemize}
    \item (Finiteness) 
        $H^\bullet (X)$, $H^\bullet_Z(X)$, $H^\bullet_c(X)$ are finite dimensional 
        with $H^i(X) = 0$ for $i < 0$ or $i > 2d$.
    \item (K\"unneth formula) 
        $H^\bullet (X\times Y) = H^\bullet (X) \otimes H^\bullet (Y)$, 
        similarly for $H^\bullet_c$ and $H^\bullet_Z$. 
    \item (Poincar\'e duality) 
        For smooth $X$, there is a trace map $H^{2d}(X) \to K$ satisfying 
        $H^\bullet(X) \otimes H^{2d-\bullet}_c(X) \to H^{2d}(X) \to K$ being a perfect pairing.
    \item (Cycle class) 
        There is a natural tranformation of functors $A^\bullet(X) \to H^{2 \bullet}(X)$.
\end{itemize}
One deduces then the Lefschetz fixed point theorem 
and apply it to the Frobenius $F^n \colon X \to X$ 
to gain information about the rational points.

The ``Riemann hypothesis" part gives information about 
the archimedean absolute value of the roots of $P_k(t)$ 
and then the number of rational points. 
It is then natural to ask about $p$-adic absolute value of those roots 
($|\alpha|_{\ell} = 1$). 
However, $p$ -adic \'etale cohomology behaves badly for characteristic $p$ varieties. 
Thus, alternative cohomology theories are necessary.

In characteristic zero, the de Rham cohomology theory 
$H^i(X) = \mathbb{H}^i(\Omega_{X/K}^\bullet)$ 
is a good cohomology theory \cite{Sta}. 
Therefore, a rough idea is to lift $X/k$ to something over $K = \Frac(W(k))$, 
and working out the de Rham cohomology. 
Two ways to realize the idea are the crystalline cohomology and the rigid cohomology.

Another benefit of $p$-adic cohomology theories is that 
the Frobenius action on \'etale cohomology can not be calculated down simply. 
However, in $p$ -adic cohomology theory, 
one can represent it as a action of a de Rham complex. 
There is then a convenient algorithm of counting rational points (\cite{Ke2}).

