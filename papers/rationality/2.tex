In this section, we relate
rationality with several other invariants of a variety.
We show that retract rationality implies that 
these invariants are trivial. 
The results are summarised in the following diagram,
although some assumptions are dropped:
\[
    \begin{array}{c}
        \text{retract} \\
        \text{rational}
    \end{array}
    \Longrightarrow
    \begin{array}{c}
        \text{universally} \\
        \text{$\CH_0$-trivial}
    \end{array}
    \Longleftrightarrow
    \begin{array}{c}
        \text{decomposition} \\
        \text{of the diagonal}
    \end{array}
    \Longrightarrow
    \begin{array}{c}
        \text{trivial} \\
        \text{Brauer group.}
    \end{array}
\]


\subsection{Chow groups}

In this subsection, we recall the definition and some basic properties of the Chow groups of a variety.
A general reference is \cite{fulton}.

In the following, let $\bbk$ be a field, and let $X$ be a $\bbk$-variety.

\begin{definition}
    Let $d$ be an non-negative integer.
    The free abelian group
    \[ \textstyle \upZ_d (X) = \bigoplus \limits _{Z \subset X} \bbZ \cdot [Z], \]
    where $Z$ runs through all $d$-dimensional integral closed subvarieties of $X$,
    is called the group of \term{$d$-cycles} of $X$.
\end{definition}

In other words, a $d$-cycle of $X$ is a finite sum
$\sum n_i [Z_i]$, where each $n_i$ is an integer,
and each $Z_i$ is a subvariety of $X$.
For example, a $0$-cycle is a linear combination of closed points.

Let $V \subset X$ be a $(d+1)$-dimensional integral closed subvariety, 
and let $f \in \bbk (V) ^\times$ be a rational function on $V$. 
The principal divisor of $V$ corresponding to $f$,
denoted by $\operatorname{div}(f)$, 
can be naturally seen as a $d$-cycle of $X$.
This defines a map of abelian groups
\[ \operatorname{div} \: \bbk (V)^\times \to \upZ_d (X). \]

\begin{definition}
    The $d$-th \term{Chow group} of $X$ is defined by
    \[ \textstyle \CH_d (X) = \operatorname{coker} \Bigl( \ 
        \bigoplus \limits _{V \subset X} \bbk (V)^\times \to \upZ_d (X) \Bigr), \]
    where $V$ runs through all $(d+1)$-dimensional integral closed subvarieties of~$X$.

    If $X$ has dimension $n$, we write
    \[ \upZ^d (X) = \upZ_{n-d} (X) \quad \text{and} \quad \CH^d (X) = \CH_{n-d} (X). \]
\end{definition}

An element of the Chow group is thus a class of cycles.
We say that the cycles in the same class are \term{rationally equivalent}.

Here we explain four operations of the Chow group:
proper pushforward, flat pullback, the intersection product, and the Gysin map.

\begin{definition}
    Let $f \: X \to Y$ be a proper map of $\bbk$-varieties,
    and let $Z \subset X$ be an integral closed subvariety of dimension $d$.
    We define 
    \[ f_* [Z] = \begin{cases}
        \ 0, & \text{if } \dim f(Z) < d, \\
        \ [\bbk(f(Z)) : \bbk(Z)] \, [f(Z)], & \text{if } \dim f(Z) = d,
    \end{cases} \]
    as a $d$-cycle of $Y$, where
    \begin{itemize}
        \item
            $f(Z)$ is an irreducible closed subset of $Y$,
            which we see as an integral closed subscheme. 
        \item
            The field extension $\bbk(f(Z)) / \bbk(Z)$ is finite
            because it is finitely generated and of transcendence degree $0$.
    \end{itemize}
    This extends linearly to a \term{pushforward map}
    \[ f_* \: \upZ_d (X) \to \upZ_d (Y). \]
\end{definition}

It turns out that proper pushforward preserves rational equivalence \cite[\S1.4]{fulton}.
We thus obtain a \term{pushforward map} of Chow groups
\[ f_* \: \CH_d (X) \to \CH_d (Y). \]

Next, we introduce the flat pullback of Chow groups.

\begin{definition}
    Let $f \: X \to Y$ be a map of $\bbk$-varieties, which is flat of relative dimension $r$.
    Let $Z \subset X$ be an integral closed subvariety of dimension $d$. We define
    \[ f^*[Z] = [f^{-1}(Z)] \]
    as a $(d+r)$-cycle of $Y$,
    where
    \begin{itemize}
        \item
            $f^{-1}(Z)$ is the scheme-theoretic inverse image, i.e., $Z \times_Y X$.
        \item
            $[f^{-1}(Z)]$ denotes the sum $\sum m_i [Z_i]$,
            where the $Z_i$ are the irreducible components of $f^{-1} (Z)$,
            and $m_i$ is the \emph{geometric multiplicity} of $Z_i$ in $f^{-1} (Z)$,
            defined as the length of the local ring $\mathscr{O}_{f^{-1} (Z), Z_i}$.
    \end{itemize}
    This extends linearly to a \term{pullback map}
    \[ f^* \: \upZ_d (Y) \to \upZ_{d+r} (X). \]
\end{definition}

It turns out again that flat pullback preserves rational equivalence \cite[\S1.7]{fulton}.
Switching to the cohomological indexing notation,
we get a \term{pullback map} of Chow groups
\[ f^* \: \CH^d (Y) \to \CH^d (X). \]

Now, we introduce the intersection product of cycles.

Let $Z_1, Z_2 \subset X$ be two integral closed subvarieties.
We can form their ``scheme-theoretic intersection''
\[ Z_1 \cap Z_2 = Z_1 \times_X Z_2. \]
However, this does not always produce cycles of the expected dimension,
as the subvarieties may not be in a general position to intersect.
To work around this difficulty, we use the Gysin map of the diagonal map, 
which acts as a pullback along a closed embedding.

The Gysin map is defined for vector bundles as follows.

\begin{theorem} \label{thm-2-gysin}
    Let $p \: E \to X$ be a vector bundle of rank $r$. Then the flat pullback
    \[ p^* \: \CH_d (X) \to \CH_{d+r} (E) \]
    is an isomorphism for all $d$. 
    Its inverse is called the \term{Gysin map}, and denoted by $i^!$,
    where $i$ is the zero section map $X \to E$.
\end{theorem}

See \cite[\S3.3]{fulton}.

Recall that a \term{regular embedding} is a
closed embedding of schemes, such that the ideal sheaf
is locally generated by regular sequences.
For example, a closed embedding of smooth varieties is always a regular embedding.

For a regular embedding, the normal cone is a vector bundle.
We can use this property to extend the definition of the Gysin map
to this case.

\begin{definition}
    Let $i \: Z \to X$ be a regular embedding of constant codimension~$e$.
    The \term{Gysin map}
    \[ i^! \: \CH_d (X) \to \CH_{d-e} (Z) \]
    is defined as follows.
    Let $N_Z X$ denote the normal bundle of $Z$ in $X$, and let
    \[ \sigma \: \upZ_d (X) \to \upZ_d (N_Z X) \]
    be the map given by
    \[ [V] \mapsto [N_{Z \cap V} V]. \]
    This map respects rational equivalence \textup{\cite[\S5.2]{fulton}}, inducing a map
    \[ \sigma \: \CH_d (X) \to \CH_d (N_Z X) \]
    We then compose this map with the Gysin map defined in Theorem~\textup{\ref{thm-2-gysin}},
    giving the desired map
    \[ i^! \: \CH_d (X) \to \CH_{d-e} (Z). \]
\end{definition}

Using the Gysin map as a pullback along the diagonal map,
we can define the intersection product of cycles.

\begin{definition}
    Let $X$ be an $n$-dimensional projective variety.
    The \term{intersection product} is a map of graded abelian groups
    \[ \cdot \: \CH^\bullet (X) \otimes \CH^\bullet (X) \to \CH^\bullet (X), \]
    defined as follows.
    \begin{itemize}
        \item
            If $X$ is smooth, we define this map by the composition
            \[ \CH_{d_1} (X) \otimes \CH_{d_2} (X) 
                \overset{\times}{\longrightarrow} \CH_{d_1 + d_2} (X \times X) 
                \overset{\Delta^!}{\longrightarrow} \CH_{d_1 + d_2 - n} (X), \]
            where $\times$ denotes the cross product map sending $[Z_1] \otimes [Z_2]$ to $[Z_1 \times Z_2]$,
            and $\Delta \: X \to X \times X$ is the diagonal map, which is a regular embedding.
        \item
            If $X$ is arbitrary, we can always embed $X$ in some $\bbP^N$,
            so that we can regard cycles of $X$ as cycles of $\bbP^N$, 
            and intersect them in $\bbP^N$.
    \end{itemize}
\end{definition}

This product equips the Chow groups with the structure of a graded ring,
called the \term{Chow ring}.


\subsection{Rationality and zero-cycles}

\begin{definition}
    We say that a map $f \: X \to Y$ of $\bbk$-varieties is
    \term{universally $\CH_0$-trivial}, if
    \begin{itemize}
        \item
            $f$ is proper.
        \item
            For any field extension $F / \bbk$, the pushforward map
            \[ f_* \: \CH_0 (X_F) \to \CH_0 (Y_F) \]
            is an isomorphism.
    \end{itemize}
    If $Y = \Spec \bbk$, then we say $X$ is
    \term{universally $\CH_0$-trivial}. This means that
    \begin{itemize}
        \item
            $X$ is complete.
        \item
            For any field extension $F / \bbk$, the degree map
            \[ \deg_F \: \CH_0 (X_F) \to \bbZ \]
            is an isomorphism.
    \end{itemize}
\end{definition}

We will show that retract rationality implies universal $\CH_0$ triviality.
The proof will need a few lemmas.
First of all, we prove a moving lemma for $0$-cycles.

\begin{lemma} \label{lem-2-moving}
    Let $X$ be a smooth projective $\bbk$-variety, with $\bbk$ infinite and perfect,
    and let $U \subset X$ be a dense open set.
    Then every $0$-cycle of $X$ is rationally equivalent to one supported in $U$.
\end{lemma}

\begin{proof}
    We follow \cite[Complément]{colliot-finitude}.
    Write $Z = X \setminus U$, and let $p \in Z$ be a closed point.
    It suffices to show that the $0$-cycle $[p]$ is equivalent to one supported in $U$.

    Let $g \in \mathscr{O}_{X,p}$ be a locally defined non-zero function that vanishes on $Z$.
    Since $X$ is smooth, we can find a regular sequence $f_1, \dotsc, f_{n-1}$
    of $\mathscr{O}_{X,p}$, where $n = \dim X$, 
    such that $g \neq 0$ in the quotient $\mathscr{O}_{X,p} / (f_1, \dotsc, f_{n-1})$.
    This can be done by working in affine coordinates and taking the $f_i$ to be linear functions.

    The ideal $(f_1, \dotsc, f_{n-1})$ defines a curve in a neighbourhood of $p$.
    Taking its closure in $X$, we obtain a closed integral curve $C$ in $X$,
    which is regular at $p$ and is not contained in $Z$. 

    Let $f \: D \to C$ be the normalisation of $C$.
    Then $D$ is regular (normality implies regularity in codimension one),
    and hence smooth since $\bbk$ is perfect.
    Also, $D$ is quasi-projective \cite[Corollary~7.4.10]{EGA2},
    and $f$ is finite \cite[Corollary~7.4.6]{EGA2}, and hence proper \cite[Corollary~6.1.11]{EGA2}.
    
    Let $q$ be the inverse image of $p$, which is a single point as $C$ is regular at $p$.
    Let $W$ be a neighbourhood of $q$ in $D$, such that $f|_W$ is an isomorphism.
    For example, we can take $W$ to be the inverse image of the smooth locus of $C$.

    The $0$-cycle $[q]$ is equivalent to a $0$-cycle supported in $W \setminus f^{-1} (Z)$.
    Indeed, we have to find a rational function on $D$ 
    that has a simple zero at $q$, and is defined on the finite set $(D \setminus W) \cup f^{-1}(Z)$.
    As $D$ is quasi-projective, this can be done by taking a suitable linear function on the projective space.

    Finally, we consider the pushforward along the proper map $D \to C \to X$.
    Since it preserves rational equivalence, we are done.
\end{proof}

\begin{remark}
    This is a special case of \cite[Proposition~7.4.9]{EGA2},
    but the proof given here is more elementary.
\end{remark}

\begin{lemma} \label{lem-2-finite}
    Let $\bbk$ be a finite field, and let $U \subset \bbP_{\bbk}^n$ be a non-empty open set.
    Then for any extension $F / \bbk$ of sufficiently large degree $d$,
    the open set $U_F \subset \bbP_{F}^n$ contains an $F$-rational point.
\end{lemma}

\begin{proof}
    Let $q$ be the cardinality of $\bbk$.
    Let $f$ be a non-zero homogeneous polynomial over~$\bbk$,
    which vanishes outside $U$.

    In $\bbP_{F}^n$, when $q^d > \deg f$, 
    there are at least $(q^d - \deg f)^n$ rational points
    where $f$ does not vanish.
    Indeed, by induction on $n$, one easily shows that a non-zero polynomial
    of degree $r$ on $\bbA_{F}^n$ does not vanish at at least $(q^d - r)^n$ rational points, provided that $q^d > r$.

    Hence $U_F$ contains a rational point whenever $q^d > \deg f$.
\end{proof}

\begin{theorem} [Colliot-Thélène and Pirutka] \label{thm-2}
    Let $X$ be a smooth projective $\bbk$-variety.
    If $X$ is retract rational, then $X$ is universally $\CH_0$-trivial.
\end{theorem}

\begin{proof}
    First, we suppose that the base field $\bbk$ is infinite.
    Since retract rationality is preserved under change of base field,
    it suffices to prove that $X$ is $\CH_0$-trivial over $\bbk$,
    i.e.\ $\deg_{\bbk} \: \CH_0 (X) \to \bbZ$ is an isomorphism.

    By definition, there exist non-empty open sets
    $U \subset X$, $V \subset \bbP_{\bbk}^n$, and maps
    \[ U \overset{f}{\longrightarrow} V \overset{g}{\longrightarrow} U, \]
    whose composition is $\id_U$.

    Let $P \in U$ be a closed point, and write $Q = f(P) \in V$.
    Then we have induced maps of residue fields
    $\kappa (P) \to \kappa (Q) \to \kappa (P)$, whose composition is $\id_{\kappa(P)}$.
    Therefore, we have
    \[ \kappa (P) \simeq \kappa (Q). \]
    Let $F$ denote this field. We then consider the diagram
    \[ \begin{tikzcd}
        \llap{$\bbP_{F}^n \supset {}$} V_F \ar[d, "p"'] \ar[r, "g_F"] &
        U_F \rlap{${} \subset X_F$} \ar[d] \\
        \llap{$\bbP_{\bbk}^n \supset {}$} V \ar[r, "g"] &
        U \rlap{${} \subset X$ .}
    \end{tikzcd} \]
    There exists $R \in p^{-1} (Q)$ such that $\kappa (R) \simeq F$.
    This is because $p^{-1} (Q) \simeq \Spec (F \otimes_{\bbk} F)$,
    and we can take $R$ to be the point defined by the maximal ideal
    which is the kernel of the multiplication map $F \otimes_{\bbk} F \to F$.

    Let $A \in V \subset \bbP_{\bbk}^n$ be a $\bbk$-rational point,
    which exists since $\bbk$ is infinite.
    Let $L \simeq \bbP_F^1 \subset \bbP_F^n$ be a line connecting $R$ and $A_F$.
    The map $g_F$ sends $L$ to a rational line $L'$ in $U_F$,
    which is a rational $F$-map $\bbP_F^1 \mathrel{\rightdasharrow} U_F$.
    This map extends to a map 
    \[ L' \: \bbP_F^1 \to X_F, \]
    which is proper since $\bbP_F^1$ is complete \cite[Corollary~5.4.3]{EGA2}.
    This is a line connecting the points $g_F(R)$ and $g_F(A_F)$. 

    As $R$ is an $F$-rational point, we have $\kappa(g_F(R)) \simeq F$
    and $\kappa( g_F (A_F) ) \simeq F$.
    Hence, the pushforward map of $L'$ on $\CH_0$ gives
    \[ [g_F(R)] = [g_F(A_F)] \quad \in \CH_0 (X_F). \]
    Pushing forward to $X$, and noticing that $\kappa( g(A) ) \simeq \bbk$, we thus have
    \[ [P] = [F : \bbk] \, [g(A)] \quad \in \CH_0 (X). \]
    
    By the moving lemma~\ref{lem-2-moving},
    every $0$-cycle of $X$ is equivalent to one supported in $U$,
    which, as we have shown, is equivalent to a multiple of $[g(A)]$.
    Since $\deg_{\bbk} {} [g(A)] = 1$,
    this shows that $X$ is $\CH_0$-trivial over $\bbk$.

    Finally, if $\bbk$ is a finite field,
    by Lemma~\ref{lem-2-finite}, we can apply the above argument
    to an extension $F / \bbk$ of sufficiently large degree $d$. Since the composition
    \[ \CH_0 (X) \overset{p^*}{\longrightarrow} \CH_0 (X_F) \overset{p_*}{\longrightarrow} \CH_0 (X) \]
    is multiplication by $d$, where $p \: X_F \to X$ is the projection,
    it follows that every $0$-cycle of $X$ of degree $0$
    is $d$-torsion in $\CH_0 (X)$.
    Hence it must be zero,
    since $d$ can be chosen to be two coprime values.
    Moreover, this also shows that $X$ has two $0$-cycles of coprime degrees,
    and hence, the degree map $\deg \: \CH_0 (X) \to \bbZ$ is surjective.
\end{proof}

We also mention the following criterion for universal $\CH_0$-triviality of a morphism,
which will be useful later.

\begin{theorem} [Colliot-Thélène and Pirutka] \label{thm-4-ch0-trivial-fibre}
    Let $f \: \widetilde{X} \to X$ be a proper morphism of $\bbk$-varieties.
    Suppose that
    \begin{itemize}
        \item
            For every point $M \in X$, not necessarily closed,
            the fibre $\widetilde{X}_M$ is universally $\CH_0$-trivial
            as a $\kappa (M)$-variety.
    \end{itemize}
    Then $f$ is universally $\CH_0$-trivial.
\end{theorem}

\begin{proof}
    It suffices to show that $f_* \: \CH_0 (\widetilde{X}) \to \CH_0 (X)$ is an isomorphism.

    By assumption, this map $f_*$ is surjective.
    Let $x$ be a $0$-cycle of $\widetilde{X}$,
    such that $f_* (x)$ is equivalent to zero.
    We need to show that $x$ is equivalent to zero.

    In this case, there exist finitely many integral curves $C_i \subset X$,
    and functions $g_i \in \bbk (C_i)$, such that
    \[ \textstyle f_* (x) = \sum_i \operatorname{div}_{C_i} (g_i). \]
    Let $\eta_i$ be the generic point of $C_i$.
    By hypothesis, each fibre $\widetilde{X}_{\eta_i}$ contains a $0$-cycle
    \[ \textstyle \sum_j n_{ij} [D_{ij}] \]
    of degree $1$, where $n_{ij} \in \bbZ$.
    We regard each $D_{ij}$ as a curve in $\widetilde{X}$.
    Then each function $g_i \circ f$ defines a rational function $g_{ij}$ on $D_{ij}$.
    Write
    \[ \textstyle x' = x - \sum_{i,j} n_{ij} \operatorname{div}_{D_{ij}} (g_{ij}). \]
    Then we have an equality of \emph{cycles} $f_* (x') = 0$.

    Let us write $x' = \sum_i x'_{\smash{Q_i}}$, where $Q_i \in X$ are distinct points, 
    and $x'_{\smash{Q_i}}$ is a $0$-cycle of $\widetilde{X}$ supported in the fibre $\widetilde{X}_{Q_i}$.
    The fact that $f_* (x') = 0$ implies that each $x'_{\smash{Q_i}}$ has degree $0$.
    It follows from the hypothesis applied to $\widetilde{X}_{Q_i}$ that $x'_{\smash{Q_i}}$ is rationally equivalent to zero.
    Therefore, $x'$ is equivalent to zero, and so is~$x$.
\end{proof}


\subsection{Decomposition of the diagonal}

We now give an equivalent characterisation of universal $\CH_0$-triviality.
We show that it is equivalent to the existence of a decomposition 
of the diagonal class in the Chow group of $X \times X$.

\begin{definition} \label{def-2-decomp}
    Let $X$ be a complete $\bbk$-variety of dimension $n$.
    A \term{decomposition of the diagonal} of $X$ is given by an equation
    \[ [\Delta_X] = D + [X] \times x_0 \quad \text{in } \CH_n (X \times X), \]
    where
    \begin{itemize}
        \item
            $[\Delta_X]$ is the pushforward of $[X] \in \CH_n (X)$
            along the diagonal map $X \to X \times X$.

        \item
            $D$ is an $n$-cycle of $X \times X$, supported in $Z \times X$
            for some closed subvariety $Z \subset X$ of codimension at least $1$.

        \item
            $x_0$ is a $0$-cycle of $X$ of degree $1$.
    \end{itemize}
\end{definition}

In order to show that this property is equivalent to $\CH_0$-triviality,
we introduce the notion of a correspondence.

\begin{definition}
    Let $X$ and $Y$ be complete $\bbk$-varieties, of dimensions $m$ and $n$, respectively.
    A \term{correspondence} from $X$ to $Y$ is an element of the set 
    \[ \operatorname{Corr} (X, Y) = \CH_m (X \times Y). \]
\end{definition}

We view a correspondence as a generalised version of
a graph of a map from $X$ to $Y$.
In this way, we can compose correspondences as if we are composing graphs of maps.
Namely, for $f \in \operatorname{Corr} (X, Y)$ and $g \in \operatorname{Corr} (Y, Z)$,
we define
\[ g \circ f = p_* \bigl( ([X] \times g) \cdot (f \times [Z]) \bigr)
    \quad \in \operatorname{Corr} (X, Z), \]
where $p \: X \times Y \times Z \to X \times Z$ is the projection map.

Moreover, we have a group homomorphism
\begin{equation} \label{eq-2-corresp-eval}
    \begin{aligned}
        \operatorname{Corr} (X, Y) \otimes_{\bbZ} \CH_\bullet (X) & \to \CH_\bullet (Y), \\
        (f, \alpha) & \mapsto p_* \bigl( f \cdot (\alpha \times [Y]) \bigr),
    \end{aligned}
\end{equation}
where $p \: X \times Y \to Y$ is the projection map.
In particular, this induces an action of $\operatorname{Corr} (X, X)$ on $\CH_\bullet (X)$.

\begin{proposition}
    Complete $\bbk$-varieties and correspondences form a category,
    which admits a functor from the category of
    complete $\bbk$-varieties and $\bbk$-maps.
    The functor $\CH_\bullet$ factors through this functor. \qed
\end{proposition}

Before proving the main theorem,
we need a lemma.

\begin{lemma} \label{lem-2-generic-pt}
    Let $X$ be an integral $\bbk$-variety, and let $\eta$ be its generic point. Consider the map
    \[ \eta \times \id_X \: \Spec ( \bbk (X) ) \times X \simeq X_{\bbk (X)} \to X \times X. \]
    The pullback of the diagonal class is the class of the generic point of $X$, which is a $0$-cycle of $X_{\bbk (X)}$ of degree $1$.
\end{lemma}

\begin{proof}
    We may assume $X = \Spec A$ is affine.
    Then the pullback of the diagonal class is the closed point
    defined by the maximal ideal, which is the kernel of
    the multiplication map
    \[ \bbk (X) \otimes A \to \bbk (X). \]
    The residue field is thus $\bbk (X)$.
\end{proof}

\begin{theorem} [Colliot-Thélène and Pirutka] \label{thm-2-decomp}
    Let $X$ be a smooth, integral, complete $\bbk$-variety. Then the following are equivalent.
    \begin{enumerate}
        \item \label{itm-thm-2-decomp-1}
            $X$ is universally $\CH_0$-trivial.
        \item \label{itm-thm-2-decomp-2}
            $X$ has a $0$-cycle of degree $1$,
            and the degree map $\deg \: \CH_0 (X_{\bbk (X)}) \to \bbZ$ is an isomorphism.
        \item \label{itm-thm-2-decomp-3}
            $X$ admits a decomposition of the diagonal.
    \end{enumerate}
\end{theorem}

\begin{proof}
    The implication \ref{itm-thm-2-decomp-1} $\Rightarrow$ \ref{itm-thm-2-decomp-2}
    follows from the definition of universal $\CH_0$-triviality.

    Assume \ref{itm-thm-2-decomp-2}.
    Let $\alpha$ be a $0$-cycle of $X$ of degree $1$,
    and let $\beta \in \CH_0 (X_{\bbk (X)})$ be the class of the generic point of $X$. 
    Then by hypothesis, we have
    \[ \alpha_{\bbk (X)} = \beta \quad \text{in } \CH_0 (X_{\bbk (X)}). \]
    By \cite[Lemma~1A.1]{bloch}, we have
    \[ \CH^n ( X_{\bbk (X)} ) \simeq \colim _{U \subset X \text{ open}} \CH^n (U \times X). \]
    By Lemma~\ref{lem-2-generic-pt}, the map 
    $\CH^n (X \times X) \to \CH^n \bigl( \Spec (\bbk(X)) \times X \bigr) \simeq \CH_0 (X_{\bbk (X)})$
    maps the diagonal class $[\Delta_X]$ to $\beta$. Therefore,
    there exists a non-empty open set $U \subset X$, such that
    \[ [U] \times \alpha = [\Delta_U] \quad \text{in } \CH_n (U \times X). \]
    Let $Z = X \setminus U$. By \cite[\S1.8]{fulton}, we have an exact sequence
    \[ \CH_n (Z \times X) \to \CH_n (X \times X) \to \CH_n (U \times X) \to 0, \]
    which implies that there exists $D \in \CH_n (Z \times X)$, such that
    \[ D = [X] \times \alpha - [\Delta_X] \quad \text{in } \CH_n (X \times X). \]
    This proves \ref{itm-thm-2-decomp-3}.

    Now assume \ref{itm-thm-2-decomp-3}, and let
    \[ [\Delta_X] = D + [X] \times x_0 \quad \text{in } \CH_n (X \times X) \]
    be a decomposition, with $D$ supported in $Z \times X$.
    Let $F$ be an extension of $\bbk$.

    Consider the action of $\operatorname{Corr} (X, X)$ 
    on $\CH_0 (X_F)$, as defined in (\ref{eq-2-corresp-eval}).
    Since $[\Delta_X]$ acts as the identity, so does $D + [X] \times x_0$.

    On the other hand, by the moving lemma \ref{lem-2-moving},
    every $0$-cycle of $X$ can be moved out of $Z$. 
    This shows that the action of $D$ is $0$,
    as in the defining equation (\ref{eq-2-corresp-eval}), we are taking the intersection product of two disjoint cycles.
    But for any $0$-cycle $\gamma$ of $X$, the action of $[X] \times x_0$ sends it to
    \[ 
        p_* \bigl( ([X] \times x_0) \cdot (\gamma \times [X]) \bigr) = p_* (\gamma \times x_0) = (\deg \gamma) \, x_0,
    \]
    where $p \: X \times X \to X$ is the second projection. This
    implies that $\CH_0 (X_F)$ is generated by $x_0$, which has degree $1$.
    This proves \ref{itm-thm-2-decomp-1}.
\end{proof}

% \begin{remark} \label{rem-2-decomp-mult-components}
%     The implication \ref{itm-thm-2-decomp-3} $\Rightarrow$ \ref{itm-thm-2-decomp-1}
%     still holds if we allow $X$ to have multiple irreducible components,
%     each being smooth, with the extra assumption that the intersection of
%     every two irreducible components is either empty or contains a rational point of degree $1$.
% \end{remark}


\subsection{The Brauer group}

In this subsection, we show that the existence of a decomposition of the diagonal 
implies the triviality of the Brauer group.

\begin{definition}
    The (cohomological) \term{Brauer group} of a scheme $X$ is the étale cohomology group
    \[ \Br (X) = H^2 (X, \Gm). \]
\end{definition}

For a field $\bbk$, the Brauer group $\Br (\bbk) = \Br (\Spec \bbk)$
coincides with the classical notion defined as the group of
equivalence classes of central simple algebras.
A classical reference is \cite{grothendieck-brauer}.

The Kummer exact sequence of étale sheaves
\[ 0 \to \sfmu_n \to \Gm \xrightarrow{(-)^n} \Gm \to 0 \]
induces a long exact sequence
\[ \cdots \to \operatorname{Pic} (X)
    \to H^2 (X, \sfmu_n) \to \Br (X) \xrightarrow{\cdot n} \Br (X) \to \cdots. \]
When $X = \Spec R$, where $R$ is a local ring,
we have $\operatorname{Pic} (X) = 0$, so that
\[ \Br (X) \, [n] \simeq H^2 (X, \sfmu_n), \]
where the left hand side denotes the $n$-torsion subgroup of $\Br (X)$.

\begin{definition} \label{def-2-unramified}
    Let $M$ be a contravariant functor from schemes to abelian groups
    (e.g.\ étale cohomology), and let $\bbk \subset \bbK$ be two fields. Let
    \[ M_{\mathrm{nr}} ( \bbK / \bbk ) = 
        \bigcap_{\bbk \subset A \subset \bbK} 
        \operatorname{image} \Bigl( M (\Spec A) \to M (\Spec \bbK) \Bigr), \]
    where $A$ runs through all discrete valuation rings with fraction field $\bbK$.
\end{definition}

This is called the \term{unramified} version of the functor $M$.
For example, one has the \emph{unramified Brauer group} $\Br_{\mathrm{nr}} (\bbK / \bbk)$,
and \emph{unramified cohomology} $H^q_{\mathrm{nr}} (\bbK / \bbk, \sfmu_n)$.

A deep result on the cohomological purity of the Brauer group
gives rise to the following theorem.

\begin{theorem} \label{thm-2-unramified-br}
    Let $X$ be a regular, complete, integral $\bbk$-variety.
    Then the natural map $\Br (X) \to \Br ( \bbk (X) )$ induces an isomorphism
    \[ \Br (X) \simeq \Br_{\mathrm{nr}} ( \bbk(X) / \bbk ). \]
\end{theorem}

See \cite[Proposition~5.2.2]{colliot-brauer}.

The discussion above immediately implies the following.

\begin{corollary}
    Let $X$ be a regular, proper, integral $\bbk$-variety. Then 
    \[ \Br (X) \, [n] \simeq H^2_{\mathrm{nr}} (\bbk (X) / \bbk, \sfmu_n). \thmqedhere \]
\end{corollary}

Since the Brauer group is a torsion group \cite[Proposition~1.3.6]{colliot-brauer},
it can thus be computed by the unramified cohomology groups.

On the other hand, the second cohomology of $\sfmu_n$ is a part of
the cycle module (in the sense of \cite{rost})
\[ \textstyle \bigoplus_{i} H^i (-, \sfmu_n^{\otimes (i - 1)}), \]
which we do not give a precise definition here.
This allows it to be regarded as a ``coefficient group'' for Chow groups.
As a result, there is an action of correspondences
\[ \operatorname{Corr} (X, Y) \otimes H^2_{\mathrm{nr}} (\bbk (X) / \bbk, \sfmu_n)
    \longrightarrow H^2_{\mathrm{nr}} (\bbk (Y) / \bbk, \sfmu_n) \]
for $\bbk$-varieties $X, Y$.
This action will relate the Brauer group
with the decomposition of the diagonal.

\begin{theorem} \label{thm-2-brauer-trivial}
    Let $X$ be a smooth projective $\bbk$-variety.
    If $X$ admits a decomposition of the diagonal,
    then the natural map induces an isomorphism
    \[ \Br (\bbk) \similarrightarrow \Br (X). \]
    In particular, if $\bbk$ is separably closed, then $\Br (X) = 0$.
\end{theorem}

\begin{proof}[Sketch of proof]
    The decomposition of the diagonal implies that 
    the identity map and the constant map (to be precise, a sum of constant maps)
    induce the same action on
    \[ H^2_{\mathrm{nr}} (\bbk (X) / \bbk, \sfmu_n). \]
    But the action of a constant map factors through 
    $H^2_{\mathrm{nr}} (\bbk / \bbk, \sfmu_n)$, via the corestriction map,
    whence the result follows.
\end{proof}

