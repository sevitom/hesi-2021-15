In this section, we describe the deformation method
which produces irrational varieties.
We show that if we have a good family of varieties,
and if one of them does not have a decomposition of the diagonal,
then a very general one in the family
will not have a decomposition, and hence, will not be retract rational.

This method was developed by C. Voisin \cite{voisin},
and modified by J.\nobreakdash-L. Colliot-Thélène and A. Pirutka \cite{CTP},
to show that a very general
quartic threefold in $\bbP_{\bbC}^4$ is not retract rational.
We will present a proof of this result.


\subsection{Families of cycles}

\begin{definition} [Kollár {\cite[Definition~3.10]{kollar}}] \label{def-3-family}
    Suppose that
    \begin{itemize}
        \item
            $\bbk$ is an algebraically closed field of characteristic $0$.
        \item
            $S$ is a $\bbk$-scheme.
        \item
            $X/S$ is a projective $S$-scheme, with a chosen relatively ample line bundle.
        \item
            $B/S$ is a reduced normal $S$-scheme.
        \item
            $d$ and $d'$ are non-negative integers.
    \end{itemize}
    A \term{well-defined family of $d$-cycles} of $X$ of degree $d'$ parametrised by $B$
    is a cycle 
    \[ \textstyle C = \sum_i m_i [C_i] \quad \text{of} \quad X \times_S B, \]
    such that
    \begin{itemize}
        \item
            Each $C_i$ is an integral closed subscheme of $X \times_S B$.
        \item
            For each $i$, the image of the projection map $g_i \: C_i \to B$
            is an irreducible component of $B$.
            In particular, $g_i$ is flat over a dense open subset of $B$.
        \item
            Each fibre of $g_i$ defines a $d$-cycle of $X$ of degree $d'$.
            This means that the fibre is either empty or of dimension $d$,
            and that $g_i$ is flat over a dense open subset of $B$.
    \end{itemize}
\end{definition}

The deep theorem below shows the existence of
a universal family of cycles,
in that every family of cycles is realised as its pullback.

\begin{theorem}
    Under the assumptions of Definition~\textup{\ref{def-3-family}},
    for an $S$-scheme $Z$, define
    \[ \mathrm{Chow}_{X/S}^{d,d'} (Z) = \biggl\{ \,
        \begin{matrix}
            \text{well-defined families of non-negative $d$-cycles} \\
            \text{of $X$ of degree $d'$ parametrised by $Z$}
        \end{matrix} \,
    \biggr\}. \]
    Then 
    \begin{itemize}
        \item
            $\mathrm{Chow}_{X/S}^{d,d'}$
            is a contravariant functor
            from the semi-normal $S$-schemes to sets.
        \item
            Moreover, this functor is represented by a
            projective semi-normal $S$-scheme $\Chow_{X/S}^{d,d'}$,
            called the \term{Chow scheme},
            so that there exists a universal
            well-defined family of non-negative $d$-cycles
            \[ \Univ_{X/S}^{d,d'} \quad \text{of $X$ parametrised by} \quad \Chow_{X/S}^{d,d'}, \]
            such that every other family of cycles is its pullback.
    \end{itemize}
\end{theorem}

See \cite[Theorem~I.3.21]{kollar}. 

We also recall the existence of Hilbert schemes.

\begin{theorem}
    Let $S$ be a locally noetherian scheme.
    Let $X \to S$ be a projective morphism.
    For an $S$-scheme $Z$, define 
    \[ \mathrm{Hilb}_{X/S}^{d,d'} (Z) = \biggl\{ \,
        \begin{matrix}
            \text{closed subschemes of $X \times_S Z$ flat over $Z$} \\
            \text{of relative dimension $d$ and relative degree $d'$}
        \end{matrix} \,
    \biggr\}. \]
    The functor $\mathrm{Hilb}_{X/S}^{d,d'}$ is represented
    by an $S$-scheme $\Hilb_{X/S}^{d,d'}$, called the \term{Hilbert scheme},
    whose irreducible components are projective over $S$.
    As a result, there exists a universal family of subschemes
    \[ U \quad \subset \quad X \times_S \Hilb_{X/S}^{d,d'}, \]
    such that every other family of subschemes is its pullback.
\end{theorem}

Below, we will write
\[ \Chow_{X/S} = \coprod_{d,d'} \Chow_{X/S}^{d,d'}
    \quad \text{and} \quad 
    \Hilb_{X/S} = \coprod_{d,d'} \Hilb_{X/S}^{d,d'}. \]


\subsection{Locus of rational equivalence}

\begin{situation} \label{sit-3-deformation}
    Suppose
    \begin{itemize}
        \item
            $\bbk$ is an algebraically closed field of characteristic $0$.
        \item
            $B$ is a smooth $\bbk$-scheme.
        \item
            $X \to B$ is a projective morphism.
    \end{itemize}
\end{situation}

\begin{lemma} \label{lem-3-divisor}
    In Situation~\textup{\ref{sit-3-deformation}}, for any non-negative integer $d$, there exists
    \begin{itemize}
        \item
            A countable family of normal, irreducible, quasi-projective $B$-schemes $\{ T_i \}$.
        \item
            For each index $i$, a family of smooth $(d+1)$-dimensional varieties $W_i \to T_i$,
            with two families of divisors $E_{i,1}, E_{i,2} \to T_i$ of~$W_i$,
    \end{itemize}
    such that
    \begin{itemize}
        \item
            For any $b \in B$ and any subvariety $V \subset X_b$ of dimension $d+1$,
            there exists a desingularisation $\widetilde{V}$,
            such that for any two effective divisors $D_1, D_2$ of $\widetilde{V}$,
            such that $D_1 - D_2$ is principal,
            there exists $i$ and $t \in (T_i)_b (\bbk)$, 
            such that the data $(\widetilde{V}, D_1, D_2)$ is identical to
            $((W_i)_t, (E_{i,1})_t, (E_{i,2})_t)$.
    \end{itemize}
\end{lemma}

The reason to consider a desingularisation of $V$, instead of $V$ itself,
is that on a smooth variety, a Weil divisor is the same thing
as a Cartier divisor, and the Weil divisor class group is the same as the Picard group.
The normality of $T_i$ is required in order to (later) satisfy
the definition of a well-defined family of cycles.

\begin{proof}
    \def\WjGj{\widetilde{W}\mkern-6mu_j/\widetilde{G}_j}
    By \cite[Theorem~9.7.7]{EGA4-3}, the set of points in the Hilbert scheme $\Hilb_{X/B}^{d+1}$
    corresponding to the geometrically integral subvarieties is locally constructible.
    Let $G$ be an irreducible component of this set, equipped with the reduced scheme structure.
    Then $G$ is quasi-projective over $S$, as the components of $\Hilb_{X/B}^{d+1}$ are projective. Let
    \[ W \subset G \times_B X \]
    be the universal family of $(d+1)$-dimensional subschemes.
    The morphism $W \to G$ is thus projective, flat, with geometrically integral fibres.

    The generic fibre $W_{\bbk(G)}$ is integral, 
    as its irreducible components correspond to irreducible components of a general fibre.
    By Hironaka's theorem, let $\widetilde{W}_{\bbk(G)} \to W_{\bbk(G)}$ be a desingularisation map. 
    This map extends to a map
    \[ \widetilde{W}_1 \to W_1 \]
    of schemes over an open set $G_1 \subset G$, where $W_1 = W|_{G_1}$.
    Shrinking $G_1$ if necessary, we can assume that for any $t \in G_1(\bbk)$, 
    the map $\widetilde{W}_{1,t} \to W_t$ of fibres over $t$ is a desingularisation map.
    
    By noetherian induction, we can find a decomposition
    \[ \textstyle G = \bigcup_{j=1}^m G_j, \]
    with $G_j$ locally closed in $G$, together with
    maps $\widetilde{W}_j \to W_j$ over $G_j$,
    where $W_j = W|_{G_j}$, such that for all $t \in G_j$,
    the map $\widetilde{W}_{j,t} \to W_t$ is a desingularisation map.

    Let $\widetilde{G}_j \to G_j$ be a desingularisation,
    and we still denote by $\widetilde{W}_j \to \widetilde{G}_j$
    the pullback of the family $\widetilde{W}_j \to G_j$.

    Since $\widetilde{W}_j$ is projective and flat over $\widetilde{G}_j$,
    with geometrically integral fibres, there exist the schemes with a morphism
    \[ \mathrm{Ab} \: \sfname{Div}_{\WjGj} \to \sfname{Pic}_{\WjGj}, \]
    where $\sfname{Div}_{\WjGj}$ is the scheme parametrising the effective Cartier divisors
    \cite[Theorem~9.3.7]{FAG},
    which is quasi-projective over $\widetilde{G_j}$, and hence over $B$,
    and $\sfname{Pic}_{\WjGj}$ is the Picard scheme \cite[Theorem~9.4.8]{FAG}.
    Let
    \[ \Delta_j \subset \sfname{Div}_{\WjGj} \times \sfname{Div}_{\WjGj} \]
    be the inverse image of the diagonal of $\sfname{Pic}_{\WjGj} \times \sfname{Pic}_{\WjGj}$,
    under the map $\mathrm{Ab} \times \mathrm{Ab}$, equipped with the reduced scheme structure.
    Let $T$ be one of its irreducible components, and let $\widetilde{T}$ be the normalisation of $T$.
    Thus $\widetilde{T}$ is quasi-projective over $B$.

    The family of all the schemes $\widetilde{T}$, together with the two universal families of divisors
    given by the $\sfname{Div}$ schemes, satisfies the requirement of the lemma.
\end{proof}

Of course, a rational equivalence of two $d$-cycles may involve
more than one $(d+1)$-dimensional subvariety.
The next lemma deals with this situation.

For simplicity, if $V \subset X_b$ is a subvariety,
we will say ``the desingularisation'' of $V$
when we refer to the variety $\widetilde{V}$ given by the previous lemma,
and we simply add a tilde to indicate this desingularisation.

\begin{lemma} \label{lem-3-divisors}
    In Situation~\textup{\ref{sit-3-deformation}}, for any non-negative integer $d$, there exists
    \begin{itemize}
        \item
            A countable family of normal irreducible $B$-schemes $\{ H_i \}$.
        \item
            For each index $i$, an integer $n_i \geq 1$,
            and $n_i$ triples $(W_{i,j},\ E_{i,j,1},\ E_{i,j,2})_{j=1}^{n_i}$,
            where for each $j$, $W_{i,j} \to H_i$ is a smooth projective family of $(d+1)$-dimensional varieties,
            and $E_{i,j,1},\ E_{i,j,2} \to H_i$ are two families of divisors of~$W_{i,j}$,
    \end{itemize}
    such that
    \begin{itemize}
        \item
            For any $b \in B(\bbk)$, and any data $(V_j,\ D_{j,1},\ D_{j,2})_{j=1}^n$,
            where each $V_j$ is an integral subscheme of $X_b$ of dimension $d+1$,
            and $D_{j,1},\ D_{j,2}$ are two effective Weil divisors on 
            the desingularisation $\widetilde{V}_j$ of $V_j$, 
            such that $D_{j,1} - D_{j,2}$ is a principal divisor on $V_j$, 
            there exists $i$ and $t \in (H_i)_b (\bbk)$, 
            such that the fibre $\bigl( (W_i)_t,\ (E_{i,j,1})_t,\ (E_{i,j,2})_t \bigr)$ 
            is identical to the given data.
    \end{itemize}
\end{lemma}

\begin{proof}
    For each $n$-tuple $(T_1, \dotsc, T_n)$ as given by the previous lemma,
    we consider the normalisation $H$ of the product $T_1 \times \cdots \times T_n$,
    equipped with the data of $n$ triples as given by the previous lemma.
    The collection of all such $H$ satisfies the requirement of this lemma.
\end{proof}

Our effort to parametrise all possibilities for a rational equivalence
allows us to prove the following result.

\begin{lemma} \label{lem-3-rat-equiv}
    In Situation~\textup{\ref{sit-3-deformation}}, let 
    \[ Z_1, Z_2 \in \mathrm{Chow}_{X/B}^{d} (B) \] 
    be two well-defined families of cycles.
    Then there exists
    \begin{itemize}
        \item
            A countable family of quasi-projective $B$-schemes $\{ M_i \}$.
        \item
            For each index $i$, the data $(W_{i,j},\ E_{i,j,1},\ E_{i,j,2})_{j=1}^{n_i}$ 
            as in Lemma~\textup{\ref{lem-3-divisors}}, with $M_i$ in place of $H_i$,
    \end{itemize}
    such that 
    \begin{itemize}
        \item
            The union of the images of $M_i(\bbk)$ in $B(\bbk)$ is exactly the set
            \[ \{ b \in B(\bbk) \mid [Z_{1,b}] = [Z_{2,b}] \text{ in } \CH_\bullet (X_b) \}. \]
        \item
            For any $b \in B(\bbk)$, and any data $(V_i,\ D_{i,1},\ D_{i,2})_{i=1}^n$
            as in Lemma~\textup{\ref{lem-3-divisors}}, such that
            \[ \textstyle Z_{1,b} + \sum_{i=1}^n [D_{i,1}] = Z_{2,b} + \sum_{i=1}^n [D_{i,2}]
                \quad \text{in } \upZ_\bullet (X_b), \]
            there exists $i$ and a point $t \in (M_i)_b (\bbk)$,
            such that the fibre $\bigl( (W_i)_t$, $(E_{i,j,1})_t$, $(E_{i,j,2})_t \bigr)$ 
            is identical to the given data.
    \end{itemize}
\end{lemma}

\begin{proof}
    Let $\{ H_i \}$ be the family in Lemma~\ref{lem-3-divisors}. We define a morphism
    \[ \begin{aligned}
        f \: H_i & \to     \Chow_{X/B} \times \Chow_{X/B}, \\
               t & \mapsto \textstyle \Bigl( Z_1 + \sum_{j=1}^{n_i} (E_{i,j,1})_t, \ Z_2 + \sum_{j=1}^{n_i} (E_{i,j,2})_t \Bigr),
    \end{aligned} \]
    where $(E_{i,j,1 \text{ or } 2})_t$ is regarded as a cycle of $X$
    via the pushforward along the desingularisation map (onto a closed subvariety of $X$).
    Now let $M_i$ be the inverse image of the diagonal along $f$,
    with the reduced scheme structure,
    equipped with the data of $n_i$ triples given by that of $H_i$.
    This proves the second statement.

    For the first statement, write $Z = Z_1 - Z_2$.
    Let $b \in B$ be a point where $Z_b$ is rationally equivalent to zero.
    Then, there exist subvarieties $V_j \subset X_b$, where $j = 1, \dotsc, n$,
    and rational functions $g_j$ on $V_j$, which give rise to
    rational functions $\widetilde{g_j}$ on a desingularisation $\widetilde{V}_j$,
    such that
    \[ \textstyle Z_b = \sum_{j=1}^n (f_j)_* (\operatorname{div} \widetilde{g_j}), \]
    where $f_j$ denotes the map $\widetilde{V}_j \to X_b$.
    Conversely, the existence of this data implies that $Z_b$ is rationally equivalent to zero,
    since $M_j$ is taken to be the inverse image of the diagonal.
    Therefore, the locus where $Z_b$ is equivalent to zero
    is exactly the union of the images of the $M_i$.
\end{proof}

We are now getting close to the main theorem,
which states that the locus where $[Z_{1,b}] = [Z_{2,b}]$
is a countable union of closed sets.
There is one further lemma needed.

\begin{lemma} \label{lem-3-specialise}
    Let $M$ be a smooth $\bbk$-variety of dimension $m$, with $\bbk$ algebraically closed,
    and let $f \: W \to M$ be a flat morphism of relative dimension $r$.
    Let $Z$ be an $n$-cycle on $W$. Suppose that
    \begin{itemize}
        \item
            There is a dense open set $M^\circ \subset M$,
            such that $Z|_{f^{-1} (M^\circ)}$ is rationally equivalent to $0$ in $f^{-1} (M^\circ)$.
    \end{itemize}
    Then
    \begin{itemize}
        \item
            For any $t \in M (\bbk)$,
            the fibre $Z_t$ is rationally equivalent to $0$ in $W_t$.
    \end{itemize}
\end{lemma}

\begin{proof}
    Let $t \in M(\bbk) \setminus M^\circ(\bbk)$ be a point.
    As in the proof of Lemma~\ref{lem-2-moving}, we can find
    a curve $C$ in $M$, passing through $t$, and not contained in $M \setminus M^\circ$.
    Taking the normalisation of this curve,
    we may thus assume that $M$ is a smooth curve.

    Let $D = M \setminus M^\circ$, which is now a finite set.
    There is an exact sequence \cite[\S1.8]{fulton}
    \[ \CH_n (f^{-1} (D)) \overset{i_*}{\longrightarrow} \CH_n (W) 
        \longrightarrow \CH_n (f^{-1} (M^\circ)) \to 0, \]
    so that $Z = i_* (z)$ for some $z \in \CH_n (f^{-1} (D))$,
    where $i \: f^{-1} (D) \to W$ denotes the inclusion.
    But by the projection formula \cite[\S2.3]{fulton}, the intersection of $i_* (z)$
    with the divisor $f^{-1} (D)$ of $W$ is 
    $i_* i^* (f^{-1}(D) \cdot z) = 0$, so that $Z_t$ is rationally equivalent to $0$ for any $t \in D$.
\end{proof}

Now we are ready to prove the main result,
and our proof follows that of \cite[Proposition~2.4]{voisin}.

\begin{theorem} [Voisin] \label{thm-3-locus-equality}
    In Situation~\textup{\ref{sit-3-deformation}}, let 
    \[ Z_1, Z_2 \in \mathrm{Chow}_{X/B}^{d} (B) \] 
    be two well-defined families of cycles.
    Then there exists a countable family $\{ B_i \}$
    of closed subschemes of $B$, such that
    \[ \textstyle
        \bigl\{ \, b \in B (\bbk) \bigm| 
        [Z_{1, b}] = [Z_{2, b}] \text{ \ in \ } {\CH_d (X_b)} \, \bigr\}
        = \bigcup_i B_i (\bbk).
    \]
\end{theorem}

\begin{proof}
    \def\Mibar{\mspace{2mu}\overline{\smash{\mspace{-2mu}M}\vphantom{t}}\mspace{-1mu}\vphantom{t}_i}
    Let $\{ M_i \}$ be as in Lemma~\ref{lem-3-rat-equiv}.
    Replacing each $M_i$ by its desingularisation, we can assume
    that all $M_i$ are smooth.

    Let $B_i \subset B$ be the closure of the image of $M_i$ in $B$,
    as a closed integral subvariety.
    By Lemma~\ref{lem-3-rat-equiv}, the equation $[Z_{1,b}] = [Z_{2,b}]$
    implies $b \in B_i$ for some $i$. 
    Thus, it suffices to show that it holds for all $b \in B_i$.

    Let $B_i^\circ \subset B_i$ be an open subset contained in the image of $M_i$,
    and let $M_i^\circ$ be the inverse image of $B_i^\circ$ in $M_i$.
    Let $X_{M_i} = X \times_B M_i$, and $Z_i$ the pullback of $Z = Z_1 - Z_2$
    along the morphism $X_{M_i} \to X$, 
    which is actually the Chow pullback of families of cycles along the map $M_i \to B$.
    Then $Z_i$ is equal to the universal cycle $\sum_{j=1}^{n_i} (E_{i,j,1} - E_{i,j,2})$ on $M_i$,
    and hence is rationally equivalent to zero.

    By taking the closure in a projective bundle, 
    the morphism $M_i \to B_i$ extends to a projective morphism $\Mibar \to B_i$.
    Again, taking a desingularisation, we may assume $\Mibar$ is smooth.
    Write $X_{\Mibar} = X \times_B \Mibar$.
    Now apply Lemma~\ref{lem-3-specialise} with $M = \Mibar$, $W = X_{\Mibar}$,
    and $M^\circ$ the inverse image of $M_i^\circ$. 
    This shows that for all $b \in B_i$, the cycle $Z_b$ is equivalent to zero.
\end{proof}


\subsection{Locus of decomposability of the diagonal}

\begin{lemma} \label{lem-3-all-cycles}
    Suppose
    \begin{itemize}
        \item
            $\bbk$ is an algebraically closed field of characteristic $0$.
        \item
            $B$ is a smooth $\bbk$-scheme.
        \item
            $X \to B$ is a projective morphism, and write $Y = X \times_B X$.
    \end{itemize}
    Then there exists
    \begin{itemize}
        \item
            A countable family of smooth irreducible $B$-schemes $\{ F_i \}$.
        \item
            For each index $i$, a well-defined family of non-negative
            $d_i$-cycles $C_i$ of $Y$ of degree $d'_i$,
            parametrised by $F_i$,
    \end{itemize}
    such that
    \begin{itemize}
        \item
            For any $b \in B$ and any non-negative $d$-cycle $C$ of $Y_b$ of degree $d'$,
            supported in $Z \times X_b$ for a codimension $1$ subset $Z \subset X_b$,
            there exists $i$ and $x \in (F_i)_b$ such that $C = (C_i)_x$.
        \item
            For any $x \in (F_i)_b$,
            the cycle $C = (C_i)_x$ is supported in $Z \times X_b$
            for a codimension $1$ subset $Z \subset X_b$.
    \end{itemize}
\end{lemma}

The condition ``supported in $Z \times X_b$'' is the main point of this lemma.
In fact, the proof would be a lot easier if we dropped this condition.
This lemma will be used to parametrise all possibilities for
the term $D$ in a decomposition of the diagonal, as in Definition~\ref{def-2-decomp}.

\begin{proof}
    First, we need to parametrise all the subschemes of $X$
    that are codimension $1$ in $X_b$ at each $b \in B$.
    Therefore, we consider an irreducible component 
    \[ H \subset \Hilb_{X/B}, \]
    parametrising the codimension $1$ subschemes.
    Let $U \subset H \times_B X$ be the universal subscheme.
    Thus if we look at the fibre at $b \in B$,
    then $H_b$ parametrises the codimension $1$ subschemes of $X_b$,
    and $U_b \subset H_b \times X_b$ is a subscheme
    whose intersection with $\{c\} \times X_b$ gives the subscheme of $X_b$ corresponding to $c$.

    Next, we want to parametrise all the subschemes of $Y$ which
    have the form $(\text{codim 1 subset}) \times X_b$ when restricted to the fibres.
    This is given by the universal subscheme
    \[ U' = U \times_B X \subset H \times_B X \times_B X, \]
    which, at $b \in B$, 
    when intersected with $\{c\} \times X_b \times X_b$,
    gives the subscheme of $X_b \times X_b$ corresponding to $c$.

    Finally, we parametrise cycles of $Y$ supported in
    a subset of the form of the previous step.
    Thus we consider an irreducible component
    \[ C \subset \Chow_{U'/H}. \]
    Let $V \in \upZ_\bullet (C \times_H U')$ be the universal family. Since
    \[ C \times_H U' \subset C \times_H H \times_B X \times_B X \simeq C \times_B X \times_B X, \]
    we can view $V$ as a family of cycles of $Y$ parametrised by $C$.

    Thus, all choices of $H$ and $C$ will give a countable set of families,
    which together parametrise all the cycles of $Y$ of the given form.

    However, the parametrising schemes need to be smooth.
    We thus apply Hironaka's desingularisation theorem to the schemes $C$.
\end{proof}

\begin{proposition} \label{lem-3-locus-decomp}
    Suppose 
    \begin{itemize}
        \item
            $\bbk$ is an algebraically closed field of characteristic $0$.
        \item
            $B$ is a smooth $\bbk$-scheme.
        \item
            $X \to B$ is a projective morphism.
    \end{itemize}
    Then there exists a countable family $\{ B_i \}$
    of closed subschemes of $B$, such that
    \[ \textstyle
        \{ b \in B (\bbk) \mid X_b \text{ has a decomposition of the diagonal} \}
        = \bigcup_i B_i (\bbk).
    \]
\end{proposition}

\begin{proof}
    Let $F_i, F_{i'}$ be two of the schemes as in Lemma~\ref{lem-3-all-cycles}, 
    with $d_i = d_{i'} = \dim (X/B)$,
    and let $C_i, C_{i'}$ be the universal cycles,
    lying in $X \times_B X \times_B F_{i \text{ or } i'}$. 
    
    Let $G_j, G_{j'}$ be irreducible components of 
    $\Chow_{X/B}^{0,d}$ and $\Chow_{X/B}^{0,d + 1}$,
    respectively, where $d$ is arbitrary,
    and let $D_j, D_{j'}$ be the universal cycles lying in $X \times_B G_{j \text{ or } j'}$.
    
    We define two cycles of $Y = F_i \times_B G_j \times_B X \times_B X \times_B G_{j'} \times_B F_{i'}$ by
    \[ \begin{aligned}
        Z_1 &= \Bigl( [G_j] \times C_i + [F_i] \times [G_j] \times [\Delta_{X/B}] + [F_i] \times [X] \times D_j \Bigr) 
            \times [G_{j'}] \times [F_{i'}], \\[-3pt]
        Z_2 &= [F_i] \times [G_j] \times \Bigl( C_{i'} \times [G_{j'}] + [X] \times D_{j'} \times [F_{i'}] \Bigr),
    \end{aligned} \]
    where $[\Delta_{X/B}]$ is the diagonal class.

    Now apply Theorem~\ref{thm-3-locus-equality}, where we take $X$ to be $Y$, 
    and take $B$ to be 
    \[ F_i \times_B F_{i'} \times_B G_j \times_B G_{j'}. \]
    At the point 
    \[ t = (t_1, t_2, x_1, x_2) \in (F_i)_b \times (F_{i'})_b \times (G_j)_b \times (G_{j'})_b, \]
    the cycle $Z_1$ gives $[\Delta_{X_b}] + z_1 + [X_b] \times x_1$,
    where $z_1$ is a non-negative cycle
    supported in $Z \times X_b$ for $Z \subset X_b$ of codimension $1$,
    and similarly, the cycle $Z_2$ gives $[X_b] \times x_2 + z_2$,
    with $z_2$ likewise.

    Therefore, Theorem~\ref{thm-3-locus-equality} implies that
    the locus where the equation
    \[ [\Delta_{X_b}] + z_1 + [X_b] \times x_1 = [X_b] \times x_2 + z_2 \quad \in \CH_{\dim X_b} (X_b \times X_b) \]
    holds (for non-negative $x_1, x_2, z_1, z_2$)
    is the union of countably many closed subsets.
\end{proof}

This result is restated as follows.

\begin{theorem} \label{thm-3-locus-decomp}
    Suppose 
    \begin{itemize}
        \item
            $\bbk$ is an algebraically closed field of characteristic $0$.
        \item
            $B$ is a smooth $\bbk$-scheme.
        \item
            $X \to B$ is a dominant projective morphism.
        \item
            There exists a $\bbk$-point $0 \in B$,
            such that the fibre $X_0$ does not have a decomposition of the diagonal.
    \end{itemize}
    Then for a ``very general'' $\bbk$-point $b \in B$,
    the fibre $X_b$ will not have a decomposition of the diagonal. \qed
\end{theorem}

By ``\term{very general}'', we mean ``except a countable union of closed sets of codimension $\geq 1$''.

This means that if we can find one example in a family of varieties,
which we can show has non-trivial Brauer group,
and hence does not have a decomposition of the diagonal,
then a very general variety in this family is not retract rational.


\subsection{Stable equivalence}

This subsection gives a variant of the above result,
concerning stable equivalence instead of retract rationality.

\begin{definition}
    Two projective $\bbk$-varieties are \term{stably equivalent}, if
    \[ X \times \bbP^m \quad \text{is birational to} \quad Y \times \bbP^n \]
    for some $m, n \in \bbN$.
\end{definition}

Stable rationality is the same as stable equivalence to a point.

\begin{lemma} \label{lem-3-stable-eq-cycles}
    Let $X,Y$ be two $\bbk$-varieties, such that there exist open sets $U \subset X$,
    $V \subset Y \times \bbP^n$, and two morphisms $p \: U \to V$, $q \: V \to U$,
    such that $q \circ p = \id_U$.
    Then there exist two correspondences
    \[ f \in \operatorname{Corr} (X, Y), \quad g \in \operatorname{Corr} (Y, X), \]
    such that
    for any field extension $\bbK/\bbk$, the induced map
    \[ (g \circ f)_* \: \CH_0 (X_{\bbK}) \to \CH_0 (X_{\bbK}) \]
    is the identity map.
    When $X$ is smooth, we have a decomposition
    \[ [\Delta_X] = D + g \circ f \quad \text{in } \operatorname{Corr} (X, X), \]
    where $D$ is supported in $Z \times X$
    for some closed subvariety $Z \subset X$ of codimension at least $1$.
\end{lemma}

\begin{proof}
    The correspondence $f$ is given by the rational map
    \[ X \overset{\supset}{\mathrel{\rightdasharrow}} U \overset{p}{\to} V \hookrightarrow Y \times \bbP^n \to Y, \]
    and $g$ is given by a rational map
    \[ Y \hookrightarrow Y \times \bbP^n \overset{\supset}{\mathrel{\rightdasharrow}} V \overset{q}{\to} U \hookrightarrow X, \]
    where the inclusion $Y \hookrightarrow Y \times \bbP^n$
    is chosen so that the composition is defined.
    To prove that $g \circ f$ induces the identity map on $\CH_0 (X_{\bbK})$, it suffices to prove that the map
    \[ Y \times \bbP^n \to Y \hookrightarrow Y \times \bbP^n \]
    induces the identity map on $\CH_0$. 
    This is because every closed point of $Y \times \bbP^n$
    is sent to another point that lives in the same slice of $\bbP^n$, and hence,
    is rationally equivalent to it as a $0$-cycle.

    For the second part, we use an argument as in the proof of Theorem~\ref{thm-2-decomp}.
    Namely, we change the base field to $\bbk(X)$, to find that
    \[ g_{\bbk (X)} \circ f_{\bbk (X)} (\beta) = \beta \quad \text{in } \CH_0 (X_{\bbk(X)}), \]
    where $\beta$ is the class of the generic point.
    The rest of the proof is analogous to the proof of Theorem~\ref{thm-2-decomp}, 
    \ref{itm-thm-2-decomp-2} $\Rightarrow$ \ref{itm-thm-2-decomp-3}.
\end{proof}

Note that the assumptions of this lemma is satisfied when $X$ and $Y$ are stably equivalent.

Using an argument as in the proof of Proposition~\ref{lem-3-locus-decomp},
we obtain the following result.

\begin{theorem} \label{thm-3-stable-eq-prototype}
    Suppose 
    \begin{itemize}
        \item
            $\bbk$ is an algebraically closed field of characteristic $0$.
        \item
            $B$ is a smooth $\bbk$-scheme.
        \item
            $X \to B$ and $Y \to B$ are two projective morphisms.
    \end{itemize}
    Then the set of all points $b \in B (\bbk)$ such that
    there exist correspondences
    \[ f \in \operatorname{Corr} ( X_b, Y_b ), \quad g \in \operatorname{Corr} ( Y_b, X_b ), \]
    and $D$ as before, such that 
    \[ [\Delta_{X_b}] = D + g \circ f, \]
    is a countable union of closed sets.
\end{theorem}

\begin{proof}
    We apply Theorem~\ref{thm-3-locus-equality}, 
    where we take $X$ to be
    \[ X \times_B X \times_B F_i \times_B F_{i'} \times_B G_j \times_B G_{j'} \times_B H_k \times_B H_{k'}, \]
    where 
    \begin{itemize}
        \item 
            $F_i$, $F_{i'}$ are given by Lemma~\ref{lem-3-divisor}.
        \item 
            $G_j$, $G_{j'}, H_k, H_{k'}$ are irreducible components of $\Chow_{X \times_B Y / B}$,
            parametrising the correspondences from $X$ to $Y$ for $G_j$ and $G_{j'}$,
            and from $Y$ to $X$ for $H_k$, $H_{k'}$.
    \end{itemize}
    The rest of the proof is analogous to the proof of Proposition~\ref{lem-3-locus-decomp}.
\end{proof}

\begin{corollary} \label{cor-3-stable-eq-prototype}
    Under the assumptions of Theorem~\ref{thm-3-stable-eq-prototype},
    the set of all points $b \in B(\bbk)$ such that $X_b$ is stably equivalent to $Y_b$
    is contained in a countable union of closed sets.
    Moreover, this union does not contain any point $b \in B(\bbk)$
    such that $X_b$ is smooth and has a decomposition of the diagonal,
    and $Y_b$ does not have a decomposition of the diagonal
\end{corollary}

\begin{proof}
    The countable union of closed sets given by Theorem~\ref{thm-3-stable-eq-prototype}
    satisfies this requirement.
    Indeed, for those $b \in B(\bbk)$ such that $X_b$ and $Y_b$ are stably equivalent,
    Lemma~\ref{lem-3-stable-eq-cycles} shows that $b$ is in this union.

    To prove the last statement, let $b \in B(\bbk)$ be such a point.
    We show that such correspondences $f,g$ as in Theorem~\ref{thm-3-stable-eq-prototype} do not exist between $X_b$ and $Y_b$.
    In fact, if they exist, then $g \circ f$ acts on $\CH_0(X)$ by the identity map.
    But since $\id_Y$ sends every $0$-cycle to its degree multiplied by a fixed $0$-cycle of degree~$1$,
    so does the correspondence $g \circ f = g \circ \id_Y \circ f$.
    Moreover, this holds over any field extension of $\bbk$. 
    By Theorem~\ref{thm-2-decomp}, $X$ has a decomposition of the diagonal,
    a contradiction.
\end{proof}

In particular, if we take $X$ to be a constant family which is smooth,
we deduce that every stable equivalence class in a family of varieties is contained
in a countable union of closed sets.

\begin{corollary} \label{thm-3-stable-eq-class}
    Suppose 
    \begin{itemize}
        \item
            $\bbk$ is an uncountable algebraically closed field of characteristic $0$.
        \item
            $B$ is a smooth $\bbk$-scheme.
        \item
            $X \to B$ is a dominant projective morphism, with smooth generic fibre.
        \item
            There exist two $\bbk$-points $b_0, b_1 \in B$,
            such that the fibre $X_{b_0}$ has a decomposition of the diagonal,
            while the fibre $X_{b_1}$ does not have a decomposition of the diagonal.
    \end{itemize}
    Then there are uncountably many stable equivalence classes of varieties
    in this family.
\end{corollary}

\begin{proof}
    For those smooth fibres $X_b$ that do not have a decomposition of the diagonal,
    we apply Corollary~\ref{cor-3-stable-eq-prototype} to the constant family $B \times X_b \to B$ and the family $X \to B$.
    It follows that the set of $b' \in B(\bbk)$ such that $X_b$ is stably equivalent to $X_{b'}$
    is contained in a countable union of closed sets, which can not coincide with the whole space.

    By Theorem~\ref{thm-3-locus-decomp},
    the locus of smooth fibres with no decomposition of the diagonal 
    is the complement of a countable union of closed subsets of $B$.
    Therefore, in order to cover this locus, 
    there must be uncountably many stable equivalence classes of fibres of $X \to B$,
    since each of these classes is contained in a countable union of closed sets.
\end{proof}
