For a projective variety, there are various notions
of rationality, describing how close a variety is to the projective space.

\begin{definition}
    Let $\bbk$ be a field,
    and let $X$ be a projective $\bbk$-variety.
    \begin{itemize}
        \item
            $X$ is \term{rational},
            if there exists $n \in \bbN$, such that 
            \[ X \quad \text{is birational to} \quad \bbP_{\bbk}^n. \]
            Equivalently, we have
            $\bbk(X) \simeq \bbk(x_1, \dotsc, x_n)$ as $\bbk$-algebras.

        \item 
            $X$ is \term{stably rational},
            if there exist $m, n \in \bbN$, such that
            \[ X \times \bbP_{\bbk}^m \quad \text{is birational to} \quad \bbP_{\bbk}^n. \]
            Equivalently, we have
            $\bbk(X)(y_1, \dotsc, y_m) \simeq \bbk(x_1, \dotsc, x_n)$ as $\bbk$-algebras.

        \item
            $X$ is \term{retract rational},
            if there exists $n \in \bbN$, and open sets $U \subset X$, $V \subset \bbP_{\bbk}^n$,
            together with two maps 
            \[ 
                f \: U \to V, \quad
                g \: V \to U,
            \]
            such that $g \circ f = \id_U$. 

        \item
            $X$ is \term{unirational}, 
            if there exists a dominant rational map 
            \[ \bbP_{\bbk}^n \mathrel{\rightdasharrow} X. \]
            Equivalently, there exists a map 
            $\bbk(X) \to \bbk(x_1, \dotsc, x_n)$ of $\bbk$-algebras.

        \item
            $X$ is \term{rationally connected},
            if for every algebraically closed field $\bbK$ containing $\bbk$,
            and any two $\bbK$-points $x, y \in X(\bbK)$,
            there exists a rational curve 
            \[ f \: \bbP_{\bbK}^1 \to X_{\bbK} \]
            joining them, i.e.\ we have $f(0) = x$ and $f(\infty) = y$.
    \end{itemize}
\end{definition}

These notions are sorted from strong to weak,
i.e.\ every notion implies its next one.
A natural question to ask is that whether these implications are strict.

In 1972, Artin and Mumford \cite{artin-mumford}
gave an example of a variety that is unirational but not retract rational.
More recently, Voisin \cite{voisin}
developed a deformation method which can show that 
a very general member of a family of varieties is not retract rational,
as long as it contains one particular example that is not retract rational.
This method was then modified by Colliot-Thélène and Pirutka \cite{CTP}
to show that over $\bbC$, a very general quartic hypersurface in $\bbP^4$
is not retract rational.
They also developed the specialisation method,
with which they can provide more general examples of smooth varieties
that are not retract rational
over number fields and local fields.

In this article, we give an exposition of
the methods and results mentioned above.

