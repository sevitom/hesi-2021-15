In this section, we construct an explicit example
of a quartic threefold which is unirational,
but has non-trivial Brauer group.
This example was originally due to Artin and Mumford \cite{artin-mumford},
and slightly modified in \cite{CTP} so that it embeds in $\bbP^4$.

\subsection{The example} \label{sect-4}

\begin{itemize}
    \item
        Let $\bbk$ be an algebraically closed field with $\Char \bbk \neq 2$.
    \item 
        Let $A \subset \bbP^2$ be a smooth conic,
        defined by the quadratic equation 
        \[ \alpha (z_0, z_1, z_2) = 0. \]
    \item
        Let $D_1, D_2 \subset \bbP^2$ be two smooth cubics, defined by
        \[ \delta_1 = 0 \quad \text{and} \quad \delta_2 = 0, \]
        such that they each meet $A$ tangentially at $3$ points,
        giving six tangent points $Q_1, \dotsc, Q_6$,
        and such that $D_1 \cap D_2$ is nine distinct points $P_1, \dotsc, P_9$.
        These nine points do not lie on $A$, since otherwise $D_1$ and $D_2$ will meet tangentially at that point.
    \item
        Let $B \subset \bbP^2$ be a cubic that intersects $A$ in the six points $Q_1, \dotsc, Q_6$.
        In fact, for any nine given points on the plane, there exists a cubic curve passing through all of them.
        We use these six points, and choose three other points which are non-collinear and not on $A$.
        This ensures that the cubic does not contain $A$, and can only intersect $A$ in these six points.

        As cycles of $A$, we have
        \[ (D_1 + D_2) \cdot A = 2B \cdot A. \]
        This means that $\alpha \mid \delta_1 \delta_2 - \beta^2$,
        where $\beta$ is the polynomial defining $B$. Thus we may write
        \[ \delta_1 \delta_2 = \beta^2 - 4 \alpha \gamma \]
        for some $\gamma$ of degree $4$.
    \item
        Let $S \subset \bbP^3$ be the quartic surface defined by
        \[ g = \alpha (z_0, z_1, z_2) \, z_3^2 + \beta (z_0, z_1, z_2) \, z_3
            + \gamma (z_0, z_1, z_2) = 0. \]
        Using the projection to $\bbP^2$ which sends $(z_0 : z_1 : z_2 : z_3)$ to $(z_0 : z_1 : z_2)$, 
        the surface $S \setminus (0:0:0:1)$ can be seen as a double cover of $\bbP^2$,
        ramified along the curves $D_1$ and $D_2$.
    \item
        After applying a linear coordinate change in $z_0, z_1, z_2$, we may assume
        \begin{multline} \label{eq-4-coord-change}
            \text{The hyperplane $z_0 = 0$ does not contain $Q_1, \dotsc, Q_6$,} \\
            \text{or any point of $M \setminus \{P_0\}$, and is not tangent to $A$,}
        \end{multline}
        where
        \[ M = \biggl\{ \, g = 0, \ \Bigl( 
            \frac {\partial g} {\partial z_1} = 0
            \ \text{or} \ 
            \frac {\partial g} {\partial z_2} = 0
        \Bigr),\ 
        \frac {\partial g} {\partial z_3} = 0 \, \biggr\}, \]
        and $P_0 = (0:0:0:1)$.
        This is a technical assumption
        which we make to avoid bad singularities.
    \item
        Now let $T \subset \bbP^4$ be the quartic threefold defined by
        \[ f = \alpha (z_0, z_1, z_2) \, z_3^2 + \beta (z_0, z_1, z_2) \, z_3
        + \gamma (z_0, z_1, z_2) + z_0^2 z_4^2 = 0. \]
        It is a double cover of $\bbP^3$,
        ramified along the surface $S$ and the hyperplane $z_0 = 0$.
\end{itemize}

We will see that $T$ has the property of
being unirational but not having a decomposition of the diagonal.


We next construct an explicit desingularisation of
the threefold $T$ constructed above,
following \cite[Appendix~A]{CTP}.
We do this in order to get a desingularisation map 
which is universally $\CH_0$-trivial,
and such that the Brauer group of the desingularisation is non-trivial.

\begin{itemize}
    \item
        We observe that the threefold $T$ is singular
        along the line
        \[ L \: z_0 = z_1 = z_2 = 0 \]
        in $\bbP^4$, and outside of this line, it has
        ordinary quadratic singularities at the points
        $P_1, \dotsc, P_9$ on the hyperplane $z_4 = 0$.

    \item
        Let $T_1 \to T$ be the blow-up along $L$.
        Standard computation shows that the exceptional divisor
        is a rational surface, and $T_1$ is singular
        along a line $L_1$ which is the inverse image of the point
        $P = (0:0:0:0:1)$.

    \item
        Let $T_2 \to T_1$ be the blow-up along $L_1$.
        Standard computation shows that the exceptional divisor
        is a rational surface, and $T_2$ only has ordinary quadratic singularities, at the inverse images of
        the nine points $P_1, \dotsc, P_9$.

    \item
        Finally, we blow up $T_2$ at the nine points $P_1, \dotsc, P_9$.
        This gives a desingularisation of $T$.
        The exceptional divisor over the nine points are rational surfaces,
        and over $L$, it is a union of two rational surfaces.
        We can thus apply Theorem~\ref{thm-4-ch0-trivial-fibre} to conclude that
        the desingularisation map is universally $\CH_0$-trivial.

    \item
        Artin and Mumford \cite[\S2]{artin-mumford}
        showed that over $\bbC$, there is a smooth projective threefold $V$, birational to $T$,
        such that the singular cohomology $H^3 (V, \bbZ)$ contains non-trivial $2$-torsion.
        It follows from the universal coefficient theorem and
        the comparison theorem \cite[Theorem~III.3.12, p.\,117]{milne}
        that the étale cohomology group $H^3 (V, \bbZ_2)$
        contains non-trivial $2$-torsion. By Lemma~\ref{lem-6-br-equals-adic} below,
        $\Br (V)$ contains non-trivial $2$-torsion,
        and so does $\Br (T_2)$ by Theorem~\ref{thm-2-unramified-br}.
        In particular, $\Br (T_2) \neq 0$.
        This shows that $T$ is not retract rational, by Theorem~\ref{thm-2-brauer-trivial}.

        For general $\bbk$ of characteristic zero, 
        a consequence of the smooth base change theorem for étale cohomology 
        \cite[Corollary~VI.4.3, p.\,231]{milne} shows that $\Br (T_2) \simeq H^2 (T_2, \Gm)$ 
        contains non-trivial $2$-torsion.
        Indeed, the cited theorem shows that this holds over $\bbQbar$,
        and applying it again shows that it holds for any algebraically closed $\bbk$ of characteristic zero,
        so that $T$ is not retract rational.
\end{itemize}

In summary, we have the following result.

\begin{theorem} \label{thm-4-main}
    Suppose that $\bbk$ is algebraically closed of characteristic zero.
    Then the quartic threefold $T$ admits a desingularisation
    \[ f \: \widetilde{T} \to T, \]
    such that $f$ is universally $\CH_0$-trivial.
    Moreover, we have $\Br (\widetilde{T}) \neq 0$. \qed
\end{theorem}

Finally, we prove the relationship between the Brauer group
and the $\ell$-adic étale cohomology group which was used above.

\begin{lemma}
    Let $X$ be a rationally connected $\bbk$-variety, with $\Char \bbk = 0$. Then
    \[ H^p (X, \mathscr{O}_X) = 0 \]
    for all $p > 0$.
\end{lemma}

\begin{proof}
    See \cite[\S3.4]{debarre}.
\end{proof}

Let $X$ be a variety over $\bbC$. 
The exact sequence of sheaves
\[ 0 \to \bbZ \to \mathscr{O}_X \to \mathscr{O}_X^* \to 0 \]
on $X$ (as an analytic space) induces a long exact sequence
\[ \cdots \to H^1(X, \mathscr{O}_X) \to \operatorname{Pic}(X) 
    \to H^2 (X, \bbZ) \to H^2 (X, \mathscr{O}_X) \to \cdots. \]
The image of $\operatorname{Pic}(X)$ in $H^2 (X, \bbZ)$ is called the
\term{Néron--Severi group}, and its rank, denoted by $\rho (X)$,
is called the \term{Picard number} of $X$.

\begin{lemma} \label{lem-6-rho-equals-b2}
    Let $X$ be a rationally connected complex variety.
    Then the Picard number $\rho (X)$ is equal to the Betti number $b_2 (X)$. \qed
\end{lemma}

We fix some notations.
For an abelian group $A$ and a prime number $\ell$,
we denote
\[ A \, \{\ell\} = \{ \, x \in A \mid \ell^n x = 0 \text{ for some } n \, \}, \]
which is naturally a $\bbZ_\ell$-module.
Suppose $M$ is a $\bbZ_\ell$-module \emph{of cofinite type}, i.e.\ one has
\[ M \simeq (\bbQ_{\ell} / \bbZ_{\ell})^{\oplus r} \oplus (\text{finite group}), \]
then we denote by $M^{\mathrm{fin}}$ its finite part, which is
the largest finite submodule of $M$ that is a direct summand.

\begin{lemma} \label{lem-6-br-equals-adic}
    Let $X$ be a rationally connected complex variety,
    and let $\ell$ be a prime number. Then
    \[ \Br (X) \, \{\ell\} \simeq H^3 (X, \bbZ_\ell (1)) \, \{\ell\},  \]
    where the right hand side is the étale cohomology of the sheaf
    $\bbZ_\ell (1) = \liminv_n \sfmu_{\ell^n}$, which may be identified with $\bbZ_\ell$
    over $\bbC$.
\end{lemma}

\begin{proof}
    By \cite[II, Theorem~3.1, p.~80]{grothendieck-brauer}, we have an exact sequence 
    \[ 0 \to \operatorname{Pic} (X) \otimes_{\bbZ} \bbQ_{\ell} / \bbZ_{\ell}
        \longrightarrow H^2 (X, \sfmu_{\ell^\infty})
        \longrightarrow \Br (X) \, \{ \ell \} \to 0. \]
    Since the first term is a finite sum of copies of $\bbQ_{\ell} / \bbZ_{\ell}$, we have
    \[ \Br (X) \, \{ \ell \} ^{\mathrm{fin}} \simeq H^2 (X, \sfmu_{\ell^\infty}) ^{\mathrm{fin}} . \]
    By Lemma~\ref{lem-6-rho-equals-b2},
    the ``corank'' of $\Br (X) \, \{ \ell \}$ (i.e.\ number of summands $\bbQ_{\ell} / \bbZ_{\ell}$)
    is $b_2 - \rho = 0$, so that
    \[ \Br (X) \, \{ \ell \} ^{\mathrm{fin}} \simeq \Br (X) \, \{ \ell \}. \]
    By \cite[III, (8.3), p.~144]{grothendieck-brauer}, we have an exact sequence 
    \[ 0 \to H^2 (X, \sfmu_{\ell^\infty}) ^{\mathrm{fin}}
        \longrightarrow H^3 (X, \bbZ_\ell (1))
        \longrightarrow \liminv_n H^3 (X, \sfmu_{\ell^\infty})\,[\ell^n] \to 0, \]
    where $[\ell^n]$ indicates the subgroup of elements killed by $\ell^n$.
    Since the last term is torsion-free, we have 
    \[ H^2 (X, \sfmu_{\ell^\infty}) ^{\mathrm{fin}} \simeq H^3 (X, \bbZ_\ell (1)) \, \{\ell\} , \]
    whence the result follows.
\end{proof}


\subsection{Consequences}

Following Colliot-Thélène and Pirutka \cite{CTP},
we present some consequences of the example given in the previous subsection.

The first result provides examples of smooth quartic threefolds over complex numbers,
which are not retract rational. 

\begin{theorem} \label{thm-6-eg-complex}
    Let $P \to \bbP^N_{\bbC}$ be the family of all quartic hypersurfaces in $\bbP^4_{\bbC}$.
    Let $t \in \bbC \setminus \bbQbar$ be a transcendental number.
    Then the set
    \[ \biggl\{ \, z \in \bbP^N_{\bbC} \biggm| \begin{matrix}
        \text{$z$ has coordinates in $\bbQbar(t)$, and the hyper-} \\
        \text{surface $P_z$ is smooth but not retract rational}
    \end{matrix} \, \biggr\} \]
    is Zariski dense in $\bbP^N_{\bbC}$.
\end{theorem}

\begin{proof}
    By Theorem~\ref{thm-4-main}, there is a quartic hypersurface $X \subset \bbP^4_{\bbQbar}$,
    with a desingularisation
    \[ f \: \widetilde{X} \to X \]
    such that $f$ is universally $\CH_0$-trivial, and $\Br (\widetilde{X}) \neq 0$.

    Let $W \subset \bbP^N_{\bbQbar}$ be the closed subset corresponding to
    the singular hypersurfaces.
    Let $M \in \bbP^N_{\bbQbar}$ be the point corresponding to $X$.
    Choose a point $M' \in \bbP^N_{\bbQbar} \setminus W$,
    and let
    \[ L \simeq \bbP^1_{\bbQbar} \subset \bbP^N_{\bbQbar} \]
    be the straight line connecting $M$ and $M'$, with generic point $\eta$.
    Then Theorem~\ref{thm-5-spe-2} implies that the quartic threefold $X^\circ$ defined by $\eta$,
    over the field $\bbQbar(x)$, is not geometrically retract rational.
    Moreover, by Theorem~\ref{thm-5-spe-3}, for any embedding
    \[ \bbQbar(x) \hookrightarrow \bbC, \]
    the base change $X^\circ_{\bbC}$ is not retract rational,
    where a desingularisation of $X^\circ$ can be obtained via Hironaka's theorem,
    as is required by Theorem~\ref{thm-5-spe-3}. 

    Let $R \in L(\bbC)$ be a point whose coordinates are in $\bbQbar(t)$, but not in $\bbQbar$.
    Then the quartic threefold $P_R$ is isomorphic to $X^\circ_{\bbC}$,
    for some embedding $\bbQbar(x) \hookrightarrow \bbC$.
    Indeed, we have a diagram of pull-back squares
    \[ \begin{tikzcd}
        P_R \ar[d] \ar[r] &
        P_{\ L_{\bbC}} \ar[d] \ar[r] &
        P_{\ \bbQbar,\ L} \ar[d] &
        X^\circ \ar[d] \ar[l] \\
        \Spec \bbC \ar[r, "R"] &
        L_{\bbC} \ar[r] &
        L_{\bbQbar} &
        \Spec \bbQbar (x) \rlap{\ ,} \ar[l, "\eta"']
    \end{tikzcd} \]
    where $P_{\ \bbQbar} \to \bbP^N_{\bbQbar}$ denotes the family of all quartic hypersurfaces in $\bbP^4_{\bbQbar}$.
    By the choice of $R$, in the diagram, the image of $\Spec \bbC$ in $L_{\smash{\bbQbar}}$ is the generic point 
    (since any other point is a $\bbQbar$-rational point, and cannot be $R$).
    Therefore, the map $\Spec \bbC \to L_{\bbQbar}$ in the diagram factors through $\Spec \bbQbar(x)$,
    giving the desired embedding.

    Therefore, $P_R$ is not retract rational.
    This shows that every line passing through $M$ and a point not in $W$
    contains infinitely many points where the hypersurface is not retract rational.
    This implies Zariski density.
\end{proof}

Together with results from \S\ref{sect-3},
this result allows us to obtain general statements 
on the irrationality of quartic threefolds over complex numbers.

\begin{theorem} [Colliot-Thélène and Pirutka] \label{thm-3-quartic-threefold}
    A very general quartic hypersurface in $\bbP_{\bbC}^4$ is not retract rational.
\end{theorem}

\begin{proof}
    By Theorem~\ref{thm-3-locus-decomp}, and by Theorem~\ref{thm-6-eg-complex}.
\end{proof}

\begin{theorem} \label{thm-3-stable-eq-threefolds}
    There are uncountably many stable equivalence classes in
    the family of quartic hypersurfaces in $\bbP_{\bbC}^4$.
\end{theorem}

\begin{proof}
    We apply Theorem~\ref{thm-3-stable-eq-class}.

    We have seen that the family contains
    a threefold with no decomposition of the diagonal.
    But the (singular) hypersurface 
    \[ X \: x_0 x_1 (x_0 + x_1) (x_0 - x_1) = 0 \]
    has a decomposition of the diagonal.
    Indeed, let $z = (0:0:0:0:1) \in X$,
    and let $X_1, \dotsc, X_4$ be the irreducible components of $X$, each isomorphic to $\bbP^3$.
    The diagonal class of $X_i \times X_i$ is rationally equivalent to $[X_i] \times [z]$,
    up to a minor term $D$ as before.
    Summing over $i$, we see that the diagonal of $X \times X$ is 
    rationally equivalent to $[X] \times [z]$, up to a minor term.
\end{proof}

\begin{remark} \label{rem-3-quaric-threefold}
    In these two theorems, the degree $4$ can be replaced by any positive multiple of $4$,
    since one can consider quartic threefolds in $\bbP^4$ with 
    multiplicity $m > 1$, 
    which will be a non-reduced hypersurface of degree $4m$.
    We use the fact that the Chow group of a ``non-reduced variety''
    is isomorphic to that of its reduction \cite[Example~1.3.1]{fulton}.
\end{remark}

Finally, we mention a result of Colliot-Thélène and Pirutka which provides examples of smooth quartic hypersurfaces 
in $\bbP_{\bbC}^4$, which are defined over $\bbQbar$, but not retract rational over $\bbC$.

\begin{theorem}
    There exist smooth quartic hypersurfaces in $\smash{\bbP_{\bbQbar}^4}$
    that are not universally $\CH_0$-trivial over any field containing~$\bbQbar$, and hence
    not retract rational over any field containing~$\bbQbar$, and in particular, over $\bbC$.
\end{theorem}

\begin{proof}
    \newcommand*{\kappavbar}{\overline{\vphantom{t}\smash{\kappa(v)}}}
    By Theorem~\ref{thm-4-main}, there is a singular
    quartic hypersurface $X \subset \bbP^4_{\bbQbar}$,
    with a desingularisation
    \[ f \: \widetilde{X} \to X \]
    such that $f$ is universally $\CH_0$-trivial, and $\Br (\widetilde{X}) \, \{2\} \neq 0$.
    By Lemma~\ref{lem-6-br-equals-adic}, we thus have 
    \[ H^3 (\widetilde{X}, \bbZ_2) \, \{2\} \neq 0. \]

    By Galois descent and by Lemma~\ref{lem-5-fin-ext},
    we choose a finite extension $K / \bbQ$,
    over which $X$, $\widetilde{X}$ and $f$ are defined,
    such that $f$ is universally $\CH_0$-trivial.

    Let $\mathscr{O}_K$ be the ring of integers of $K$.
    Let $U \subset \Spec \mathscr{O}_K$ be an open set,
    such that there exists a map of $U$-schemes
    \[ \mathscr{f} \: \widetilde{\mathscr{X}} \to \mathscr{X}, \]
    such that it coincides with $f \: \widetilde{X} \to X$ over the generic point of $U$.
    Indeed, we define them to be cut out by the same set of equations as $X$ and $\widetilde{X}$
    in the projective space.

    Shrinking $U$ if necessary, we assume that $U$ contains no $2$-adic points,
    and that $\widetilde{\mathscr{X}}$ is smooth over $U$.

    Shrinking $U$ again, we assume for any closed point $v \in U$, the map of geometrical fibres
    \[ \mathscr{f}_{\kappavbar} \, \: \,
        \widetilde{\mathscr{X}}_{\kappavbar} 
        \ \to \ \mathscr{X}_{\kappavbar} \]
    is a desingularisation map which is universally $\CH_0$-trivial.
    In fact, it is what we get if we start with 
    $\bbk = \kappavbar$ in \S\ref{sect-4}.

    Applying the smooth specialisation property of étale cohomology \cite[Corollary~VI.4.2, p.~230]{milne},
    we see that for any $v \in U$,
    \[ H^3 (\widetilde{\mathscr{X}}_{\kappavbar}, \bbZ_2) 
        \simeq H^3 (\widetilde{X}_{\bbQbar}, \bbZ_2). \]
    It follows that $H^3 (\widetilde{\mathscr{X}}_{\kappavbar}, \bbZ_2) \, \{2\} \neq 0$,
    and hence $\Br (\widetilde{\mathscr{X}}_{\kappavbar}) \neq 0$ by Lemma~\ref{lem-6-br-equals-adic}.

    Fix a point $v \in U$. We regard $v$ as a discrete valuation on $K$.
    By Lemma~\ref{lem-5-dvr}, there is an extension of discrete valuation rings 
    \[ \mathscr{O}_{K,v} \subset A, \]
    such that the residue field of $A$ is $\kappavbar$.
    Let $L$ be the fraction field of $A$.

    Finally, there exists a smooth $A$-scheme 
    whose special fibre is $\mathscr{X}_{\kappavbar}$.
    Indeed, in the projective space $\smash{\bbP_L^N}$ 
    parametrising the quartic hypersurfaces in $\smash{\bbP_L^4}$,
    the set of points in $\smash{\bbP_L^N}$ with coordinates in $\mathscr{O}_L$ 
    lying over $\mathscr{X}_{\kappavbar}$ is Zariski dense.
    But there is an open set of $\smash{\bbP_L^N}$ whose points correspond to smooth hypersurfaces.

    Now we apply Theorem~\ref{thm-5-spe-3} to complete the proof.
\end{proof}

