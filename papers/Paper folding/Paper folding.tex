\documentclass[twoside]{article}

\usepackage{enumitem}
\usepackage{graphics}

\usepackage{geometry}
\geometry{
    paperwidth = 155mm,
    paperheight = 235mm,
    outer = 20mm,
    inner = 20mm,
    top = 25mm,
    bottom = 20mm
}

% fonts & unicode
\usepackage[PunctStyle=kaiming]{xeCJK}
\usepackage{amsmath}
\usepackage{unicode-math}

\setCJKmainfont{NotoSerifCJKsc-Regular.otf}[
    Path            = ../../fonts/,
    BoldFont        = NotoSansCJKsc-Medium.otf,
    ItalicFont      = fzktk.ttf,
    Scale           = .97,
    ItalicFeatures  = {Scale = 1}
]

\setCJKsansfont{NotoSansCJKsc-DemiLight.otf}[
    Path            = ../../fonts/,
    BoldFont        = NotoSansCJKsc-Bold.otf,
    Scale           = .97
]

\setCJKmonofont{NotoSansCJKsc-DemiLight.otf}[
    Path            = ../../fonts/,
    BoldFont        = NotoSansCJKsc-Bold.otf,
    Scale           = .9
]

\newCJKfontfamily{\KaiTi}{fzktk.ttf}[
    Path            = ../../fonts/,
    BoldFont        = NotoSansCJKsc-Medium.otf,
    BoldFeatures    = {Scale = .97},
    ItalicFont      = NotoSerifCJKsc-Regular.otf,
    ItalicFeatures  = {Scale = .97}
]

\setmainfont{XITS}[
    Path            = ../../fonts/,
    Extension       = .otf,
    UprightFont     = *-Regular,
    BoldFont        = *-Bold,
    ItalicFont      = *-Italic,
    BoldItalicFont  = *-BoldItalic
]

\setsansfont{Lato}[
    Path            = ../../fonts/,
    Scale           = MatchUppercase,
    Extension       = .ttf,
    UprightFont     = *-Regular,
    BoldFont        = *-Bold,
    ItalicFont      = *-Italic,
    BoldItalicFont  = *-BoldItalic
]

\setmonofont{FiraMono}[
    Path            = ../../fonts/,
    Scale           = .9,
    Extension       = .otf,
    UprightFont     = *-Regular,
    BoldFont        = *-Bold
]

\setmathfont{XITSMath-Regular.otf}[
    Path            = ../../fonts/,
    BoldFont        = XITSMath-Bold.otf
]

\setmathfont{latinmodern-math.otf}[
    Path            = ../../fonts/,
    range           = {frak, bffrak},
    BoldFont        = latinmodern-math.otf
]

\setmathfont{LatoMath.otf}[
    Path            = ../../fonts/,
    Scale           = .95,
    BoldFont        = LatoMath.otf,
    version         = sf
]

\setmathfont{LatoMath.otf}[
    Path            = ../../fonts/,
    Scale           = .95,
    BoldFont        = LatoMath.otf,
    range           = {bb, sfup -> up, sfit -> it, bfsfup -> bfup, bfsfit -> bfit}
]

\setmathfont{STIX2Math.otf}[
    Path            = ../../fonts/,
    BoldFont        = STIX2Math-Bold.otf,
    range           = {\int, \sum, \prod, \coprod, \bigoplus, \bigotimes, \bigcup, \bigcap, \bigvee, \bigwedge}
]

\Umathcode`/  =  "0 "0 "2215    % / -> U+2215 division slash

% headers and footers
\usepackage{fancyhdr}
\fancyhf{}
\fancyhead[CE]{\sf\mathversion{sf}\theheadertitle}
\fancyhead[CO]{\sf\mathversion{sf}\nouppercase{\leftmark}}
\fancyhead[LE,RO]{\textbf{\textsf{\thepage}}}
\headsep=8mm
\headheight=6mm

\AtBeginDocument{
    \renewcommand{\thepage}{\roman{page}}
    \pagestyle{fancy}\thispagestyle{empty}
}

% spacing
\AtBeginDocument{
    \hfuzz=2pt
    \emergencystretch 2em
    \setlength{\belowdisplayshortskip}{\belowdisplayskip}
}

% parts
\usepackage{tikz}

\renewcommand{\titlepage}[2]{%
    \clearpage%
    \thispagestyle{empty}%
    \vspace*{20mm}%
    \centerline{\begin{tikzpicture}
        \node [scale = 3] at (0, 0) {\sffamily 荷\hspace{.5em}思};
        \node [scale = 1.8] at (0, -94.5mm) {\sffamily #1};
        \node [scale = 1.2] at (0, -105mm) {\sffamily 第 #2 期};
        \draw (-16mm, -8mm) -- (16mm, -8mm);
        \draw (-14mm, -100mm) -- (14mm, -100mm);
        \draw (-8mm, -110mm) -- (8mm, -110mm);
    \end{tikzpicture}}%
    \clearpage%
}

\newcommand{\committee}{%
    \clearpage%
    \thispagestyle{empty}%
    \vspace*{120mm}%
}

\newcommand{\committeeitem}[2]{%
    \par%
    {%
        \leftskip=3em%
        \rightskip=8em%
        \parindent=-3em%
        {\bfseries\sffamily#1}\quad%
        {\sffamily#2}%
        \par\vspace{6pt}%
    }%
}

\newcommand{\toctitle}{%
    \clearpage%
    \thispagestyle{empty}%
    \vspace*{15mm}%
    \noindent{\huge\sffamily 目录}%
    \par\vspace{10mm}%
}

\newcommand{\tocsection}[1]{%
    \par\vspace{6mm}%
    \noindent{\large\sffamily #1}%
    \par\vspace{4mm}%
}

\newcommand{\tocitem}[3]{%
    \par%
    {%
        \leftskip=3em%
        \parindent=-3em%
        \makebox[2em][r]{\textbf{\textsf{#3}}}%
        \quad#1%
        \hfill\mbox{}\hfill\phantom{#2}\hfill\makebox[0em][r]{#2}%
        \par\vspace{8pt}%
    }%
}

\usepackage{indentfirst}
\parindent=2em

% names in chinese
\def\abstractname{摘\quad 要}
\def\contentsname{目录}
\def\proofname{证明}

% bibliography
\DefineBibliographyStrings{english}{
    references = {参考文献},
}

% spacing
\usepackage{setspace}
\setdisplayskipstretch{} %https://tex.stackexchange.com/q/529214

\AtBeginDocument{
    \spacing{1.25}
}
\AtBeginBibliography{
    \spacing{1}
}

% lengths
\AtBeginDocument{
    \setlength{\abovedisplayskip}{8pt plus 4pt minus 4pt}
    \setlength{\belowdisplayskip}{8pt plus 4pt minus 4pt}
    \setlength{\belowdisplayshortskip}{8pt plus 4pt minus 4pt}
}

% theorems & proofs
\newtheoremstyle{cjk-theorem}{}{}{\KaiTi}{}{\bfseries}{.}{.5em}{}
\newtheoremstyle{cjk-definition}{}{}{}{}{\bfseries}{.}{.5em}{}
\newtheoremstyle{cjk-remark}{}{}{}{}{\KaiTi}{.}{.5em}{}

\theoremstyle{cjk-theorem}
\renewtheorem{theorem}{定理}[section]
\renewtheorem{lemma}[theorem]{引理}
\renewtheorem{corollary}[theorem]{推论}
\renewtheorem{proposition}[theorem]{命题}

\theoremstyle{cjk-definition}
\renewtheorem{definition}[theorem]{定义}
\renewtheorem{remark}[theorem]{注}
\renewtheorem{example}[theorem]{例}
\theoremstyle{cjk-theorem}

\numberwithin{equation}{theorem}

\makeatletter
\renewenvironment{proof}[1][\proofname]{\par
    \pushQED{\qed}%
    \normalfont \topsep6\p@\@plus6\p@\relax
    \trivlist
    \item\relax{\bfseries#1}\hspace{1em}\ignorespaces
}{\popQED\endtrivlist\@endpefalse}
\makeatother

% customized numbering for Step
\usepackage{cleveref}
\newtheorem{mstep}{Step}
\newtheorem*{convention}{convention}
\newenvironment{step}[1]{%
	\renewcommand\themstep{#1}%
	\mstep
}{\endmstep}
\crefname{step}{Step}{Steps}

% math letters
\newcommand{\Ab}{{\mathbb{A}}}
\newcommand{\Bb}{{\mathbb{B}}}
\newcommand{\Cb}{{\mathbb{C}}}
\newcommand{\Db}{{\mathbb{D}}}
\newcommand{\Eb}{{\mathbb{E}}}
\newcommand{\Fb}{{\mathbb{F}}}
\newcommand{\Gb}{{\mathbb{G}}}
\newcommand{\Hb}{{\mathbb{H}}}
\newcommand{\Ib}{{\mathbb{I}}}
\newcommand{\Jb}{{\mathbb{J}}}
\newcommand{\Kb}{{\mathbb{K}}}
\newcommand{\Lb}{{\mathbb{L}}}
\newcommand{\Mb}{{\mathbb{M}}}
\newcommand{\Nb}{{\mathbb{N}}}
\newcommand{\Ob}{{\mathbb{O}}}
\newcommand{\Pb}{{\mathbb{P}}}
\newcommand{\Qb}{{\mathbb{Q}}}
\newcommand{\Rb}{{\mathbb{R}}}
\newcommand{\Sb}{{\mathbb{S}}}
\newcommand{\Tb}{{\mathbb{T}}}
\newcommand{\Ub}{{\mathbb{U}}}
\newcommand{\Vb}{{\mathbb{V}}}
\newcommand{\Wb}{{\mathbb{W}}}
\newcommand{\Xb}{{\mathbb{X}}}
\newcommand{\Yb}{{\mathbb{Y}}}
\newcommand{\Zb}{{\mathbb{Z}}}
\newcommand{\Ac}{{\mathcal{A}}}
\newcommand{\Bc}{{\mathcal{B}}}
\newcommand{\Cc}{{\mathcal{C}}}
\newcommand{\Dc}{{\mathcal{D}}}
\newcommand{\Ec}{{\mathcal{E}}}
\newcommand{\Fc}{{\mathcal{F}}}
\newcommand{\Gc}{{\mathcal{G}}}
\newcommand{\Hc}{{\mathcal{H}}}
\newcommand{\Ic}{{\mathcal{I}}}
\newcommand{\Jc}{{\mathcal{J}}}
\newcommand{\Kc}{{\mathcal{K}}}
\newcommand{\Lc}{{\mathcal{L}}}
\newcommand{\Mc}{{\mathcal{M}}}
\newcommand{\Nc}{{\mathcal{N}}}
\newcommand{\Oc}{{\mathcal{O}}}
\newcommand{\Pc}{{\mathcal{P}}}
\newcommand{\Qc}{{\mathcal{Q}}}
\newcommand{\Rc}{{\mathcal{R}}}
\newcommand{\Sc}{{\mathcal{S}}}
\newcommand{\Tc}{{\mathcal{T}}}
\newcommand{\Uc}{{\mathcal{U}}}
\newcommand{\Vc}{{\mathcal{V}}}
\newcommand{\Wc}{{\mathcal{W}}}
\newcommand{\Xc}{{\mathcal{X}}}
\newcommand{\Yc}{{\mathcal{Y}}}
\newcommand{\Zc}{{\mathcal{Z}}}

% general notation
\newcommand{\id}{{\mathrm{id}}}
\newcommand{\im}{\operatorname{im}}
\newcommand{\coker}{\operatorname{coker}}
\newcommand{\coim}{\operatorname{coim}}
\newcommand{\codim}{\operatorname{codim}}
\newcommand{\Hom}{\operatorname{Hom}}
\newcommand{\End}{\operatorname{End}}
\newcommand{\gfr}{{\mathfrak{g}}}
\newcommand{\GL}{{\mathrm{GL}}}
\newcommand{\SU}{{\mathrm{SU}}}
\newcommand{\ad}{\operatorname{ad}}
\newcommand{\Ad}{\operatorname{Ad}}

% shorthands
\newcommand{\la}{\langle}
\newcommand{\ra}{\rangle}
\newcommand{\fr}{\forall\,}
\newcommand{\exi}{\exists\,}
\newcommand{\pa}{\partial}
\newcommand{\pab}{{\bar{\partial}}}
\newcommand{\ph}{\phantom{{}={}}}
\newcommand{\wt}[1]{\widetilde{#1}}
\newcommand{\wh}[1]{\widehat{#1}}
\newcommand{\ol}[1]{\overline{#1}}
\newcommand{\pfrac}[2]{\frac{\partial #1}{\partial #2}}

% special notation
\newcommand{\Crit}{\operatorname{Crit}}
\newcommand{\ind}{\operatorname{ind}}
\newcommand{\Sym}{\mathrm{Sym}}
\newcommand{\reg}{{\mathrm{reg}}}
\newcommand{\Morse}{{\mathrm{M}}}

\addbibresource{Paper.bib}

\begin{document}

\title{折纸问题的进一步讨论}
\author{黄虞来\footnote{清华大学数学系数92班.}}

\begin{abstract}
    在《折纸的代数结构》\cite{Hesi}一文中, 作者给出了折纸的六条公理.
    本文将完整地给出折纸构造点的刻画,
    并给出折纸的两个有趣实例: 三等分角与折正 $N$ 边形.
\end{abstract}

\section{折纸公理}

Let us fix the notations: $p$ a prime, $\ell$ a prime different from $p$, 
$k$ a field of characteristic $p$ and $K$ a field of characteristic $0$.

The motivation of a ``good" cohomology theory 
(more precisely, Weil cohomology theory, \cite{Sta}) 
for algebraic varieties is from Weil conjecture, 
which is about the zeta function of algebraic variety $X$ of characteristic $p$,
\[
    \zeta(X, t)=\sum_{n = 1} ^ {\infty} |X(\mathbb{F} _ {q^n})| \frac {t^n} {n}.
\]
\begin{theorem}[Weil conjecture]
    The following holds:
    \begin{itemize}
        \item (Rationality) 
            \[
                \zeta(X, t) = \prod _ {i = 0} ^ {2d} P_i(t) ^ {(-1) ^ {i + 1}},
            \]
            with $P_i(t)\in \BQ[t]$ and $d$ the dimension of $X$.
        \item (Functional equation) 
            For proper smooth $X$, 
            \[
                \zeta(X, \frac {1} {q^d t})=\pm q^{\frac {dE} {2}} t^E \zeta(X, t), 
                \quad E = \sum_{i = 0}^{2d} \deg P_i(t).
            \]
        \item (Purity) 
            The inverse of roots of $P_i(t)$ are algebraic integers 
            with absolute value $q^{\frac{k}{2}}$, 
            where $k$ is an integer between $0$ and $2i$. 
            When $X$ is proper and smooth, $k=i$. 
            (It is an analog of purity of mixed Hodge structures.)
    \end{itemize}
\end{theorem}
To prove the theorem one needs a Weil cohomology theory, 
that is, contravariant functors $H^*(X)$, $H^*_c(X)$, $H^*_Z(X)$ 
(the usual cohomology, cohomology with compact support and cohomology supported on a closed set) 
from varieties over a field $k$ of characteristic $p$ 
to vector spaces over some characteristic $0$ field, satisfying:
\begin{itemize}
    \item (Finiteness) 
        $H^*(X)$, $H^*_Z(X)$, $H^*_c(X)$ are finite dimensional 
        with $H^i(X) = 0$ for $i < 0$ or $i > 2d$.
    \item (K\"unneth formula) 
        $H^*(X\times Y) = H^*(X) \otimes H^*(Y)$, 
        similarly for $H^*_c$ and $H^*_Z$. 
    \item (Poincar\'e duality) 
        For smooth $X$, there is a trace map $H^{2d}(X) \to K$ satisfying 
        $H^*(X) \otimes H^{2d-*}_c(X) \to H^{2d}(X) \to K$ being a perfect pairing.
    \item (Cycle class) 
        There is a natural tranformation of functors $A^*(X) \to H^{2*}(X)$.
\end{itemize}
One deduces then the Lefschetz fixed point theorem 
and apply it to the Frobenius $F^n \colon X \to X$ 
to gain information about the rational points.

The ``Riemann hypothesis" part gives information about 
the archimedean absolute value of the roots of $P_k(t)$ 
and then the number of rational points. 
It is then natural to ask about $p$-adic absolute value of those roots 
($|\alpha|_{\ell} = 1$). 
However, $p$ -adic \'etale cohomology behaves badly for characteristic $p$ varieties. 
Thus, alternative cohomology theories are necessary.

In characteristic zero, the de Rham cohomology theory 
$H^i(X) = \mathbb{H}^i(\Omega_{X/K}^\cdot)$ 
is a good cohomology theory \cite{Sta}. 
Therefore, a rough idea is to lift $X/k$ to something over $K = \Frac(W(k))$, 
and working out the de Rham cohomology. 
Two ways to realize the idea are the crystalline cohomology and the rigid cohomology.

Another benefit of $p$-adic cohomology theories is that 
the Frobenius action on \'etale cohomology can not be calculated down simply. 
However, in $p$ -adic cohomology theory, 
one can represent it as a action of a de Rham complex. 
There is then a convenient algorithm of counting rational points (\cite{Ke2}).



\section{代数结构}

In this chapter we will give some trivial but useful lemmas for this problem.

\begin{defn}
	For any point $P\in A$, we call the set $\iT_P := \{ K \colon K\in T ,
	|KP| = \min_{i=1}^{n} |KA_i| \} $ the domain of $p$ and we call $P$
	the origin point of $\iT_{P}$.
\end{defn}	

\begin{defn}
	If $T$ and $n$ satisfies the condition in question, we call the
	pair $(T, n)$ a good pair.
\end{defn}	

In this section we just consider the properties of good pairs.

\begin{lem}
	The intersection line of two adjacent domains is the perpendicular
	bisector of the two origin points they correspond to.
\end{lem}
This lemma can be shown by the definition of domain.

\begin{lem}
	Domains are all convex.
\end{lem}	

\begin{center}
	\begin{tikzpicture}[scale=0.7]
	
	\draw[fill][lightgray](1.91,-3.56) circle [radius=0.05];
	\draw(1.91,-3.56) circle [radius=0.05];
	\node at (2.11,-3.76) {$A_2$};
	
	
	\draw[fill][lightgray](-5,-2.42) circle [radius=0.05];
	\draw(-5,-2.42) circle [radius=0.05];
	\node at (-5.3,-2.52) {$P$};
	
	
	\draw[fill][lightgray](-1.5,-0.26) circle [radius=0.05];
	\draw(-1.5,-0.26) circle [radius=0.05];
	\node at (-1.7,-0.16) {$Q$};
	
	
	\draw[fill][lightgray](4.59,-0.28) circle [radius=0.05];
	\draw(4.59,-0.28) circle [radius=0.05];
	\node at (4.89,-0.28) {$R$};
	
	
	\draw[fill][lightgray](1.93,3.02) circle [radius=0.05];
	\draw(1.93,3.02) circle [radius=0.05];
	\node at (2.2,3.0) {$A_1$};
	
	\draw[thick][black](-5,-2.42)--(-1.5,-0.26);
	\draw[thick][black](-1.5,-0.26)--(4.59,-0.28);
	\end{tikzpicture}
\end{center}

\begin{proof}
	If $\iT_{A_1}$ is not convex, we may as well assume that $\angle PQR$
	is more than $\pi$, here $P , Q , R$ are adjacent vertexes of
	$\iT_{A_1}$. Because $T$ is convex, $PQ$ can't be the side of $T$.
	So the symmetric point of $A_1$ about $QR$ must
	be another origin point. We call it $A_2$. Since $\angle PQR > \pi$,
	the length of $|PA_1|$ is larger than $|PA_2|$. So P couldn't
	be on the side of $\iT_{A_1}$. Then we get the contradiction.
\end{proof}

There are some lemmas about the area of each domain. For convenience,
we sometimes use $S_i$ to denote $S_{\iT_i}$,  
$S_{A_i}$ to denote $S_{\iT_{Ai}}$

\begin{lem}
	For any domain $\iT_{A_i}$ and $\epsilon > 0$, we can choose a
	point as one of points in $B$ and let the point change almost
	half area of $A$ into green. (i.e. it can change the domain
	with area at least $\frac{S_{\iT_{Ai}}}{2} - \epsilon$ into
	green.). And we can also choose two points $B_1$ and $B_2$
	so that they change almost the whole $\iT_{Ai}$ into green.
\end{lem}


\begin{center}
	\begin{tikzpicture}[scale=0.4]
	
	\draw[fill][lightgray](2.91,0.21) circle [radius=0.05];
	\draw(2.91,0.21) circle [radius=0.05];
	\draw[fill][lightgray](-0.56,5.32) circle [radius=0.05];
	\draw(-0.56,5.32) circle [radius=0.05];
	\draw[fill][lightgray](-7.49,4.44) circle [radius=0.05];
	\draw(-7.49,4.44)circle [radius=0.05];
	\draw[fill][lightgray](-9.02,-0.79) circle [radius=0.05];
	\draw(-9.02,-0.79) circle [radius=0.05];
	\draw[fill][lightgray](-7.12,-4.58) circle [radius=0.05];
	\draw(-7.12,-4.58)  circle [radius=0.05];
	\draw[fill][lightgray](0.45,-5.64) circle [radius=0.05];
	\draw(0.45,-5.64) circle [radius=0.05];
	\draw[fill][lightgray](-2.73,0.45) circle [radius=0.05];
	\draw(-2.73,0.45) circle [radius=0.05];
	\draw[fill][lightgray](-2.76,-0.20) circle [radius=0.05];
	\draw(-2.76,-0.20) circle [radius=0.05];
	\draw[fill][lightgray](-2.67,1.52) circle [radius=0.05];
	\draw(-2.67,1.52) circle [radius=0.05];
	
	\draw[thick][black](2.91,0.21)--(-0.56,5.32)--(-7.49,4.44)--(-9.02,-0.79)--(-7.12,-4.58)--(0.45,-5.64)--cycle ;
	
	\node at (-3.03,0.45) {$A_1$};
	\node at (-2.56,-0.4) {$B_1$};
	\node at (-2.47,1.76) {$B_2$};
	\node at (1.43,0.65) {$l$};
	\node at (-2.07,3.26) {$k$};
	
	\draw[dashed,thick][black](-2.34,7.46)--(-3.18,-7.94);
	\draw[dashed,thick][black](-10.45,0.87)--(5.08,0.02);
	
	\end{tikzpicture}	
\end{center}


\begin{proof}
	For any line $l$ passing through $A_1$, it divides $A$ into two parts.
	We consider the line $k$ perpendicular to $l$ and go through $A_1$. 
	
	$k$ divides $A_1$ into two parts $\iT_1$, $\iT_2$ (We assume 
	$S_1 \geq S_2 $). Let the diameter of $A_1$ be $d$, 
	choose a point $B_1$ on $k$ in $S_1$ such that 
	$|B_1A_1| \leq \frac{\epsilon}{2d}$. This
	point meets the first requirement. Then choose another 
	point $B_2$ on $k$ in $S_2$ such that $|B_2A_1| \leq 
	\frac{\epsilon}{2d}$. The two points, $B_1$ and $B_2$,  
	meet the second requirement.
\end{proof}

\begin{cor}
	The areas of all the domains are equal.
\end{cor}

\begin{proof}
	Suppose that $S_{A_1} \geq S_{A_2} \geq ... \geq S_{A_n}$. 
	By lemma 1.3, for every $\epsilon$ we can choose two points 
	in $\iT_{A_1}$ one point in $\iT_{A_2}$ and one point in 
	$\iT_{A_3}$ ... until one point in $\iT_{A_{n-1}}$ so that 
	these points change the domain of area at least $S_{A_1} + 
	\frac{S_{A_2} + ... S_{A_{n-1}}}{2} - \epsilon $ into green.
	Then we get that $S_{A_1} + \frac{S_{A_2} + ... S_{A_{n-1}}}{2}
	- \epsilon \leq \frac{S_{A_1} + ... S_{A_n}}{2} $ is true for
	every $\epsilon$. So $S_{A_1} = S_{A_n}$, 
	we get the corollary.
\end{proof}

\begin{lem}
	Every line goes through the origin point divides the 
	domain into two parts of equal area.
\end{lem}

\begin{proof}
	Using the same notation in lemma 3, for each line passing through 
	$A_i$, choose a point $B$ which changes the area of area 
	$\frac{S_1-S_2}{2}-\epsilon$ into green and we choose other $n-1$ 
	points which change at almost half area left into green. 
	We can get that $S_1 \leq S_2$, so $S_1 = S_2$.
\end{proof}	

We can get the most useful conclusion from lemma 4.

\begin{lem}
	All the domains are centrally symmetric.
\end{lem}

\begin{center}
	\begin{tikzpicture}[scale=0.45]
	
	\draw[thick][black](0.66,-0.24)--(-2.96,3.84)--(-9.68,4.29)
	--(-11.35,-0.74)--(-7.59,-5.29)--(0.32,-3.49)--cycle ;
	
	
	\draw[fill][lightgray](0.6,-0.77) circle [radius=0.05];
	\draw(0.6,-0.77) circle [radius=0.05];
	\draw[fill][lightgray](0.5,-1.75) circle [radius=0.05];
	\draw(0.5,-1.75) circle [radius=0.05];
	\draw[fill][lightgray](-4.6,0.21) circle [radius=0.05];
	\draw(-4.6,0.21) circle [radius=0.05];
	\draw[fill][lightgray](-10.31,2.4) circle [radius=0.05];
	\draw(-10.31,2.4) circle [radius=0.05];
	\draw[fill][lightgray](-10.66,1.36) circle [radius=0.05];
	\draw(-10.66,1.36) circle [radius=0.05];
	
	\node at (0.9,-0.77) {$C_2$};
	\node at (0.8,-1.75) {$B_2$};
	\node at (-4.4,0.41) {$A_1$};
	\node at (-10.61,2.4) {$B_1$};
	\node at (-10.96,1.36) {$C_1$};
	
	\draw[dashed,thick][black](0.5,-1.75)--(-10.31,2.4);
	\draw[dashed,thick][black](0.6,-0.77)--(-10.66,1.36);
	
	\end{tikzpicture}	
\end{center}


\begin{proof}
	For each line passing through $A_1$, it intersects the boundary 
	of the domain at two points $B_1$ and $B_2$. If $|A_1B_1| > |A_2B_2|$, 
	we rotate the line by a small angle $\theta$ and let the two new 
	points of intersection $C_1$ and $C_2$ satisfies $|A_1C_1| \geq |A_2C_2|$. 
	Then we know  $S_{A_1B_1C_1} = \frac{1}{2}|A_1B_1||A_1C_1|sin\theta >  \frac{1}{2}|A_1B_2||A_1C_2|sin\theta = S_{A_1B_2C_2}$. 
	From lemma 4
	$S_{A_1B_1C_1}=S_{A_1B_2C_2}$, then we get the 
	contradiction. Hence $|A_1B_1|=|A_2B_2|$ for any line, 
	which implies that $\iT_{A_1}$ is centrally symmetric.
\end{proof}

Now we have got all tools we need. Then we will discuss an example
to use these tools.


\section{第六公理}

\subsection{Basic definitions and comparison results}
In this section, let $k$ be a perfect field with characteristic $p$, 
$\CO = W(k)$ the Witt ring and $K = \Frac(W(k))$. 
$X, Y$ etc. are varieties over $k$ and 
$\CP$ a formal scheme over $\CO$. 
$\CP_K, \CP_k$ the generic fibre and special fibre of $\CP$.
We firstly define the convergent cohomology for a basic approach.
\begin{definition}
    Let $X \to \CP$ be an immersion, 
    $\Sp \colon \CP_K \to \CP_k$ the specialization map. 
    The tube of $X$ is the inverse image $\Sp^{-1}(X) \subset \CP$, denoted $]X[_\CP$.
\end{definition}

\begin{definition}
    If there exists an immersion $X \to \CP$ such that 
    $\XP$ is smooth, the convergent cohomology of $X$ is defined as
    \[
        \HCd = R\Gamma(\DRX),
    \]
    where $\DRX$ is the de Rham complex.
\end{definition}

\begin{proposition}[Weak fibration theorem \cite{St}]
    Suppose there is a commutative diagram 
    \[
        \begin{tikzcd}
            & \CP'\ar[d, "p"]\\
            X \ar[r] \ar[ur] \ar[dr] & \CP \ar[d] \\
            & S,
        \end{tikzcd}
    \]
    where $\CP' \to \CP$ is smooth in a neighbourhood of $X$. 
    Then $]X[_{\CP'} = ]X[_{\CP} \times B^\circ(0,1)^d$ locally on $X$, 
    for $d$ the relative dimension of $\CP' \to \CP$.
\end{proposition}

\begin{proposition}
    The cohomology $\HCd$ is independent of the choice of $\CP$, 
    which means the convergent cohomology is well-defined. 
\end{proposition}

\begin{proof}
    For two immersions $X \to \CP_1$ and $X\to \CP_2$ with $\CP_1$ and $\CP_2$ smooth, 
    $X \to \CP = \CP_1 \times \CP_2$ maps to both of them by projections $p_1$ and $p_2$. 
    By the method of Gauss--Manin filtration and 
    Poincar\'e lemma mentioned in \ref{comparison}, 
    one proves that
    \[
        Rp_{1*} \DRX \simeq \Omega_{\XP/K}^\bullet ]X[_{\CP_1}.
    \] 
    Taking global sections gets the result.
\end{proof}

\begin{remark}
    Such an immersion exists locally: for affine $X$, $X$ can be embedded into $\BA_k^n$. 
    Then into $\widehat{\BA^n_{\CO}}$. 
    We may use the cohomological descent method 
    to define the convergent cohomology in general. 
\end{remark}

The convergent cohomology behaves badly for the non-proper varieties.
\begin{example}
    \label{counter ex}
    Let $X = \BA_k^1$ be the affine line, $\CP = \Spf (\BZ_p \langle T \rangle)$, 
    where $\BZ_p \langle T \rangle$ consists of all the convergent series. 
    Then $\CP_K = \Spp (K \langle T \rangle)$ and $\XP = \CP_K$. 
    By the "Theorem B" in rigid geometry, all the sheaves $\DRXn$ are acyclic. 
    Then the convergent cohomology is the cohomology of the complex
    \[
        \begin{tikzcd}
            K \langle T \rangle \ar[r,"d"] & K \langle T \rangle dT.
        \end{tikzcd}
    \]
    However, $H^1$ is not finite dimensional 
    for there are infinitely many linearly independent convergent series 
    that no longer converge after integration.
\end{example}

We then introduce the rigid cohomology to resolve the problem mentioned above.
\begin{definition}
    A frame of $X$ contains arrows,
    \[
        \begin{tikzcd}
            X \ar[r,"j"] & Y \ar[r,"i"] & \CP
        \end{tikzcd}
    \]
    such that $j$ is an open immersion and $i$ a closed immersion. 
    A proper smooth frame is a frame such that $Y$ is proper and $\YP$ is smooth. 
\end{definition}

\begin{definition}
    Let $\CF$ be an abelian sheaf on $]Y[_\CP$, 
    the overconvergent sheaf $j_X^\dagger \CF$ is 
    \[
        j_X^\dagger \CF = \colim j_{V*} j_{V}^* \CF,
    \]
    where $V$ runs through all the strict neighbourhood of $\XP$ in $\YP$ 
    and $j_V$ the inclusion $V \to \YP$. 
\end{definition}

\begin{definition}
    If there exists a proper smooth frame of $X$, the rigid cohomology $\HRd$ is 
    \[
        \HRd = R\Gamma(j_X^ \dagger \DRY),
    \]
    where $\DRY$ is the de Rham complex.
\end{definition}

\begin{proposition}
    The rigid cohomology of $X$ is independent of the choice of the frame if such frame exists.
\end{proposition}

\begin{proof}
    It is similar to the case of convergent cohomology. 
    See chapter 6 in \cite{St} for details.
\end{proof}

\begin{remark}
    A proper smooth frame of $X$ exists locally. 
    That is, for affine $X$, $X$ can be embedded into $\BA_k^n$ and then $\BP_k^n$. 
    Then we may take $Y$ be $\bar{X} \subset \BP_k^n$ and $\CP$ be $\widehat{\BP_\CO^n}$. 
    Then we may use cohomological descent to define rigid cohomology in general (\cite{Tsu}).
\end{remark}

The rigid cohomology resolves the problem above.
\begin{example}
    Let $X = \BA_k^1$, $Y = \BP_k^1$, $\CP = \widehat{\BP_{\CO}^1}$. 
    Then $\XP = \Spp (K \langle T \rangle)$, $\YP = \BP^{1,an}_K$. 
    Then $\{V^\lambda = \Spp (K \langle \lambda T \rangle) \mid |\lambda| < 1 \}$ 
    forms a cofinal system of strict neighbourhoods of $\XP$ in $\YP$. 
    Then the cohomology becomes
    \[
        \begin{tikzcd}
            K \langle T \rangle^\dagger \ar[r] & K \langle T \rangle^\dagger dT,
        \end{tikzcd}
    \]
    where $K \langle T \rangle^ \dagger$ contains all the overconvergent series 
    (i.e. converges in some radius $\mu > 1$). 
    Then $H^1 = 0$ as the integration of overconvergent series remains overconvergent.

    In general, for an affine smooth variety $X = \Spec (A)$, 
    $A$ can be lifted to a finitely presented formally smooth algebra 
    $\CA = \CO \langle T_1, \cdots, T_n \rangle/(f_1, \cdots, f_m)$. 
    Then the rigid cohomology of $X$ becomes the cohomology of the de Rham complex 
    $\Omega_{\CA_K^\dagger/K}^\bullet$, 
    where $\CA_K^\dagger \simeq K \langle T_1, \cdots,T_n \rangle^\dagger/(f_1, \cdots, f_m)$. 
    That is, the Monsky--Washnitzer cohomology.
\end{example}

\begin{remark}
    The way considering overconvergent series can be justified by the language of adic spaces. 
    Indeed, $\XP$ is not closed in $\YP$ when viewing them as adic spaces 
    and we should not expect that a non-closed thing can have a good cohomology theory. 
    Thus we should consider the closure of $\XP$ in $\YP$ and work out its cohomology. 
    Indeed, the strict neighbourhoods are neighbourhoods of the closure of $\XP$, $\overline{\XP}$. 
    Thus, for $i \colon \overline{\XP} \to \YP$,
    \[
        \HRd = R\Gamma_{\YP} i_* i^* \DRY.
    \]
\end{remark}

In proper case, the convergent and rigid cohomology concide.
\begin{proposition}
    For proper $X$, there is a canonical isomorphism in derived category
    \[
        \HCd \simeq \HRd.
    \]
\end{proposition}

\begin{proof}
    The proposition is trivial for projective $X$, 
    for there exists a closed immersion $X \to \BP_k^n$, 
    then an immersion $X \to \widehat{\BP_\CO^n}$ and a frame $X \to X \to \widehat{\BP_\CO^n}$. 
    In general, one checks that both convergent cohomology and rigid cohomology functors 
    are right Kan extensions from the category of projective varieties 
    to the category of proper varieties 
    (for a proper variety, it has a simplicial resolution by projective varieties). 
    See \cite{Tsu} for details. 
\end{proof}

We are now able to prove the comparison theorem 
between crystalline cohomology and convergent/rigid cohomology.
\begin{theorem}
    There exists an canonical isomorphism for smooth $X$:
    \[
        \Hkd \otimes K \simeq \HCd.
    \]
\end{theorem}

\begin{proof}
    By cohomological descent, it suffices to prove that it holds locally. 
    For affine $X$, there exists a formally smooth lift $\CP=\CX$. Then $\XP=\CX_K$ and 
    \[
        \Hkd \simeq \Omega_{\CX/\CO}^\bullet, \quad \HCd \simeq \Omega_{\CX_K/K}^\bullet.
    \]
    Hence the result follows from the fact that 
    \[
        \Omega_{\CX/\CO}^\bullet \otimes K \simeq \Omega_{\CX_K/K}^\bullet.
    \]
\end{proof}

The comparison between crystalline and rigid cohomology 
is a direct consequence of the above two facts
\begin{corollary}[\cite{Be1}]
    For proper smooth $X$, there exists an isomorphism 
    \[
        \Hkd \otimes K \simeq \HRd.
    \]
\end{corollary}

As crystalline cohomology can be defined for crystals, 
rigid cohomology can be defined for "overconvergent isocrystals". 
As the categorical equivalence mentioned above, 
the following abelian categories are equivalent 
(chapter 7 in \cite{St}) (let $X \to Y \to \CP$ be a proper smooth frame):
\begin{itemize}
    \item 
        Coherent $j^\dagger \CO_{\YP/K}$ -modules with integrable connection.
    \item 
        Coherent $j^\dagger \CO_{\YP/K}$ -modules with stratification,
    \item 
        $j^\dagger \CO_{\YP/K}$-modules 
        which are coherent as $j^\dagger \CO_{\YP/K}$ -modules,
    \item 
        Overonvergent isocrystals, denoted by $Isoc^\dagger(X \to Y)$. 
        That is, associating each morphism of frames
        \[
            \begin{tikzcd}
                X' \ar[r] \ar[d] & Y' \ar[r] \ar[d] & \CP' \ar[d] \\
                X \ar[r] & Y \ar[r]& \CP
            \end{tikzcd}
        \]
    a coherent $j^\dagger \CO_{\YP/K}$-module $\CE_\CP'$, 
    satisfying the crystal relations as above.
\end{itemize}

Moreover, those categories are "stacks" in some sense. 
That is, taking the fifth for example, 
there is an equivalence ($\CU = \{Y_i \to Y\}$ be an open covering):
\[
    Isoc^\dagger (X \to Y) \to \lim_{\CU} Isoc^\dagger (X. \to Y.)
\]
where $X.,Y.$ means the simplicial scheme with respect to the open covering $\CU$. 
Then we can define those concepts for the case $X$ 
does not admit a proper smooth frame by descent.

Then for an overconvergent isocrystal $\CE$ 
(with the corresponding coherent sheaf with integrable connection also denoted $\CE$), 
we may define the rigid cohomology with coefficients
\[
    H^n_{rig}(X,\CE) := \BH^n(\CE \otimes \Omega_{\YP/K}^\bullet).
\]
Then there is a comparison theorem between crystalline cohomology and rigid cohomology.
\begin{theorem}
    There is a functor for smooth $X$:
    \[
        \{\textit{coherent crystals on } (X/W(k))_{Cris}\} \to Isoc(X)
    \]
    and a functor for proper smooth $X$
    \[
        \{\textit{coherent crystals on }(X/W(k))_{Cris}\} \to Isoc^\dagger(X),
    \]
    ($Isoc(X)$ means convergent isocrystals), 
    denoted $\CE \mapsto \CE_K$ such that for smooth $X$,
    \[
        R^i\Gamma((X/S)_{Cris}, \CE) \otimes K \simeq H^i_{conv}(X, \CE_K),
    \]
    and for proper smooth $X$,
    \[
        R^i\Gamma((X/S)_{Cris}, \CE) \otimes K \simeq H^i_{rig}(X, \CE_K).
    \]
\end{theorem}

\begin{proof}
    For the first functor, when $X$ admits a smooth lifting $\CX$ (this is the local case), 
    $\CE$ corresonds a $\CO_{\CX}$-module with an integrable connection, denoted $\CE$ also. 
    $CE_K := \CE \otimes K$ is then a coherent sheaf 
    over $\CX_K$ with induced integrable connection. 
    The comparison then holds. 
    We may use descent theory to work out in general case.

    For the second functor, note that for projective $X$, 
    there is a frame $X \to X \to \CP$ and thus $Isoc^\dagger(X) \simeq Isoc(X)$. 
    Resolving general proper smooth varieties 
    by projective smooth verieties proves the general case. 
\end{proof}

\begin{remark}
    In practice, we shall use overconvergent $F$-isocrystals only. 
    That is, overconvergent isocrystals with a compatible Frobenius action.
\end{remark}

One can define the cohomology with compact support and cohomology 
supported on a closed set in rigid cohomology theory.
\begin{definition}
    Let $X \to Y \to \CP$ be a proper smooth frame. 
    View $\YP$ as an adic space and then $\XP$ is an open subspace of it. 
    Denote $i\colon \overline{\XP} \to \YP$ and $j$ its complement.
    \[
        H^\bullet_{c,rig}(X) = R\Gamma_{\YP} (i_*i^! \DRY) 
        \simeq R\Gamma_{\YP} (\DRY \to j_*j^*\DRY).
    \]
\end{definition}

\begin{definition}
    Let $X \to Y \to \CP$ be a proper smooth frame and $Z$ a closed subvariety of $X$. 
    Denote $j \colon \ZP \to \YP$ and $i$ its complement.
    \[
        H^\bullet_Z(X) = R\Gamma_{\YP} (j_! j^* \DRY) 
        \simeq R\Gamma(\DRY \to i_*i^* \DRY).
    \]
\end{definition}

Then we have exact sequences for $X=U\cup Z$, 
with $U$ open and $Z$ closed 
(we shall omit the subscript ``rig'' if there is no confusion):
\[
    \cdots \to H^\bullet_Z(X) \to H^\bullet(X) \to H^\bullet(U) \to H^{\bullet+1}_Z(X) \to \cdots
\]
and 
\[
    \cdots \to H^\bullet_c(U) \to H^\bullet_c(X) \to H^\bullet_c(Z) \to H^{\bullet+1}_c(U) \cdots,
\]
and some excision theorems. 
For example, for $Z \subset Y \subset X$ being closed varieties, 
there is an exact sequence
\[
    \cdots \to H^\bullet_Y(X) \to H^\bullet_Z(X) \to H^\bullet_Z(Y) \to H^{\bullet+1}_Y(X) \to \cdots.
\]

\subsection{Verification of the desired properties}
We verify the assuptions in the Weil cohomology theory in remains of this section. 
\begin{proposition}[Gysin isomorphism, \cite{St}]
    For a smooth variety $X$ and its smooth, closed subvariety $Z$, 
    if $X$ is liftable, there is an isomorphism
    \[
        H^\bullet_Z(X) \simeq H^{\bullet-2c}(Z),
    \]
    called the Gysin isomorphism, where $c$ is the codimension of $Z$ in $X$.
\end{proposition}

\begin{proof}
    For $X$ is liftable, $Z$ is also. 
    One represents those rigid cohomology by de Rham cohomology 
    and then define the Gysin map. 
    It can be checked that it is an isomorphism.
\end{proof}

\begin{theorem}[Finiteness theorem, \cite{Be3}]
    For any variety $X / k$ and overconvergent $F$-isocrystal $\CE$ on it, 
    the cohomology $H^\bullet_{rig}(X, \CE)$ is finite dimensional.
\end{theorem}

\begin{proof}
    We shall prove the case that $\CE$ is trivial only. 
    Firstly consider the case that $X$ is smooth. 
    We apply induction. Consider the following two propositions:
    \begin{itemize}
        \item [(a)\textsubscript{$n$}] 
            $H^\bullet(X)$ is finite dimensional for all smooth varieties $X$ 
            with dimension no greater than $n$.
        \item [(b)\textsubscript{$n$}] 
            $H^\bullet_Z(X)$ is finite dimensional for all varieties $Z$ 
            with dimension no greater than $n$ and $X$ being smooth.
    \end{itemize}
    The proposition (a)\textsubscript{$0$} is clear. 
    (b)\textsubscript{$0$} follows from the Gysin ismorphism. 
    (One can choose an affine open set $U$ containing $Z$, 
    $H^\bullet_Z(U) \simeq H^\bullet_Z(X)$ by excision. $U$ is liftable.)

    The implication (b)\textsubscript{$n-1$} $\Rightarrow$ (a)\textsubscript{$n$}: 
    for such $X$, by de Jong's alternation theorem (theorem 4.1 in \cite{dJ}), 
    there is a projective smooth variety $X'$ and its open set $U$, 
    such that $p \colon U \to X$ is proper and generically \'etale. 
    There is then a dense open set $U_1$ of $X$ such that 
    $p \colon p^{-1}(U_1) \to U_1$ is \'etale (hence finite). 
    By finiteness theorem for crystalline cohomology, 
    the cohomology of $X'$ is of finite dimension. 
    By the long exact sequence above and (b)\textsubscript{$n-1$}, 
    the cohomology of $p^{-1}(U_1)$ is also. 
    There is a trace map for finite morphisms 
    (it can be defined locally via de Rham cohomology, and then globally via gluing), 
    becoming a section (up to a scalar) of the pull back map. 
    Thus the cohomology of $U_1$ is finite dimensional. 
    By (b)\textsubscript{$n-1$} and the long exact sequence again, 
    the cohomology of $X$ is finite dimensional.

    The implication (b)\textsubscript{$n-1$} and 
    (a)\textsubscript{$n$} $\Rightarrow$ (b)\textsubscript{$n$}: 
    one applies excision to reduce to the case that $X, Z$ are smooth and $X$ is liftable. 
    Then apply (a)\textsubscript{$n$} and the Gysin isomorphism to conclude.

    For general $X$, it is concluded by choosing a simplicial resolution of $X$ 
    by smooth varieties and applying cohomological descent. 
    See \cite{Tsu} for details. 
\end{proof}

Using similar methods, one proves that 
the cohomology with compact support is of finite dimension.
\begin{theorem}[K\"unneth formula, \cite{Be4}]
    For $X,Y$ arbitrary varieties, $\CE_1$ an overconvergent isocrystal on $X$ 
    and $\CE_2$ an overconvergent isocrystal on $Y$, 
    there is an isomorphism 
    (then $\CE_1 \boxtimes \CE_2$ is an overconvergent isocrystal on $X \times Y$):
    \[
        H^\bullet(X, \CE_1) \otimes H^\bullet(Y, \CE_2) \simeq H^\bullet(X \times Y, \CE_1 \boxtimes \CE_2).
    \]
    Similarly for cohomology with compact supports.
\end{theorem}

\begin{proof}
    It is almost the same as the case for crystalline cohomology. 
    One can check it by first working locally 
    via the de Rham complex and then noticing that the map is canonical. 
\end{proof}

\begin{theorem}[Poincar\'e duality, \cite{Be4}]
    For smooth $X$ of dimension $d$, 
    there is a trace map $\tr \colon H^{2d}(X) \to K$ such that 
    for an overconvergent $F$-isocrystal $\CE$, the pairing
    \[
        H^\bullet(X, \CE) \otimes H^{2d-\bullet}_c(X, \CE^\vee) \to H^{2d}(X) \to K
    \]
    is perfect.
\end{theorem}

\begin{proof}
    The trace map is defined almost the same as the case for crystalline cohomology. 
    We shall prove the case that $\CE$ is trivial. We consider two propositions:
    \begin{itemize}
        \item [(a)\textsubscript{$n$}] 
            Poincar\'e duality holds 
            for all smooth varieties of dimension no greater than $n$.
        \item [(b)\textsubscript{$n$}] 
            The pairing ($d$ is the dimension of $Z$)
            \[
                H^\bullet_Z(X) \times H^{2d-\bullet}_c(Z) \to K
            \] 
            (the trace map can be similarly defined) is perfect 
            for $Z$ (may not be smooth) of dimension no greater than $n$ and $X$ smooth.
    \end{itemize}
    The case (a)\textsubscript{$0$} is trivial. 
    (b)\textsubscript{$0$} is from the Gysin isomorphism. 
    (b)\textsubscript{$n-1$} $\Rightarrow$ (a)\textsubscript{$n$} is done 
    by de Jong's alternation theorem 
    and the Poincar\'e duality for crystalline cohomology 
    (note that the trace map for finite morphisms are compatible with all those constructions). 
    (b)\textsubscript{$n-1$} and 
    (a)\textsubscript{$n$} $\Rightarrow$ (b)\textsubscript{$n$} is done 
    by excision and Gysin isomorphism. 
    The whole process is similar to the proof of finiteness and we shall omit it.
\end{proof}

\begin{remark}
    The case for general overconvergent $F$-isocrystals $\CE$ 
    can be proved by ``$p$-adic local monodromy theorem'', 
    which claims that after a suitable base change, 
    the isocrystal is unipotent, i.e., 
    has a filtration with each subquotient becoming trivial. 
    See \cite{Ke1} for details.
\end{remark}

\begin{remark}
    The cycle class map can be defined 
    firstly by de Rham cohomology locally 
    and then applying cohomological descent.
\end{remark}

All requirements for a good cohomology theory are therefore proved. 
We end this article by remarking an example.
\begin{example}
    If a family of varieties over $k$ comes from reduction of a proper smooth family, 
    the cohomology of all varieties in the family has the same dimension. 
    The family of hypersurfaces in the projective space is the case. 
    For example, all plane cubic curves have 
    $h^0 = 1$, $h^1 = 2$, $h^2=1$ and $h^k = 0$ for $k > 2$. 
    Here $h^k := \dim H^k$.
\end{example}

\begin{proof}
    The first statement is from the fact that 
    Betti numbers are invariant over a family and 
    the comparison theorem between de Rham cohomology and Betti cohomology. 

    The second statement is from the fact that 
    for any homogenous polynomial $f\in k[x_0, \dots, x_n]$, 
    there is a element $\tilde{f} \in W(k)[x_0, \dots, x_n]$ 
    defining a smooth hypersurface of $\BP_K^n$, 
    as the set of singular hypersurfaces is Zariski closed in the parameter space 
    and all $\tilde{f}$ is Zariski dense in the the parameter space.

    The last statement follows from computations of Betti numbers of cubic curves. 
\end{proof}

The example illustrates that for singular varieties, 
rigid cohomology may not be the ``intuitive'' one. 
For example, the first Betti number of a cuspidal cubic curve over $\BC$ is $0$ 
and the first rigid cohomology of a cuspidal cubic curve over $k$ is of dimension 2. 

\section{折纸三等分角}

In this section two main theorems are stated.
	
\begin{thm}
	For a fixed $n$ there exists a $T$ which makes $(T, n)$ a good pair.
\end{thm}
	
\begin{thm}
	For a fixed $T$ and a large enough $n$, $(T, n)$ is not a good pair.
\end{thm}
	
The two theorems show that $(T, n)$ may possibly be good or bad when $T$ 
or $n$ changes. And we can see another fact by the proof then: the 
condition is better when a domain would not 'interfere' with
another domain.\newline
	
We prove the first theorem first as it is easier.
	
\begin{proof}[proof of Theorem 4.1]
	We just need to construct the $T$ for each $n$.
		
	Let $T$ be a rectangle of length $8n$ and width $2$, $A_k$ 
	be $(8k-4, 1)$(let the lower left corner of $T$ be the origin 
	point and establish coordinate system).
		
	To prove such an example meets the requirement, we need to prove
	that each $B$ changes at most an area of measure $8$ into green.
		
	Note that a point $B$ changes the points in at most three different 
	domains into green. So we just discuss by the number of domains $B$ changes.
		
	If $B$ changes two domains. We have two cases.\newline
		
	The first case: The area that $B$ changes is a trapezoid.
		
\begin{center}
	\begin{tikzpicture}[scale=0.6]
		\draw[ultra thick][blue][fill=yellow](8,0)--(8,3)--(6.29,3)
		--(6.79,0)--cycle;
		\draw[ultra thick][blue](0,0)--(20,0)--(20,3)--(0,3)--cycle ;
		\draw[fill][red](0,-0) circle [radius=0.07];
		\draw(0,-0) circle [radius=0.07];
		\draw[fill][red](20,0) circle [radius=0.07];
		\draw(20,0) circle [radius=0.07];
		\draw[fill][red](20,3) circle [radius=0.07];
		\draw(20,3) circle [radius=0.07];
		\draw[fill][red](0,3) circle [radius=0.07];
		\draw(0,3) circle [radius=0.07];
		
		\draw[ultra thick][blue](5,0)--(5,3);
		
		\draw[ultra thick][blue](8,0)--(8,3);
		
		\draw[ultra thick][blue](10,0)--(10,3);
		
		
		\draw[fill][yellow](5,1.5) circle [radius=0.07];
		\draw(5,1.5) circle [radius=0.07];
		\node at (4.6,1.5){$A_1$};
		\draw[fill][yellow](15,1.5) circle [radius=0.07];
		\draw(15,1.5) circle [radius=0.07];
		\node at (15.4,1.5){$A_2$};
		
		
		
		\draw[fill][green](8,2) circle [radius=0.07];
		\draw(8,2) circle [radius=0.07];
		\node at (8.3,2.3){$B$};
		
		\draw[fill][red](8,1.5) circle [radius=0.07];
		\draw(8,1.5) circle [radius=0.07];
		\node at (8.3,1.4){$E$};
		
		\draw[fill][red](6.5,1.75) circle [radius=0.07];
		\draw(6.5,1.75) circle [radius=0.07];
		\node at (6.2,2){$T$};
		
		\draw[fill][red](6.29,3) circle [radius=0.07];
		\draw(6.29,3) circle [radius=0.07];
		\draw[fill][red](6.79,0) circle [radius=0.07];
		\draw(6.79,0) circle [radius=0.07];
		
		\draw[fill][red](6.54,1.5) circle [radius=0.07];
		\draw(6.54,1.5) circle [radius=0.07];
		\node at (6.24,1.2){$D$};
		
		\draw[ultra thick][blue](8,2)--(5,1.5)--(8,1.5);
		\draw[ultra thick][blue](8,2)--(6.54,1.5);
		
	\end{tikzpicture}	
\end{center}
		
	Label the points with letters as it shown in the picture above, 
	we figure out that $HI$ $LM$ are the perpendicular bisectors of 
	$A_1B$ and $A_2B$. We need to prove that $S_{HIML} \leq 4$. Note 
	that $|A_1D| =|BD| \geq |DE|$. Then we get $S_{HIGK} \leq S_{FHIJ}$. 
	Similarly we have that $S_{GKML} \leq S_{MLNO}$ and then 
	get the conclusion.\newline
		
	The second case: The area that $B$ changes is a triangle.
		
\begin{center}
	\begin{tikzpicture}[scale=0.6]
		\draw[thick][blue][fill=lime](8,3.33)--(2.29,0)--(8.69,0)--cycle;
	
		\draw[thick][blue](0,0)--(16,0)--(16,4)--(0,4)--cycle ;
		\draw[fill][red](0,-0) circle [radius=0.07];
		\draw(0,-0) circle [radius=0.07];
		\draw[fill][red](16,0) circle [radius=0.07];
		\draw(16,0) circle [radius=0.07];
		\draw[fill][red](16,4) circle [radius=0.07];
		\draw(16,4) circle [radius=0.07];
		\draw[fill][red](0,4) circle [radius=0.07];
		\draw(0,4) circle [radius=0.07];
	
		\draw[thick][blue](8,0)--(8,4);
	
		\draw[thick][blue](4,2)--(4.87,0.5)--(12,2);
		\draw[thick][blue](8,4)--(0,0);
		\draw[thick][blue](8,3.33)--(0,0.67)--(2.29,0)--(4,2);
	
		\draw[fill][yellow](4,2) circle [radius=0.07];
		\draw(4,2) circle [radius=0.07];
		\node at (3.7,2.3){$A_1$};
		\draw[fill][yellow](12,2) circle [radius=0.07];
		\draw(12,2) circle [radius=0.07];
		\node at (12.3,2.3){$A_2$};
	
		\draw[fill][green](4.87,0.5) circle [radius=0.07];
		\draw(4.87,0.5) circle [radius=0.07];
		\node at (4.81,0.1){$B$};
	
		\draw[fill][red](0,0) circle [radius=0.07];
		\draw(0,0) circle [radius=0.07];
		\node at (-0.3,-0.3){$H$};
	
		\draw[fill][red](8,0) circle [radius=0.07];
		\draw(8,0) circle [radius=0.07];
		\node at (8,-0.4){$I$};
	
		\draw[fill][red](8,4) circle [radius=0.07];
		\draw(8,4) circle [radius=0.07];
		\node at (8,4.4){$J$};
	
		\draw[fill][red](0,0.67) circle [radius=0.07];
		\draw(0,0.67) circle [radius=0.07];
		\node at (-0.3,0.67){$L$};
	
		\draw[fill][red](2.29,0) circle [radius=0.07];
		\draw(2.29,0) circle [radius=0.07];
		\node at (2.29,-0.4){$E$};
	
		\draw[fill][red](4.43,1.25) circle [radius=0.07];
		\draw(4.43,1.25) circle [radius=0.07];
		\node at (4.13,1.2){$K$};
	
		\draw[fill][red](8.69,0) circle [radius=0.07];
		\draw(8.69,0) circle [radius=0.07];
		\node at (8.69,-0.4){$F$};
	
		\draw[fill][red](8,3.33) circle [radius=0.07];
		\draw(8,3.33) circle [radius=0.07];
		\node at (8.4,3.33){$D$};
	\end{tikzpicture}	
\end{center}
		
	We may as well let the coordinate of the two origin points be 
	$(4, 1)$, $(12, 1)$. And the coordinate of $B$ be $(4+a, 1-b)$.
		 
	\begin{equation}
		\begin{split}
		&S_{DIF} = \frac{1}{2}|DI||IF| = |DI| \times \frac{|DI|}{|k_{DF}|}
		\leq 2|DI||k_{BA_2}| = 2|DE|\frac{a}{\sqrt{a^2+b^2}}\frac{b}{8-a}\\ 
		\leq & |DE|\frac{b}{2}
		\leq |ED||KA_1|=2S_{EDA1}
		\leq S_{EDA1} + S_{LEA1} \\
		\leq & S_{EDA1}+S_{LHEA1} = S_{HEDJ}
		\end{split}
	\end{equation}
	
	So $S_{DEF}\leq S_{IJH}$.\newline
		
	If $B$ changes three domains.
		
\begin{center}
	\begin{tikzpicture}[scale=0.6]
		
		\draw[thick][blue][fill=orange](6,1.03)--(5.88,0)--
		(12.04,0)--(12,0.26)--cycle;
		\draw[thick][blue](0,0)--(18,0)--(18,2)--(0,2)--cycle ;
		\draw[fill][red](0,-0) circle [radius=0.07];
		\draw(0,-0) circle [radius=0.07];
		\draw[fill][red](18,0) circle [radius=0.07];
		\draw(18,0) circle [radius=0.07];
		\draw[fill][red](18,2) circle [radius=0.07];
		\draw(18,2) circle [radius=0.07];
		\draw[fill][red](0,2) circle [radius=0.07];
		\draw(0,2) circle [radius=0.07];
		
		\draw[thick][blue](6,0)--(6,2);
		
		\draw[thick][blue](12,0)--(12,2);
		
		
		
		
		\draw[fill][yellow](9,1) circle [radius=0.07];
		\draw(9,1) circle [radius=0.07];
		\draw[fill][yellow](3,1) circle [radius=0.07];
		\draw(3,1) circle [radius=0.07];
		\draw[fill][yellow](15,1) circle [radius=0.07];
		\draw(15,1) circle [radius=0.07];
		
		\draw[fill][green](8.92,0.29) circle [radius=0.07];
		\draw(8.92,0.29) circle [radius=0.07];
		
		\draw[fill][red](8.96,0.65) circle [radius=0.07];
		\draw(8.96,0.65) circle [radius=0.07];
		
		
		\draw[fill][red](6,1.03) circle [radius=0.07];
		\draw(6,1.03) circle [radius=0.07];
		\node at (5.7,1){$A$};
		
		
		\draw[fill][red](5.88,0) circle [radius=0.07];
		\draw(5.88,0) circle [radius=0.07];
		\node at (5.88,-0.4){$B$};
		
		
		\draw[fill][red](12.04,0) circle [radius=0.07];
		\draw(12.04,0) circle [radius=0.07];
		\node at (12.04,-0.4){$C$};
		
		
		\draw[fill][red](12,0.26) circle [radius=0.07];
		\draw(12,0.26) circle [radius=0.07];
		\node at (12.4,0.26){$D$};
		
		
		\draw[thick][blue](9,1)--(8.92,0.29)--(3,1);
		\draw[thick][blue](8.92,0.29)--(15,1);
		
	\end{tikzpicture}	
\end{center}
	Let the coordinate of the three origin points be $(4, 1)$, $(12, 1)$, 
	$(20, 1)$. And coordinate of $B$ be $(12-a, 1-b)$, here $a$ and $b$ 
	are positive. We can calculate the area of $ABCD$.
	
	\begin{equation}
		\begin{split} &\frac{4b - 2b^2 - 2a^2}{b} + \frac{1}{2k_{AB}}
		(\frac{8a - a^2 + 2b - b^2}{2b})^2 + \frac{1}{2k_{CD}}
		(\frac{ - 8a - a^2 + 2b - b^2}{2b})^2 
		\\
		\leq& \frac{4b - 2b^2 - 2a^2}{b} + \frac{b}{7}(\frac{8a - a^2 + 2b
		- b^2}{2b})^2 + \frac{b}{7}(\frac{ - 8a - a^2 + 2b - b^2}{2b})^2
		\\
		\leq&4 - 2b + \frac{b}{7}(\frac{2b + 2b}{2b})^2 +
		\frac{b}{7}(\frac{2b}{2b})^2\\
		\leq& 4
		\end{split}
	\end{equation}
		
	In this inequality we use the conclusion that $4a \leq b$ and 
	$b \leq 1$. The first inequality can be proved by geometric 
	relationship $|BC|<|CA_2|$(or the B will just change two domains), 
	from the fact that $B$ changes three domains we can also get the 
	inequality that $a^2+b^2\geq 8a-2b$.
		
	This finish the proof of theorem4.1
	\end{proof}
	
\begin{rem}
	In fact we haven't discussed all the cases, such as the case
	we discuss in the last of section3. But we know that the other cases 
	can be treated as a part of these three cases we have discussed.
\end{rem}
	
\begin{rem}
	It's clear that the length of small rectangles is not the best 
	constant here. The best constant of $length:width$ is around 2.
\end{rem}
	
Now it's time to go to our last main theorem, the main idea of proof 
is that when $n$ is big enough there must be a lot of quadrilateral 
and hexagonal domains. We need to find a contradiction in it.
	
At first we prove that there can't be quadrilaterals in most cases.
	
\begin{thm}
	If $A_1, A_2 ... A_n$ satisfy the condition and there is 
	a quadrilateral in $\iT_{Ai}$ , then the all the domains 
	must be like the case in theorem4.1 (a row of rectangles).
\end{thm}
	
\begin{proof}
	By lemma 5 the quadrilateral must be a parallelogram. 
	At first we prove that it must be a rectangle. For the origin 
	point of the quadrilateral $\iT_{A1}$, we notice that the symmetric 
	point of $A_1$ about the side of $\iT_{A1}$ which is not the bound 
	of T must be another origin point, we call it $A_2$, by lemma4 and 
	lemma5 we know that $\iT_{A_2}$ is also a quadrilateral. If $\iT_{A_1}$ 
	is not a rectangle, then two of its angle are obtuse angles, we select 
	the angle that one side is also the side of $\iT_{A_2}$ and label 
	it with letter $S$. Then we label other points like the picture.
		
\begin{center}
	\begin{tikzpicture}[scale=0.6]
		\draw[ultra thick][teal][fill=violet](0,0) circle [radius=0.5];
		
		\draw[fill][gray](8,0)--(7.5,1)--(-0.5,1)--(0,0)
		--(-0.5,-1)--(7.5,-1)--cycle;
		
		\draw[thick][blue](0,0)--(8,0)--(6,4)--(-2,4)
		--(0,0)--(-2,-4)--(6,-4)--(8,0) ;
		\draw[fill][red](0,-0) circle [radius=0.07];
		\draw(0,-0) circle [radius=0.07];
		\draw[fill][red](8,0) circle [radius=0.07];
		\draw(8,0) circle [radius=0.07];
		\draw[fill][red](6,4) circle [radius=0.07];
		\draw(6,4) circle [radius=0.07];
		\draw[fill][red](-2,4) circle [radius=0.07];
		\draw(-2,4) circle [radius=0.07];
		\draw[fill][red](-2,4) circle [radius=0.07];
		\draw(-2,4) circle [radius=0.07];
		\draw[fill][red](-2,4) circle [radius=0.07];
		\draw(6,4) circle [radius=0.07];
		\draw[fill][red](-2,4) circle [radius=0.07];
		\draw(6,4) circle [radius=0.07];
		
		\draw[fill][yellow](3,2) circle [radius=0.07];
		\draw(3,2) circle [radius=0.07];
		\node at (3.4,2){$A_2$};
		
		\draw[fill][yellow](3,-2) circle [radius=0.07];
		\draw(3,-2) circle [radius=0.07];
		\node at (3.4,-2){$A_1$};
		
		\draw[fill][green](3,0) circle [radius=0.07];
		\draw(3,0) circle [radius=0.07];
		\node at (3,-0.4){$B_1$};
		
		\draw[fill][green](2.7,2.3) circle [radius=0.07];
		\draw(2.7,2.3) circle [radius=0.07];
		\node at (2.4,2.6){$B_1$};
		
		\draw[fill][red](7.5,1) circle [radius=0.07];
		\draw(7.5,1) circle [radius=0.07];
		\node at (7.9,1){$I$};
		
		\draw[fill][red](7.5,-1) circle [radius=0.07];
		\draw(7.5,-1) circle [radius=0.07];
		\node at (7.9,-1){$J$};
		
		\draw[fill][red](-0.5,1) circle [radius=0.07];
		\draw(-0.5,1) circle [radius=0.07];
		\node at (-0.9,1){$H$};
		
		\draw[fill][red](-0.5,-1) circle [radius=0.07];
		\draw(-0.5,-1) circle [radius=0.07];
		\node at (-0.9,-1){$K$};	
		
		\draw[thick][blue](3,2)--(0,0);
		\node at (8.4,0){$U$};
		\node at (0.2,-0.4){$S$};
	\end{tikzpicture}	
\end{center}
		
	Call intersection of $US$ and $A_1A_2$ $B$. Here $HS = \frac{|ST|}{4}$ 
	then $B_1$ changes the blue area into green. Consider the disk $D(S,r)$ 
	whose center is at $S$ with radius of $r = \frac{|SA_2-SB1|}{2}$. 
	The (purple) area of $D(S,r) \cap (\iT_{A1} \cup \iT_{A2})^c$ 
	is also turned to green by $B$ because for any point $T$ in $D(S,r)$ and
	another origin point $A$ we have that $|AT|\leq
	|AS|-\frac{|SA_2 - SB_1|}{2} \leq |SA_1| - \frac{|SA_2 - SB_1|}{2} 
	\leq |SB_1|$. So $B$ changes more than half (both gray and purple 
	part) of the quadrilateral. Then we use the trick we have used many 
	times to select other $B_i$ Then we get the contradiction.
		
	If $\iT_{A_1}$ is a rectangle, using the trick we used in section3 
	we know that all domains are rectangles and they make up a table. 
	Using the proof in section3 again we can know that the table only 
	have one row(or one column) thus we get the conclusion.
	\end{proof}
	
So only if $T$ is a rectangle then there could be quadrilateral 
in $\iT_{A_1}, \iT_{A_2} ... \iT_{A_n}$. But when $T$ is a rectangle 
and $n$ becomes big enough it becomes a case in section3 which turns 
out to be bad. 
	
The we prove that there would be a lot of hexagons. Euler's Formula 
: $V-E+F=1$ to prove this thing.
	
\begin{thm}
	We divide a polygon $T$ into small polygons with even edges. 
	If there aren't quadrilaterals, the numbers of `points on edges', 
	`the polygons which has more than eight edges', `the points in $T$ 
	who is the endpoint of more than four lines' can be controlled 
	by a constant $c$ only depended on $T$.
\end{thm}
	
\begin{rem}
	By lemma 6 $T_i$ must be polygons (we even know it's a centrally 
	symmetric polygon), and all the numbers of edges of polygons must
	be even. So we naturally consider theorem4.4. 
\end{rem}
	
\begin{proof}
	Let $n$ be the number of edges of $T$ , $p$ be the number of points
	on edges, $r$ be the number of points in $T$, $h$ be the number of 
	hexagons, $t$ be the number of polygons which has more than seven 
	edges, $l$ be the number of lines in $T$, $f$ be the the number of 
	the points in $T$ who is the endpoint of more than four lines.
		
	Then we have
		
\[
	\begin{cases}
		n - 2 + p + 2r \geq 4h + 6t\\
		h + t + p + r - l - p = 1 \\
		l \geq \frac{p + 3r + f}{2}
	\end{cases}
\]
		
	The first inequality is gotten by calculating degrees, 
	the second one comes from  Euler's Formula and the third one 
	is because every point inside $T$ has at least 3 lines from it 
	and every point on edges has at least one.
		
	solve the inequalities system we get that $n - 6 \geq p + 2t + 2f$ and
	we get the conclusion.
\end{proof}	
	
This theorem gives out a way leading to our goal. It tells us 
that there are enough hexagons which are not on the edges. It's 
interesting that a hexagon will not cause contradiction directly 
but when a hexagon is surrounded by others hexagons we can get 
contradiction.
	
	
By theorem4.4 we know that if $n$ is big enough we can select a 
lot of adjacent hexagons, every of which is surrounded by other 
six hexagons and each vertex of the original hexagon is just the
vertex of three domains.
	
Consider such a hexagon:
	
\begin{thm}
	Such a hexagon must be a diagonal parallel hexagon.
\end{thm}	 
	
We give out the definition of parallel hexagon then.

\begin{defn}
	A diagonal parallel hexagon is a hexagon that each main 
	diagonal is parallel to the opposite.
\end{defn}
	

\begin{center}
	\begin{tikzpicture}[scale=0.6]
	
		\draw[thick][blue](4,0)--(1,5)--(-3,5)--(-4,0)--(-1,-5)--(3,-5)--cycle ;
		\draw[fill][red](4,-0) circle [radius=0.07];
		\draw(4,-0) circle [radius=0.07];
		\draw[fill][red](1,5) circle [radius=0.07];
		\draw(1,5) circle [radius=0.07];
		\draw[fill][red](-3,5) circle [radius=0.07];
		\draw(-3,5) circle [radius=0.07];
		\draw[fill][red](-4,0) circle [radius=0.07];
		\draw(-4,0) circle [radius=0.07];
		\draw[fill][red](-1,-5) circle [radius=0.07];
		\draw(-1,-5) circle [radius=0.07];
		\draw[fill][red](3,-5) circle [radius=0.07];
		\draw(3,-5) circle [radius=0.07];
		\draw[fill][red](0,0) circle [radius=0.07];
		\draw(0,0) circle [radius=0.07];
		\draw[thick][blue](4,0)--(-4,0);
		\draw[thick][blue](1,5)--(-1,-5);
		\draw[thick][blue](-3,5)--(3,-5);
	
		\node at (-3.3,5.3){$A$};
		\node at (1.3,5.3){$B$};
		\node at (4.4,0){$C$};
		\node at (3.3,-5.3){$D$};
		\node at (-1.3,-5.3){$E$};
		\node at (-4.4,0){$F$};
	
		\node at (5,5){$AB\parallel FC\parallel ED$};
		\node at (5,4){$AF\parallel BE\parallel CD$};
		\node at (5,3){$FE\parallel AD\parallel BC$};	
	\end{tikzpicture}	
\end{center}
	
Now we give out the proof, the idea is based on translating $B$ 
a little distance from $A$, because the calculating of areas is a 
bit complicated, we use different colors to sign different domain 
for convenience. We clarify our notation, let $\T$ be any function such that $\lim_{\epsilon \to 0}\frac{\T(\epsilon)}{\epsilon} = 1$, and
$o(\epsilon)$ be higher order infinitesimal of $\epsilon$.
	
	
\begin{center}
	\begin{tikzpicture}[scale=0.3]
		
		\draw[blue][fill=yellow](-13.97,6.55)--(6.82,7.42)--(6.03,5.95)--(-14.76,5.08)--cycle;
		\draw[blue][fill=orange](-14.76,5.08)--(6.03,5.95)--(1.06,-3.28)--(-11.62,-3.81)--(-15.19,4.29)--cycle;
		\draw[blue][fill=cyan](-11.62,-3.81)--(1.06,-3.28)--(1.69,-4.69)--(-12.72,-5.29)--cycle;
		\draw[blue][fill=gray](-15.28,4.26)--(-15.19,4.29)--(-11.62,-3.81)--(-11.98,-4.29)--(-14.23,0.82)--cycle;
		\draw[blue][fill=violet](-14.23,0.82)--(-11.98,-4.29)--(-12.47,-4.95)--cycle;
		\draw[blue][fill=black](-15.28,4.26)--(-14.23,0.82)--(-15.71,4.17)--cycle;
		
		
		\draw[fill][yellow](-25.01,-2.47) circle [radius=0.06];
		\draw(-25.01,-2.47) circle [radius=0.06];
		\node at (-25.01,-2){$A_2$}; 
		\draw[fill][red](-15.71,4.17) circle [radius=0.06];
		\draw(-15.71,4.17) circle [radius=0.06];
		\node at (-16.21,4.17){$M$};
		\draw[fill][red](-15.28,4.26) circle [radius=0.06];
		\draw(-15.28,4.26) circle [radius=0.06];	
		\node at (-15.58,4.56){$Q$};
		\draw[fill][red](-13.97,6.55) circle [radius=0.06];
		\draw(-13.97,6.55) circle [radius=0.06];
		\node at (-14.37,6.65){$G$};
		\draw[fill][red](-14.76,5.08) circle [radius=0.06];
		\draw(-14.76,5.08) circle [radius=0.06];
		\node at (-15,5.3){$I$};
		\draw[fill][red](-15.19,4.29) circle [radius=0.06];
		\draw(-15.19,4.29) circle [radius=0.06];
		\node at (-14.7,4.29){$U$};
		\draw[fill][red](-14.29,2.26) circle [radius=0.06];
		\draw(-14.29,2.26) circle [radius=0.06];
		\node at (-13.89,2.3){$S$};
		\draw[fill][red](-14.23,0.82) circle [radius=0.06];
		\draw(-14.23,0.82) circle [radius=0.06];
		\node at (-14.73,0.7){$O$};
		\draw[fill][red](-13.75,1.03) circle [radius=0.06];
		\draw(-13.75,1.03) circle [radius=0.06];
		\node at (-13.3,1){$R$};
		\draw[fill][red](-12.72,-5.29) circle [radius=0.06];
		\draw(-12.72,-5.29) circle [radius=0.06];
		\node at (-12.92,-5.593){$K$};
		\draw[fill][red](-12.47,-4.95) circle [radius=0.06];
		\draw(-12.47,-4.95) circle [radius=0.06];
		\node at (-12.17,-5.25){$P$};
		\draw[fill][red](-11.98,-4.29) circle [radius=0.06];
		\draw(-11.98,-4.29) circle [radius=0.06];
		\node at ((-11.68,-4.59){$N$};
		\draw[fill][red](-11.62,-3.81) circle [radius=0.06];
		\draw(-11.62,-3.81) circle [radius=0.06];
		\node at (-11.32,-4.11){$T$};
		\draw[fill][yellow](-3.57,6.98) circle [radius=0.06];
		\draw(-3.57,6.98) circle [radius=0.06];
		\node at (-3.57,7.48){$A_1$};
		\draw[fill][red](-3.51,5.55) circle [radius=0.06];
		\draw(-3.51,5.55) circle [radius=0.06];
		\node at (-3.01,5.55){$V$};
		\draw[fill][green](-3.45,4.12) circle [radius=0.06];
		\draw(-3.45,4.12) circle [radius=0.06];
		\node at (-3,4.12){$B$};
		\draw[fill][red](-3.14,-3.46) circle [radius=0.06];
		\draw(-3.14,-3.46) circle [radius=0.06];
		\node at (-2.64,-3.46){$W$};
		\draw[fill][red](-3.08,-4.89) circle [radius=0.06];
		\draw(-3.08,-4.89) circle [radius=0.06];
		\node at (-2.58,-4.89){$X$};
		\draw[fill][red](1.69,-4.69) circle [radius=0.06];
		\draw(1.69,-4.69) circle [radius=0.06];
		\node at (2.19,-4.69){$L$};
		\draw[fill][red](1.06,-3.28) circle [radius=0.06];
		\draw(3.3,-5.3) circle [radius=0.06];
		\node at (1.56,-3.2){$Y$};
		\draw[fill][red](6.03,5.95) circle [radius=0.06];
		\draw(6.03,5.95) circle [radius=0.06];
		\node at (6.53,5.95){$J$};
		\draw[fill][red](6.82,7.42) circle [radius=0.06];
		\draw(6.82,7.42) circle [radius=0.06];
		\node at (7.32,7.42){$H$};
		\draw[fill][red](8.04,9.68) circle [radius=0.06];
		\draw(8.04,9.68) circle [radius=0.06];
		\node at (8.54,9.68){$Z$};
		\draw[fill][yellow](12.18,-1.5) circle [radius=0.06];
		\draw(12.18,-1.5) circle [radius=0.06];
		\node at (12.68,-1.5){$A_4$};
		\draw[fill][red](-8.2,17.25) circle [radius=0.06];
		\draw(-8.2,17.25)circle [radius=0.06];
		\draw[fill][red](4.47,17.78) circle [radius=0.06];
		\draw(4.47,17.78) circle [radius=0.06];
		\draw[fill][yellow](-2.7,-13.9) circle [radius=0.06];
		\draw(-2.7,-13.9) circle [radius=0.06];	
		\node at (-2.7,-14.4){$A_3$};
		
		
		
		
		\draw[blue](-8.2,17.25)--(4.47,17.78)--(8.04,9.68)--(1.06,-3.28)--(-11.62,-3.81)--(-15.19,4.29)--cycle;
		\draw[blue](-13.97,6.55)--(6.82,7.42);
		\draw[blue](-14.76,5.08)--(6.03,5.95);
		\draw[blue](13.14,10.87)--(-24.24,2.18);
		\draw[blue](-3.57,6.98)--(-25.01,-2.47);
		\draw[blue](-3.57,6.98)--(-14.96,-8.3);
		\draw[blue](-3.57,6.98)--(-2.7,-13.9);
		\draw[blue](-3.57,6.98)--(3.48,-8.65);
		\draw[blue](-3.57,6.98)--(12.18,-1.5);
		\draw[blue](-14.29,2.26)--(-14.23,0.82)--(-13.75,1.03);
		
		
		\node at (0,6.3){yellow};
		\node at (-6,0){orange};
		\node at (-6,-4.3){blue};
		\node at (-12.8,-0.5){gray};
		\node at (-13.5,-2){purple};
		\node at (-16,2.5){black};
	\end{tikzpicture}
\end{center}
	
\begin{proof}
	At first we tell the construction of points in the picture.
	We translate $B$ a infinitesimal $\epsilon$ from $A$ on the line $AW$
	which is the vertical line from $A$ to $ST$. $A_2$, $A_3$ and $A_4$ are
	the symmetric point of $A$ about $UT$, $TS$ and $SR$. $IJ$, $MN$ and 
	$KL$ are perpendicular bisectors of$A_1B$, $A_1A_2$ and $BA_3$. 
	The intersection of $BA_2$ and $MN$ are $O$, $PQ$ is the lines which 
	vertical to $BA_2$ and passed $O$.
		
	$B$ changes the orange, blue, gray, purple and the symmetrical part
	of gray and purple(call it $D$) into its own domain. 
	
	What we want to prove now is that $S_{blue} + S_{gray} +
	S_{purple} + S_{D} - S_{yellow} = c\T(\epsilon) $ 
	Here $c\geq0$ and is $0$ iff the
	hexagon is a parallel hexagon.
		
	Notice that $S_{yellow} = \frac{|GH|}{2}\T(\epsilon)$ $S_{blue} = \frac{|TS|}{2}\T(\epsilon)$ because we have that that $|A_1V| = |WX| = \frac{\epsilon}{2}$
		
	Now calculate $S_{gray}+S_{purpul}$. Let $f_{UT} = |TS| - |SU|$ 
	and $\theta$ be $\pi - \angle UTS$
		
	At first we notice that 
		
	\begin{equation}
		\begin{split}
		&S_{gray} + S_{black}\\
		=&|UT|(|OR| + o(|OR))\\
		=&|UT|(|SO||\cos \angle SOR + o(SO)|\\
		=&|UT|\cos\theta \frac{\T(\epsilon)}{2}
		\end{split}
	\end{equation}
		
		and	
		
	\begin{equation}
		\begin{split}
		&S_{purple} - S_{gray}\\
		=&\T(\frac{1}{2}|NO^2 - MO^2||\sin\angle PON|)\\
		=&\T(\frac{1}{2}|TS^2 - SU^2|\sin \angle A_1A_2B)\\
		=&\T(d_{UT}|UT|\epsilon\frac{\sin \angle A_2A_1B}{A_1A_2})
		\end{split}
	\end{equation}
		
	So $B$ changes the domain of area.
	($\theta'$ means the $\pi - \angle RST$)
	\begin{equation}
		\begin{split}
		&\frac{S_{hexagon}}{2} + |UT|\cos\theta \frac{\T(\epsilon)}{2} + \T(d_{UT}|UT|\epsilon\frac{\sin \angle A_2A_1B}{A_1A_2} + 
		|RS|\cos\theta' \frac{\T(\epsilon)}{2}\\
		-& \T(d_{SR}|SR|\epsilon\frac{\sin \angle A_4A_1B}{A_4A_2}) - \frac{|GH|}{2}\T(\epsilon) + 
		\frac{|TS|}{2}\T(\epsilon)
		\\
		= & \frac{S_{hexagon}}{2} + \T((\epsilon)
		(\frac{1}{2}|UT|\cos\theta )
		+\frac{1}{2}|RS|\cos\theta' +(f_{UT}|UT|\frac{\sin 
		\angle A_2A_1B}{A_1A_2})\\
		- &(f_{SR}|SR|\frac{\sin \angle A_4A_1B}{A_4A_2}) + 
		\frac{|TS|}{2} - \frac{|GH|}{2})
		\end{split}
	\end{equation}
		
	It's the case that $B$ moves a little distance $\epsilon$ 
	on the vertical line from $A$ to $ST$, we let this $B$ be $B_1$. 
	Similarly if $B$ moves $\epsilon$ on the perpendicular lines to 
	other edges and we calculate the average of them we get that $B$ 
	changes the domain of area 
	
	\[ \frac{S_{hexagon}}{2} + \frac{1}{6} \T(\epsilon)(\frac{1}{2}\sum _{cyc}(|UT|\cos\theta+|RS|\cos\theta'+|TS|-|GH|)) \]
		
	Here cyc means sum interchangeably.
		
	We notice that$|UT|\cos\theta + |RS|\cos\theta' + |TS|\geq|GH|$ 
	and the equation established iff $GH \backslash \backslash TS$. 
	We can prove this by geometric relationship. There are three cases.
	First of two are:
		
\begin{center}
	\begin{tikzpicture}[scale=0.6]
		
		\draw[thick][blue](3,-1)--(1,-2)--(-2,-2)--(-3,1)
		--(-1,2)--(2,2)--cycle ;
		\draw[fill][red](3,-1) circle [radius=0.07];
		\draw(3,-1)circle [radius=0.07];
		\node at (3.4,-1){$R$};
		\draw[fill][red](1,-2) circle [radius=0.07];
		\draw(1,-2) circle [radius=0.07];
		\node at (1,-2.4){$S$};
		\draw[fill][red](-2,-2) circle [radius=0.07];
		\draw(-2,-2) circle [radius=0.07];
		\node at (-2,-2.4){$T$};
		\draw[fill][red](-3,1) circle [radius=0.07];
		\draw(-3,1) circle [radius=0.07];
		\node at (-3.4,1){$U$};
		\draw[fill][red](-1,2) circle [radius=0.07];
		\draw(-1,2) circle [radius=0.07];
		\node at (-1,2.4){$P$};
		\draw[fill][red](2,2) circle [radius=0.07];
		\draw(2,2) circle [radius=0.07];
		\node at (2,2.4){$Q$};
		\draw[fill][red](0,0) circle [radius=0.07];
		\draw(0,0) circle [radius=0.07];
		\node at (0.4,-0.2){$A_1$};
		\draw[fill][red](2.66,0) circle [radius=0.07];
		\draw(2.66,0) circle [radius=0.07];
		\node at (2.16,0.3){$H$};
		\draw[fill][red](-2.66,0) circle [radius=0.07];
		\draw(-2.66,0) circle [radius=0.07];
		\node at (-2.56,0.4){$G$};
		\draw[fill][red](3,-2) circle [radius=0.07];
		\draw(3,-2) circle [radius=0.07];
		\node at (3,-2.4){$K$};
		\draw[fill][red](-3,-2) circle [radius=0.07];
		\draw(-3,-2) circle [radius=0.07];
		\node at (-3,-2.4){$L$};
			
		\draw[thick][blue](2,2)--(-2,-2);
		\draw[thick][blue](1,-2)--(-1,2);
		\draw[thick][blue](2.66,0)--(-2.66,0);
		\draw[dashed, thick][blue](3,-1)--(3,-2);
		\draw[dashed, thick][blue](-3,1)--(-3,-2);
		\draw[thick][blue](-3,-2)--(3,-2);
		
		\draw[thick][blue](10,1)--(11,-2)--(7,-2)--(6,-1)
		--(5,2)--(9,2)--cycle ;
		\draw[fill][red](10,1) circle [radius=0.07];
		\draw(10,1)circle [radius=0.07];
		\node at (10.4,1){$R$};
		\draw[fill][red](11,-2) circle [radius=0.07];
		\draw(11,-2)circle [radius=0.07];
		\node at (11,-2.4){$S$};
		\draw[fill][red](7,-2) circle [radius=0.07];
		\draw(7,-2) circle [radius=0.07];
		\node at (7,-2.4){$T$};
		\draw[fill][red](6,-1) circle [radius=0.07];
		\draw(6,-1) circle [radius=0.07];
		\node at (5.6,-1){$U$};
		\draw[fill][red](5,2) circle [radius=0.07];
		\draw(5,2) circle [radius=0.07];
		\node at (5,2.4){$P$};
		\draw[fill][red](9,2) circle [radius=0.07];
		\draw(9,2) circle [radius=0.07];
		\node at (9,2.4){$Q$};
		\draw[fill][red](8,0) circle [radius=0.07];
		\draw(8,0) circle [radius=0.07];
		\node at (8.4,-0.2){$A_1$};
		\draw[fill][red](10.33,0) circle [radius=0.07];
		\draw(10.33,0) circle [radius=0.07];
		\node at (10.66,0.3){$H$};
		\draw[fill][red](5.66,0) circle [radius=0.07];
		\draw(5.66,0) circle [radius=0.07];
		\node at (5.9,0.4){$G$};
		\draw[fill][red](10,-2) circle [radius=0.07];
		\draw(10,-2) circle [radius=0.07];
		\node at (10,-2.4){$K$};
		\draw[fill][red](6,-2) circle [radius=0.07];
		\draw(6,-2) circle [radius=0.07];
		\node at (6,-2.4){$L$};
			
		\draw[thick][blue](5.66,0)--(10.33,0);
		\draw[thick][blue](9,2)--(7,-2);
		\draw[thick][blue](11,-2)--(5,2);
		\draw[dashed, thick][blue](6,-1)--(6,-2);
		\draw[dashed, thick][blue](10,1)--(10,-2);
		\draw[thick][blue](5.66,-2)--(11,-2);
		\draw[dashed, thick][blue](5.66,0)--(5.66,-2);
		\draw[dashed, thick][blue](10.33,0)--(10.33,-2);
	\end{tikzpicture}	
\end{center}
		
	For the first case we know that 
	\[ |UT|\cos\theta+|RS|\cos\theta' + |TS| \geq |JT| + |TS| + |SK|
	\geq|TS| \]
		
	For the second case we know that
	\[ LHS\geq|JT| + |TS| + |SK| = |JW| + |WS|\geq |JW| + 
	|WK| = |TS| \]
		
	So $|UT|\cos\theta+|RS|\cos\theta'+|TS|-|GH|\geq 0$.
	
	If the equation didn't establish, then we can always find a $B$ 
	in the hexagon and $B$ changes more than half area of the hexagon 
	into its own domain.And then use the tricks we have used many 
	times to select other $B_i$. 
	So the equation all established.
	
	There is one case left: when $UT\perp TS$, the projection of $GH$ 
	is the same as $UR$. In that case, we can change the direction of $B$ 
	moved from $A$ to get contradiction. 
\end{proof}
	
When we have such a strong conclusion, we can select three adjacent 
hexagons and calculate the angles to get further conclusion.
	
\begin{thm}
	If 3 hexagons are adjacent, they are all regular hexagons.
\end{thm}
	
\begin{center}
	\begin{tikzpicture}[scale=0.6]
	
	
		\draw[thick][blue](0,8)--(3.5,6)--(3.5,2)--(7,0)--(7,-4
		)--(3.5,-6)--(0,-4)--(-3.5,-6)--(-7,-4)--(-7,0)--(-3.5,2)
		--(-3.5,6)--cycle ;
		\draw[thick][blue](0,4)--(3.5,2)--(3.5,-2)--(0,-4)--
		(-3.5,-2)--(-3.5,2)--cycle ;
		\draw[thick][blue](0,8)--(0,4)--(0,0)--(0,-4) ;	
		\draw[thick][blue](-7,-4)--(-3.5,-2)--(0,0)--(3.5,2);
		\draw[thick][blue](7,-4)--(3.5,-2)--(0,0)--(-3.5,2);
		\draw[thick][blue](-3.5,6)--(0,4)--(3.5,6);
		\draw[thick][blue](-7,0)--(-3.5,-2)--(-3.5,-6);
		\draw[thick][blue](7,0)--(3.5,-2)--(3.5,-6);
	
		\draw[fill][red](0,-0) circle [radius=0.07];
		\draw(0,-0) circle [radius=0.07];
		\node at (0.2,0.4){$T$};
		\draw[fill][red](0,4) circle [radius=0.07];
		\draw(0,4) circle [radius=0.07];
		\node at (0.2,4.4){$A_1$};
		\draw[fill][red](-3.5,2) circle [radius=0.07];
		\draw(-3.5,2) circle [radius=0.07];
		\node at (-3.8,2.3){$C$};
		\draw[fill][red](-3.5,-2) circle [radius=0.07];
		\draw(-3.5,-2) circle [radius=0.07];
		\node at (-3.8,-2.3){$A_2$};
		\draw[fill][red](0,-4) circle [radius=0.07];
		\draw(0,-4) circle [radius=0.07];
		\node at (0,-4.4){$B$};
		\draw[fill][red](3.5,-2) circle [radius=0.07];
		\draw(3.5,-2) circle [radius=0.07];
		\node at (3.8,-2.3){$A_3$};
		\draw[fill][red](3.5,2) circle [radius=0.07];
		\draw(3.5,2) circle [radius=0.07];
		\node at (3.8,2.3){$D$};
	\end{tikzpicture}	
\end{center}
	
	
\begin{proof}
	We just need to calculate angles. In the equation we use the facts 
	that $A_1BTC$, $A_3DTB$ and $A_2DTC$ are parallelograms, 
	$A_1$ and $A_2$ and $A_3$ are symmetric points about 
	$TC$, $TD$, and $TB$.
	
	\begin{equation}
		\begin{split}
		&\angle TA_3B = \angle DTA_3 = \pi - \angle CTB = \angle TBA_1
		\end{split} 
	\end{equation}
		
	Similarly we have that $\angle TA_1B = \angle TBA_3$. Then we 
	get that $\angle BTA_1 = \angle BTA_3$ So we can get that 
	$\angle BTA_1 = \angle BTA_3 = \angle DTA_3 = \angle DTA_2 = 
	\angle CTA_2 = \angle CTA_1$ We can get that $\angle BTA_3 = \angle TBA_3=\frac{\pi}{3}$ then we can get that $\iT_{A_3}$ 
	is a regular hexagon. Similarly, $\iT_{A_2}$ and $\iT_{A_3}$ 
	are regular hexagons.
\end{proof}
	
Consider four adjacent regular hexagons, we can show that it's 
not good by date in this picture. 
	
\begin{center}
	\begin{tikzpicture}[scale=0.6]
		
		\draw[blue][fill=lightgray](4.5,7.8)--(3,10.4)--(0,10.4)--
		(-1.5,7.8)--(0,5.2)--(3,5.2)--cycle;
	
		\draw[blue][fill=cyan](-0.52,4.3) --(3.34,4.6)--(3.8,6.58)
		--(-1.37,7.57)--cycle;
	
	
		\draw[thick][blue](0,0)--(3,0)--(4.5,2.6)--(7.5,2.6)--(9,5.2)
		--(7.5,7.8)--(4.5,7.8)--(3,10.4)--(0,10.4)--(-1.5,7.8)
		--(-4.5,7.8)--(-6,5.2)--(-4.5,2.6)--(-1.5,2.6)--cycle ;
		\draw[thick][blue](4.5,2.6)--(3,5.2)--(0,5.2)--(-1.5,7.8) ;
		\draw[thick][blue](-1.5,2.6)--(0,5.2)--(3,5.2)--(4.5,7.8) ;
	
		\draw[fill][yellow](1.5,2.6) circle [radius=0.07];
		\draw(1.5,2.6) circle [radius=0.07];
		\node at (1.6,2.3){$A_1$};
		\draw[fill][green](1.212,6.363) circle [radius=0.07];
		\draw(1.212,6.363) circle [radius=0.07];
		\node at (1.512,6.063){$B$};
		\draw[fill][yellow](-3,5.2) circle [radius=0.07];
		\draw(-3,5.2) circle [radius=0.07];
		\node at (-3.3,4.9){$A_2$};
		\draw[fill][yellow](6,5.2) circle [radius=0.07];
		\draw(6,5.2) circle [radius=0.07];
		\node at (6.4,5.2){$A_3$};
		\draw[fill][yellow](1.5,7.8) circle [radius=0.07];
		\draw(1.5,7.8) circle [radius=0.07];
		\node at (1.5,8.2){$A_4$};
	
		\draw[fill][red](-0.52,4.3) circle [radius=0.07];
		\draw(-0.52,4.3) circle [radius=0.07];
		\node at (-0.9,4.3){$O$};
		\draw[fill][red](3.34,4.6) circle [radius=0.07];
		\draw(3.34,4.6) circle [radius=0.07];
		\node at (3.74,4.6){$P$};
		\draw[fill][red](3.8,6.58) circle [radius=0.07];
		\draw(3.8,6.58) circle [radius=0.07];
		\node at (4.1,6.22){$Q$};
		\draw[fill][red](-1.37,7.57) circle [radius=0.07];
		\draw(-1.37,7.57) circle [radius=0.07];
		\node at (-1.7,7.3){$R$};
	
		\node at (-5,10.22){$A_1$:(0.500,0.866)};
		\node at (-5,9.52){$B$:(0.404,2.121)};
		\node at (-5,8.82){$S_{hexagon}$=2.598};
		\node at (-5,8.02){$S_{OPQR}$=1.309};
		\node at (-5,7.12){$\frac{S_{OPQR}}{S_{hexagon}}$=0.504};
	\end{tikzpicture}	
\end{center}

We can choose a $B:(0.404,2.121)$ such that it changes more than half a 
hexagon, so this situation is also not good. \newline

We have finished all the theorems and given out some conclusions about 
this question. But it's a pity that we haven't solved the problem completely, 
we will give out some ideas about the question in next section.

\section{折纸正 \texorpdfstring{$N$}{N} 边形}

Firstly we recall the conclusions we have gotten: for initial question,
$(square, n)$ is good if and only if $n=1$. For any $n$ there exists 
$T$ such that $(T,n)$ is good. For any $T$, when $n$ is large enough 
$(T,n)$ are always bad.
 	
	
A fact is that it's hard to construct another good $(T, n)$ when $n$ 
is more than 2 and $T$ is not a rectangle(except those in 
theorem4.1). The difficulty is that we can't divide a convex polygon 
into centrally symmetric polygons which has more than five edges easily. 
Moreover we need to use area-equivalent polygons, even 
diagonal parallel(polygons with 4n+2 edges) or every edge has two edges
vertical to it(polygons with 4n edges) when a polygon is surrounded by 
other polygons. It's hard to achieve all such things. It can be conjectured 
there are no other good pairs except the cases in theorem4.1. 
But in other hand, as we don't have the condition that $n$ is big enough, 
it's not easy to consider this problem locally. An available  idea 
to prove that a polygon can't be divided into such polygons. We may prove
this by considering the lines between origin points and using the length 
and angles to calculating areas. It's also possible to calculating 
integral in some certain area locally around an origin point to prove 
the existence of $B_i$. But they all need nontrivial calculation.
	
As a generalization of the question, we can change $T$ into Riemannian 
manifolds with constant curvature, for example, $S^2$ with canonical metric.
It's easy to see that $(S^2, 1)$ and $(S^2, 2)$ are good but $(S^2, 3)$ not.
However, $(S^2,4)$ is good again. The question on Riemannian manifolds 
with constant curvature may be much more complicated. It may be connected 
with isometric group of them.
The condition of constant curvature is necessary here since vertical bisector
is well defined in this case.
	
At the end of this article, we figure out what we have proved in fact, 
in such a game the second person  has advantage most of time. We can 
easily get the corollary that even if the number of points two people 
choose are different, the average of area of the second is almost always 
bigger than the first. It can be seen that in such situations it's always 
important to know the other people's position which will bring advantage 
to you.

\printbibliography

\end{document}