\documentclass[twoside]{article}

\input{../common/preamble}
\newcommand{\Hom}{\operatorname{Hom}}
\newcommand{\Sp}{\mathbf{sp}}
\newcommand{\colim}{\mathrm{colim}}
\newcommand{\im}{\operatorname{im}}
\newcommand{\vp}{\varphi}
\newcommand{\supp}{\mathrm{supp}}
\newcommand{\tr}{\operatorname{tr}}
\newcommand{\YP}{]Y[_\CP}
\newcommand{\XP}{]X[_\CP}
\newcommand{\ZP}{]Z[_\CP}
\newcommand{\HSn}{H^n_{cris}(X/S)}
\newcommand{\Hkn}{H^n_{cris}(X/W(k))}
\newcommand{\HSd}{H^\cdot_{cris}(X/S)}
\newcommand{\Hkd}{H^\cdot_{cris}(X/W(k))}
\newcommand{\HCd}{H^\cdot_{conv}(X)}
\newcommand{\HCn}{H^n_{conv}(X)}
\newcommand{\HRd}{H^\cdot_{rig}(X)}
\newcommand{\HRn}{H^n_{rig}(X)}
\newcommand{\DRX}{\Omega_{\XP/K}^\cdot}
\newcommand{\DRY}{\Omega_{\YP/K}^\cdot}
\newcommand{\Spec}{\mathrm{Spec}}
\newcommand{\Spf}{\mathrm{Spf}}
\newcommand{\Spp}{\mathrm{Sp}}
\newcommand{\Frac}{\mathrm{Frac}}
\newcommand{\DRXn}{\Omega_{\XP/K}^n}
\newcommand{\BA}{{\mathbb {A}}}
\newcommand{\BB}{{\mathbb {B}}}
\newcommand{\BC}{{\mathbb {C}}} 
\newcommand{\BD}{{\mathbb {D}}}
\newcommand{\BE}{{\mathbb {E}}} 
\newcommand{\BF}{{\mathbb {F}}}
\newcommand{\BG}{{\mathbb {G}}} 
\newcommand{\BH}{{\mathbb {H}}}
\newcommand{\BI}{{\mathbb {I}}} 
\newcommand{\BJ}{{\mathbb {J}}}
\newcommand{\BK}{{\mathbb {K}}} 
\newcommand{\BL}{{\mathbb {L}}}
\newcommand{\BM}{{\mathbb {M}}} 
\newcommand{\BN}{{\mathbb {N}}}
\newcommand{\BO}{{\mathbb {O}}} 
\newcommand{\BP}{{\mathbb {P}}}
\newcommand{\BQ}{{\mathbb {Q}}} 
\newcommand{\BR}{{\mathbb {R}}}
\newcommand{\BS}{{\mathbb {S}}} 
\newcommand{\BT}{{\mathbb {T}}}
\newcommand{\BU}{{\mathbb {U}}} 
\newcommand{\BV}{{\mathbb {V}}}
\newcommand{\BW}{{\mathbb {W}}} 
\newcommand{\BX}{{\mathbb {X}}}
\newcommand{\BY}{{\mathbb {Y}}} 
\newcommand{\BZ}{{\mathbb {Z}}}
\newcommand{\CA}{{\mathcal {A}}} 
\newcommand{\CB}{{\mathcal {B}}}
\newcommand{\CC}{{\mathcal {C}}} 
\newcommand{\CD}{{\mathcal {D}}}
\newcommand{\CE}{{\mathcal {E}}} 
\newcommand{\CF}{{\mathcal {F}}}
\newcommand{\CG}{{\mathcal {G}}} 
\newcommand{\CH}{{\mathcal {H}}}
\newcommand{\CI}{{\mathcal {I}}} 
\newcommand{\CJ}{{\mathcal {J}}}
\newcommand{\CK}{{\mathcal {K}}} 
\newcommand{\CL}{{\mathcal {L}}}
\newcommand{\CM}{{\mathcal {M}}} 
\newcommand{\CN}{{\mathcal {N}}}
\newcommand{\CO}{{\mathcal {O}}} 
\newcommand{\CP}{{\mathcal {P}}}
\newcommand{\CQ}{{\mathcal {Q}}} 
\newcommand{\CR}{{\mathcal {R}}}
\newcommand{\CS}{{\mathcal {S}}} 
\newcommand{\CT}{{\mathcal {T}}}
\newcommand{\CU}{{\mathcal {U}}} 
\newcommand{\CV}{{\mathcal {V}}}
\newcommand{\CW}{{\mathcal {W}}} 
\newcommand{\CX}{{\mathcal {X}}}
\newcommand{\CY}{{\mathcal {Y}}} 
\newcommand{\CZ}{{\mathcal {Z}}}

\addbibresource{kfh.bib}

\begin{document}

\title{An Introduction to Factorization Algebras and Factorization Homology}
\author{Kong Fanhao\footnote{孔繁浩,清华大学数学系数 72 班.}}

\begin{abstract}
	We introduce the theory of factorization algebras and factorization homology.
    They are based on the theory on higher operads and algebras over them.
    We also give two applications of factorization algebras and factorization homology:
    the Deligne conjecture and the Bar constructions.
\end{abstract}

\tableofcontents

\clearpage

It has been many years since people try to study quantum field theories
using the tool of mathematics. People introduce a number of physical ideas into mathematics,
including fields, phase spaces, quantization, Feynman diagrams, observables,
etc. Roughly speaking, we know that observables on small sets induce
observables on large sets. The study of this property of observables
yields a new structure, called the ``factorization algebra''. Since the
start of the 21st century, where the notion of factorization algebras was
introduced, it has been studied by a number of mathematicians, and many
notable results have been developed.

On the other hand, there exists another mathematical structure which is 
closely related to the factorization algebra, called ``factorization
homology''. It is a structure to describe homology theories specific
to manifolds of a specific dimension, with some specific structures. 
The specific structures can be smooth structures, framings, or simply
no structure at all. They are introduced by axioms similar to the Eilenberg-Steenrod
axioms for homology theory for spaces, including the monoidal axiom and the
excision axiom.

In order to examine factorization homology and factorization algebras, 
a number of preliminaries are required; they include the theory of operads
and $E_n$-algebras. The theory of operads is used to study the structures of
multimorphisms and the relations that the multimorphisms must satisfy,
while the $E_n$-algebras are introduced to study iterated loop spaces and
configuration spaces, and have been heavily studied in algebraic topology
ever since the seventies. 

It has been proved that factorization homology and factorization algebras
are highly related. This leads to more studies on $E_n$-algebras, together
with related concepts in algebraic topology related to quantum mechanics, including
deformations of $E_n$-algebras, the (higher) Deligne conjecture, (higher) 
string topology and Bar constructions of iterated loop spaces. It is also
related to Hoschchild homology and cohomology, which arise as specific
examples of factorization homology.

In this article, we will present the preliminaries we need for defining
factorization algebras and factorization homology in Section 1.
In Section 2, we will give the definition of
factorization algebras, and present some basic operations, such as pullbacks,
pushforwards, extending from a basis and gluing. The definition of 
homology theories for manifolds, factorization homology, together with their
properties, relationships with factorization algebras and basic examples,
will be given in Section 3. In the final section, two applications will be 
introduced, including the Bar constructions and the Deligne conjecture.

\begin{notation}
    We fix the following notations throughout the whole article:
    \begin{itemize}
    \item All manifolds are assumed to be smooth.
    \item When we talk about a ``topological space'', we mean a $K$-space, whose
    definition and basic properties may be found in \cite{Hov99}. The category
    of $K$-spaces is demoted by $\K$.
    \item $\SSet$ is the category of simplicial sets. By \cite{Hov99},
    there exists a singular functor $\Sing:\K\to\SSet$.
    \item The definitions and basic properties of model categories are from
    \cite{Hov99} as well. Note that the definitions are slightly different from
    other references.
    \item An introduction to the language of $\infty$-categories can be found in \cite{Lur09}.
    Further reference for $\infty$-categories include \cite{Lur16}.
    \item $\S$ denotes the $\infty$-category of topological spaces, and $\H$
    denotes the homotopy category of topological spaces. 
    The $1$-morphisms in $\S$ are continuous maps between spaces, and
    the $n$-morphisms in $\S$ are homotopies between $(n-1)$-morphisms;
    and the morphisms in $\H$ are homotopy classes of continuous maps.
    \item For any commutative ring with unit $R$, we use $\Ch(R)$ to denote
    the (ordinary) category of chain complexes over $R$, and we use $\Chinf(R)$
    to denote the $\infty$-category of chain complexes over $R$. 
    For a minimal supply to $\infty$-category of chain complexes, the first chapter of
    \cite{Lur16} may be helpful.
    \item For a topological space $X$, $C_\bullet(X)$ denotes the singular chain
    complex of $X$, and $C^\bullet(X)$ denotes the singular cochain
    complex of $X$.
    \end{itemize}
\end{notation}

\section{Preliminaries: Operads, higher operads and algebras over them}

In this section, we introduce the language of operads, that will be used in
the sections below. We will see how these operads represent different
structures on multilinear operators in a single, simple language.
We will also introduce the language of higher operads, which allows us to
intepret multilinear operators in different categories in one
single language, as well as homotopy coherent algebras. Finally, we introduce 
the little cubes operad, and $E_n$-algebras, as we will use them
in the third section.

\subsection{Basic definitions}

Operads are first introduced by Peter May in his book \cite{May72}. 
We first introduce its definition and give some evident generalizations.

\begin{definition}
    Suppose $(\cal C,\ox,\mathbb1)$ is a symmetric monoidal category.
    An \term{operad} with values in $\cal C$ consists of:
    \begin{enumerate}[i)]
        \item A sequence of objects $F(n)\in\cal C$ for all nonnegative integers $n$;
        \item A unit map $e\:\mathbb1\to F(1)$;
        \item A right action $\rho_n$ of the symmetric group $S_n$ on $F(n)$;
        \item A composition map 
        
        \[c\:F(k)\ox F(n_1)\ox\cdots\ox F(n_k)
        \to F(n_1+\cdots+n_k)\]
        
        for nonnegative integers $k,n_1,\cdots,n_k$;
    \end{enumerate}
    such that the following conditions are satisfied:
    \begin{enumerate}[i)]
        \item The unital identity: The following maps are the identity maps:
        
        \[F(n)\xrightarrow{\cong}\mathbb1\ox F(n)\xrightarrow{e\ox\mathbb1}
        F(1)\ox F(n)\xrightarrow{c}F(n),\]
        \[F(n)\xrightarrow{\cong}F(n)\ox\mathbb1^{\ox n}
        \xrightarrow{\mathbb1\ox e^{\ox n}}F(n)\ox F(1)^{\ox n}\xrightarrow{c}F(n);\]

        \item The associativity diagram: The following diagram is commutative:
        
        \begingroup
        \footnotesize
        \[\xymatrix{
        F(k)\ox\9\bigox_{i=1}^k F(n_i)\0\ox\9\bigox_{i=1}^k\bigox_{j=1}^{n_i}F(m_{ij})\0
        \ar[dd]_{c\ox\9\bigox_{i=1}^k\bigox_{j=1}^{n_i}\mathbb1\0}\ar[r]^\cong&
        F(k)\ox\bigox_{i=1}^k\9F(n_i)\ox\bigox_{j=1}^{n_i}F(m_{ij})\0
        \ar[d]^{\mathbb1\ox\bigox_{i=1}^kc}\\&
        F(k)\ox\bigox_{i=1}^kF\9\sum\limits_{j=1}^{n_i}m_{ij}\0\ar[d]^c\\
        F\9\sum\limits_{i=1}^kn_i\0\ox\9\bigox_{i=1}^k\bigox_{j=1}^{n_i}F(m_{ij})\0
        \ar[r]^c&F\9\sum\limits_{i=1}^k\sum\limits_{j=1}^{n_i}m_{ij}\0;}\]
        \endgroup

        \item Compatibility with the $S_n$-action: The following diagram
        is commutative: 
        
        \[\xymatrix{
        F(k)\ox\bigox_{i=1}^k F(n_i)\ar[r]^\cong\ar[d]_{\rho_\sigma\ox\mathbb1}&
        F(k)\ox\bigox_{i=1}^k F(n_{\sigma^{-1}(i)})\ar[d]^{c}\\
        F(k)\ox\bigox_{i=1}^k F(n_i)\ar[r]^c&F(n_1+\cdots+n_k)
        }\]
        
        for any $\sigma\in S_k$; and the following diagram is commutative:

        \[\xymatrix{F(k)\ox\bigox_{i=1}^k F(n_i)\ar[d]_{\mathbb1\ox
        \bigox_{i=1}^k\rho_{\sigma_i}}\ar[r]^c&F(n_1+\cdots+n_k)
        \ar[d]^{\rho_{(\sigma_1,\cdots,\sigma_r)}}\\F(k)\ox\bigox_{i=1}^k F(n_i)
        \ar[r]^c&F(n_1+\cdots+n_k)}\]
        
        for any $\sigma_i\in S_{n_i}$, where
        $(\sigma_1,\cdots,\sigma_r)$ denotes its image under the evident inclusion
        $S_{n_1}\times\cdots\times S_{n_k}\to S_{n_1+\cdots+n_k}$.
    \end{enumerate}
    If $\mathcal C=\SSet$, we call such an operad with values in $\SSet$
    a \term{simplicial operad}.
\end{definition}

What an operad describes is some sort of ``multimorphisms on a single object'';
that is, we may regard $F(n)$ as some sort of ``the space of $n$-ary operations'', 
$e$ to be the ``identity operation'', $\rho$ to be ``permuting the inputs'';
$c$ to be the ``composition'', and these operations satisfy a number of
coherent diagrams. To be more explicitly, we have the following example:

\begin{example}
    Suppose $\cal C$ is a symmetric monoidal category, $\cal D$ is a
    symmetric monoidal $\cal C$-enriched category, and $X\in\cal D$.
    There exists a canonical operad given by \[F(n)=\Map(X^{\ox n},X),\]
    where $e$ is the identity map \[\mathbb1_X\:\mathbb1_{\cal C}\to\Map(X,X)\]
    (regarded as an object in $\Map(X,X)$), and the $S_n$-actions and 
    compositions are given in the obvious way. This operad indeed describes 
    multimorphisms on the object $X$. We denote this operad by $\End(X)$,
    and call it the \term{endomorphism operad} of $X$.
\end{example}

More examples of operads include Examples \ref{eg1} and \ref{eg2}.

In many cases we also need to consider multimorphisms on not only
one object but maybe more than one: for example, observables in small sets
lead to observables in large sets. In order to deal with this situation,
we introduce a structure called colored operads. The following definition
is from \cite{nLab}.

\begin{definition}
    Suppose $C$ is a set, called the set of \term{colors}. 
    A \term{colored operad} with values taken in a symmetric monoidal category
    $\cal C$ consists of: 
    \begin{enumerate}[i)]
        \item An object $\Hom(c_1,\cdots,c_n;c)$ in $\cal C$ for every nonnegative integer $n$
        and $c_1,\cdots,c_n,c\in C$;
        \item A unit map $e\:\mathbb1\to\Hom(c;c)$ for all $c\in C$;
        \item An action by the symmetric group 
        \[\rho_\sigma\:\Hom(c_1,\cdots,c_n;c)
        \to\Hom(c_{\sigma(1)},\cdots,c_{\sigma(n)};c)\] 
        for $\sigma\in S_n$ and $c_1,\cdots,c_n,c\in C$;
        \item A composition law \[\Hom(c_1,\cdots,c_k;c)\ox\bigox_{i=1}^k
        \Hom(c_{i1},\cdots,c_{in_i};c_i)\to\Hom(c_{11},\cdots,c_{kn_k};c);\]
    \end{enumerate}
    such that the composition is associative, unital and compatible
    with the $S_n$-action.

    If $\mathcal C=\SSet$, we call such a colored operad with values in $\SSet$
    a \term{simplicial colored operad}.
\end{definition}

From this definition, we see that an operad is a special case
of a colored operad: it is a colored operad with only one color.
In this case, we have 
\[F(n)=\Hom(\underbrace{*,\cdots,*}_{n\text{ copies}};*).\]
For this reason, unless focusing on specific examples, we aim at
colored operads instead of single-colored operads.

The endomorphism operad can also be generalized to the colored case,
as follows:

\begin{example}
    Suppose $\cal C$ is a symmetric monoidal category, $\cal D$ is a
    symmetric monoidal $\cal C$-enriched category, $X=\{X_c\}_{c\in C}$ is a set of objects
    in $\cal D$, where $C$ is the set of colors. Consider the colored operad given by
    \[\Hom(c_1,\cdots,c_n;c)=\Map(X_{c_1}\ox\cdots\ox X_{c_n},X_c),\]
    where the identity maps, $S_n$-actions and compositions are given in the obvious way.
    We denote this operad by $\End(X)$, and call it the \term{endomorphism operad}
    of $X$. If $X$ is the set of all objects in $\cal D$, we also write it as
    $\End(\cal D)$.
\end{example}

Next we discuss morphisms between (colored) operads.

\begin{definition}
    Suppose $\mathcal O_1,\mathcal O_2$ are colored operads,
    with colorings $C_1,C_2$, respectively. A map $F$ from $\mathcal O_1$ to 
    $\mathcal O_2$ consists of a map $F:C_1\to C_2$ of colors, and a map
    \[F\:\Hom(c_1,\cdots,c_n;c)\to\Hom(F(c_1),\cdots,F(c_n);F(c))\] of morphisms
    for any objects $c_1,\cdots,c_n,c\in C_1$, such that the map of morphisms
    preserves unit, composition, and is $S_n$-equivariant.
\end{definition}

Thus, we may define the category of single-colored operads and colored operads
with values in $\cal C$, which we will denote by $\Op(\cal C)$ 
and $\COp(\cal C)$.

Now we are ready to introduce what an algebra over a (colored) operad is.

\begin{definition}
    Suppose $\cal O$ is a colored operad, taking values in a symmetric
    monoidal category $\cal C$, with coloring $C$, and $\cal D$ is a 
    symmetric monoidal $\cal C$-enriched category. 
    
    An \term{algebra over $\cal O$}
    in $\cal D$ is a pair $(X,\theta)$, where $X=\{X_c\}_{c\in C}$ is a collection
    of objects in $\cal D$ indexed by $C$, and $\theta:\mathcal O\to\End(X)$
    is a map of colored operads, such that $\theta(c)=X_c$ for all $c\in C$.
    Equivalently, an algebra over $\cal O$ in $\cal D$ is a map 
    $\mathcal O\to\End(\cal D)$ of colored operads. 
    
    A map $f$ between algebras
    $(X,\theta)$ and $(Y,\xi)$ over $\cal O$ in $\cal D$ consists a map
    $f\:X_c\to Y_c$ for every $c\in C$, such that the following diagram
    is commutative for any colors $c_1,\cdots,c_n,c\in C$:
    \[\xymatrix{
    \Hom(c_1,\cdots,c_n;c)\ar[r]^\theta\ar[d]_\xi&\Hom(X_{c_1},\cdots,X_{c_n};X_c)
    \ar[d]^{f_*}\\\Hom(Y_{c_1},\cdots,Y_{c_n};Y_c)\ar[r]^{f^*}&
    \Hom(X_{c_1},\cdots,X_{c_n};Y_c).}\]
    The category of algebras over $\cal O$ in $\cal D$ will be denoted by
    $\Alg(\cal O,D)$.
\end{definition}

Loosely speaking, an algebra over an operad is a structure such that the
``formal operations'' in an operad turn into ``actual operations'' 
between certain objects.

The language of operads allows us to inteprete different structures
in a common language. We first state a lemma, whose proof is simply reinterpreting
the definitions:

\begin{lemma}
    Suppose $\cal O$ is a colored operad, taking values in a symmetrical
    monoidal category $\cal C$, with coloring $C$, and $\cal D$ is a 
    symmetric monoidal $\cal C$-enriched category that is tensored over $\cal C$. Giving an algebra
    $(X,\theta)$ over $\cal O$ in $\cal D$ is equivalent to giving a set of objects
    $X=\{X_c\}_{c\in C}$ in $\cal D$ indexed by $C$, and maps 
    \[\theta\:\Hom(c_1,\cdots,c_n;c)\ox X_{c_1}\ox\cdots\ox X_{c_n}\to X_c,\]
    such that the unital, $S_n$-equivariance and associativity diagrams are satisfied.
\end{lemma}

We now state some basic examples.

\begin{example}\label{eg1}
    Suppose $\cal C$ is a symmetric monoidal cocomplete category,
    $\cal D$ is a symmetric monoidal $\cal C$-enriched category. Define the 
    (single-colored) operad $\Assoc$ to be the operad with:
    \begin{enumerate}[i)]
        \item $\Assoc(n)=\coprod_{\sigma\in S_n}(\mathbb1_{\cal C})_{\sigma}$;
        \item The unit map is the identity map of the unit object;
        \item The $S_n$-action is given by right multiplication;
        \item The composition map is induced by the inclusion 
        $S_k\times S_{n_1}\times\cdots\times S_{n_k}\to S_{n_1+\cdots+n_k}$,
        which is again induced by an $S_k\times S_{n_1}\times\cdots\times S_{n_k}$%
        -action on $\{1,\cdots,n_1+\cdots+n_k\}$, where we first divide
        the set into $k$ block with the $i$-th block has size $n_i$, permute the blocks
        by the element in $S_k$, and permute each block by the element in $S_{n_i}$.
    \end{enumerate}
    Then an algebra over $\Assoc$ is equivalent to a monoidal object in $\cal D$,
    and a map of algebras over $\Assoc$ is equivalent to a homomorphism
    between two monoidal objects. In particular, if $\mathcal D=\Mod_R$ for some
    ring $R$, the two assertions are equivalent to associative $R$-algebras
    and homomorphisms between associative $R$-algebras; if $\mathcal D=\Ch_R$
    for some ring $R$, the two assertions are equivalent to
    associative dg-$R$-algebras and homomorphisms between 
    associative dg-$R$-algebras.
\end{example}

\begin{example}\label{eg2}
    Suppose $\cal C$ is a symmetric monoidal category, 
    $\cal D$ is a symmetric monoidal $\cal C$-enriched category. Define the 
    (single-colored) operad $\Comm$ to be the operad with:
    \begin{enumerate}[i)]
        \item $\Comm(n)=\mathbb1_{\cal C}$;
        \item The unit map and composition map is the identity map of the unit object;
        \item The $S_n$-action is the trivial action.
    \end{enumerate}
    Then an algebra over $\Comm$ is equivalent to a commutative monoidal object
    in $\cal D$, and a map of algebras over $\Comm$ is equivalent to a homomorphism
    between two commutative monoidal objects. In particular, if $\mathcal D=\Mod_R$
    for some ring $R$, the two assertions are equivalent to
    commutative $R$-algebras and homomorphisms between commutative $R$-algebras;
    if $\mathcal D=\Ch_R$ for some ring $R$, the two assertions are equivalent to
    commutative dg-$R$-algebras and homomorphisms between 
    commutative dg-$R$-algebras.
\end{example}

\subsection{Higher operads}

In practice we often have to change the underlying category of an operad. 
For example, we sometimes face the problem of changing
the underlying category of an operad $\cal O$ from the category of
topological spaces to some other category (for example $\Ch(R)$ for a ring $R$).
If there exists a symmetric monoidal
functor between the underlying categories (in the case $\Ch(R)$
we have the singular chain complex functor), then we may take the pushforward of $\cal O$
along the symmetric monoidal functor; however this is not always the case.
In this section, we will introduce the notion of a higher operad,
which solves this problem nicely for topological operads.

To do this, we first review how we get a symmetric monoidal 
$\infty$-category from an ordinal symmetric monoidal 
category. 

\begin{construction}
    For a symmetric monoidal category $\cal C$, we may construct a 
    category $\cal C^\ox$, as follows:

    \begin{enumerate}[i)]
        \item The objects of $\cal C^\ox$ are finite (possibly empty) sequences
        of objects in $\cal C$, which will be expressed in the form $[C_1,\cdots,C_n]$;

        \item For any $[C_1,\cdots,C_n],[D_1,\cdots,D_m]\in\cal C^\ox$, a morphism
        \[f\:[C_1,\cdots,C_n]\to[D_1,\cdots,D_m]\] consists of a subset 
        $S\subseteq\{1,\cdots,n\}$, a map $f\:S\to\{1,\cdots,m\}$, and a map 
        \[f_i\:\bigotimes_{j\in f^{-1}(i)}C_j\to D_i\] for any $1\le i\le m$;

        \item Suppose \[f\:[C_1,\cdots,C_n]\to[D_1,\cdots,D_m],\hspace{2em}
        g\:[D_1,\cdots,D_m]\to[E_1,\cdots,E_l]\] are morphisms in $\cal C^\ox$,
        determining subsets $S\subseteq\{1,\cdots,n\}$ and $T\subseteq\{1,\cdots,m\}$.
        The composition $g\circ f$ consists of the set 
        $S'=f^{-1}(T)\subseteq\{1,\cdots,n\}$, the map \[g\circ(f|_{S'})\:
        S'\to\{1,\cdots,l\},\] and the composition \[(g\circ f)_{i}\:
        \bigox_{k\in(g\circ(f|_{S'}))^{-1}(i)}C_k\cong\bigox_{j\in g^{-1}(i)}
        \bigox_{k\in f^{-1}(j)}C_k\to\bigox_{j\in g^{-1}(i)}D_j\to E_i\] for every
        $1\le i\le l$.
    \end{enumerate}
\end{construction}

Next, take the category $\Fin_*$, the category of pointed finite sets,
where the objects are $\<n\>:=\{*,1,\cdots,n\}$ for all
nonnegative integers $n$, and the morphisms are all maps that preserve the base point.
Then we have a forgetful functor 
$p:\cal C^\ox\to\Fin_*$, that carries an object $[C_1,\cdots,C_n]$ to the set
$\<n\>$. Moreover, $p$ satisfies the following properties:

\begin{proposition}
    \begin{enumerate}[i)]
        \item $p$ is a \term{coCartesian fibration}, which means for any 
        \[\alpha\:\<n\>\to\<m\>\in\Fin_*,\hspace{2em}C=[C_1,\cdots,C_n]\in\cal C^\ox,\]
        there exists a map \[C\to D=[D_1,\cdots,D_m]\in\cal C^\ox\] lifting $\alpha$
        that is \term{coCartesian} in the following sense: for any $E\in\cal C^\ox$
        the map \[\Hom_{\cal C^\ox}(D,E)\to\Hom_{\cal C^\ox}(C,E)
        \mathop{\operatorname{\times}}\limits_{\Hom_{\Fin_*}(\<n\>,p(E))}
        \Hom_{\Fin_*}(\<m\>,p(E))\] is an isomorphism. This implies that there exists
        a functor $\alpha_!\:\mathcal C^\ox_{\<n\>}\to\mathcal C^\ox_{\<m\>}$
        between the fibers.
    
        \item The map \[\mathcal C^\ox_{\<n\>}\to(\mathcal C^\ox_{\<1\>})^n\cong {\cal C}^n\] induced by
        the $n$ maps $\rho_i\:\<n\>\to\<1\>$ for every $1\le i\le n$, where $\rho_i(i)=1$
        and $\rho_i(j)=*$ if $i\ne j$, is an equivalence of categories.
    \end{enumerate}
\end{proposition}

Thus the following definition is reasonable:

\begin{definition}
    A \term{symmetric monoidal $\infty$-category} is an $\infty$-category 
    $\cal C^\ox$ equipped with a coCartesian fibration $p\:\cal C^\ox\to\N(\Fin_*)$
    (where $\N$ is the nerve functor that takes a category to its nerve),
    such that for every $n\ge 0$, the map $\mathcal C^\ox_{\<n\>}
    \to(\mathcal C^\ox_{\<1\>})^n$ induced by the $n$ maps $\rho_i\:\<n\>\to\<1\>$
    for every $1\le i\le n$, is an equivalence of $\infty$-categories.
    The \term{underlying category} of $\cal C^\ox$
    is defined to be $\mathcal C:=\mathcal C^\ox_{\<1\>}$.
\end{definition}

We now turn back to the case of (colored) operads. 

\begin{construction}\label{constr}
    For any colored operad
    $\cal O$ over $\Set$, we may define a category $\cal O^\ox$ as follows:
    
    \begin{enumerate}[i)]
        \item The objects of $\cal O^\ox$ are finite (possibly empty) sequences
        of colors in $\cal O$, which will be expressed in the form $[C_1,\cdots,C_n]$;
    
        \item For any $[C_1,\cdots,C_n],[D_1,\cdots,D_m]\in\cal O^\ox$, a morphism
        \[f\:[C_1,\cdots,C_n]\to[D_1,\cdots,D_m]\] consists of a map $f\:\<n\>\to\<m\>$, 
        and a map $f_i\in\Hom((C_j)_{j\in f^{-1}(i)};D_i)$ for any $1\le i\le m$;
    
        \item Composition of morphisms in $\cal O^\ox$ is determined by
        the composition on $\Fin_*$ and on the colored operad $\cal O$.
    \end{enumerate}
\end{construction}

Then we have a forgetful functor $p\:\cal O^\ox\to\Fin_*$, 
that carries an object $[C_1,\cdots,C_n]$ to the set
$\<n\>$. Moreover, $p$ satisfies properties similar to the $\cal C^\ox$
constructed above. Consequencely, we may define $\infty$-operads in a way
similar to that of symmetric monoidal $\infty$-categories. We make the following
definition, which is due to \cite{Lur16}:

\begin{definition}
    An \term{$\infty$-operad} is an $\infty$-category $\cal O^\ox$ equipped with 
    a functor $p\:\cal O^\ox\to\N(\Fin_*)$, such that:
    \begin{enumerate}[i)]
        \item For any \term{inert} morphism $f\:\<n\>\to\<m\>$, which means that
        $f^{-1}(\{1,\cdots,m\})\to\{1,\cdots,m\}$ is an isomorphism, 
        and any $C\in\cal O^\ox_{\<n\>}$, there exists
        a $p$-coCartesian morphism $C\to D\in\cal O^\ox$ lifting $f$;

        \item For every $C_1,\cdots,C_n\in\mathcal O^\ox_{\<1\>}$, there exists 
        $C\in\mathcal O^\ox_{\<n\>}$ and $p$-coCartesian morphisms $C\to C_i$ covering 
        $\rho_i$;

        \item For any objects $C\in\mathcal O^\ox_{\<n\>},C'\in\mathcal O^\ox_{\<m\>}$,
        and a morphism $f\in\Hom_{\Fin_*}(\<n\>,\<m\>)$, take
        \[\Map_{\cal O^\ox}^f(C,C')=\Map_{\cal O^\ox}(C,C')
        \mathop{\operatorname{\times}}\limits_{\Map_{\N(\Fin_*)}(\<n\>,\<m\>)}\{f\},\]
        take $p$-coCartesian morphisms $C'\to C'_i$ covering $\rho_i$ 
        for every $1\le i\le m$, then the induced map \[\Map_{\cal O^\ox}^f(C,C')\to
        \prod_{1\le i\le m}\Map_{\cal O^\ox}^{\rho_i\circ f}(C,C_i')\] is an isomorphism
        in $\H$.
    \end{enumerate}
\end{definition}

\begin{remark}
    \begin{enumerate}[i)]
        \item By this definition, for any $\Set$-valued colored operad $\cal O$,
        $\N(\cal O^\ox)$ is an $\infty$-operad. Thus this definition
        generalizes the definition of a ($\Set$-valued colored) operad
        to the higher categorial sense. 

        \item If $\cal O^\ox$ is an $\infty$-operad, we define the \term{underlying
        category} of $\cal O^\ox$ to be $\mathcal O:=\mathcal O^\ox_{\<1\>}$. This is an
        $\infty$-category, since $p$ must be an inner fibration by \cite{Lur09},
        Proposition 2.3.1.5.

        \item By definition, for every $n\ge 0$ the map $\mathcal O^\ox_{\<n\>}
        \to\mathcal O^n$, induced by the $n$ maps $\rho_i:\<n\>\to\<1\>$
        for every $1\le i\le n$, is an equivalence of $\infty$-categories. Thus,
        as in the ordinary case, we may regard objects in $\mathcal O^\ox_{\<n\>}$ as
        finite sequences of objects in $\mathcal O$ of length $n$.

        \item Any symmetric monoidal $\infty$-category $\cal C^\ox$ is an $\infty$-operad.
        This is because the condition that $p$ is a coCartesian fibration gaurantees
        i) and ii), and the equivalence between $\mathcal C^\ox_{\<n\>}$ and 
        $\mathcal C^n$ gaurantees iii).
    \end{enumerate}
\end{remark}

Next we define morphisms between $\infty$-operads.

\begin{definition}
    Suppose $\cal O^\ox,O'^\ox$ are $\infty$-operads. A \term{map 
    of $\infty$-operads} from $\cal O^\ox$ to $\cal O'^\ox$ is a map
    of $\infty$-categories $f:\cal O^\ox\to O'^\ox$, such that:
    \begin{enumerate}[i)]
        \item $f$ is compatible with the forgetful maps to $\N(\Fin_*)$;
        \item $f$ preserves inert morphisms, where a morphism in $\cal O^\ox$
        is \term{inert} if it is $p$-coCartesian, and its image in $\N(\Fin_*)$
        is inert.
    \end{enumerate}
    The full sub-$\infty$-category of $\Fun_{\N(\Fin_*)}(\cal O^\ox,O'^\ox)$
    spanned by maps of $\infty$-operads will be denoted $\Alg(\cal O,\cal O')$.
    A map of $\infty$-operads from $\cal O^\ox$ to $\cal O'^\ox$ is also called
    an \term{algebra over $\cal O^\ox$ in $\cal O'^\ox$}, if $\cal O'^\ox$
    is a symmetric monoidal $\infty$-category.
\end{definition}

\begin{remark}\label{a}
    \begin{enumerate}[i)]
        \item Since by definition, if $f\:\cal O\to O'$ is a map of (ordinary) operads,
        the map $\N(f^\ox)\:\N(\cal O^\ox)\to\N(O'^\ox)$ is a map of $\infty$-operads,
        this is really a generalization of the original definition.

        \item By this definition, we obtain an $\infty$-category of algebras
        over $\cal O^\ox$ in $\cal O'^\ox$.

        \item If $\cal O'^\ox$ is a symmetric monoidal $\infty$-category,
        then the symmetric monoidal structure on $\cal O'^\ox$ induces a 
        symmetric monoidal structure on $\Alg(\cal O,\cal O')$, where for any 
        $f,g\in\Alg(\cal O,\cal O')$ and $p\in\cal O^\ox$, $(f\ox g)(p)=f(p)\ox g(p)$.
        Verifications of the axioms are straightforward.
    \end{enumerate}
\end{remark}

We now show how these definitions solve the problem mentioned in the beginning of
the section. We first notice that the construction of $\cal O^\ox$ also makes
sense in the case of a simplicial colored operad, except that we need to
change ii) of Construction \ref{constr} into the following construction:

\begin{enumerate}[ii')]
\item For any $[C_1,\cdots,C_n],[D_1,\cdots,D_m]\in\cal O^\ox$, we define
\[\begin{aligned}&\Hom_{\cal O^\ox}([C_1,\cdots,C_n],[D_1,\cdots,D_m])\\
&\hspace{3em}=\coprod_{f\in\Hom_{\Fin_*}(\<n\>,\<m\>)}\prod_{1\le i\le m}
\Hom((C_j)_{j\in f^{-1}(i)},D_i).\end{aligned}\]
\end{enumerate}

Since $\cal O^\ox$ is a simplicial category, we may take its simplicial nerve
to obtain a simplicial set $\N^\ox(\cal O)$, which we will call 
the \term{operadic nerve} of $\cal O$. This simplicial set may or may not be an
$\infty$-category, but we have the following proposition:

\begin{proposition}\label{d}
Suppose all the Hom-spaces of $\cal O$ are Kan complexes. Then $\N^\ox(\cal O)$
is an $\infty$-operad.
\end{proposition}

We refer the reader to \cite{Lur16}, Proposition 2.1.1.27 for the proof.

In the case that $\cal O$ is a topological operad,
we may take the pushforward of $\cal O$ along the symmetric monoidal functor 
$\Sing\:\K\to\SSet$, and repeat the above constructions, obtaining
a simplicial set $\N^\ox(\Sing\cal O)$, which we will simply denote by
$\N^\ox(\cal O)$, called the \term{operadic nerve} of $\cal O$.
Now the operadic nerve of a topological operad is \textit{always} 
an $\infty$-operad, since $\Sing(X)$ is always a Kan complex for any topological space $X$.

Since any symmetric monoidal category $\cal C$ is a symmetric monoidal 
$\infty$-category, we may regard an algebra over $\cal O$ in $\cal C$
as an algebra over $\N^\ox(\cal O)$ in $\N^\ox(\cal C)$.

\begin{remark}
    If a map of topological colored operads induces weak equivalences on all
    morphism spaces, then the operadic nerve of the topological colored operads
    are equivalent $\infty$-categories, thus the associated $\infty$-categories
    of algebras over the topological colored operads are equivalent.
\end{remark}

\subsection{The little cubes operads and \texorpdfstring{$E_n$}{En}-algebras}

The little cubes operads are one of the most often discussed types of operads. We
give the definitions of the little cubes operads and $E_n$-algebras,
and basic examples of $E_n$-algebras.

\begin{definition}
A \term{rectilinear morphism} is a map $f\:\R^n\to\R^n$, such that $f$
can be written as the form $f_1\times\cdots\times f_n$, where each $f_i\:\R\to\R$
is an affine morphism. 
\end{definition}

\begin{definition}
    Suppose $n\ge 1$ is an integer. We define the topological single-colored
    operad $\E_n$, called the \term{little $n$-cubes operad}, as follows:
    \begin{enumerate}[i)]
        \item For any $r\ge 0$, the morphism space $\E_n(r)$ is the configuration space
        of $r$ ordered disjoint open rectangles whose edges are parallel to the coordinate axes (each called a \term{little $n$-cube})
        lying in the unit open cube; in other words,
        $\E_n(r)$ is the space of rectilinear embeddings
        from $\coprod_{i=1}^r(0,1)^n$ to $(0,1)^n$, which will be denoted by \[\Rect\9\coprod_{i=1}^r(0,1)^n,(0,1)^n\0;\]
        \item The $S_r$-action is given by permuting the $r$ little $n$-cubes;
        \item The unit map is the identity map of $(0,1)^n$;
        \item Composition is given by the composition of the rectilinear embeddings.
    \end{enumerate}
    An \term{$E_n$-algebra} is an algebra over $\E_n$. The $\infty$-category
    of $E_n$-algebras in a symmetric monoidal $\infty$-category $\cal C$
    will be denoted $\E_n\-\Alg({\cal C})=\Alg(\N^\ox(\E_n),{\cal C})$.
\end{definition}

By definition, there exists a canonical map $i\:\E_n\to \E_{n+1}$,
given by \[\begin{aligned}i(r)\:\Rect\9\coprod_{i=1}^r(0,1)^n,(0,1)^n\0&\to
\Rect\9\coprod_{i=1}^r(0,1)^{n+1},(0,1)^{n+1}\0,\\
f&\mapsto f\times\mathbb1_{(0,1)},\end{aligned}\] which is an inclusion of topological spaces,
for all $r\ge 0$. Taking colimits, we obtain an operad, which we will call
the \term{little $\infty$-cubes operad}, denoted $\E_\infty$.
Explicitly, we have $\E_\infty(r)=\colim_{n\to\infty}\E_n(r)$.
An \term{$E_\infty$-algebra} is an algebra over $\E_\infty$. 
The $\infty$-category of $E_\infty$-algebras in a symmetric monoidal
$\infty$-category $\cal C$ will be denoted $\E_\infty\-\Alg({\cal C})=\Alg(\N^\ox(\E_\infty),{\cal C})$.

Note that the inclusion $i\:\E_n\to \E_{n+1}$ induces a functor \[i^*\:\E_n\-\Alg(\mathcal C)\to 
{\E}_{m}\-\Alg(\cal C)\] for any $1\le m\le n\le\infty$, which shows that
any $E_n$-algebra can be regarded as an $E_m$-algebra if $n\ge m$.

For one of the basic properties of $\E_n$ we have the following proposition,
which is stated and proved in Theorem 4.8, \cite{May72}:

\begin{proposition}
For any $1\le n\le\infty,r\ge 0$, the space $\E_n(r)$ is $S_r$-equivariantly
homotopy equivalent to $\Conf(\R^n,r)$, the configuration space of $r$ ordered points in $\R^n$.
\end{proposition}

It is well known that the configuration space $\Conf(\R^n,r)$ is 
$(n-2)$-connected for $n\ge 2$ (See, for instance, \cite{FN62}).
This proposition shows that an $E_n$-algebra may be regarded as an algebra
that is ``homotopy commutative up to the $(n-1)$-th degree''. Also, note that
$\E_1(r)$ is a disjoint union of $r!$ contractible spaces.
Therefore, the maps $\pi_0\:\E_1\to\Assoc$ and $\pi_0\:\E_\infty\to\Comm$ induce
weak equivalences in all morphism spaces. Thus, $E_1$-algebras
and $E_\infty$-algebras can be regarded as ``homotopy refinements''
of associative algebras and commutative algebras.

\begin{example}
Iterated loop spaces. This is one of the first examples of $E_n$-algebras
that have been studied. Suppose $(X,*)$ is a pointed topological space.
We define a map \[\theta\:\E_n(r)\times(\Omega^nX)^r\to\Omega^nX\] as follows:
for any \[\E_n(r)\ni f\:\coprod_{i=1}^r(0,1)^n\to(0,1)^n\] given by 
$f_1,\cdots,f_r\:(0,1)^n\to(0,1)^n$, and $\alpha_1,\cdots,\alpha_r\:[0,1]^n\to X$
representing $r$ elements in $\Omega^nX$, $\theta(f,\alpha_1,\cdots,\alpha_r)$
is the map $\beta\:[0,1]^n\to X$ such that for any $u\in(0,1)^n$, if $u=f_k(v)$
for some $1\le k\le r$ and $v\in(0,1)^n$, then $\beta(u)=\alpha_k(v)$;
otherwise $\beta(u)=*$. It is easy to verify that this gives the iterated
loop space $\Omega^nX$ an $E_n$-algebra structure. Moreover, the singular chain
complex $C_\bullet(\Omega^nX)$ is an $E_n$-algebra in chain complexes.
\end{example}

$E_n$-algebras play an important role in the study of homology theories
of manifolds and factorization homology; we will see this in the following 
sections.

\section{Factorization algebras}

Factorization algebras are a way to describe multilinear structures 
that satisfy certain homotopy coherent properties. It is a bit similar to
an algebra over the little cubes operad, but this time the operad is replaced
by a colored operad, with the cubes $(0,1)^n$ replaces by manifolds.
We shall now present the definitions and the basic properties,
which are mostly based on \cite{CG16} and \cite{Gin13}.

\subsection{Basic definitions}

\begin{definition}
    Suppose $X$ is a topological space (which in the case of a factorization
    algebra is often taken to be a manifold). Define the colored operad
    $\Fact(X)$ to be the following operad:
    \begin{enumerate}[i)]
        \item The colors of $\Fact(X)$ are the connected open sets in $X$。
        
        \item For any connected open sets $U_1,\cdots,U_n,U$ in $X$,
        define $\Hom(U_1,\cdots,U_n;U)$ to be the single-element set, if
        the $U_i$'s are pairwise disjoint and all contained in $U$;
        and the empty set if otherwise。

        \item The unit morphism, $S_n$-action, and the composition are given in the
        obvious way.
    \end{enumerate}
\end{definition}

The morphism from $(U_1,\cdots,U_n)$ to $U$ is some sort of ``putting an element
into each of the open sets $U_i$ and get an element in $U$''. We now state this
precisely.

\begin{definition}
    Suppose $\cal C$ is a symmetric monoidal $\infty$-category, $X$ is a
    topological space. A \term{prefactorization algebra $A$ over $X$ with values in
    $\cal C$} is an algebra over $\Fact(X)$ in $\cal C$. The $\infty$-category of 
    prefactorization algebras over $X$ in $\cal C$ is denoted $\PreFA(X,\cal C)$.
    If $U=\coprod_{i\in I}U_i$ is an open set in $X$ where each $U_i$ is connected,
    we define $A(U)$ to be $\bigotimes_{i\in I}A(U_i)$; and we define 
    $A(\varnothing)$ to be the tensor identity. A morphism between 
    prefactorization algebras is called an \term{equivalence} if it induces
    an equivalence on every open subset.
\end{definition}

Loosely speaking, a prefactorization algebra assigns an open set with an object,
and a morphism from a small open set to a large one. 

\begin{example}
    Suppose $A$ is an associative algebra. We construct a prefactorization algebra
    on the real line $\R$: for any open interval $(a,b)\subseteq\R$ we set 
    $A(a,b)=A$; for any \[-\infty\le a\le a_1<b_1\le a_2<b_2\le\cdots\le a_n<b_n\le b
    \le\infty\] we set the map \[A(a_1,b_1)\ox\cdots\ox A(a_n,b_n)\to A(a,b)\]
    to be the map \[A^{\ox n}\to A,(x_1,\cdots,x_n)\mapsto x_1\cdots x_n.\]
    It is easy to show that this forms a prefactorization algebra on $\R$.
\end{example}
    
We now discuss the definition of a factorization algebra. 
Factorization algebras are to prefactorization algebra as cosheaves
are to precosheaves. We first introduce a special type of cover, called a 
factorization cover.

\begin{definition}
    Suppose $U$ is an open set in a topological space $X$. A cover 
    $\{U_i\hookrightarrow U\}_{i\in I}$ is called a \term{factorizing cover}
    if for any $x_1,\cdots,x_n\in U$ there exists a finite collection of pairwise
    disjoint open sets $U_{i_1},\cdots,U_{i_m}$ in the open cover such that
    $x_1,\cdots,x_n\in U_{i_1}\coprod\cdots\coprod U_{i_m}$. 
\end{definition}

\begin{remark}
    Note that a nontrivial factorization cover exists whenever $X$ is Hausdorff.
    If $X$ is a manifold, we may take a factorization cover by take a neighborhood
    basis at every point of $X$.
\end{remark}

For a factorization cover $\{U_i\hookrightarrow U\}_{i\in I}$, we denote 
$PI$ to be the set of finite subsets of $I$, such that for every $\alpha\in PI$
and $i,j\in\alpha$, the sets $U_i$ and $U_j$ are disjoint. $PI$ describes
the tuples of open sets that appear in the structure maps of a 
prefactorization algebra. 

Now, for any $A$ a prefactorization algebra and any $\alpha_1,\cdots,\alpha_n
\in PI$, we denote \[A(\alpha_1,\cdots,\alpha_n)
=A\9\coprod_{i_1\in\alpha_1,\cdots,i_n\in\alpha_n}
U_{i_1}\cap\cdots\cap U_{i_n}\0,\] then there exists a natural map
\[d_i\:A(\alpha_1,\cdots,\alpha_n)\to A(\alpha_1,\cdots,\widehat{\alpha_i},
\cdots,\alpha_n)\] for every integer $1\le i\le n$, and a natural map
\[s_i\:A(\alpha_1,\cdots,\alpha_n)\to A(\alpha_1,\cdots,\alpha_i,\alpha_i,
\cdots,\alpha_n)\] for every integer $1\le i\le n$. Together this defines
a simplicial object $\check{C}_\bullet(I,A)$ in $\cal C$, where
\[\check{C}_n(I,A)=\coprod_{\alpha\in PI^{n+1}}A(\alpha),\] and the face and
degeneracy maps are given above. We call this the \term{simplicial \v{C}ech
complex} of $I$. Note that there exists a map of simplicial objects
$\check{C}_\bullet(I,A)\to A(U)$, where we regard $A(U)$ as the
constant simplicial object.

We now assume that $\cal C$ is a stable $\infty$-category, whose tensor product
is compatible with colimits over simplicial diagrams (say, for example,
$\Chinf(\Bbbk)$). Then by the $\infty$-categorical Dold-Kan correspondence 
and taking the direct colimit, we obtain an object $\check{C}(I,A)$
from the simplicial \v{C}ech complex. We call this the \term{\v{C}ech complex}
of $I$. Moreover we have a canonical map $\check{C}(I,A)\to A(U)$.

We are now ready to state the definition of a factorization algebra:

\begin{definition}
    A prefactorization algebra $A$ is said to be a \term{homotopy factorization
    algebra}, or a \term{factorization algebra} for short, if for any 
    factorization covering $\{U_i\hookrightarrow U\}_{i\in I}$, the morphism
    $\check{C}(I,A)\to A(U)$ is an equivalence in $\cal C$.
    The full subcategory of $\PreFA(X,\cal C)$ spanned by factorization algebras
    will be denoted $\FA(X,\cal C)$. 
\end{definition}

Note that in the case that ${\cal C}=\Chinf(\Bbbk)$ where $\Bbbk$ is a field,
the above definition is equivalent to that the chain map
$C_\bullet(I,A)\to A(U)$ being a quasi-isomorphism; a morphism between
factorization algebras is an equivalence if and only if it induces a
quasi-isomorphism on every open subset.

Naively speaking, homotopy factorization algebras are to prefactorization
algebras as homotopy cosheaves are to precosheaves, where we change the
underlying site to the category of open subsets of $X$ with open coverings
being factorization coverings.

In the case that $X$ is a manifold, there exists a notion of a locally constant
factorization algebra, which we will give as follows:

\begin{definition}
    Suppose $X$ is an $n$-dimensional topological manifold. We say that an open
    subset $U$ of $X$ is a \term{disk} if it is homeomorphic to $\R^n$.
    A (pre-)factorization algebra $A$ on $X$ is called \term{locally constant},
    if for any inclusion $U\hookrightarrow V$ of disks, the map $A(U)\to A(V)$
    is an equivalence.
\end{definition}

\begin{remark}\label{g}
    In Example \ref{h}, we will prove that a locally constant prefactorization
    algebra can be reformed to a factorization algebra, using 
    factorization homology. We denote the full subcategory of $\FA(X,\cal C)$
    spanned by locally constant factorization algebras by $\FA^{lc}(X,\cal C)$.
\end{remark}

\subsection{Basic properties}

We now discuss basic operations and properties of factorization algebras. Most
of them will be useful in the next section.

Firstly, we have the following proposition.

\begin{proposition}
    The categories $\PreFA(X,\mathcal C),\FA(X,\mathcal C),\FA^{lc}(X,\mathcal C)$
    are symmetric monoidal $\infty$-categories.
\end{proposition}

\begin{proof}
    By Remark \ref{a}, the $\infty$-category $\PreFA(X,\cal C)$
    is symmetric monoidal. In order to verify $\FA(X,\cal C),$ $\FA^{lc}(X,\cal C)$
    are symmetric monoidal, it suffices to verify for any $f,g\in\FA(X,\cal C)$
    (resp. $\FA^{lc}(X,\cal C)$), the tensor product of $f$ and $g$ again satisfies
    the axiom of $\FA(X,\cal C)$ (resp. $\FA^{lc}(X,\cal C)$). But this is
    straightforward since the tensor product in $\cal C$ is compatible with
    colimits over simplicial diagrams and equivalences are compatible with
    the tensor product.
\end{proof}

Next we discuss changing the underlying topological space $X$. Suppose we are
given a continuous map $f:X\to Y$ of topological spaces. Just like precosheaves
and cosheaves, we can take the \term{pushforward} of a (pre-)factorization algebra $A$
on $X$ to a (pre-)factorization algebra $f_*(A)$ on $Y$, given by
\[f_*(A)(V)=A(f^{-1}(V))\] for every open set $V\subseteq Y$. If $A$
is a prefactorization algebra, then $f_*(A)$ is a prefactorization algebra, since
it is the composite \[\Fact(Y)\xrightarrow{f^{-1}}\Fact(X)\xrightarrow{A}
\cal C^\ox;\] If $A$ is a factorization algebra, then since the pullback of 
any factorizing cover on $Y$ by $f$ is again a factorizing cover, 
$f_*(A)$ is a factorization algebra. Furthermore, we have
the following property:

\begin{proposition}\label{b}
    Suppose $f:X\to Y$ is a locally trivial fibration between manifolds. Then
    the pushforward of any locally constant factorization algebra $A$ is again
    locally constant.
\end{proposition}

\begin{proof}
    Suppose $U\hookrightarrow V$ is an inclusion of disks in $Y$. Since $V$
    is contractible, by assumption, we may assume that $f^{-1}(V)=V\times F$
    for some manifold $F$. Take a factorizing cover of $F$ by disks, namely
    $\{F_i\to F\}_{i\in I}$. Then $\{V\times F_i\to V\times F\}
    _{i\in I}$ and $\{U\times F_i\to U\times F\}_{i\in I}$ are factorizing covers
    by disks. On the other hand, for any $i\in I$, the map $A(U\times F_i)
    \to A(V\times F_i)$ is an equivalence. Thus the map $\check{C}(I,U\times F)
    \to\check{C}(I,V\times F)$ is an equivalence, which means that the map
    $f_*(A)(U)\to f_*(A)(V)$ is an equivalence.
\end{proof}

On the other hand, if we have an inclusion $i:Y\to X$ such that $Y$ is an
open subspace of $X$, and $A$ a (locally constant) (pre-)factorization algebra,
then we may \term{restrict} $A$ to $Y$; namely, reduce the domain of $A$
to the smaller colored operad $\Fact(Y)$. We denote this restriction
by $A|_{Y}$. When $A$ is (locally constant) (pre-)factorization algebra,
so is $A|_{Y}$.

We next discuss how to extend a ``partially defined'' factorization algebra
to a ``real'' factorization algebra.

\begin{definition}
    Suppose $\mathscr U$ is an open cover of $X$. Define $\Fact(\mathscr U)$
    to be the full suboperad of $\Fact(X)$ spanned by those colors in $\mathscr U$.
    A \term{$\mathscr U$-prefactorization algebra} is an algebra over 
    $\Fact(\mathscr U)$. A \term{$\mathscr U$-factorization algebra}
    is a $\mathscr U$-prefactorization algebra $A$ such that for any 
    factorization covering $\{U_i\hookrightarrow U\}_{i\in I}$ with all open sets
    in $\mathscr U$, the morphism $\check{C}(I,A)\to A(U)$ is an equivalence.
\end{definition}

Now, suppose $\mathscr U$ is a topological basis of $X$ that is stable
under finite intersections (such $\mathscr U$ is called a 
\term{factorizing basis}). We state the following proposition, that allows us
to extend a $\mathscr U$-factorization algebra to an ordinary
factorization algebra:

\begin{proposition}\label{c}
    Suppose $A$ is a $\mathscr U$-factorization algebra. For any open subset $V$
    of $X$, define $$i_*^{\mathscr U}(A)(V):=\check{C}({\mathscr U}_V,A),$$
    where ${\mathscr U}_V$ consists of those open sets in ${\mathscr U}$ that is
    contained in $V$. Then $i_*^{\mathscr U}(A)$ is a factorization algebra,
    whose restriction to $\Fact(\mathscr U)$ is equivalent to $A$. Moreover,
    if $A'$ is a factorization algebra, whose restriction to $\Fact(\mathscr U)$
    is equivalent to $A$, then $A'$ is equivalent to $i_*^{\mathscr U}(A)$.
    Furthermore, the construction $i_*^{\mathscr U}$ is functorial.
\end{proposition}

The proof to this is really technical, so we refer the interested readers to Appendix B, \cite{CG16}.
We note that the uniqueness directly follows from
the definition of a factorization algebra.

This proposition tells us that a factorization algebra is ``in some sense''
defined by its values on a few of its open subsets. This proposition is
in particular very useful in the case of a smooth manifold, 
since open disks form a factorizing basis, and the value of 
a factorization algebra on a disk is usually easy to determine.

Next, we discuss what happens if we take the product of two spaces. Suppose $X,Y$
are topological spaces and $\pi\:X\times Y\to X$ is the projection. 
By Proposition \ref{b}, $\pi$ induces a functor $\pi_*\:\PreFA(X\times Y)
\to\PreFA(X)$. We make a stronger claim:

\begin{proposition}
    There exists a canonical functor \[\underline\pi_*\:\PreFA(X\times Y)\to\PreFA(X,\PreFA(Y))\]
    such that the following diagram is commutative:
    \[\xymatrix{
        \PreFA(X\times Y)\ar[r]^{\underline\pi_*}\ar[rd]_{\pi_*}
        &\PreFA(X,\PreFA(Y))\ar[d]^{\PreFA(X,p_*)}
        \\&\PreFA(X).
    }\]
    Here $p\:Y\to*$ is the canonical map.
\end{proposition}

\begin{proof}
    We do the construction as follows. Suppose $A$ is a prefactoriation algebra on 
    $X\times Y$. For any $U\subseteq X$, we set $\underline\pi_*(A)(U)$ to be
    the prefactoriation algebra on $Y$, which asserts an open set $V\subseteq Y$
    with $\underline\pi_*(A)(U)(V)=A(U\times V)$, and for any $V_1,\cdots,V_n,V$
    being connected open sets in $Y$, such that the $V_i$'s are pairwise disjoint
    and all contained in $V$, define the map \[\underline\pi_*(A)(U)(V_1)\ox\cdots
    \ox\underline\pi_*(A)(U)(V_n)\to\underline\pi_*(A)(U)(V)\] to be the structure
    map \[A(U\times V_1)\ox\cdots\ox A(U\times V_n)\to A(U\times V).\] Since 
    $A$ is a prefactoriation algebra, the axioms of a prefactoriation algebra
    ensure that $\underline\pi_*(A)(U)$ is a prefactoriation algebra on $Y$. 
    Moreover, the axioms of a prefactoriation algebra ensure that for any
    $U_1,\cdots,U_n,U$ being connected open sets in $Y$, such that the $U_i$'s are
    pairwise disjoint and all contained in $U$, the collection of the maps
    \[\begin{aligned}&\underline\pi_*(A)(U_1)(V)\ox\cdots\ox\underline\pi_*(A)(U_n)(V)
    \\&\hspace{3em}\cong A(U_1\times V)\ox\cdots\ox A(U_n\times V)\to A(U\times V)\cong
    \underline\pi_*(A)(U)(V)\end{aligned}\] for all $V\subseteq Y$ forms a map
    \[\underline\pi_*(A)(U_1)\ox\cdots\ox\underline\pi_*(A)(U_n)
    \to\underline\pi_*(A)(U)\] of prefactoriation algebras on $Y$, the collection
    of which again forms a prefactoriation algebra on $X$ with values in 
    prefactoriation algebras on $Y$. Finally, since $p_*$ sends a prefactoriation
    algebra on $Y$ to its global sections, the identity $\PreFA(X,p_*)
    \circ\underline\pi_*=\pi_*$ is straightforward.
\end{proof}

The above construction also holds for factorization algebras, and has a
more interesting property:

\begin{proposition}\label{j}
    There exist functors \[\underline\pi_*\:\FA(X\times Y)\to\FA(X,\FA(Y))\]
    and \[\underline\pi_*\:\FA^{lc}(X\times Y)\to\FA^{lc}(X,\FA^{lc}(Y))\] between
    $\infty$-categories, and the latter is an equivalence.
\end{proposition}

\begin{proof}
    We first check that if $A$ is a factorization algebra on $X\times Y$, then for
    any $U\subset X$, $\underline\pi_*(A)(U)$ is a factorization algebra on $Y$.
    To prove this, take any factorizing cover $\mathscr V$ of $V\subseteq Y$ where
    $V$ is open, then $U\times\mathscr V$ forms a factorizing cover of $U\times V$,
    and the \v{C}ech complex $\check{C}(\mathscr V,\underline\pi_*(A)(U))$
    is equal to $\check{C}(U\times\mathscr V,A)$. Hence the map 
    $\check{C}(\mathscr V,\underline\pi_*(A)(U))\to\underline\pi_*(A)(U)(V)$
    is equal to $\check{C}(U\times\mathscr V,A)\to A(U\times V)$, which is
    an equivalence since $A$ is a factorization algebra. 

    Next we check that if $A$ is a factorization algebra on $X\times Y$,
    then $\underline\pi_*(A)$ is a factorization algebra on $X$ with values in
    $\FA(Y)$. To prove this, take any factorizing cover $\mathscr U$ 
    of $U\subseteq X$ where $U$ is open, it suffices to show that the map
    $\check{C}(\mathscr U,\underline\pi_*(A)(-)(V))\to\underline\pi_*(A)(U)(V)$
    is an equivalence for all $V\subseteq Y$. But this follows from the same argument
    as above. Therefore, $\underline\pi_*$ restricts to a functor 
    $\FA(X\times Y)\to\FA(X,\FA(Y))$.

    If $A$ is locally constant, applying Proposition \ref{b} to the two natural
    projections, it is shown that $\underline\pi_*(A)\in\FA^{lc}
    (X,\FA^{lc}(Y))$.

    We shall now show that $\underline\pi_*$ is an equivalence in the
    locally constant case by giving an inverse to it. Take a metric on $X,Y$,
    and take $\mathscr U,\mathscr V$ to be the collection of all
    strictly convex subset of $X,Y$, respectively. Then $\mathscr U,\mathscr V$
    are stable under intersection, form bases of $X,Y$, and are factorization bases
    of $X,Y$, respectively. By Proposition \ref{b}, in order to give a
    factorization algebra on $X\times Y$, it suffices to give a
    factorization algebra on $\mathscr U\times\mathscr V$,
    since this is a factorizing basis for $X\times Y$. Now for any
    $U\in\mathscr U,V\in\mathscr V$, $U,V$ must be homeomorphic to the euclidean
    disks. Thus the construction of the structure maps for opens in 
    $\mathscr U\times\mathscr V$ that lies in $U\times V$, restricts the problem
    to the case that $X=\R^m,Y=\R^n$ for some integers $m,n$. By 
    Theorem \ref{k}\footnote{The only usage of the proposition
    is to prove the Fubini formula in Example \ref{u}, so there is no circularity.}
    and the Dunn's theorem (Theorem \ref{v}), the functor
    \[\underline\pi_*\:\FA^{lc}(\R^{m+n})\to\FA^{lc}(\R^m,\FA^{lc}(\R^n))\] is indeed
    an equivalence, making it possible to define a
    $\mathscr U\times\mathscr V$-factorization algebra $j(B)$ for arbitrary
    manifolds $X,Y$ from any $B\in\FA^{lc}(X,\FA^{lc}(Y))$, by previous discussion.
    We denote $j(B)$ again to be the factorization algebra on $X\times Y$ obtained
    from the $\mathscr U\times\mathscr V$-factorization algebra $j(B)$ by extending
    from a basis. Again, by Dunn's theorem, $j(B)$ is locally constant.

    It remains to show that $j$ is an inverse to $\underline\pi_*$. This is
    again from Proposition \ref{b}, which tells us that the factorization algebra
    extending a factorization algebra on a factorizing basis is unique.
\end{proof}

We next discuss how to define the pullback of a factorization algebra.
We do this in the case that $f\:X\to Y$ is an open immersion of smooth manifolds.
Suppose $B$ is a factorization algebra on $Y$. Define \[\mathscr U_f=
\{U\subseteq X\mid U\text{ open, }f\:U\xrightarrow{\cong}f(U)\}.\]
Then $\mathscr U_f$ is a factorizing basis for $X$. For any $U\in\mathscr U_f$,
define $f^*(B)(U)=B(f(U))$. The axioms of a factorization algebra ensure that
$f^*(B)$ is a $\mathscr U_f$-factorization algebra. By Proposition \ref{c}
it can be extended to a factorization algebra on $X$. We denote this 
factorization algebra again by $f^*(B)$ and call it the \term{pullback}
of $B$ along $f$. By the functoriality of the construction above and 
Proposition \ref{c}, $f^*\:\FA(Y)\to\FA(X)$ is a functor. Moreover by the 
uniqueness property of Proposition \ref{c}, for any open immersions 
$f\:X\to Y, g\:Y\to Z$ and $C$ a factorization algebra on $Z$, the restriction
of $(g\circ f)^*(C)$ and $f^*g^*(C)$ on \[\mathscr U:=\{U\subseteq X\mid 
U\in\mathscr U_f,U\in \mathscr U_{g\circ f},f(U)\in \mathscr U_g\}\] are 
equal, since $\mathscr U$ is a factorizing basis, $(g\circ f)^*(C)$ 
and $f^*g^*(C)$ are equivalent. Since the above construction is functorial,
the functors $(g\circ f)^*$ and $f^*g^*(C)$ are equivalent.

\begin{remark}
    We note that the functors $f_*$ and $f^*$ are \textit{not} adjoint functors.
    To see this, take $X=\{c,d\}$ to be the discrete space with two points,
    and $Y=*$. A factorization algebra $B$ on $Y$ in $\cal C$ is an object
    $B\in\cal C$ and a map $b\:\mathbb1\to B$ in $\cal C$; while a
    factorization algebra $A$ on $X$ in $\cal C$ consists of two objects 
    $C,D\in\cal C$, two maps $c:\mathbb1\to C,d:\mathbb1\to D$ in $\cal C$,
    corresponding to the objects $A(c)$ and $A(d)$. By direct computation,
    the factorization algebra $f_*(f^*(B))$ is $B\ox B$, and the 
    factorization algebra $f^*(f_*(A))$ is $A\ox A$. But there does not exist a
    natural map $A\ox A\to A$, nor does there exist a natural map $B\ox B\to B$,
    thus $f_*$ and $f^*$ are not adjoint functors.

    Nevertheless, if $f\:X\to Y$ is an open immersion of smooth manifolds,
    $A$ is a factorization algebra on $A$, then for any $U\in\mathscr U_f$
    there exists a map $$A(U)\xrightarrow{\text{induced by }U\hookrightarrow
    f^{-1}(f(U))}A(f^{-1}(f(U)))\cong (f^*(f_*(A)))(U),$$ thus inducing a map of
    $\mathscr U_f$-factorization algebras $A\to f^*(f_*(A))$. By Proposition
    \ref{c}, it induces a map of factorization algebras $A\to f^*(f_*(A))$.
    Moreover this construction is compatible with morphisms between
    factorization algebra. Therefore we \textit{do} obtain a natural transformation
    between functors $\mathbb1\to f^*f_*\:\FA(X)\to\FA(X)$.
\end{remark}

We end this section by gluing factorization algebras. Suppose 
$\mathscr U=\{U_i\}_{i\in I}$ is an open cover of a space $X$,
such that any point is contained in only finitely many open sets 
in $\mathscr U$. We define a set of ``gluing data'' to be as follows:

\begin{definition}
    A set of \term{gluing data} for $\mathscr U$ consists of a factorization algebra $A_J$
    on the intersection $V_j$ of the elements of the collection $\{U_j\}_{j\in J}$ 
    for every finite set $J\subseteq I$, an equivalence $r_{J,j}:A_J
    \to(A_{J-\{j\}})|_{V_J}$ for any $J$ and $j\in J$, such that the following
    diagram commutes for every $J$ and $j,k\in J$: \[\xymatrix{
    A_J\ar[r]^{r_{J,j}}\ar[d]_{r_{J,k}}&(A_{J-\{j\}})|_{V_J}\ar[d]^{r_{J-\{j\},k}}\\
    (A_{J-\{k\}})|_{V_J}\ar[r]^{r_{J-\{k\},j}}&A_{J-\{j,k\}}.}\]
\end{definition}

Now, if we have a set of gluing data, we may take a factorizing basis $\mathscr V$ 
of $X$, which is given by all open subsets of $U_i$ for all $i\in I$. We define
$A(V)=A_J(V)$ where $J$ is the largest subset of $I$ satisfying 
$V_J\supseteq V$. It is easy to verify that the coherent property of the
structure maps $r_{J,j}$ gives $A$ a structure of a $\mathscr V$-factorization
algebra. Moreover, if we define $r_J$ to be the canonical restriction 
equivalence $A_J\to A|_{V_J}$, the maps $r_J$ and $r_{J-\{j\}}\circ r_{J,j}$
are equal for any $J$ and $j\in J$. Combining Proposition \ref{b}, we obtain
the following proposition:

\begin{proposition}\label{t}
    Given a set of gluing data, there exists a factorization algebra $A$ on $X$ that is unique up to equivalence,
    whose restriction to each $V_J$ is canonically equivalent to $A_J$.
\end{proposition}

In particular, given two open sets $U,V$ of $X$ such that $X=U\cup V$, there 
exists an equivalence of $\infty$-categories \[\FA(X)\cong\FA(U)
\mathop{\operatorname{\times}}\limits_{\FA(U\cap V)}\FA(V).\]

In the case of locally constant factorization algebras, we state the
following proposition, which says that being locally constant is some sort of a
``local property'':
\begin{proposition}\label{l}
    Suppose $M$ is a topological manifold and $A$ is a factorization algebra on $M$.
    If there exists an open cover $\mathscr U$ of $M$ such that $A|_U$ is
    locally constant for every $U\in\mathscr U$, then $A$ is locally constant.
\end{proposition}

This is Proposition 13, \cite{Gin13}, whose proof is given in Section 9.1
of the paper. It is really technical, and does not give much enlightenment
to the proof of other propositions; so we will omit it here.

Combining the two propositions above, we see that if we have a gluing data
consisting of locally constant factorization algebras, the resulting global
factorization algebra is locally constant. 

\section{Factorization homology}

Factorization homology is a type of homology theory on
manifolds. They produce invariants of manifolds, just as classical
homology theories produce invariants of topological spaces. In some specific
manifolds, factorization homology reduces to some well-known structures,
such as Hoschchild homology. In this section, we will introduce homology
theories on manifolds and factorization homology, and study their properties,
as well as their relationships with factorization algebras, especially locally
constant factorization algebras.

\subsection{Homology theory for manifolds}

We first specify an interesting category of manifolds that we will study.

\begin{definition}
    Consider the topological category of $n$-dimensional manifolds
    (without boundaries), with morphism spaces the spaces of all topological embeddings.
    We define the nerve of the topological category to be $\Mfld_n$, and regard
    this $\infty$-category as the \term{$\infty$-category of
    $n$-dimensional manifolds}.
\end{definition}

Note that we consider topologial manifolds instead of smooth manifolds.
However, we may also consider smooth manifolds, whose terminology will be
introduced later.

We may also add other structures that we may be interested in, such as
orientations, framings, etc. These structures may be interpreted as follows:

\begin{definition}
    Suppose $E\to X$ is a topological $n$-dimensional vector bundle, which is 
    equivalent to a space $X$ together with a homotopy class of maps
    $e\in[X,B\Aut(\R^n)]$, classifying $E$, where $\Aut(\R^n)$ is the
    automorphism topological group of $\R^n$. We define an \term{$(X,e)$-structure}
    on an $n$-dimensional manifold $M$ to be a map $f\:M\to X$, such that
    the tangent bundle $TM$ is classified by the map $e\circ f$.
    (In other words, if we pullback $E$ along $f$, we obtain $TM$.)
    We consider the (homotopy) pullback \[\Mfld_n^{(X,e)}:=\Mfld_n
    \mathop{\operatorname{\times}}\limits_{\S_{/B\Aut(\R^n)}}\S_{/X},\]
    we regard this $\infty$-category as the \term{$\infty$-category of 
    $n$-dimensional manifolds with an $(X,e)$-structure}. We define
    $\Emb^{(X,e)}(M,N)$ to be the space of $(X,e)$-embeddings from $M$ to $N$.
\end{definition}

\begin{example}
    \begin{enumerate}[i)]
        \item If $X=*$, then $E$ is trivial, and an $(X,e)$-structure on $M$ is
        a trivialization of $TM$. Hence $\Mfld_n^{(X,e)}$ consists of framed manifolds.
        We define $\Mfld_n^{\mathrm{fr}}$ to be $\Mfld_n^{(*,e)}$, and call it the 
        \term{$\infty$-category of framed $n$-dimensional manifolds}. Note that
        a morphism in $\Mfld_n^{\mathrm{fr}}$ from $(M,f)$ to $(N,g)$ is an embedding from $M$
        to $N$ together with a homotopy between the framing that already exists on 
        $TM$ and the pullback of the framing on $TN$.

        \item If $X=BO(n)$, and $e$ is the canonical map $BO(n)\to B\Aut(\R^n)$,
        then $\Mfld_n^{(X,e)}$ consists of smooth manifolds. This is essentially because
        the map from $O(n)$ to the topological group of all diffeomorphisms on $\R^n$ and the characterization
        of smooth manifolds in terms of their microbundle structure, which is shown in
        \cite{KS77}. We define $\Mfld_n^{\mathrm{un}}$ to be $\Mfld_n^{(BO(n),e)}$,
        and call it the \term{$\infty$-category of smooth $n$-dimensional manifolds}.
        A morphism in $\Mfld_n^{\mathrm{un}}$ is equivalent to a smooth embedding.

        \item If $X=BSO(n)$, and $e$ is the canonical map $BSO(n)\to B\Aut(\R^n)$,
        then $\Mfld_n^{(X,e)}$ consists of oriented smooth manifolds. We define
        $\Mfld_n^{\mathrm{or}}$ to be $\Mfld_n^{(BSO(n),e)}$, and call it the 
        \term{$\infty$-category of smooth oriented $n$-dimensional manifolds}.
        A morphism in $\Mfld_n^{un}$ is equivalent to a smooth oriented embedding.

        \item If $X$ is an $n$-dimensional manifold, then we may take $e$ to be the
        canonical map characterizing $TX$. We define $\Mfld_n^{(X,TX)}$ to be the
        associated $\infty$-category with respect to $(X,e)$. Note that every open set
        of $X$ is canonically an object of $\Mfld_n^{(X,TX)}$.
    \end{enumerate}
\end{example}

For any $M,N\in\Mfld_n^{(X,e)}$, $M\amalg N$ is canonically a manifold with
an $(X,e)$-structure. therefore, $(\Mfld_n^{(X,e)},\amalg)$
is a symmetric monoidal $\infty$-category, where the axioms are easily
verified. Note that there is no embedding $M\amalg M\to M$ in general, 
so that $\amalg$ is not a coproduct on $\Mfld_n^{(X,e)}$. Indeed, 
$\Mfld_n^{(X,e)}$ does not admit finite coproducts.

We now take a special $n$-dimensional manifold, $\R^n$. Since it is
contractible, for any $(X,e)$ there exists a canonical framing on $\R^n$
such that it lies in the $\infty$-category $\Mfld_n^{(X,e)}$.
Moreover, unlike other manifolds, there exists $(X,e)$-structured embeddings
$\coprod_{i=1}^m\R^n\to\R^n$ for any integers $m,n$ and any $(X,e)$. We define 
$\Disk_n^{(X,e)}$ to be the full subcategory of $\Mfld_n^{(X,e)}$,
spanned by the disjoint unions of $\R^n$. We observe the following property:

\begin{proposition}
    $\Disk_n^{(X,e)}$ is an $\infty$-operad, with only one color, being
    $\R^n$.
\end{proposition}

\begin{proof}
    This is by Proposition \ref{d}, since it is the nerve of a topological
    single colored operad, with morphism space $\Disk_n^{(X,e)}(r)$ the subspace
    of $\Hom(\coprod_r\R^n,\R^n)$ consisting of all $(X,e)$-structured embeddings
    $\coprod_r\R^n\to\R^n$.
\end{proof}

We now specify a case: $X=*$. In this case $\Disk_n^{\mathrm{fr}}(r)$ is the subspace
of $\Hom(\coprod_r\R^n,\R^n)$ consisting of all framed embeddings 
$\coprod_r\R^n\to\R^n$. In this case we have the following proposition, see \cite{MO181828}:

\begin{proposition}
    The $\infty$-operads $\Disk_n^{\mathrm{fr}}$ and $\E_n$ are equivalent.
\end{proposition}

Thus, the $\infty$-category of algebras over the $\infty$-operad $\Disk_n^{\mathrm{fr}}$
is equivalent to the $\infty$-category of $E_n$-algebras.

\begin{definition}
    Suppose $\cal C$ is a symmetric monoidal $\infty$-category.
    We define $\Disk_n^{(X,e)}\-\Alg(\cal C)$ to be the $\infty$-category of 
    algebras over $\Disk_n^{(X,e)}$ in $\cal C$. In some cases we give 
    the $\infty$-category $\Disk_n^{(X,e)}\-\Alg(\cal C)$ some special names:
    \begin{enumerate}[i)]
        \item If $X=BO(n)$, and $e$ is the canonical map $BO(n)\to B\Aut(\R^n)$,
        we call the objects of $\Disk_n^{\mathrm{un}}\-\Alg$ the \term{unoriented (smooth) $E_n$-algebras};
        \item If $X=BSO(n)$, and $e$ is the canonical map $BSO(n)\to B\Aut(\R^n)$,
        we call the objects of $\Disk_n^{\mathrm{or}}\-\Alg$ the \term{oriented (smooth) $E_n$-algebras}.
    \end{enumerate}
\end{definition}
    
We are now able to present what a homology theory for ($n$-dimensional)
manifolds is. We interprete this in a way similar to the Eilenberg--Steenrod
axioms. First, we consider the excision axiom. 

\begin{proposition}\label{s}
    Suppose $H\:(\Mfld_n^{(X,e)},\coprod)\to\cal C$ is a symmetric monoidal functor. Then:
    \begin{enumerate}[i)]
        \item For any $(n-s)$-dimensional manifold $N$ such that $N\times\R^s$ 
        is endowed with an $(X,e)$-structure, $H(N\times\R^s)$ is an $E_s$-algebra in
        $\cal C$;

        \item Suppose $M$ is an $(X,e)$-structured manifold such that an end of $M$
        is trivialized as $N\times\R$ such that $N$ is of codimension $1$ and the open
        part of $M$ lies in the neighborhood of $N\times\{-\infty\}$, then
        $H(M)$ is a left (pointed) module\footnote{We assume once and for all that
        all modules are pointed.} of the $E_1$-algebra $H(N\times\R)$; dually, if
        the open part of $M$ lies in the neighborhood of $N\times\{+\infty\}$, then
        $H(M)$ is a right module of the $E_1$-algebra $H(N\times\R)$;

        \item $H(\R^n)$ is a $\Disk_n^{(X,e)}$-algebra.
    \end{enumerate}
\end{proposition}

\begin{proof}
    \begin{enumerate}[i)]
        \item Notice that for any finite sets $I,J$, any framed embedding 
        $\coprod_I\R^s\to\coprod_J\R^s$ induces a $(X,e)$-embedding
        $\coprod_I(N\times\R^s)\to\coprod_J(N\times\R^s)$, thus it induces a map 
        $\gamma_{I,J}^N$ from the space of framed embeddings from $\coprod_I\R^s$ to
        $\coprod_J\R^s$ to the space of $(X,e)$-embeddings from 
        $\coprod_I(N\times\R^s)$ to $\coprod_J(N\times\R^s)$. Therefore, $N\times\R^s$
        is an $E_s$-algebra object in the $\infty$-category $\Mfld_n^{(X,e)}$. Since $H$ 
        is symmetric monoidal, $H(N\times\R^s)$ is an $E_s$-algebra in $\cal C$.

        \item It suffices to give a map $$\Emb^{\mathrm{fr}}_0\9\9\coprod_I\R\0
        \amalg(-\infty,0],(-\infty,0]\0\ox\bigox_IH(N\times\R)\ox H(M)\to H(M),$$
        where $\Emb^{\mathrm{fr}}_0$ is the subspace of $\Emb^{\mathrm{fr}}$ consisting of maps
        that map $0$ in the last component to $0$. Using i), the map is given by
        $$\begin{aligned}&\Emb^{\mathrm{fr}}_0\9\9\coprod_I\R\0\amalg(-\infty,0],(-\infty,0]\0
        \ox\bigox_IH(N\times\R)\ox H(M)\\\xrightarrow{\gamma^N_{I\amalg*,*}}&
        \Emb^{\mathrm{fr}}\9\9\coprod_I(N\times\R)\0\amalg(N\times(-\infty,0]),
        N\times(-\infty,0]\0\\&\hspace{3em}\ox\bigox_IH(N\times\R)\ox H(M)\\\to&\Emb^{\mathrm{fr}}_0
        \9\9\coprod_I(N\times\R)\0\amalg M,M\0\ox H\9\9\coprod_I(N\times\R)\0
        \amalg M\0\\\to& H(M).\end{aligned}$$

        \item This is because by definition $\R^n$ is an $\Disk_n^{(X,e)}$-algebra
        object in the $\infty$-category $\Mfld_n^{(X,e)}$ and $H$ is symmetric monoidal.
    \end{enumerate}
\end{proof}

With this proposition, we are able to express the excision axiom for
homology theories for manifolds, which is again due to \cite{Gin13}:

\begin{definition}
    A \term{homology theory} for $(X,e)$-manifolds with values in a symmetric
    monoidal stable $\infty$-catgeory $\cal C$ is a functor 
    $$H:\Mfld_n^{(X,e)}\to\mathcal C,\qquad M\mapsto H(M),$$ 
    such that $H$ is symmetric monoidal, preserves sequential colimits,
    and satisfies the \term{excision axiom}:
    \begin{itemize}
        \item For any $(X,e)$-manifold $M$, if there exists a codimension $1$
        submanifold $N$ of $M$ and a trivialization $N\times\R$ of its neighborhood
        such that $M$ is decomposible as $R\amalg_{N\times\R}L$, where $R, L$ are
        submanifolds of $M$ glued along $N\times\R$, then the natural map 
        $$H(L)\mathop{\operatorname{\ox}}\limits_{H(N\times\R)}H(R)\to H(M)$$
        is an equivalence.
    \end{itemize}
    The \term{$\infty$-category of homology theories} is defined to be the
    full subcategory of $\Alg(\Mfld_n^{(X,e)})$ spanned by homology theories,
    denoted $\HT(\Mfld_n^{(X,e)})$.
\end{definition}

\subsection{Factorization homology}

Now we will introduce a specific homology theory, called factorization
homology. We will discuss the properties of factorization homology,
and we will prove the following result, which is from \cite{AF12}:

\begin{theorem}\label{e}
    For any $\Disk_n^{(X,e)}$-algebra $A$, there exists, up to contractible choices,
    a \textnormal{unique} homology theory for $(X,e)$-manifolds, satisfying the
    \term{dimension axiom}:
    \begin{itemize}
        \item The value of the homology theory on $\R^n$ is naturally equivalent to $A$;
    \end{itemize} 
    which is exactly the factorization homology with coeffecients in $A$
    (whose definition will be given below).
\end{theorem}

We start with the definition of the factorization homology.

\begin{definition}
    Any $\Disk_n^{(X,e)}$-algebra $A$ defines a functor 
    $\Disk_n^{(X,e)}\to\cal C$. On the other hand, if $M$ is in $\Mfld_n^{(X,e)}$,
    then $M$ defines a functor $E_M:(\Disk_n^{(X,e)})^{\mathrm{op}}\to\K$, which maps
    an object in $\Disk_n^{(X,e)}$ to the space of $(X,e)$-embeddings from that
    object to $M$. Together $M$ and $A$ defines a functor:
    $$E_M\ox A\:(\Disk_n^{(X,e)})^{op}\times\Disk_n^{(X,e)}\to\K\times\cal C
    \xrightarrow{\ox}\cal C.$$ We define the
    \term{factorization homology} of $M$ with coeffecients in $A$ to be the
    (homotopy) coend of the above functor: $$\int_MA:=E_M\oxx_{\Disk_n^{(X,e)}}A.$$
\end{definition} 

Note that by the definition of a coend, factorization homology can also be
interpreted as $$\int_MA=\colim((\Disk_n^{(X,e)})_{/M}\to\Disk_n^{(X,e)}
\xrightarrow{A}\cal C).$$ Thus, factorization homology preserves
sequential colimits.

We shall now verify some properties of factorization homology. 
The fully faithful symmetric monoidal functor $i\:\Disk_n^{(X,e)}
\to\Mfld_n^{(X,e)}$ induces a functor $$i^*\:\Alg(\Mfld_n^{(X,e)},\mathcal C)
\to\Disk_n^{(X,e)}\-\Alg(\cal C).$$ The following proposition identifies
factorization homology as a left adjoint to this functor provided $\cal C$
admits some additional properties:

\begin{proposition}\label{r}
    Suppose $\cal C$ is \term{$\ox$-presentable} (which means that $\cal C$
    is presentable and the monoidal structure distributes over small colimits). Then
    there exists a left adjoint $$i_!\:\Disk_n^{(X,e)}\-\Alg(\mathcal C)\to
    \Alg(\Mfld_n^{(X,e)},\mathcal C)$$ to $i_*$, which is given by 
    $$i_!(A)(M)=\int_MA.$$
\end{proposition}

\begin{proof}
    From the theory of $\infty$-categories, we know that there exists a functor
    $$i_!\:\Fun(\Disk_n^{(X,e)},\mathcal C)\to\Fun(\Mfld_n^{(X,e)},\mathcal C),$$
    that is the left adjoint to the restriction functor $$i^*\:\Fun(\Mfld_n^{(X,e)},
    \mathcal C)\to\Fun(\Disk_n^{(X,e)},\mathcal C).$$ Moreover, $i_!(A)$ sends
    every manifold $M$ to $$\colim((\Disk_n^{(X,e)})_{/M}\to\Disk_n^{(X,e)}
    \xrightarrow{A}\cal C),$$ which coincides with the definition of 
    factorization homology. Therefore, it suffices to show that the image of 
    $\Disk_n^{(X,e)}\-\Alg(\mathcal C)$ under $i_!$ lies in $\Alg(\Mfld_n^{(X,e)},
    \mathcal C)$.

    To prove this, it is equivalent to prove that for any $A\in\Disk_n^{(X,e)}
    \-\Alg(\mathcal C)$, and for any map $f\:\<n\>\to\<m\>$, the following diagram
    commutes, up to an equivalence: $$\xymatrix{(\Mfld_n^{(X,e)})^n
    \ar[rrr]^{(i_!(A))^n}\ar[d]_{f_*}&&&\mathcal C^n\ar[d]^{f_*}\\
    (\Mfld_n^{(X,e)})^m\ar[rrr]^{(i_!(A))^m}&&&\mathcal C^m.}$$ 
    Since every map $f$ can be factored to be the composition
    of a surjective active map followed by an injective active map followed by
    an inert map (where an \term{active map} is a map such that the inverse image
    of $*$ is $*$), it suffices to prove this in these three cases.

    If $f$ is inert, then $f_*$ is a projection, the commutativity is obvious.

    If $f$ is injective and active, since $A$ is symmetric monoidal, $A$
    preserves the monoidal unit, thus $i_!(A)$ also preserves the monoidal unit,
    therefore $f_*$ is the canonical inclusion, the commutativity follows.

    If $f$ is surjective and active, by taking all fibers of the map $f$,
    we may assume that $m=1$. Then the map $f_*$ is the $n$-fold tensor product.
    For any $(M_1,\cdots,M_n)\in(\Mfld_n^{(X,e)})^n$, we consider the following 
    diagram:
    $$\xymatrix{\prod\limits_{i=1}^n(\Disk_n^{(X,e)})_{/M_i}\ar[r]^{\coprod}\ar[d]
    &(\Disk_n^{(X,e)})_{/\coprod\limits_{i=1}^nM_i}\ar[d]\\(\Disk_n^{(X,e)})^n
    \ar[r]^{\coprod}\ar[d]_{A^n}&\Disk_n^{(X,e)}\ar[d]^A\\\mathcal C^n
    \ar[r]^{\bigox}&\mathcal C.}$$
    This diagram is commutative. Moreover the top row is fully faithful and
    essentially surjective, which means that it is an equivalence. Therefore, 
    we obtain the following natural equivalence $$\begin{aligned}&\colim\9(\Disk_n^{(X,e)})
    _{/\coprod\limits_{i=1}^nM_i}\to\Disk_n^{(X,e)}\xrightarrow{A}\mathcal C\0
    \\&\hspace{2em}\cong\colim\9\prod_{i=1}^n(\Disk_n^{(X,e)})_{/M_i}\to
    (\Disk_n^{(X,e)})^n\xrightarrow{A^n}\mathcal C^n\xrightarrow{\bigox}
    \mathcal C\0.\end{aligned}$$ On the other hand, since $\ox$ and $\colim$ commute
    in $\cal C$, there exists a natural equivalence $$\begin{aligned}&\colim\9\prod_{i=1}^n
    (\Disk_n^{(X,e)})_{/M_i}\to(\Disk_n^{(X,e)})^n\xrightarrow{A^n}\mathcal C^n
    \xrightarrow{\bigox}\mathcal C\0\\&\hspace{2em}\cong\bigox_{i=1}^n\colim
    \9(\Disk_n^{(X,e)})_{/M_i}\to\Disk_n^{(X,e)}\xrightarrow{A}\mathcal C\0.\end{aligned}$$
    The composition of the two equivalences yields the commutativity of the diagram,
    as desired.
\end{proof}

The above proposition shows that factorization homology can be expressed as
symmetric monoidal left Kan extension, at least when $\cal C$ is
$\ox$-presentable. In particular, factorization homology is symmetric monoidal
on $M$.

Next we show the independence of factorization homology with respect to $X$.

\begin{proposition}
    Suppose $\vp\:(X,e)\to(X',e')$ is a map of spaces over $B\Aut(\R^n)$, $M$ is an
    $(X,e)$-manifold, $A$ is an $\Disk_n^{(X',e')}$-algebra Let $\vp M$ denote the
    $(X',e')$-manifold $M$, $\vp A$ denote the $\Disk_n^{(X,e)}$-algebra $A$,
    then there exists a natural equivalence $$\int_{\vp M}A\cong\int_M\vp A.$$
\end{proposition}

\begin{proof}
It suffices to show that the $\infty$-categories $(\Disk_n^{(X,e)})_{/M}$
and $(\Disk_n^{(X',e')})_{/M}$ are equivalent. By definition of 
$\Disk_n^{(X,e)}$ and $\Disk_n^{(X',e')}$, it suffices to show that
the map $$((\S_{/B\Aut(\R^n)})_{/X})_{/M}\to((\S_{/B\Aut(\R^n)})_{/X'})_{/M}$$
is an equivalence of $\infty$-categories. But this map is an isomorphism.
\end{proof}

\begin{remark}
Factorization homology \textit{does} depend on the choice of the map 
$M\to B\Aut(\R^n)$. For example, if $M=\R^n$ is equipped with its standard
framing and $A$ is an $E_n$-algebra, then $\int_MA=A$. However, if $M=\R^n$
is equipped with the opposite framing, then $\int_MA=A^{op}$.
In particular, factorization homology is not homotopy invariant on manifolds.
\end{remark}

Next we discuss the excision property of factorization homology.

\begin{proposition}
    Factorization homology satisfies the excision property:
    for any $(X,e)$-manifold $M$, if there exists a codimension $1$
    submanifold $N$ of $M$ and a trivialization $N\times\R$ of its neighborhood
    such that $M$ is decomposable as $R\coprod_{N\times\R}L$, where $R, L$ are
    submanifolds of $M$ glued along $N\times\R$, then the map
    $$\int_LA\oxx_{\int_{N\times\R}A}\int_RA\to\int_MA$$ is an equivalence.
\end{proposition}

\begin{proof}
    First, we write the definition of a factorization algebra in another way. The
    objects in $(\Disk_n^{(X,e)})_{/M}$ are those open subsets in $M$, that is
    homeomorphic to a disjoint union of disks. Then if we define $\Ball(M)$
    to be the collection of the open subsets in $M$ that are homeomorphic to a disk, then $$\int_MA
    =\colim_{U_1,\cdots,U_l\in\Ball(M),\text{ pairwise disjoint}}\bigox_{i=1}^l
    \int_{U_i}A.$$

    Now, the tensor product $$\int_LA\oxx_{\int_{N\times\R}A}\int_RA$$ can be
    interpreted as $$\colim\9\int_LA\ox\int_{N\times\R}A\ox\int_RA\rightrightarrows
    \int_LA\ox\int_RA\0,$$ where the two maps are the two module structures. 
    Define $L_0$ and $R_0$ to be $L-N\times(-\infty,0]$ and $R-N\times[0,\infty)$,
    respectively. Then we have canonical equivalences $$\int_LA\cong\int_{L_0}A,
    \qquad\int_RA\cong\int_{R_0}A.$$ Under this equivalence, using
    the definition of the module structure, the colimit above is equivalent to
    $$\colim_{U_1,\cdots,U_l\in\widetilde{\Ball},\text{ pairwise disjoint}}
    \bigox_{i=1}^l\int_{U_i}A,$$ where $\widetilde{\Ball}=\Ball(L_0)\cup\Ball(R_0)
    \cup\Ball(N\times\R)$. Therefore, it suffices to show that the map
    $$\colim_{U_1,\cdots,U_l\in\widetilde{\Ball},\text{ pairwise disjoint}}
    \bigox_{i=1}^l\int_{U_i}A\to\colim_{U_1,\cdots,U_l\in\Ball(M),
    \text{ pairwise disjoint}}\bigox_{i=1}^l\int_{U_i}A$$ is an equivalence.
    But if $U\in\Ball(M)$ intersects $N\times\{0\}$, we may take
    an open set $U'\subseteq U$ such that $U'\in\Ball(N\times\R)$, thus inducing
    a map $\int_{U'}A\to\int_UA$, which is an equivalence since both sides
    are naturally equivalent to $A$. Thus the colimit on the right hand side 
    is equivalent to the colimit on the left hand side, completing the proof.
\end{proof}

Combining the results above, we obtain the following result:

\begin{proposition}
    For any $\Disk_n^{(X,e)}$-algebra $A$, the functor $\int_- A$ is a
    homology theory for $(X,e)$-manifolds.
\end{proposition}

\begin{example}[The relationship with Hochschild homology]
    Hochschild homology is an example of factorization homology.
    Take $M=S^1$ and take a framing on $M$ induced by its Lie group structure. 
    Then $M$ can be viewed as the gluing: $M=\R\coprod_{\{1,-1\}\times\R}\R$,
    where $\R$ is endowed with the trivial framing. Now, let $A$ be an associative
    algebra (or more generally an $A_\infty$-algebra), then $A$ is a
    $\Disk_1^{\mathrm{fr}}$-algebra, and the factorization homology on $M$ with coefficient
    $A$ is $$\int_MA\cong\int_{\R}A\oxx_{\int_{\{1\}\times\R}A\ox\int
    _{\{-1\}\times\R}A}\int_{\R}A=A\oxx_{A\ox A^{op}}A,$$ which is the 
    Hochschild homology of $A$.
\end{example}

We may now prove Theorem \ref{e}:

\begin{theorem}
    If $\cal C$ is a symmetric monoidal $\ox$-presentable $\infty$-category,
    then there exists a pair of equivalence 
    $$\int\:\Disk_n^{(X,e)}\-\Alg(\mathcal C)\rightleftarrows\HT(\Mfld_n^{(X,e)},
    \mathcal C):\!\mathrm{ev}_{\R^n}$$ between $\Disk_n^{(X,e)}$-algebras
    and homology theories for $(X,e)$-manifolds.
\end{theorem}

\begin{proof}
    By Proposition \ref{r}, $\int$ is a left adjoint to the functor 
    $\mathrm{ev}_{\R^n}$. The unit of the adjunction is an equivalence, because
    $\Disk_n^{(X,e)}\to\Mfld_n^{(X,e)}$ is fully faithful, and the Kan extension
    along a fully faithful functor restricts as the original functor. The counit
    of the adjuntion evaluates on a symmetric monoidal functor $F$ as a morphism
    $\int F(\R^n)\to F$. It suffices to show that this map is an equivalence, which
    means we only have to prove that the map $\int_M F(\R^n)\to F(M)$ is an
    equivalence for any $(X,e)$-manifold $M$. We denote $A:=F(\R^n)$.

    If $M$ is the disjoint union of some copies of $\R^n$, this is obvious. Suppose that it is true for
    $M=S^i\times\R^{n-i}$ for some $0\le i<n$. Then for the case 
    $S^{i+1}\times\R^{n-i-1}$, it can be decomposed as $\R^{i+1}_-\times\R^{n-i-1}
    \coprod_{S^{i+1}\times\R^{n-i}}\R^i_+\times\R^{n-i-1},$ since both
    $F$ and $\int_- A$ have the excision property, we have
    $$\begin{aligned}\int_{S^{i+1}\times\R^{n-i-1}}A
    &\cong\int_{\R^{i+1}_-\times\R^{n-i-1}}A
    \mathop{\operatorname{\ox}}\limits_{\int_{S^i\times\R^{n-i}}A}
    \int_{\R^{i+1}_+\times\R^{n-i-1}}A\\&\cong F(\R^{i+1}_-\times\R^{n-i-1})
    \mathop{\operatorname{\ox}}\limits_{F(S^i\times\R^{n-i})}
    F(\R^{i+1}_+\times\R^{n-i-1})\\&\cong F(S^{i+1}\times\R^{n-i-1}).\end{aligned}$$
    Thus the map is an equivalence for $M=S^{i+1}\times\R^{n-i-1}$. By induction,
    for any $0\le i\le n$, the above map is an equivalence 
    for $M=S^i\times\R^{n-i}$.

    Now, for any manifold $M$ that is obtained from a manifold $N$ making the 
    counit map an equivalence by adding a handle of index $i+1$, $M$ can be
    expressed as the coproduct $N\coprod_{S^i\times\R^{n-i}}\R^n$. Similarly as above,
    since $N,S^i\times\R^{n-i},\R^n$ all make the counit map an equivalence,
    so does $M$. Therefore, by induction, the counit map is an equivalence for all
    handlebodies.

    We finish the proof by noticing that all smooth manifolds can be regarded as the direct
    colimit of inclusions of handlebodies, and the functors $\int A$ and $F$
    preserve sequential colimits.
\end{proof}

\subsection{Relation to factorization algebra and applications to
\texorpdfstring{$E_n$}{En}-algebras}

We now explain the relation between factorization homology and
locally constant factorization algebras. We will prove the following theorem,
which is from \cite{GTZ10}:

\begin{theorem}\label{f}
    There exists an equivalence of $\infty$-categories from the $\infty$-category of
    $\Disk_n^{(M,TM)}$-algebras to the $\infty$-category of locally constant
    factorization algebras on $M$, for any manifold $M$.
\end{theorem}

We will need a lemma, which is stated in \cite{Lur16} and will not be proved
here:

\begin{lemma}\label{newb}
    Suppose $A$ is an $E_n$-algebra, $\mathscr U$ is a collection
    of open sets in $\R^n$ covering $\R^n$, each homeomorphic to $\R^n$. The natural
    $\mathscr U$-prefactorization algebra with respect to $A$, given by
    $U\mapsto A(U)$, is a $\mathscr U$-factorization algebra.
\end{lemma}

We will also use the following lemma,
which presents the excision property of locally constant factorization algebras:

\begin{lemma}\label{newa}
    Suppose $M$ is a manifold such that it can be decomposed as 
    $R\coprod_{N\times\R}L$, and $A$ is a locally constant factorization algebra
    on $M$. Then $A(R)$ and $A(L)$ are right and left $E_1$-modules over $A(N\times\R)$,
    and $$A(M)\cong A(R)\mathop{\operatorname{\ox}}_{A(N\times\R)}A(L).$$
\end{lemma}

\begin{proof}
    By the proof of Proposition \ref{s}, to prove that $A(R)$ and $A(L)$ are
    right and left $E_1$-modules over $A(N\times\R)$, it suffices to show that the map
    $A(R-N\times[t,+\infty))\to A(R)$ is an equivalence for all $t\in\R$. This is
    since by Proposition \ref{b}, the map $A(N\times(a,b))\to A(N\times(a',b'))$
    is an equivalence for all $a'\le a<b\le b'$, thus the \v{C}ech complexes
    of the covers $$\{R-N\times[t-1,+\infty)\}\cup\{N\times(a,b)\mid t-2\le a<b\}$$
    and $$\{R-N\times[t-1,+\infty)\}\cup\{N\times(a,b)\mid t-2\le a<b\le t\}$$
    are naturally equivalent, since $A$ is a factorization algebra, the map
    $A(R-N\times[t,+\infty))\to A(R)$ is an equivalence for all $t\in\R$. Using
    the same argument in Proposition \ref{s}, it is shown that
    $A(R)$ and $A(L)$ are right and left $E_1$-$A(N\times\R)$-modules.

    To show that the excision property holds, we take the following
    factorizing cover of $M$: $$\begin{aligned}\mathscr U&:=\{R-N\times[t,+\infty)
    \mid-\infty<t\le\infty\}\\&\qquad\cup\{L-N\times(-\infty,t]
    \mid-\infty\le t<\infty\}\cup\{N\times(a,b)\mid-\infty\le a<b\le\infty\}.
    \end{aligned}$$
    Then we know that $A(M)$ is equivalent to the \v{C}ech complex 
    $\check{C}(\mathscr U,A)$. But by direct computation, the \v{C}ech complex
    is given by the classical two-sided Bar construction
    $B(A(R),A(N\times\R),A(L))$, which is equivalent to the tensor product
    $A(R)\ox_{A(N\times\R)}A(L)$ (by \cite{CG16}). Thus $A(M)$ is equivalent to
    $A(R)\ox_{A(N\times\R)}A(L)$, completing the proof.
\end{proof}

Using these lemmas, we state the following proposition:

\begin{proposition}\label{i}
    Suppose $M$ is an $(X,e)$-manifold. Then for any $\Disk_n^{(X,e)}$-algebra
    $A$, the map $U\mapsto\int_UA$ for any open set $U$ of $M$ defines a 
    locally constant factorization algebra, which we will denote by $i(A)$.
\end{proposition}

\begin{proof}
    Since the inclusion of open sets of $M$ are 
    $(X,e)$-embeddings, the map $U\mapsto\int_UA$ is 
    a prefactorization algebra. To prove that it is a factorization algebra, we 
    construct a factorization algebra that is locally constant, and show that
    it is naturally equivalent to $i(A)$. We deduce this by a path similar to the
    proof of Theorem \ref{e}.

    We begin with the case $M\cong\R^n$. Then $A$ is a
    $(\Disk_n^{(X,e)})_{/M}$-algebra. But by definition $(\Disk_n^{(X,e)})_{/M}$
    is naturally isomorphic to $\Disk_n^{(M,TM)}$, thus if $M=\R^n$, then $A$ is
    naturally a $\Disk_n^{(\R^n,T\R^n)}$-algebra, which is equivalent to a
    $\Disk_n^{\mathrm{fr}}$-algebra. By Lemma \ref{newb} and Proposition \ref{c}, we obtain
    a factorization algebra on $\R^n$, whose value on each open subset that is
    homeomorphic to a disk is $A$. We will denote it by $i'(A)$. 

    To prove that $i(A)$ and $i'(A)$ are naturally equivalent, we again mimic
    the proof of Theorem \ref{e}. We recall that the essential part of the proof
    of Theorem \ref{e} is the excision property. By Lemma \ref{newa},
    similar to the proof of Theorem \ref{e}, we can show
    that $i(A)(U)$ and $i'(A)(U)$ are naturally equivalent firstly for $U$
    homeomorphic to some $S^i\times\R^{n-i}$, secondly for all handlebodies,
    and finally for all open subsets. (Note that the \v{C}ech complex functor
    preserves sequential colimits of inclusions of classes of coverings
    by its definition.) Thus $i(A)$ is a factorization algebra when $M\cong\R^n$.

    Now we show that $i(A)$ is a factorization algebra for all manifolds $M$.
    By the proof of Theorem \ref{e}, we only have to consider the
    following two cases:

    \begin{enumerate}[i)]
        \item $M$ can be decomposed as the gluing $R\amalg_{N}L$, where $R,N,L$ are
        open subsets of $M$ such that $i(A)|_R,i(A)|_N,i(A)|_L$ are
        factorization algebras. In this case $i(A)|_R,i(A)|_N,i(A)|_L$ together with
        the restriction maps form a gluing data, thus defining a factorization algebra
        $i'(A)$, that is locally constant by Proposition \ref{l}; which means that
        the values of $i'(A)$ and $i(A)$ are naturally equivalent for all open disks
        in $M$. Again using the lemma, $i(A)(U)$ and $i'(A)(U)$ are naturally equivalent
        firstly for $U$ homeomorphic to some $S^i\times\R^{n-i}$, secondly for
        all handlebodies, and finally for all open subsets. Thus $i(A)$ is a
        factorization algebra.

        \item $M$ is the colimit of a sequential colimit $\colim(M_1\hookrightarrow M_2
        \hookrightarrow\cdots)$ of inclusions of open sets, such that each $M_n$
        satisfies that $i(A)|_{M_n}$ is a factorization algebra. Although in this case
        $i(A)|_{M_n}$ do not form a gluing data, the proof of Proposition \ref{t}
        is still available in the case of sequential colimit of inclusions of open sets.
        Thus the data $i(A)|_{M_n}$ defines a factorization algebra $i'(A)$, that is
        locally constant by Proposition \ref{l}; which means that the values of 
        $i'(A)$ and $i(A)$ are naturally equivalent for all open disks in $M$. 
        Again using the lemma, $i(A)(U)$ and $i'(A)(U)$ are naturally equivalent
        firstly for $U$ homeomorphic to some $S^i\times\R^{n-i}$, secondly for
        all handlebodies, and finally for all open subsets. Thus $i(A)$ is a
        factorization algebra.
    \end{enumerate}

    After discussing the two cases, we discovered that $i(A)$ is a factorization
    algebra firstly for manifolds homeomorphic to some $S^i\times\R^{n-i}$,
    secondly for all handlebodies, and finally for all $(X,e)$-manifolds,
    completing the proof.
\end{proof}

\begin{proof}[Proof of Theorem \ref{f}]
    By the above proposition, it suffices to show that the functor 
    $$i\:\Disk_n^{(M,TM)}\-\Alg\to\FA^{lc}(M),\qquad A\mapsto\9U\mapsto\int_UA\0$$
    induces an equivalence of $\infty$-categories. We take the functor $j$ from
    $\FA^{lc}(M)$ to $\Disk_n^{(M,TM)}\-\Alg$ to be the evaluation at any
    disk inside $M$, and verify that this functor is the inverse of the above one.

    \begin{enumerate}[i)]
        \item For any $A$ being a $\Disk_n^{(M,TM)}$-algebra, $j(i(A))$ is 
        $\int_{\R^n}A$, which is naturally equivalent to $A$.

        \item For any locally constant factorization algebra $B$, $i(j(B))$ sends
        every subset $U$ of $M$ to $\int_UB(\R^n)$. Now, take a metric on $M$, and take
        $\mathscr U$ to be the collection of all strictly convex subset of $M$.
        Then $\mathscr U$ is stable under intersection, forms a basis of $M$, and
        is a factorization basis of $M$. Moreover the values of $i(j(B))$ and $B$ on
        $\mathscr U$ are naturally equivalent, since $\int_{\R^n}B(\R^n)$ and $B(\R^n)$
        are naturally equivalent. By Proposition \ref{c}, $i(j(B))$ and $B$ are
        naturally equivalent.
    \end{enumerate}

    The above discussion shows that $j$ is an inverse to $i$, which completes
    the proof. 
\end{proof}

Using this theorem, we may make some interesting applications:

\begin{example}
    We take $M=\R^n$. Then a preferred choice of framing for $\R^n$ together
    with the projective map $\R^n\to*$ induces an equivalence of $\infty$-categories
    $\Disk_n^{(\R^n,T\R^n)}\to\Disk_n^{\mathrm{fr}}$, which induces equivalences of
    $\infty$-categories $$\FA^{lc}(\R^n)\cong\Disk_n^{(\R^n,T\R^n)}\-\Alg
    \cong\Disk_n^{\mathrm{fr}}\-\Alg\cong\E_n\-\Alg.$$ Hence we obtain the following theorem,
    which also appears in \cite{Lur16}:

    \begin{theorem}\label{k}
        There exists an equivalence of $\infty$-categories $$\FA^{lc}(\R^n)\cong\E_n
        \-\Alg.$$
    \end{theorem}
\end{example}

\begin{example}\label{h}
    Suppose $A$ is a locally constant prefactorization algebra. Then we may take its
    value on any open subset of the underlying space that is homeomorphic to a disk,
    which we denote by $A_0$. By Theorem \ref{f}, $i(A_0)$ is a factorization
    algebra, whose value on open disks is naturally equivalent to $A$. This shows
    that a locally constant prefactorization algebra can be reformed to
    a factorization algebra, as stated in Remark \ref{g}.
\end{example}

\begin{example}
    The notion of a factorization algebra is in some sense a generalization of 
    factorization homology. Indeed, suppose $A$ is a $\Disk_n^{(X,e)}$-algebra,
    and $M$ is an $(X,e)$-manifold. Then by Proposition \ref{i},
    $A$ corresponds to a locally constant factorization algebra on $M$, which
    is $i(A)$, in the language of Theorem \ref{f}. Take the map $p\:M\to *$,
    we see that $$p_*(i(A))(*)=i(A)(M)=\int_AM.$$ From this observation, 
    it is meaningful to make the following definition:

    \begin{definition}
        Suppose $A$ is a factorization algebra on a space $X$, $p\:X\to *$ is the 
        unique map. The \term{factorization homology} of $A$ is defined to be 
        $\int_XA:=p_*(A)(*)$.
    \end{definition}

    When $X$ is a manifold and $A$ is locally constant, this restricts to the 
    original definition of factorization homology.
\end{example}

We now discuss the behavior of factorization homology in the case that
the underlying space changes, again using Theorem \ref{f}.

\begin{example}
    Suppose $X$ and $Y$ are spaces, $f\:X\to Y$ is a continuous map. Then, if we set
    $p_1\:X\to *$ and $p_2\:Y\to *$, we have $p_1=p_2\circ f$.
    Thus for any factorization algebra $A$ on $X$, we have $$\int_XA=(p_1)_*(A)(*)
    =(p_2)_*f_*(A)(*)=\int_Yf_*(A),$$ which we will call the \term{pushforward
    formula}.
\end{example}

\begin{example}\label{u}
    Suppose $X$ and $Y$ are spaces, $p_1\:X\to *$ and $p_2\:Y\to *$ are the canonical
    maps. If we take $\pi\:X\times Y\to X$ to be the projection, then by Proposition
    \ref{j}, there exists an equivalence of $\infty$-categories $\underline\pi_*\:
    \FA^{lc}(X\times Y)\to\FA^{lc}(X,\FA^{lc}(Y))$, such that the following
    diagram is commutative: $$\xymatrix{\FA^{lc}(X\times Y)\ar[rr]^{\pi_*}\ar[rd]
    _{\underline\pi_*}^{\cong}&&\FA^{lc}(X)\\&\FA^{lc}(X,\FA^{lc}(Y))\ar[ru]
    _{\FA^{lc}(X,(p_2)_*)}}$$ If we composite the diagram with $(p_1)_*$, 
    we obtain that for any locally constant factorization algebra $A$ on $X\times Y$, there is
    an equivalence $$\begin{aligned}\int_{X\times Y}A&=(p_1\times p_2)_*(A)(*)
    \\&=(p_1)_*(\pi_*(A))(*)
    =\int_X\FA^{lc}(X,(p_2)_*(A))(*)=\int_X\int_YA.\end{aligned}$$ Restricting to the case
    of manifolds yields the \term{Fubini formula}:

    \begin{proposition}
        Suppose $M,N$ are manifolds with dimensions $m,n$, respectively, $A$
        is a $\Disk_{m+n}^{(M\times N,TM\times TN)}$-algebra. Then $\int_NA$
        is canonically a $\Disk_m^{(M,TM)}$-algebra, and $$\int_{M\times N}A
        \cong\int_M\int_NA.$$
    \end{proposition}

    Furthermore, if we set $M=\R^m$, $N=\R^n$, using Theorem \ref{k}, we see that
    the example is a generalization of the \term{Dunn's Theorem}, which is stated
    in \cite{Dun88}, rewritten in \cite{Lur16}:

    \begin{theorem}\label{v}
        There exists an equivalence of $\infty$-categories $$\E_{m+n}\-\Alg\cong
        \E_m\-\Alg(\E_n\-\Alg)$$ for any positive integers $m,n$.
    \end{theorem}
\end{example}

\subsection{Stratified spaces and applications to \texorpdfstring{$E_n$}{En}-modules}

In this section, we briefly discuss locally constant factorization algebras on
stratified spaces, and take some important examples. The notion of
stratified spaces is a slight generalization of manifolds without boundaries.
Pointed manifolds (which are manifolds with a distinguished point),
manifolds with or without boundaries can all be regarded as
stratified spaces. Most results in this section can be proved
using the strategies given in the previous sections, so we will mostly omit
the proofs.

\begin{definition}
    By a \term{stratified space} of dimension $n$, we mean a Hausdorff paracompact
    topological space $X$, together with a filtration $\varnothing=X_{-1}
    \subseteq X_0\subseteq\cdots\subseteq X_n=X$ of closed subsets, such that for
    any $x\in X_i-X_{i-1}$, there exists a neighborhood $U_x$ of $x$ that is
    homeomorphic to $\R^i\times C(L)$, where $C(L)$ is the open cone on a
    stratified space of dimension $n-i-1$ if $i<n$, $C(L)=*$ if $i=n$,
    and the homeomorphism preserves
    the filtration; and $X-X_{n-1}$ is dense in $X$. The connected components of
    $X_i-X_{i-1}$ are called the \term{dimension $i$-strata} of $X$. In our
    discussion, we always assume that $X$ has at most countable strata.
\end{definition}

For example, a manifold with boundary can be regarded as a stratified space,
with its $(n-1)$-dimensional strata being its boundary; a pointed manifold
can be regarded as a stratified space with its $0$-dimensional strata being the
base point.

\begin{definition}
    An open subset $D$ of $X$ is called a \term{(stratified) disk}, if it is 
    homeomorphic to $\R^i\times C(L)$, where $C(L)$ is the open cone on a
    stratified space of dimension $n-i-1$ if $i<n$, $C(L)=*$ if $i=n$, 
    and the homeomorphism preserves
    the filtration, and further $D\cap X_i\ne\varnothing$ and 
    $D\subseteq X-X_{i-1}$. The integer $i$ is called the \term{index} of $D$.

    A (stratified) disk $D$ is called a \term{good neighborhood} at $X_i$
    if $D$ has index $i$, and $D$ intersects only one connected component
    of $X_i-X_{i-1}$.

    A factorization algebra $A$ on a stratified space $X$ is called \term{locally
    constant} if for any inclusion of (stratified) disks $U\hookrightarrow V$
    such that $U$ and $V$ have the same index $i$ and $U,V$ are good neighborhoods
    at $X_i$, the map $A(U)\to A(V)$ is an equivalence.
\end{definition}

\begin{remark}
    In the case of pointed manifolds and manifolds with or without boundaries,
    all (stratified) disks are homeomorphic to the euclidean plane or half plane.
\end{remark}

Some results on locally constant factorization algebras in the usual case
remain true in the stratified case. We state some useful ones; the proofs
are similar.

\begin{proposition}[c.f.\ Proposition \ref{b}]
    Suppose $f\:X\to Y$ is a locally trivial fibration between stratified spaces,
    that is \term{adequately stratified}, in the sense that $Y$ has an open cover by
    trivializing (stratified) disks $V$ which are good neighborhoods satisfying:
    \begin{enumerate}[i)]
        \item $f^{-1}(V)\cong V\times F$ has a cover by (stratified) disks of the form
        $V\times D$ which are good neighborhoods in $X$;
        \item For sub-disks $T\subseteq U$ which are good neighborhoods (in $V$) with
        the same index, $T\times D$ is a good neighborhood in $X$ of same index as
        $U\times D$.
    \end{enumerate}
    Then pushforward along $f$ preserves local constantness.
\end{proposition}

\begin{proposition}[c.f.\ Proposition \ref{j}]
    Suppose $X,Y$ are stratified spaces with finitely many strata. Then the 
    projections $X\times Y\to X$ and $X\times Y\to Y$ are adequately stratified,
    and the functor $\underline\pi_*\:\FA(X\times Y)\to\FA(X,\FA(Y))$ restricts to
    a functor $\underline\pi_*\:\FA^{lc}(X\times Y)\to\FA^{lc}(X,\FA^{lc}(Y))$,
    which is an equivalence.
\end{proposition}

\begin{proposition}[c.f.\ Proposition \ref{l}]\label{m}
    Suppose $X$ is a stratified space, $A$ is a factorization algebra on $X$.
    If there exists an open cover $\mathscr U$ of $X$ such that $A|_U$ is
    locally constant for every $U\in\mathscr U$, then $A$ is locally constant.
\end{proposition}

We furthermore have the following proposition:

\begin{proposition}
    Suppose $i:X\to Y$ is a stratified embedding of stratified spaces such
    that $i(X)$ is a union of strata of $Y$. Then taking the pushforward along $i$ preserves
    local constantness.
\end{proposition}

We will now focus on some examples.

\begin{example}\label{n}
    The half line, $X=[0,+\infty)$, with the dimension-$0$ stratum given by $\{0\}$.
    Then all connected open subsets of $X$ form a factorization basis $\mathscr I$
    of $X$.

    An example of stratified locally constant prefactorization algebra
    on $\mathscr I$ arises as follows: Choose an $E_1$-algebra $E$, a right
    $E$-module $M$, and define the factorization algebra $A_{E,M}$ as follows:
    \begin{enumerate}[i)]
        \item For any $0<a<b$, define $A_{E,M}((a,b))=E$;
        \item For any $0<a$, define $A_{E,M}([0,a))=M$;
        \item The structure maps are given by the multiplication on $E$ and
        the structure of $M$ being a right $E$-module.
    \end{enumerate}

    It is easy to verify that this forms an $\mathscr I$-prefactorization algebra,
    and satisfies the locally constant condition. Furthermore, we have:

    \begin{proposition}
        \begin{enumerate}[i)]
            \item The $\mathscr I$-prefactorization algebra $A_{E,M}$ given above is an 
            $\mathscr I$-factorization algebra, and hence extends uniquely into a factorization
            algebra on $X$, which we will still denote by $A_{E,M}$;
            \item $A_{E,M}$ is locally constant on $X$;
            \item Moreover, all locally constant factorization algebras on $X$ is equivalent
            to some $A_{E,M}$;
            \item Finally, there exists an equivalence between $\FA^{lc}([0,\infty))$
            and the $\infty$-category $\E_1\-\RMod$ of (pointed) right modules over 
            $E_1$-algebras (whose objects are ordered pairs $(E,M)$ such that $E$ is an
            $E_1$-algebra and $M$ is a right $E_1$-module);
            \item The equivalence of $\infty$-categories satisfies the following commutative
            diagram up to equivalence: $$\xymatrix{\FA^{lc}([0,\infty))\ar[r]^{\cong}\ar[d]&\E_1\-\RMod
            \ar[d]\\\FA^{lc}((0,\infty))\ar[r]^{\cong}&\E_1\-\Alg.}$$
        \end{enumerate}
    \end{proposition}

    Hence this equivalence can be used to characterize right $E_1$-modules,
    which may be regarded as a generalization of Theorem \ref{k} in the
    one dimensional case. Similarly, there exists an equivalence between
    $\FA^{lc}((-\infty,0])$ and the $\infty$-category $\E_1\-\LMod$ of
    (pointed) left modules over $E_1$-algebras.
\end{example}

\begin{example}
    The unit interval, $X=[0,1]$, with the dimension-$0$ stratum given by $\{0,1\}$.
    Proposition \ref{m} shows that $$\FA^{lc}(X)\cong\FA^{lc}([0,1))
    \mathop{\operatorname{\times}}\limits_{\FA^{lc}((0,1))}\FA^{lc}((0,1])
    \cong\E_1\-\RMod\mathop{\operatorname{\times}}\limits_{\E_1\-\Alg}\E_1\-\LMod.$$
    Thus, giving a locally constant factorization algebra $A$ on $X$ is equivalent
    to giving a triple $(E,M_r,M_l)$, where $E$ is an $E_1$-algebra (where for any
    $0\le a< b\le 1$ we have $A((a,b))=E$), $M_r$ is a 
    right $E$-module (where for any $0<a\le 1$ we have $A([0,a))=M_r$), and 
    $M_l$ is a left $E$-module (where for any $0\le a< 1$ we have $A((a,1])=M_l$).
    Furthermore by discussion in the previous section, the factorization homology
    of $A$ is $$\int_{[0,1]}A\cong M_r\mathop{\operatorname{\ox}}\limits_EM_l.$$
\end{example}

\begin{example}
    The pointed euclidean space, $X=\R^n_*$, with the dimension-$0$ stratum given by 
    $\{0\}$. Similar to the case above, we will discuss the relationship between 
    locally constant factorization algebras on $X$ and $E_n$-modules.

    To do this, we notice that there exists a factorization basis of $X$, the open
    convex sets, which we will denote by $\mathscr C$. Suppose $M$ is a module
    over an $E_n$-algebra $E$. We define a $\mathscr C$-prefactorization algebra
    $A_{E,M}$ as follows:

    \begin{enumerate}[i)]
        \item For any $U\in\mathscr C,0\in U$, define $A_{E,M}(U)=M$;
        \item For any $U\in\mathscr C,0\notin U$, define $A_{E,M}(U)=E$;
        \item The structure maps are given by the multiplication on $E$ and
        the structure of $M$ being an $E$-module.
    \end{enumerate}

    It is easy to verify that this forms a $\mathscr C$-prefactorization algebra,
    and satisfies the locally constant condition. Similar to Proposition \ref{m},
    we have the following result:

    \begin{proposition}\label{o}
        \begin{enumerate}[i)]
            \item The $\mathscr C$-prefactorization algebra $A_{E,M}$ given above is a 
            $\mathscr C$-factorization algebra, and hence extends uniquely to a factorization
            algebra on $X$, which we will still denote by $A_{E,M}$;
            \item $A_{E,M}$ is locally constant on $X$;
            \item We obtain a functor $\psi\:\E_n\-\Mod\to\FA^{lc}(X)$, which fits into
            the following commutative diagram: $$\xymatrix{\E_n\-\Alg\ar[r]^{\cong}
            \ar@{^(->}[d]&\FA^{lc}(\R^n)\ar[d]\\\E_n\-\Mod\ar[r]^\psi&\FA^{lc}(X);}$$
            \item The map $$\E_n\-\Mod\xrightarrow{\psi}\FA^{lc}(X)\to\FA^{lc}(\R^n-\{0\})$$
            and $$\E_n\-\Mod\to\E_n\-\Alg\cong\FA^{lc}(\R^n)\to\FA^{lc}(\R^n-\{0\})$$
            are equivalent, and identify $\E_n\-\Mod$ with the pullback $\FA^{lc}(X)\times_
            {\FA^{lc}(\R^n-\{0\})}\FA^{lc}(\R^n)$.
        \end{enumerate}
    \end{proposition}

    Hence this proposition can be used to characterize $E_n$-modules,
    which may be regarded as a generalization of Theorem \ref{k} in the
    $n$-dimensional case. In particular, by taking the fiber we have the following
    corollary:

    \begin{corollary}
        The functor $$\E_n\-\Mod(A)\to\FA^{lc}(\R^n_*)\mathop{\operatorname{\times}}
        _{\FA^{lc}(\R^n-\{0\})}\{A\}$$ is an equivalence for any $E_n$-algebra $A$.
    \end{corollary}

    We may also use this example to characterize $E_1$-bimodules. We take $n=1$,
    then there exists two (instead of one) strata of maximal dimension in $X$,
    unlike the case of $n\ge 2$. Mimicking Example \ref{n}, we have the following proposition:

    \begin{proposition}
        There exists an equivalence of $\infty$-categories between $\FA^{lc}(\R_*)$
        and the $\infty$-category $\E_1\-\BiMod$ of $E_1$-bimodules. The equivalence
        assigns an $(L,R)$-bimodule $M$ to a locally constant factorization algebra $A$,
        whose value on $(a,b)$ is $L$ if $b\le 0$, $R$ if $a\ge 0$, $M$ otherwise,
        and the structure maps are given by the structure maps of a bimodule.
    \end{proposition}
\end{example}

\begin{example}\label{q}
    The closed unit disk, $X=D^n$, with the $(n-1)$-dimensional strata given by 
    $S^{n-1}$. Then by Proposition \ref{m}, there exists an equivalence of 
    $\infty$-categories $$\FA^{lc}(D^n)\cong\FA^{lc}(D^n-\{0\})
    \mathop{\operatorname{\times}}\limits_{\FA^{lc}(D^n-S^{n-1}-\{0\})}
    \FA^{lc}(D^n-S^{n-1}).$$ Furthermore, we have $$\begin{aligned}
    \FA^{lc}(D^n-S^{n-1})&\cong\FA^{lc}(\R^n)\cong\E_n\-\Alg,\\
    \FA^{lc}(D^n-S^{n-1}-\{0\})&\cong\FA^{lc}(\R^n-\{0\})\\&\hspace{2em}\cong
    \FA^{lc}(S^{n-1}\times\R)\cong\E_1\-\Alg(\FA^{lc}(S^{n-1})),\\
    \FA^{lc}(D^n-\{0\})&\cong\FA^{lc}(S^{n-1}\times(-\infty,0])\cong
    \E_1\-\LMod(\FA^{lc}(S^{n-1})),
    \end{aligned}$$ we obtain an equivalence $$\FA^{lc}(D^n)\cong\E_n\-\Alg
    \mathop{\operatorname{\times}}\limits_{\E_1\-\Alg(\FA^{lc}(S^{n-1}))}
    \E_1\-\LMod(\FA^{lc}(S^{n-1})).$$

    Now, for any map $f\:E\to F$ of $E_n$-algebras, $F$ has an $E_n$-module
    structure over $E$ induced by $f$. Furthermore, $E,F$ can be regarded as locally
    constant factorization algebras on $\R^n$, hence on $\R^n-\{0\}\cong
    S^{n-1}\times\R$. Define $q\:S^{n-1}\times\R\to S^{n-1}$ to be the projection,
    then $q_*(E)$ is an $E_1$-algebra, and $q_*(F)$ is an $E_1$-module over $q_*(E)$.
    Thus the equivalence above assigns the map $f$ with a locally constant
    factorization algebra on $D^n$, which we will denote $\omega(f)$. Its value
    on a good neighborhood of index $(n-1)$ is $F$, and its value
    on a good neighborhood of index $n$ is $E$. Moreover, we may collapse
    the boundary of $D^n$ into one point, making the locally constant factorization
    algebra on $D^n$ be a locally constant factorization algebra on $S^n_*$.
    We denote this locally constant factorization algebra on $S^n_*$ by
    $\omega'(f)$.
\end{example}

\section{Further applications}

In this section, we will briefly talk about two applications of the above
results on factorization algebras and factorization homology: the Deligne
conjecture and the Bar constructions. We will solve the Deligne conjecture
by the tool of centralizers, and extend the study of Bar constructions on
associative algebras to arbitrary $E_n$-algebras. We will mostly present ideas
and sketches of proofs; the details can be found in \cite{Fra11} and
\cite{GTZ12}.

\subsection{Centralizers and (higher) Deligne conjecture}

The original Deligne conjecture on the structure of Hochschild cohomology of
an associative algebra (more generally an $A_\infty$-algebra) states that
Hochschild cohomology naturally has the structure of a BV-algebra. 
More generally, the full Hochschild cohomology complex has the structure of
an algebra over the (framed) little disk operad. We will here prove
the following result, which is mostly stated as the ``higher
Deligne conjecture'':

\begin{theorem}[Higher Deligne conjecture]\label{p}
    Suppose $A$ is an $E_n$-algebra. Then the $E_n$-Hochschild cohomology 
    $\HH_{E_n}(A,A)$ has an $E_{n+1}$-algebra structure.
\end{theorem}

We will first review the definition of the $E_n$-Hochschild cohomology.

\begin{definition}
    Suppose $\cal C$ is a closed symmetric monoidal $\infty$-category.
    Let $M$ be an $E_n$-module over an $E_n$-algebra $A$. The \term{$E_n$-Hochschild
    cohomology} of $A$ with values in $M$ is defined to be
    $$\HH_{E_n}(A,M):=\Map_A^{E_n}(A,M):=\Map_{\E_n\-\Mod(A)}(A,M)\in \cal C.$$
\end{definition}

\begin{remark}
    Notice that an $E_1$-$A$-module is equivalent to a left 
    $E_1$-module over $A\ox A^{op}$, we find out that $$\HH_{E_1}(A,M)
    =\Map_{\E_1\-\Mod(A)}(A,M)\cong\Map_{\E_1\-\LMod(A\ox A^{op})}(A,M)$$ coincides
    with the usual definition of  Hochschild cohomology.
\end{remark}

Now we introduce the language of the center of a map of $E_n$-algebras, which is
due to \cite{Lur16} and can be used to prove the Deligne conjecture:

\begin{definition}
    Suppose $f\:A\to B$ is an map of $E_n$-algebras. The \term{centralizer} of $f$
    is an $E_n$-algebra $\mathfrak z_n(f)$ together with a map 
    $e_f\:A\ox\mathfrak z_n(f)\to B$ of $E_n$-algebras, such that 
    $e_f\circ(\mathbb1\ox 1_{\mathfrak z_n(f)})=f$; and for any $E_n$-algebra $C$
    together with a map $\vp\:A\ox C\to B$ of $E_n$-algebras, such that
    $\vp\circ(\mathbb1\ox 1_{C})=f$, there exists a unique map 
    $\kappa\:C\to\mathfrak z_n(f)$ such that $\vp=e_f\circ(\mathbb1\ox\kappa)$.
    We define the \term{center} of $A$ to be $\mathfrak z_n(A):=
    \mathfrak z_n(\mathbb1_A)$.
\end{definition}

\begin{example}
    Suppose $G$ is an $E_1$-algebra in the category $\Set$; that is a monoid.
    Then $\mathfrak z_1(G)$ is the usual center $Z(G)$. If $f\:H\to G$ is
    an inclusion of monoids, then $\mathfrak z_1(G)$ is the usual centralizer 
    $Z_H(G)$. This explains the name ``centralizer''. Notice that in this example,
    $Z(G)$ is commutative, meaning that it is an $E_2$-algebra.
\end{example}

\begin{remark}
    By the definition, for any maps of $E_n$-algebras $f\:A\to B,\ g\:B\to C$,
    there exists a commutative diagram $$\xymatrix{
    &&A\ox\mathfrak z_n(f)\ox\mathfrak z_n(g)\ar[rd]^{e_f\ox\mathbb1}&&\\
    &A\ox\mathfrak z_n(f)\ar[ru]^{\mathbb1\ox1_{\mathfrak z_n(g)}}
    \ar[rd]^{e_f}&&B\ox\mathfrak z_n(g)\ar[rd]^{e_g}&\\
    A\ar[rr]^f\ar[ru]^{\mathbb1\ox1_{\mathfrak z_n(f)}}&&B\ar[rr]^g
    \ar[ru]^{\mathbb1\ox1_{\mathfrak z_n(g)}}&&C,}$$ which induces a natural map
    $$\mathfrak z_n(\circ)\:\mathfrak z_n(f)\ox\mathfrak z_n(g)
    \to\mathfrak z_n(g\circ f)$$ of $E_n$-algebras by the universal property.
\end{remark}

The following theorem, also known as the ``relative Deligne conjecture'',
claims that the centralizer is computed by $E_n$-Hochschild cohomology:

\begin{theorem}[Relative Deligne conjecture]
    Suppose $f\:A\to B$ is a map of $E_n$-algebras. Denote $B_f$ to be $B$, endowed
    with the $E_n$-module structure over $A$ induced by $f$. Then there is an
    $E_n$-algebra structure on $\HH_{E_n}(A,B_f)$, making it the centralizer of
    $f$. In particular, $\mathfrak z_n(f)$ exists. Moreover for any $g\:B\to C$
    another map of $E_n$-algebras, the following diagram is commutative:
    $$\xymatrix{
    \mathfrak z_n(f)\ox\mathfrak z_n(g)\ar[d]_{\cong}\ar[rr]^{\mathfrak z_n(\circ)}
    &&\mathfrak z_n(g\circ f)\ar[d]_{\cong}\\\Map_A^{E_n}(A,B_f)
    \ox\Map_B^{E_n}(B,C_g)\ar[rr]^\circ&&\Map_A^{E_n}(A,C_{g\circ f}),}$$ where 
    the lower arrow is induced by composition of maps.
\end{theorem}

\begin{proof}[Sketch of proof]
    Details can be found in \cite{GTZ12}. By the
    characterization given in Proposition \ref{o}, $A,B$ can be regarded as
    locally constant factorization algebras on $\R^n_*$, and a map of 
    $E_n$-modules over $A$ from $A$ to $B_f$ is equivalent to a map of factorization
    algebras on $\R^n_*$ from $A$ to $B$, whose restriction on $\R^n-\{0\}$ is 
    $f$.

    We first give the $E_n$-algebra structure on $\HH_{E_n}(A,B_f)
    =\Map_A^{E_n}(A,B_f)$. To do so, it suffices to give a locally constant
    factorization algebra on $\R^n$, whose global section is $\Map_A^{E_n}(A,B_f)$.
    To do so, it will be enough to define it on $\mathscr C$, the basis of convex
    open sets. To any $U\in\mathscr C$, we take a central point $x_U$, and associate
    $U$ with the object $Z(U)=\Map_A^{E_n}(A,B_f)$, represented by all maps of
    factorization algebras from $A|_U$ to $B|_U$ whose restriction to $U-\{x_U\}$
    is $f$; i.e. $$\Map_{\FA(U)}(A|_U,B|_U)\mathop{\operatorname{\times}}\limits
    _{\Map_{\FA(U-\{x_U\})}(A|_U,B|_U)}\{f\}.$$ For any $U_1,\cdots,U_r,V
    \in\mathscr C$ such that $U_1,\cdots,U_r$ are pairwise disjoint and contained in
    $V$, we need to define a map $$\rho_{U_1,\cdots,U_r,V}\:Z(U_1)\ox\cdots\ox Z(U_r)
    \to Z(V).$$ For any $g_i:A|_{U_i}\to B|_{U_i}\in Z(U_i)$ for all $i$,
    we define $\rho_{U_1,\cdots,U_r,V}(g_1,\cdots,g_r)$ to be a map of factorization
    algebras from $A$ to $B$ on a factorizing basis on $V$. We take the following
    basis:

    \begin{itemize}
        \item All open sets in $V$ such that it is either in some $U_i$, or is in
        $V-\{x_{U_1},\cdots,x_{U_r}\}$;
    \end{itemize}

    and define the following map $\rho_{U_1,\cdots,U_r,V}(g_1,\cdots,g_r)$
    of factorization algebras:
    
    \begin{itemize}
        \item Its value on an open set $D$ contained in $U_i$ is given by $g_i$;
        \item Its value on an open set $D$ contained in $V-\{x_{U_1},\cdots,x_{U_i}\}$
        is given by $f$.
    \end{itemize}

    Then we observe that if we take a sufficient large closed disk $D'$ contained in
    $V$ containing all $x_{U_i}$'s, and collapsing $D'$, $\rho_{U_1,\cdots,U_r,V}
    (g_1,\cdots,g_r)$ turns into a map of factorization algebras on $V/D'$
    from $A$ to $B$, whose restriction on $V-D'$ is $f$, hence lies in 
    $\Map_A^{E_n}(A,B_f)=Z(V)$.

    It can be checked that the constructions given above make $Z$ a 
    locally constant $\mathscr C$-factorization algebra, thus extends to a locally
    constant factorization algebra on $\R^n$, making $\Map_A^{E_n}(A,B_f)$ an
    $E_n$-algebra. Moreover using the constructions above,
    one can check that the map of evaluation
    $\mathrm{ev}\:A\ox\Map_A^{E_n}(A,B_f)\to B$ is a map of $E_n$-algebras.

    To prove that $\HH_{E_n}(A,B_f)$ is equivalent to $\mathfrak z_n(f)$, we choose
    an arbitrary $E_n$-algebra $C$ together with a map $\vp\:A\ox C\to B$ of 
    $E_n$-algebras, such that $\vp\circ(\mathbb1\ox 1_{C})=f$. Then $\vp$
    induces a map $\theta_\vp\:C\to\Map_{\E_n\-\Alg}(A,B)$. On the other hand, there
    exists a canonical functor $\Map_{\E_n\-\Alg}(A,B)\to
    \Map_{\E_n\-\Mod(A)}(A,B_f)$; composition of the two maps yields a map 
    $\tilde\theta_\vp\:C\to\HH_{E_n}(A,B_f)$, which, by the definition of
    an adjunction, fits the identity $\mathrm{ev}\circ(\mathbb1\ox\tilde\theta_\vp)
    =\vp$. Using the identity that $$\begin{aligned}\Map_A^{E_n}(A,B_f)
    &\xrightarrow{\Map_A^{E_n}(A,A\ox\mathbb1)}
    \Map_A^{E_n}(A,A\ox\Map_A^{E_n}(A,B_f))\\&\xrightarrow{\Map_A^{E_n}(A,\mathrm{ev})}
    \Map_A^{E_n}(A,B_f)\end{aligned}$$ is the identity map, the uniqueness of the map
    $\tilde\theta_\vp$ follows.

    The commutativity of the diagram is immediately from the universal property.
\end{proof}

We can now prove Theorem \ref{p}.

\begin{proof}[Proof of Theorem \ref{p}]
    We first notice that for any object $X$ in $\cal C$, the composition map
    $\circ\:\Map(X,X)\ox\Map(X,X)\to\Map(X,X)$
    makes $\Map(X,X)$ an $E_1$-algebra. Now, by the relative Deligne conjecture,
    $\HH_{E_n}(A,A)$ is an $E_1$-algebra in the $\infty$-category $\E_n\-\Alg$,
    which means it is an $E_{n+1}$-algebra, by Dunn's Theorem.
\end{proof}

We state some examples.

\begin{example}
    Suppose $\cal A$ is a monoidal category, which is an $E_1$-algebra in
    the category of categories. Then $\mathfrak z_1(\cal A)$ is an $E_2$-algebra in
    the category of categories, which is a braided monoidal category. It is shown
    in \cite{Lur16} that $\mathfrak z_1(\cal A)$ is actually the Drinfeld center
    of $\cal A$.
\end{example}

\begin{example}
    Suppose $A$ is an $E_n$-algebra, then the map $1_A:\mathbb1\to A$ has center
    $\mathfrak z_n(1_A)=A$. The composition map $\mathfrak z_n(1_A)\ox
    \mathfrak z_n(A)\to\mathfrak z_n(1_A)$ exhibits $\mathfrak z_n(1_A)=A$
    a left module of $\mathfrak z_n(A)$ in the $\infty$-category $\E_n\-\Alg$,
    which means that $A$ has a natural action on the $E_{n+1}$-algebra 
    $\HH_{E_n}(A,A)$. This is also called the ``Swiss-Cheese version of
    Deligne conjecture''.
\end{example}

\subsection{An introduction to the Bar constructions}

Bar constructions have been introduced in topology as a model for the 
coalgebra structure on the cochains on $\Omega X$, the loop space of a pointed
space $X$. In this section, we will give a short introduction of the definition
and basic properties of the Bar constructions. We refer to \cite{Fra11} and
\cite{GTZ12} for those who are interested in more properties of the
Bar constructions and its relationship to iterated loop spaces.

We recall the original definition of the Bar construction.

\begin{definition}
    Suppose $A$ is an augmented associative algebra in a symmetric monoidal $\infty$-category
    $\cal C$; that is, an associative algebra
    together with an algebra homomorphism $A\to\mathbb1$, where $\mathbb1$ is the 
    unit of the tensor product of $\cal C$. The 
    \term{Bar object}\footnote{This is often called the ``Bar complex'', since
    the underlying category is often taken to be the derived $\infty$-category
    of chain complexes, $\Chinf(\Bbbk)$.} is given by $B(A):=\mathbb1\ox_A\mathbb1.$
\end{definition}

\begin{remark}
    In the sense of chain complexes, if $A$ is flat over the base field
    $\Bbbk$, $B(A)$ is computed by the usual standard chain complex 
    $\bigoplus_{n\ge 0}\overline{A}^{\ox n}$, where $\overline A$ is the kernel
    of the augmentation.
\end{remark}

It is well known that the Bar construction has a standard coalgebra structure.
Moreover, if $A$ is a commutative algebra, then the Bar construction has a
standard commutative algebra structure. We now generalize the results into the case of
$E_n$-algebras.

\begin{definition}
    An \term{augmented $E_n$-algebra} is an $E_n$-algebra $A$ together with a map
    $\ve:A\to\mathbb1$ of $E_n$-algebras, which is called the \term{augmentation}.
    The $\infty$-category of augmented $E_n$-algebras is denoted $\E_n\-\Alg^{aug}$.
\end{definition}

Now, according to Example \ref{q}, an augmented $E_n$-algebra defines a locally
constant factorization algebra on the closed disk $D^n$. Moreover, we have the
following functors $$\omega_i\:\E_n\-\Alg^{aug}\to\E_i\-\Alg^{aug}
(\E_{n-i}\-\Alg^{aug})\to\FA^{lc}(D^i,\E_{n-i}\-\Alg^{aug}),$$ where the first
is induced by the equivalence $\E_n\-\Alg\cong\E_i\-\Alg(\E_{n-i}\-\Alg)$.

\begin{definition}
    Suppose $A$ is an augmented $E_n$-algebra. Define its \term{Bar object}
    to be $$B(A):=\int_{I\times\R^{n-1}}A\mathop{\operatorname{\ox}}\limits_
    {\int_{S^0\times\R^{n-1}}A}\mathbb1.$$
\end{definition}

It can be directly verified that the right hand side of the definition
is equivalent to $p_*(\omega_1(A))$, where $p\:I\to *$ is the unique map.
Thus the definition is just the factorization homology of the associated
factorization algebra on $D^1$. In particular, this definition agrees with
the definition for augmented associative algebras.

Hence, the above discussion shows that $B(A)$ has a natural structure of an
$E_{n-1}$-algebra. We can therefore iterate this construction at most $n$ times:

\begin{definition}
    Suppose $A$ is an augmented $E_n$-algebra, and $0\le i\le n$. The \term{$i$-th
    iterated Bar object} of $A$ is the augmented $E_{n-i}$-algebra $$B^i(A)
    =B(B(\cdots(B(A))\cdots)).$$
\end{definition}

By the definition of the Bar object, and the excision axiom of factorization
homology, we may prove the following proposition:

\begin{proposition}
    Suppose $A$ is an augmented $E_n$-algebra, and $0\le i\le n$. There exist
    natural equivalences of $E_{n-i}$-algebras $$B^i(A)\cong
    \int_{D^i\times\R^{n-i}}A\mathop{\operatorname{\ox}}\limits_
    {\int_{S^{i-1}\times\R^{n-i}}A}\mathbb1\cong p_*(\omega_i(A)),$$
    where $p\:D^i\to *$ is the unique map. 
\end{proposition}

In particular, taking $n=\infty$, we obtain the $E_\infty$-structure on
the iterated Bar objects $B^i(A)$ of an augmented $E_\infty$-algebra, which 
restricts to the standard commutative algebra structure on the Bar object if $A$
is a commutative algebra.

We now sketch the construction of the $E_i$-coalgebra structure on the $i$-th
iterated Bar construction of an $E_n$-algebra. To do so, we only
need to construct a locally constant ``cofactorization algebra'' on $\R^i$,
whose global section is $B^i(A)$, which, similar to the previous section,
suffices to do so on $\mathscr C$, the basis of convex open sets.

We take the locally constant factorization algebra on $\R^i$ with values in
$\E_{n-i}\-\Alg^{aug}$ determined by $A$, which will be still denoted $A$.
For any $V\in\mathscr C$, by Example \ref{q}, the augmentation
gives a stratified locally constant factorization algebra $\omega'(A|_V)$
on $V\cup\{\infty\}$. (Notice that $V\cup\{\infty\}$ is homeomorphic
to $S^n_*$.) We assign $V$ with the object $$B^i(A)(V):=\int_{V\cup\{\infty\}}
\omega'(A|_V).$$ By definition, $B^i(A)(V)$ is equivalent to $B^i(A)$. 
For any $U_1,\cdots,U_r,V\in\mathscr C$ such that $U_1,\cdots,U_r$ are
pairwise disjoint and contained in $V$, there exists a continuous map
$\pi\:V\cup\{\infty\}\to U_1\cup\cdots\cup U_n\cup\{\infty\}$, and
the augmentation defines a map $$\ve\:\pi_*(\omega'(A|_{V}))\to\bigox_{i=1}^r
\omega'(A|_{U_i}).$$ We then obtain a structure map $$\begin{aligned}\nabla_{U_1,\cdots,U_s,V}\:
B^i(A)(V)&=\int_{V\cup\{\infty\}}\omega'(A|_V)\\&\xrightarrow{\int\ve}
\bigox_{k=1}^r\int_{U_i\cup\{\infty\}}\omega'(A|_{U_k})
=\bigox_{k=1}^rB^i(A)(U_k).\end{aligned}$$ 

The following theorem from \cite{GTZ12} proved that this indeed gives the
$E_i$-coalgebra structure on $B^i(A)$:

\begin{theorem}
The maps $\nabla_{U_1,\cdots,U_s,V}$ form the structure maps of a 
locally constant factorization coalgebra, making $B^i(A)$ into an 
$E_i$-coalgebra. Thus $B^i$ lifts to a functor $$B^i\:\E_n\-\Alg^{aug}\to
\E_i\-\coAlg(\E_{n-i}\-\Alg^{aug}).$$
\end{theorem}

In particular, taking $i=1$, we obtain the $E_1$-coalgebra structure on
the Bar object $B(A)$ of an augmented $E_n$-algebra, which 
restricts to the standard coalgebra structure on the Bar object if $A$
is an associative algebra. 

\printbibliography

\end{document}