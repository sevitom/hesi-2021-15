\subsection{Genericity of Morse functions}
As a baby example, we prove the classical result 
that Morse functions are generic:

\begin{theorem}
	Let $M$ be a compact smooth manifold. 
	Then a generic smooth function on $M$ is Morse.
\end{theorem}

\begin{proof}
	We apply the general transversality theorem with
	\begin{itemize}
		\item[$\Xc$:] 
			the manifold $M$;
		\item[$\Yc$:] 
			the Banach space $C^k (M)$ of $C^k$ functions on $M$;
		\item[$\Ec$:] 
			the pullback $\pi^*T^*M$ via the projection 
			$\pi \colon M \times C^k (M) \to M$;
		\item[$\Sc$:] 
			the section $M \times C^k (M) \to \pi^*T^*M,\, (x, f) \mapsto df_x$.
	\end{itemize}
	Here $k\geq2$. 
	For $(x, f) \in \Sc^{-1} (0)$, $D_2\Sc_{(x, f)} (h) = dh_x$, 
	$D_1\Sc_{(x, f)}(v) = \nabla_v df$, where $\nabla$ is any 
	fixed connection on $T^*M$. 
	Note that the latter is well-defined since $df_x=0$. 
	The assumptions are trivial to verify: 
	(a) $D_2\Sc_{(x, f)}$ is already surjective; 
	(b) $D_1\Sc_{(x, f)}$ is Fredholm with index $0$ since it is a map 
	between spaces of the same finite dimension. 
	Thus the general transversality theorem implies that 
	for a generic $f \in C^k (M)$, $v \mapsto \nabla_v df$ is surjective 
	whenever $df_x = 0$, i.e., $f$ is Morse.

	To pass to $C^\infty$, we use the argument of Taubes. 
	Let $C^k_\Morse (M)$ be the space of $C^k$ Morse functions on $M$. 
	We have proved $C^k_\Morse (M)$ is dense in $C^k (M)$. 
	Since Morse functions are stable, $C_\Morse^k (M)$ is open 
	in $C^k (M)$ and hence in $C^\infty (M)$. 
	It remains to show $C^\infty_\Morse (M)$ is dense in $C^\infty (M)$, 
	so that $C^\infty_\Morse (M) = \bigcap _{k=2} ^\infty C^k_\Morse (M)$ 
	is residual. 
	This is done by a $1 / k$-argument: 
	Let $f \in C^\infty (M)$. 
	For each $k$, take $f_k \in C^k_\Morse (M)$ with 
	$\| f_k - f \|_{C^k} \leq 1 / k$, take $\delta_k > 0$ such that 
	$\| \wt{f} - f_k \|_{C^k} < \delta_k$ implies $\wt{f} \in C^k_\Morse (M)$, 
	then take $\wt{f}_k \in C^\infty (M)$ with 
	$\| \wt{f}_k - f_k \|_{C^k} < \min (1 / k, \delta_k)$, 
	which is possible by the Whitney approximation theorem. 
	Then $\wt{f}_k \in C^\infty_\Morse (M)$ and 
	$\wt{f}_k \to f$ in $C^\infty (M)$.
\end{proof}

\subsection{Moduli spaces of flow lines}
Let $M$ be a smooth manifold, $f$ a $C^2$ Morse function on $M$. 
We denote by $\Crit(f)$ the set of critical points of $f$, and 
for $p \in \Crit(f)$, by $\ind(p) = \ind_f (p)$ the Morse index of $f$ at $p$. 
Fix a Riemannian metric $g$ on $M$. 
Consider the negative gradient flow generated by $f$. 
A \textit{flow line} for this flow is a curve $\gamma \colon \Rb \to M$ 
such that $\gamma' = - \nabla f \circ \gamma$. 
For $p, q \in \Crit(f)$, the (\textit{parametrized}) 
\textit{moduli space of flow lines} from $p$ to $q$ for the pair $(f, g)$, 
denoted by $\Mc(p, q, f; g)$, is defined to be the space of all flow lines 
$\gamma$ such that $\lim_{t \to -\infty} \gamma(t) = p$, 
$\lim_{t \to \infty} \gamma(t) = q$.

\begin{remark}
	Once we know it is a smooth manifold, it is not hard to show that 
	$\gamma \mapsto \gamma(0)$ gives a diffeomorphism 
	between the moduli space consider here and the usual moduli space 
	defined as the intersection of stable and unstable manifolds.
\end{remark}

\medskip

\noindent This subsection is devoted to proving the following theorem:

\begin{theorem}
	Let $M$ be a compact smooth manifold, 
	$f$ a $C^{k + 1}$ Morse function on $M$, $p, q \in \Crit(f)$. 
	Then for a generic Riemannian metric $g$ on $M$, the moduli space 
	$\Mc(p, q, f; g)$ is a smooth manifold of dimension $\ind(p) - \ind(q)$.
\end{theorem}

If $p = q$, then $\Mc(p, p, f; g)$ is a singleton. Assume $p\neq q$.

\subsubsection{Setting the stage}
We shall apply the general transversality theorem with

\begin{itemize}
	\item[$\Xc$:] 
		the space $\Pc^{1, 2}$ of $W^{1, 2}$ curves on $M$ from $p$ to $q$;
	\item[$\Yc$:] 
		the space $\Gc^\ell$ of $C^\ell$ Riemannian metrics on $M$;
	\item[$\Ec$:] 
		the bundle $\Ec^{0, 2} \to \Pc^{1, 2} \times \Gc^\ell$ 
		with fiber $\Ec^{0, 2}_{(\gamma, g)} = L^2(\gamma^*TM)$;
	\item[$\Sc$:] 
		the section $\Pc^{1, 2} \times \Gc^\ell \to \Ec^{0,2},\, 
		(\gamma, g) \mapsto \gamma' + \nabla f \circ \gamma$.
\end{itemize}

Here $\ell$ is sufficiently large. 
Clearly $\Mc(p, q, f; g) = \Sc^{-1} (0) \cap ( \Pc^{1, 2} \times \{g\} )$.

\medskip

Let us explain the Banach manifold structures on these spaces. 
Since $\Gc^\ell$ is an open subset of the Banach space 
$C^\ell (\Sym^2 T^*M)$, where $\Sym^2 T^*M$ is the vector bundle 
of symmetric $(0, 2)$-tensors on $M$, it is a smooth Banach manifold 
with $T_g \Gc^\ell = C^\ell (\Sym^2 T^*M)$. 
The manifold structure on $\Pc^{1, 2}$ is more complicated. 
Recall that a $W^{1, 2}$ map on $\Rb$ is continuous 
by the Sobolev embedding theorem. 
Thus to be more precise, we define 
\[
	\Pc^{1, 2} := \{ \gamma \in W^{1, 2} (\Rb, M) \cap C^0 (\ol{\Rb}, M)
		\colon \gamma(-\infty) = p,\, \gamma(\infty) = q \},
\] 
equipped with the $W^{1, 2}$ topology. 
One can show that $\Pc^{1, 2}$ is a smooth Banach manifold 
with $T_\gamma \Pc^{1, 2} = W^{1, 2} (\gamma^*TM)$. 
Charts on $\Pc^{1, 2}$ are given by 
$T_\gamma \Pc^{1,2} \to \Pc^{1,2},\, \xi \mapsto \exp_\gamma \xi$, 
where the exponential map is with respect to 
some fixed Riemannian metric on $M$. 
One can also show that $\Ec^{0, 2} \to \Pc^{1, 2} \times \Gc^\ell$ 
is a smooth Banach vector bundle and $\Sc$ is a smooth section. 
See \cite[Appendix A]{Sch}.

\medskip

For a flow line $\gamma$, write 
\[
	D_\gamma := D_1\Sc_{(\gamma, g)} 
		\colon W^{1, 2} (\gamma^*TM) \to L^2 (\gamma^*TM)
\] 
and its formal adjoint 
\[
	D_\gamma^* \colon W^{1, 2} (\gamma^*TM) \to L^2 (\gamma^* TM).
\] 
By definition, 
\[
	D_h \nabla f := D_2\Sc_{(\gamma, g)} (h) 
		= \pfrac{}{t} \nabla^{g + th} f \,\Big|_{t=0}\,,
\] 
where $\nabla^g$ denotes gradient with respect to $g$. 
To compute it, applying $\pfrac{}{t} \big|_{t=0}$ to 
$(g + th) (\nabla^{g + th} f, \xi) = \xi f$ gives 
$\la D_h \nabla f, \xi \ra = - h (\nabla f, \xi)$ for $\xi$ vector field.

\subsubsection{Fredholm property}

\begin{proposition}
	For $(\gamma, g) \in \Sc^{-1} (0)$, 
	\[
		D_1\Sc_{(\gamma, g)} = 
			D_\gamma \colon W^{1, 2} (\gamma^*TM) \to L^2 (\gamma^*TM)
	\] 
	is Fredholm with index $\ind(p) - \ind(q)$.
\end{proposition}

This follows from the theory of spectral flows, cf.\ \cite[Section 2.2]{Sch}.

\begin{proof}[Sketch of proof]
	Since $\gamma$ is an embedding and $\Rb$ is contractible, 
	$\gamma^*TM$ is a trivial bundle, so $D_1\Sc_{(\gamma,g)}$ 
	is similar to an operator of the form 
	$\pa_t + A \colon W^{1, 2} (\Rb, \Rb^n) \to L^2 (\Rb, \Rb^n)$ 
	where $A \in C^0_b (\ol{\Rb}, \End(\Rb^n))$. 
	One checks that such operators are all Fredholm. 
	Since the space of such operators with fixed $A(\pm\infty)$ 
	is convex, the Fredholm index is constant on it, 
	i.e., depends only on $A(\pm\infty)$. 
	Thus to compute $\ind(\pa_t + A)$, one takes a nice $A$ 
	such that one can write down explicitly the solutions 
	to the ODE $(\pa_t + A) s = s'$, so that $\ind(\pa_t + A)$ 
	is directly computable.
\end{proof}

\subsubsection{Regular value}

\begin{lemma}
	For $\eta \in \Rb^n \setminus \{0\}$, 
	$\Sym^2 \Rb^n \to \Rb^n,\, h \mapsto h \eta = 
	(\eta^\intercal h)^\intercal$\footnote{We view elements in $\Rb^n$ 
	as $n \times 1$ column vectors,
	elements in $\Sym^2 \Rb^n$ as $n \times n$ matrices, and 
	$^\intercal$ denotes matrix transposition.} 
	is surjective.
\end{lemma}
\begin{proof}
	Conjugating by an orthogonal matrix, we may assume 
	$\eta = (t, 0, \ldots, 0)^\intercal$ with $t>0$, 
	in which case it is obvious.
\end{proof}

\begin{proposition}
	For $(\gamma, g) \in \Sc^{-1} (0)$,
	\[
		D\Sc_{(\gamma, g)} \colon W^{1, 2} (\gamma^*TM) 
			\times C^\ell (\Sym^2 T^*M) \to L^2(\gamma^*TM)
	\] 
	is surjective.
\end{proposition}

\begin{proof}
	Since $D_1\Sc_{(\gamma, g)}$ is Fredholm, $\im D_1\Sc_{(\gamma, g)}$ 
	and hence $\im D\Sc_{(\gamma, g)}$ is closed and has finite codimension.
	Thus it suffices to show $(\im D\Sc_{(\gamma, g)})^\perp = 0$. 
	Suppose $\eta \in L^2 (\gamma^*TM)$ and 
	\begin{gather*}
		\int_M \la \eta, D_\gamma \xi \ra, 
			\quad \fr \xi \in W^{1, 2} (\gamma^*TM) \\
		\int_M \la \eta, D_h \nabla^g f \ra = 0,
			\quad \fr h \in C^\ell(\Sym^2 T^*M).
	\end{gather*} 
	By elliptic regularity for $D_\gamma^*$,\footnote{This is 
	an ODE, so it is straightforward to check this regularity.} 
	the first equation implies $\eta \in W^{1, 2}$. 
	Thus $\eta$ is continuous by the Sobolev embedding theorem. 
	If $\eta(t) \neq 0$ for some $t \in \Rb$, then by the lemma, 
	there exists $h \in \Sym^2 T_{\gamma(t)}^*M$ with 
	$\la \eta(t), D_h \nabla f(\gamma(t)) \ra > 0$. 
	Extend $h$ to a section $\in C^\ell(\Sym^2 T^*M)$. 
	By continuity, $\la \eta, D_h \nabla f \ra \neq 0$ 
	in some neighborhood $I$ of $t$. 
	Since $\gamma$ is a flow line between distinct critical points, 
	it is an embedding. 
	If $\phi$ is a smooth cutoff function supported 
	in an open set $U \subset M$ with $\gamma^{-1} (W) = I$, 
	then $\la \eta, D_{\phi h} \nabla f \ra \geq 0$ on $\Rb$, $>0$ near $t$. 
	This contradicts the second equation.
\end{proof}

\subsubsection{Passage to \texorpdfstring{$C^\infty$}{C^infty}}
We omit this.