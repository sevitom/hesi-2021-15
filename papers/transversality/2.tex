In this section, we establish the theoretical foundations for proving 
transversality theorems. 
Since we shall be mostly working with infinite-dimensional manifolds, 
we first recall the relevant theory and fix our notation. 
After that, we prove the Sard--Smale theorem and deduce from it 
a general transversality theorem, which we shall apply 
in the sections that follow.

For a Banach space $\Xc$, we denote by 
$\mathclose{\|} \cdot \mathclose{\|} _\Xc$ the norm defining $\Xc$.

\subsection{Banach manifolds}
Let $\Xc, \Yc$ be Banach spaces, $U \subset \Xc$ an open set. 
A map $F \colon U \to \Yc$ is (\textit{Fr{\'e}chet}) \textit{differentiable} 
at $x \in U$ if there is a continuous linear map $A \colon \Xc \to \Yc$ 
such that
\[
    \lim_{x \to x_0} \frac {\| F(x) - F(x_0) + A(x - x_0) \|_\Yc} {\| x - x_0 \|_\Xc}
        = 0.
\]
In this case, $A$ is called the \textit{differential} of $F$ at $x_0$ 
and denoted by $DF_{x_0}$. 
If $F$ is differentiable at each point in $U$, we consider 
$DF \colon U \to \Lc (\Xc, \Yc)$, where $\Lc (\Xc, \Yc)$ is 
the Banach space of continuous linear maps $X \to Y$. 
We say $F$ is \textit{continuously differentiable} if $DF$ is continuous. 
Iterating this definition, $F$ is $C^k$ is $DF$ is $C^{k-1}$.

With this notion of differentiability, we define $C^k$ diffeomorphisms, 
immersions, submersions, regular values, etc., just as 
in the finite-dimensional case. 
Similarly, a $C^k$ \textit{Banach manifold} modeled 
on a Banach space $\Xc$ is a Hausdorff space obtained by piecing together 
open sets in $\Xc$ by transition maps that are $C^k$ diffeomorphisms. 
Note that we do \textit{not} require a Banach manifold to be 
second countable, since the model space $\Xc$ itself 
may not be second countable. 
Tangent vectors are defined as equivalence classes of ordered triples 
so that $T_x U = \Xc$ for an open set $U \subset \Xc$. 
Much of the basic theory of finite-dimensional $C^k$ manifolds 
carries over to Banach manifolds. 
A standard reference in this direction is \cite{Lan}.

\begin{theorem}[Inverse mapping theorem]
    Let $\Xc, \Yc$ be $C^k$ Banach manifolds, 
    $F \colon \Xc \to \Yc$ a $C^k$ map. 
    If $x \in \Xc$ is such that $DF_x \colon T_x \Xc \to T_{F(x)} \Yc$ 
    is an isomorphism, then there are open neighborhoods $U, V$ 
	of $x, F(x)$ in $\Xc, \Yc$, respectively, such that 
	$F \colon U \to V$ is a $C^k$ diffeomorphism.
\end{theorem}

\begin{theorem}[Implicit mapping theorem]
    Let $\Xc, \Yc$ be $C^k$ Banach manifolds, 
	$F \colon \Xc \to \Yc$ a $C^k$ map. 
	If $y \in \Yc$ is a regular value of $F$, 
	then $F^{-1} (y)$ is a $C^k$ Banach submanifold of $\Xc$ 
	with $T_x F^{-1} (y) = \ker DF_x$.
\end{theorem}

For our applications, the maps are always sufficiently smooth, and 
we shall omit the technical verifications of smoothness for reasons of space.

\subsubsection{Interlude: Examples in geometry}
Let $M$ be a compact smooth manifold, 
$E \to M$ a smooth vector bundle. 
Fix a Riemannian metric on $M$, a bundle metric on $E$, 
and a metric connection on $E$. 
For a smooth section $s \in \Gamma (M, E)$, define
\begin{gather*}
    \|s\|_{C^k} := \sum _{i = 0} ^k \sup_M |\nabla^i s|, \\
    \|s\|_{W^{k, p}} 
		:= \Bigg( \sum _{i = 0} ^k \int_M |\nabla^i s|^p \Bigg) ^{1 / p}.
\end{gather*}
The completions of $\Gamma (M, E)$ with respect to these norms 
are Banach spaces which we denote by $C^k (M, E)$ or $W^{k, p} (M, E)$. 
We also have the space $C^\infty (M, E)$, which is a Fr{\'e}chet space 
defined by the sequence $( \mathclose{\|} \cdot \mathclose{\|} _{C^k} )_k$ 
of norms, i.e., a sequence $s_n \to s$ in $C^\infty (M, E)$ if and only if 
$\| s_n - s \|_{C^k (M, E)} \to 0$ for all $k$. 
Since $M$ is compact, it is easy to check that the topology 
on these spaces do not depend on the metrics and connections chosen.

It is technically more difficult to define manifold structures 
on spaces of sections of general fiber bundles, 
due to lack of linear structures. 
The case of trivial fiber bundles, which amounts to spaces of maps 
between two manifolds, is discussed in, e.g., \cite[Remark B.1.24]{MS}. 
We shall not bother with this, since knowledge of their tangent spaces 
already suffices in most applications.

\subsection{Fredholm theory}
The Fredholm property is a crucial technical condition 
for generalizing finite-dimensional phenomena. 
Fortunately, Fredholm maps abound in geometry. 
For example, elliptic differential operators are Fredholm. 
In this article, we omit the analytical verification of the Fredholm property 
and the computation of the Fredholm index for reasons of time. 
However, we do recall the definitions:

Let $\Xc, \Yc$ be Banach spaces. 
A continuous linear map $A \colon \Xc \to \Yc$ is \textit{Fredholm} 
if it has closed image and finite-dimensional kernel and cokernel. 
In this case, its (\textit{Fredholm}) \textit{index} is defined to be 
$\ind A := \dim \ker A - \dim \coker A$. 
It is standard that the space of 
Fredholm linear maps is open in the norm topology, and 
the Fredholm index is locally constant on this space.

Let $\Xc ,\Yc$ be Banach manifolds. 
A $C^1$ map $F \colon \Xc \to \Yc$ is \textit{Fredholm} if 
$DF_x \colon T_x \Xc \to T_{F(x)} \Yc$ is Fredholm for all $x \in \Xc$. 
If $\Xc$ is connected, then by continuity, $\ind DF_x$ is independent 
of $x \in \Xc$, so $F$ has a well-defined Fredholm index.

\subsection{The Sard--Smale theorem}
In 1965, Smale \cite{Sma} proved the following 
generalization of Sard's theorem:

\begin{theorem}[Sard--Smale]
	Let $\Xc, \Yc$ be $C^k$ Banach manifolds with $\Xc$ separable, 
	$F \colon \Xc \to \Yc$ a $C^k$ map Fredholm with index $m$, 
	where $k \geq \max (1, m+1)$. 
	Then a generic $y \in \Yc$ is a regular value of $F$, 
	so that $F^{-1} (y)$ is an $C^k$ submanifold of $\Xc$ 
	of dimension $m$.
\end{theorem}

The proof is by reducing to the finite-dimensional case 
using the Fredholm property. 
To describe this reduction, we introduce the following notion.

Let $\Xc, \Yc$ be Banach spaces, 
$A \colon \Xc \to \Yc$ a continuous linear map. 
A \textit{pseudo-inverse} of $A$ is a continuous linear map 
$B \colon \Yc \to \Xc$ such that $ABA = A$, $BAB = B$. 
In this case, we have 
$\Xc = \Xc_0 \oplus \Xc_1$, 
$\Yc = \Yc_0 \oplus \Yc_1$, 
where 
$\Xc_0 := \ker A$, $\Xc_1 := \im B$, 
$\Yc_0 := \ker B$, $\Yc_1 := \im A$. 
With respect to these splittings,
\[
	A = 
		\begin{pmatrix}
			0 & 0 \\
			0 & A_1
		\end{pmatrix},
	\quad 
	B = 
		\begin{pmatrix}
			0 & 0 \\
			0 & B_1
		\end{pmatrix},
\]
where $A_1 := A |_{\Xc_1}$ and $B_1 := B |_{\Yc_1}$ 
are mutual inverses.

It is easy to see that $A$ has a pseudoinverse if and only if 
$A$ has closed image, $\ker A$ has a complement in $\Xc$, and 
$\im A$ has a complement in $\Yc$. 
In fact, if 
$\Xc = \ker A \oplus \Xc_1$, 
$\Yc = \Yc_0 \oplus \im A$, 
we simply define $B$ by 
$B |_{\Yc_0} = 0$, 
$B |_{\im A} = (A |_{\Xc_1})^{-1}$. 
Since closed subspaces of finite dimension or codimension are 
always complemented, linear Fredholm maps admit pseudoinverses.

\begin{lemma}
	Let $\Xc, \Yc$ be Banach spaces, 
	$U \subset \Xc$ an open neighborhood of $0$, 
	$F \colon U \to \Yc$ a $C^k$ map with $F(0) = 0$. 
	Suppose $A := DF_0$ has a pseudoinverse $B$. 
	Then there is an open neighborhood $V$ of $0$ in $\Xc$, 
	a $C^k$ diffeomorphism $\Phi \colon V \to \Phi(V)$ 
	onto an open set $\Phi(V) \subset U$, 
	and a $C^k$ map $F_0 \colon V \to \Yc_0 = \ker B$ 
	such that $( F \circ \Phi ) (x) = F_0 (x) + Ax$ for $x \in V$, 
	and $\Phi(0) = 0$, $D\Phi_0 = \id_\Xc$, $F_0(0) = 0$, $(DF_0)_0 = 0$.
\end{lemma}

In other words, after changing the chart on the domain, 
$F$ takes the form $(F_0, A)$ with respect to the splitting 
$\Yc \cong \Yc_0 \oplus \Yc_1$.

\begin{proof}[Proof of lemma]
	We apply the inverse mapping theorem to 
	$\Psi \colon U \to \Xc,\, x \mapsto x + B ( F(x) - Ax )$ at $0$. 
	Since $\Psi(0) = 0$, $D\Psi_0 = \id_\Xc$, there is 
	an open neighborhood $W$ of $0$ such that 
	$\Psi \colon W \to \Psi(W)$ is a $C^k$ diffeomorphism. 
	Let $V := \Psi(W)$, $\Phi := \Psi^{-1} \colon V\to W$, 
	$F_0 := ( \id_{\Yc} - AB ) ( F \circ \Phi )$. 
	Since $A\Psi = A + ABF - ABA = ABF$ on $W$, 
	we have $A = ABF \circ \Phi$ on $V$, so 
	$F \circ \Phi = F_0 + ABF \circ \Phi = F_0 + A$ on $V$. 
	The rest is easy to verify.
\end{proof}

We now commence the proof proper of the Sard--Smale theorem.

\begin{step}{1}
	Any point in $X$ has a neighborhood $V$ such that 
	the set of regular values of $F |_V$ is dense in $\Yc$.
\end{step}
\begin{proof}
	Working in local charts, we are in the situation of the lemma. 
	Thus we may assume $F = (F_0, A)$ on $V$ with respect to 
	$\Yc = \Yc_0 \oplus \Yc_1$, where we retain the notation above. 
	The equation $F(x) = y$ becomes $y_0 = F_0 (x_0, x_1)$, 
	$y_1 = A_1 x_1$ for $x = x_0 + x_1 \in \Xc_0 \oplus \Xc_1$, 
	$y = y_0 + y_1 \in \Yc_0 \oplus \Yc_1$. 
	Thus $y$ is a regular value for $F |_V$ if and only if 
	$y_0$ is a regular value of $x_0 \mapsto F_0 (x_0, B_1 y_1)$, 
	which is a finite-dimensional map. 
	Then the claim follows from Sard's theorem for $C^k$ maps.
\end{proof}

Note that we cannot conclude that the set fo regular values 
of $F |_V$ is residual, since Sard's theorem only implies that 
the set of regular values of $F |_V$ is residual on each section 
$\Yc_0 \oplus \{ y_1 \}$.

\begin{step}{2}
	With $V$ constructed in Step 1, the set of regular values 
	of $F |_K$\footnote{We say $y \in \Yc$ is a regular value 
	of $F |_V$ if $DF_x \colon T_x \Xc \to T_y \Yc$ is surjective 
	for $x \in V \cap F^{-1} (y)$.} 
	is open for any closed set $K \subset V$.
\end{step}
\begin{proof}
	We may assume $V$ is bounded. 
	Let $(x_n)_n \subset K$ be a sequence of critical points of $F$ 
	with $F(x_n) \to y \in \Yc$. 
	Write $x_n = x_{n, 0} + x_{n, 1} \in \Yc_0 \oplus \Yc_1$. 
	Then $x_{n, 1} = BA x_n = B y_n \to By$, since $BF_0 = 0$. 
	In particular, $( x_{n, 0} )_n$ is bounded. 
	Since $\Xc_0$ is finite-dimensional, passing to a subsequence, 
	we may assume $x_{n, 0} \to x_0 \in \Xc_0$. 
	Then $x_n \to x := x_0 + By$. 
	Since the set of surjective continuous linear maps 
	between Banach spaces is open in the norm topology, 
	$DF_x = \lim DF_{x_n}$ cannot be surjective, 
	so $y = F(x)$ is a critical value.
\end{proof}

\begin{step}{3}
	Conclusion of the proof.
\end{step}
\begin{proof}[Proof of Sard--Smale]
	We have proved that any point in $\Xc$ has a neighborhood $V$ 
	such that the set of regular values of $F |_K$ is dense and open 
	for any closed set $K \subset V$. 
	Since $\Xc$ is separable, it can be covered 
	by countably such open sets. 
	But the set of regular values of $F$ is the intersection of those 
	of $F |_V$ as $V$ ranges over this countable cover.
\end{proof}

\subsection{A general transversality theorem}
Instead of applying directly the Sard--Smale theorem, 
we shall streamline our proofs using 
the following general transversality theorem:

\begin{theorem}[General transversality theorem]
	Let $\Xc, \Yc$ be separable Banach manifolds, 
	$\Ec \to \Xc \times \Yc$ a Banach vector bundle, 
	$\Sc \colon \Xc \times \Yc \to \Ec$ a section. 
	Suppose they are sufficiently smooth 
	and for all $(x, y) \in \Sc^{-1} (0)$,
	\begin{enumerate}
		\item[\emph{(a)}]
			$D_1\Sc_{(x, y)} \colon T_x \Xc \to \Ec_{(x, y)}$ 
			is Fredholm with index $m$;
		
		\item[\emph{(b)}] 
			$D\Sc_{(x, y)} \colon T_x \Xc \times T_y \Yc \to
			\Ec_{(x, y)}$\footnote{More precisely, this is the 
			differential $D\Sc_{(x, y)}$ followed by the vertical 
			projection $T_{\Sc(x, y)} \Ec \to \Ec_{(x, y)}$.
			This is possible since $T_{\Sc(x, y)} \Ec \cong 
			T_{(x, y)} (\Xc \times \Yc) \oplus \Ec_{(x, y)}$ 
			splits naturally where $\Sc(x,y)=0$.} 
			is surjective.
	\end{enumerate}
	Then for a generic $y \in \Yc$, 
	$\Sc^{-1} (0) \cap ( \Xc \times \{y\} )$ is an appropriately 
	smooth submanifold of dimensional $m$, and 
	$D_1 \Sc_{(x, y)} \colon T_x \Xc \to \Ec_{(x, y)}$ is surjective 
	for all $(x, y) \in \Sc^{-1} (0)$.
\end{theorem}

In applications, $\Sc$ is usually a nonlinear differential operator 
on the space $\Xc$ that depends on an auxiliary structure 
given by elements of $\Yc$, and the moduli space in question 
is the solution space of $\Sc = 0$. 
The desired transversality theorem is that the moduli space 
is a smooth manifold for a generic auxiliary structure. 
Incidentally, $\Sc^{-1} (0)$ is sometimes called 
the \textit{universal moduli space} in the literature.

\begin{proof}
	By the implicit mapping theorem, 
	(b) implies $\Sc^{-1} (0)$ is a Banach manifold. 
	We apply the Sard--Smale theorem to the projection 
	$\pi \colon \Sc^{-1} (0) \to \Yc$. 
	It suffices to show $\pi$ is Fredholm with index $m$. 
	For $(x, y) \in \Sc^{-1} (0)$, since 
	$T_{(x, y)} \Sc^{-1} (0) = \{ (v, w) \in T_x \Xc \times T_y \Yc \mid 
	D_1\Sc_{(x, y)} (v) + D_2\Sc_{(x, y)} (w) = 0 \}$, 
	$D\pi_{(x, y)} \colon (v, w) \mapsto w$, we see that 
	$\ker D\pi_{(x, y)} = \ker D_1\Sc_{(x, y)}$, 
	$\im D\pi_{(x, y)} = (D_2\Sc_{(x, y)})^{-1} (\im D_1\Sc_{(x, y)})$ is closed, 
	and $D_2\Sc_{(x, y)}$ induces an isomorphism 
	$\coker D\pi_{(x, y)} \to \coker D_1\Sc_{(x, y)}$, 
	so $D\pi_{(x, y)}$ is Fredholm with 
	$\ind D\pi_{(x, y)} = \ind D_1\Sc_{(x, y)}$ by (a). 
	For the last conclusion, 
	$T_x \Xc + T_{(x, y)} \Sc^{-1} (0) = T_{(x, y)} (\Xc \times \Yc)$ 
	since $y$ is a regular value of $\pi$, but $D\Sc_{(x, y)}$ vanishes 
	on $T_{(x, y)} \Sc^{-1} (0)$, so 
	$\im D_1\Sc_{(x, y)} = \im D\Sc_{(x, y)} = \Ec_{(x, y)}$.
\end{proof}

\begin{remark}
	Note that in the finite-dimensional case, this reduces to 
	the familiar parametric transversality theorem 
	with $\Yc$ the space of parameters that provides room for perturbation.
\end{remark}

\medskip

We have made a deliberate effort to recast the proofs 
of transversality theorems into the following four-step streamline:

\begin{enumerate}
	\item Setting the stage.
		
		Specify $\Xc, \Yc, \Ec, \Sc$ and their Banach manifold structures.
	
	\item Fredholm property.
		
		Check that $D_1\Sc_{(x, y)} \colon T_x \Xc \to \Ec_{(x, y)}$ is Fredholm.
	
	\item Regular value.
		
		Check that $D\Sc_{(x, y)} \colon T_x \Xc \times T_y \Yc \to \Ec_{(x, y)}$ 
		is surjective.
	
	\item Passage to $C^\infty$.
		
		See the next subsection.
\end{enumerate}

\subsection{Remarks on smoothness}
So far we have been solely concerned with Banach manifolds. 
However, many spaces in geometry do \textit{not} 
admit a natural Banach structure. 
This includes the space of smooth objects or objects on noncompact 
manifolds, which are merely Fr{\'e}chet manifolds. 
Fortunately, it is often possible to remedy this problem. 
We shall presently describe two ways to pass to smooth objects, 
i.e., to prove transversality theorems of the form 
``For a generic smooth ..., we have ...''

The first approach is due to Clifford H. Taubes. 
It is based on the observation that 
the set of good $C^\ell$ objects is open. 
Sometimes further modification is needed. 
Since we shall follow this approach here, 
the reader will see this in action.

The second approach is due to Andreas Floer. 
Instead of $C^\ell$ objects, one considers 
for $(\varepsilon_\ell)_\ell \subset \Rb^{>0}$ 
the Banach space $C_\varepsilon^\infty$ of $C^\infty$ objects 
with norm 
\[
	\mathclose{\|} \cdot \mathclose{\|} _{C_\varepsilon^\infty} := 
		\sum _{\ell = 0} ^\infty \varepsilon_\ell 
		\mathclose{\|} \cdot\mathcal{\|} _{C^\ell}.
\]
This approach is adopted by, e.g., \cite{Sch}.