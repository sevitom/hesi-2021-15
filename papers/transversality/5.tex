Mathematical gauge theory was inspired by physics and 
is now a basic tool in low-dimensional geometry. 
In this section, we sketch the proof of the fundamental transversality 
theorem that for a generic Riemannian metric on a compact $4$-manifold, 
the space of irreducible self-dual connections on a fixed principal 
$\SU(2)$-bundle has the structure of a smooth $5$-manifold. 
This is a major step in the proof of the celebrated 
diagonalizability theorem of S.\ K.\ Donaldson.

\subsection{Background material}
The purpose of this subsection is mainly to fix notation.

\subsubsection{Connections on principal bundles}
Let $G$ be a Lie group, $\gfr$ its Lie algebra. 
Let $M$ be a smooth manifold, 
$P \to M$ a smooth principal $G$-bundle. 
Given an open cover $(U_\alpha)_\alpha$ of $M$ on which 
$P \to M$ is trivial, a connection on $P$ can be described 
by connection $1$-forms $A_\alpha \in \Omega^1 (U_\alpha; \gfr)$. 
The difference of two connection $1$-forms on $P$ 
is a section $\in \Omega^1 (\ad P)$, so the space of connections 
on $P$ is an affine space modeled on $\Omega^1 (\ad P)$. 
For a connection $D$ on $P$, we denote by 
$F_D \in \Omega^2 (\ad P)$ the curvature $2$-form of $D$.

\subsubsection{Anti-self-dual connections}
Let $M$ be a compact smooth $4$-manifold, 
$P \to M$ a principal $\SU(2)$-connection. 
We shall usually identify it with the associated vector bundle 
$P \times_\rho \Cb^2$, where $\rho$ is 
the standard representation of $\SU(2)$ on $\Cb^2$. 
A connection on $P$ is \textit{reducible} if $P \times_\rho \Cb^2$ 
splits into the direct sum of two line bundles and 
the induced connection on it also splits. 
Otherwise it is \textit{irreducible}.

Fix a Riemannian metric $g$ on $M$. 
This induces the Hodge star operator $* \colon \Lambda^2 \to \Lambda^2$ 
with $*^2 = \id$. 
The subbundles $\Lambda^2_\pm := \ker(* \mp 1)$ are called 
the bundle of (\textit{anti-})\textit{self-dual} forms. 
This definition extends to forms with coefficients in any vector bundle. 
In particular, a connection $D$ on $P$ is (\textit{anti-})\textit{self-dual} 
if so is $F_D$. 
Such connections are automatically \textit{Yang--Mills} 
in the sense that they are critical points of the Yang--Mills functional. 
Since we shall make no use of this functional, we omit its definition.

\subsection{Moduli spaces of self-dual connections}
This subsubsection is devoted to sketching the proof of 
the following equivariant transversality theorem:

\begin{theorem}
	Let $M$ be a compact smooth $4$-manifold, 
	$P \to M$ a smooth principal $\SU(2)$-bundle. 
	Then for a generic Riemannian metric $g$ on $M$, 
	the moduli space $\Mc^*(P; g)$ is a smooth manifold of dimension $5$.
\end{theorem}

\noindent We intend to apply the general transversality theorem with
\begin{itemize}
	\item[$\Xc$:] 
		the space $\Ac^{k - 1, 2}$ of irreducible 
		$W^{k - 1, 2}$ connections on $P$;
	\item[$\Yc$:] 
		the space $\Cc^\ell := C^\ell (\GL(TM))$;
	\item[$\Ec$:] 
		the product bundle $\Ec^{k - 2, 2}$ with fiber 
		$W^{k - 2, 2} (\Lambda^2_- \otimes \ad P)$;
	\item[$\Sc$:] 
		the map $\Ac^{k - 1, 2} \times \Cc^\ell \to \Ec^{k - 2, 2},\, 
		(D, \varphi) \mapsto (D, \varphi, P_- ((\varphi^{-1})^*F_D))$.
\end{itemize}
Here $\ell \gg k$, $*$ is with respect to 
a fixed Riemannian metric $g$ on $M$. 
As $\varphi$ ranges over $\Cc^\ell$, 
$\varphi^*g$ ranges over all $C^\ell$ metrics on $M$.\footnote{This 
trick is from \cite{FU}. The technical advantage of this parameter space 
is that it turns $\Ec$ into a trivial bundle. 
One could also proceed with $\Yc$ the space of 
$C^\ell$ Riemannian metrics, much like in the previous subsection.} 
The Hodge star operator with respect to $\varphi^*g$ is 
$\varphi^* P_- (\varphi^{-1})^*$, so 
$\Mc^*(P; \varphi^*g) = 
\Sc^{-1} (0) \cap (\Ac^{k - 1, 2} \times \{\varphi\}) / \Gc^k$.

Let us explain the Banach manifold structures on these spaces. 
Since $\Cc^\ell$ is an open subset of the Banach space $C^\ell(\End TM)$, 
it is a smooth Banach manifold. 
In fact, it is a Banach Lie group with 
Lie algebra $T_{\id} \Cc^\ell = C^\ell(\Sym^2 T^*M)$. 
Recall that $\Ac^{k - 1, 2}$ is an affine space modeled on 
$W^{k - 1, 2} (\Lambda^1 \otimes \ad P)$, so it is a Banach manifold 
with $T_D \Ac^{k - 1, 2} = W^{k - 1, 2} (\Lambda^1 \otimes \ad P)$.

\medskip

Now we pause for a moment. 
The general transversality theorem only gives that 
$\Sc^{-1} (0) \cap (\Ac^{k - 1, 2} \times \{\varphi\})$ is a smooth 
manifold, and we have to quotient out by $\Gc^\ell$ to get the theorem. 
However, it is not feasible to do this in the end. 
Instead, we follow the following steps:
\begin{enumerate}
	\item Regular value.
		
		$0$ is a regular value of $\Sc$. 
		Thus the universal moduli space $\Sc^{-1} (0)$ is a Banach manifold.
		
	\item Slices.
		 
		For $D \in \Ac^{k - 1, 2}$, $\Ac^{k - 1, 2}$ is locally 
		$\Gc^\ell$-equivariantly diffeomorphic to $\ker D^* \times \Gc^k$ 
		near $D$, where $\Gc^k$ is the space of $C^k$ gauge transformations. 
		Thus the orbit space $\Xc^{k - 1, 2} := \Ac^{k - 1, 2} / \Gc^k$ 
		is a Banach manifold. 
		This is a simple application of the implicit mapping theorem. 
		Similarly, $\Sc^{-1} (0) / \Gc^\ell$ is a Banach manifold.
		
	\item Fredholm property.
		
		The projection $\pi \colon \Sc^{-1} (0) / \Gc^k \to \Cc^\ell$ 
		is Fredholm with index $5$. 
		This follows from the Atiyah--Singer index theorem 
		for the elliptic complex 
		\[
			0 \to \Omega^0 (\ad P) \xrightarrow{D} \Omega^1 (\ad P) 
			\xrightarrow{D_1\Sc_{(D, \varphi)}} \Omega^2(\ad P) \to 0.
		\]
		
	\item Application of Sard--Smale.
		
		The theorem now follows immediately from the Sard--Smale theorem 
		since $\Mc^*(P; \varphi^*g) = \pi^{-1} (\varphi)$.
\end{enumerate}

See \cite{FU} for the complete proof.