A transversality theorem is a theorem stating that a certain desirable 
property can be achieved by an arbitrarily small perturbation, or that 
the property in question is \textit{generic}. 
We begin by making this notion precise. 

Let $X$ be a topological space, $P$ a property that points of $X$ may 
or may not satisfy. We say $P$ holds for \textit{generic} $x \in X$ if 
the set 
\[
    \{ x \in X \mid P \text{ holds for } x \}
\]
is \textit{residual} in the sense of Baire, i.e., it contains a countable intersection 
of dense open sets. 
Note that residual sets in complete metric spaces are dense, 
as a consequence of the Baire category theorem.

In classical differential topology, the prototypical example of a 
transversality theorem is the result that transversal intersection of 
two submanifolds is generic. 
This follows from a general parametric transversality theorem, 
which, in turn, relies on Sard's theorem. 
The same technique has been extended to the infinite-dimensional setting 
so that it applies to, e.g., spaces of maps between manifolds and spaces 
of sections of fiber bundles.
Explaining and illustrating the use of this extension is the purpose of the present paper.

Most of the transversality theorems that we shall prove fit into a larger 
program, which can be abstractly formulated as follows. 
Here one is interested in the moduli space $\Mc$ of a class of geometric 
objects, typically defined by a differential equation. Such a program 
studies the structure of $\Mc$ in the following aspects:

\begin{enumerate}
    \item Transversality.
		
        In the generic case, $\Mc$ is a finite-dimensional smooth manifold. 
        The present paper concerns itself with how one proves such a result.
    
    \item Orientation.
		
        The moduli space $\Mc$ is orientable.
		This seemingly innocent fact could be highly nontrivial. 
        Incidentally, an important tool for proving this is the determinant 
		line bundle on the space of Fredholm operators.
    
    \item Compactness.
		
        Any sequence in $\Mc$ has a subsequence
        ``converging'' in a suitable sense. 
        Adding all possible limits to $\Mc$, 
        we obtain a compactification $\ol{\Mc}$. 
        Of course, such results are necessarily quite analytic in nature.
		
    \item Gluing.
		 
		The compactified moduli space $\ol{\Mc}$ is a smooth manifold 
		with an expected boundary: 
		Taking hint from the previous step, we expect the boundary to be 
		the space of another class of geometric objects. 
		That all such objects arise as boundary points 
		is proved by a gluing procedure.
\end{enumerate}

As an example, $\Mc$ could be the moduli space of flow lines in Morse 
or Floer homology, in which case the boundary (or corner) points are 
broken flow lines.