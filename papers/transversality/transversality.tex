\documentclass[twoside]{article}

\input{../common/preamble}
\newcommand{\Hom}{\operatorname{Hom}}
\newcommand{\Sp}{\mathbf{sp}}
\newcommand{\colim}{\mathrm{colim}}
\newcommand{\im}{\operatorname{im}}
\newcommand{\vp}{\varphi}
\newcommand{\supp}{\mathrm{supp}}
\newcommand{\tr}{\operatorname{tr}}
\newcommand{\YP}{]Y[_\CP}
\newcommand{\XP}{]X[_\CP}
\newcommand{\ZP}{]Z[_\CP}
\newcommand{\HSn}{H^n_{cris}(X/S)}
\newcommand{\Hkn}{H^n_{cris}(X/W(k))}
\newcommand{\HSd}{H^\cdot_{cris}(X/S)}
\newcommand{\Hkd}{H^\cdot_{cris}(X/W(k))}
\newcommand{\HCd}{H^\cdot_{conv}(X)}
\newcommand{\HCn}{H^n_{conv}(X)}
\newcommand{\HRd}{H^\cdot_{rig}(X)}
\newcommand{\HRn}{H^n_{rig}(X)}
\newcommand{\DRX}{\Omega_{\XP/K}^\cdot}
\newcommand{\DRY}{\Omega_{\YP/K}^\cdot}
\newcommand{\Spec}{\mathrm{Spec}}
\newcommand{\Spf}{\mathrm{Spf}}
\newcommand{\Spp}{\mathrm{Sp}}
\newcommand{\Frac}{\mathrm{Frac}}
\newcommand{\DRXn}{\Omega_{\XP/K}^n}
\newcommand{\BA}{{\mathbb {A}}}
\newcommand{\BB}{{\mathbb {B}}}
\newcommand{\BC}{{\mathbb {C}}} 
\newcommand{\BD}{{\mathbb {D}}}
\newcommand{\BE}{{\mathbb {E}}} 
\newcommand{\BF}{{\mathbb {F}}}
\newcommand{\BG}{{\mathbb {G}}} 
\newcommand{\BH}{{\mathbb {H}}}
\newcommand{\BI}{{\mathbb {I}}} 
\newcommand{\BJ}{{\mathbb {J}}}
\newcommand{\BK}{{\mathbb {K}}} 
\newcommand{\BL}{{\mathbb {L}}}
\newcommand{\BM}{{\mathbb {M}}} 
\newcommand{\BN}{{\mathbb {N}}}
\newcommand{\BO}{{\mathbb {O}}} 
\newcommand{\BP}{{\mathbb {P}}}
\newcommand{\BQ}{{\mathbb {Q}}} 
\newcommand{\BR}{{\mathbb {R}}}
\newcommand{\BS}{{\mathbb {S}}} 
\newcommand{\BT}{{\mathbb {T}}}
\newcommand{\BU}{{\mathbb {U}}} 
\newcommand{\BV}{{\mathbb {V}}}
\newcommand{\BW}{{\mathbb {W}}} 
\newcommand{\BX}{{\mathbb {X}}}
\newcommand{\BY}{{\mathbb {Y}}} 
\newcommand{\BZ}{{\mathbb {Z}}}
\newcommand{\CA}{{\mathcal {A}}} 
\newcommand{\CB}{{\mathcal {B}}}
\newcommand{\CC}{{\mathcal {C}}} 
\newcommand{\CD}{{\mathcal {D}}}
\newcommand{\CE}{{\mathcal {E}}} 
\newcommand{\CF}{{\mathcal {F}}}
\newcommand{\CG}{{\mathcal {G}}} 
\newcommand{\CH}{{\mathcal {H}}}
\newcommand{\CI}{{\mathcal {I}}} 
\newcommand{\CJ}{{\mathcal {J}}}
\newcommand{\CK}{{\mathcal {K}}} 
\newcommand{\CL}{{\mathcal {L}}}
\newcommand{\CM}{{\mathcal {M}}} 
\newcommand{\CN}{{\mathcal {N}}}
\newcommand{\CO}{{\mathcal {O}}} 
\newcommand{\CP}{{\mathcal {P}}}
\newcommand{\CQ}{{\mathcal {Q}}} 
\newcommand{\CR}{{\mathcal {R}}}
\newcommand{\CS}{{\mathcal {S}}} 
\newcommand{\CT}{{\mathcal {T}}}
\newcommand{\CU}{{\mathcal {U}}} 
\newcommand{\CV}{{\mathcal {V}}}
\newcommand{\CW}{{\mathcal {W}}} 
\newcommand{\CX}{{\mathcal {X}}}
\newcommand{\CY}{{\mathcal {Y}}} 
\newcommand{\CZ}{{\mathcal {Z}}}

\addbibresource{transversality.bib}

\begin{document}

\title{Transversality Theorems by Example}
\author{Xie Yuxiao\footnote{谢雨潇, 清华大学数学系数 80 班.}}

\begin{abstract}
	We prove a general transversality theorem in the setting of Fredholm 
	maps between Banach manifolds. After that, we illustrate its power by
	describing applications in three geometric contexts.
\end{abstract}

\tableofcontents
\bigskip

\begin{convention}
	Unless otherwise stated, the word ``manifold'' refers to one which 
	has no boundary but which is not necessarily smooth. 
	Whenever we write $C^k$ or $W^{k, p}$, it is tacitly understood 
	that $k \geq 1$, $1 < p < \infty$.
\end{convention}

\section{Introduction}

A transversality theorem is a theorem stating that a certain desirable 
property can be achieved by an arbitrarily small perturbation, or that 
the property in question is \textit{generic}. 
We begin by making this notion precise. 

Let $X$ be a topological space, $P$ a property that points of $X$ may 
or may not satisfy. We say $P$ holds for \textit{generic} $x \in X$ if 
the set 
\[
    \{ x \in X \mid P \text{ holds for } x \}
\]
is \textit{residual} in the sense of Baire, i.e., it contains a countable intersection 
of dense open sets. 
Note that residual sets in complete metric spaces are dense, 
as a consequence of the Baire category theorem.

In classical differential topology, the prototypical example of a 
transversality theorem is the result that transversal intersection of 
two submanifolds is generic. 
This follows from a general parametric transversality theorem, 
which, in turn, relies on Sard's theorem. 
The same technique has been extended to the infinite-dimensional setting 
so that it applies to, e.g., spaces of maps between manifolds and spaces 
of sections of fiber bundles.
Explaining and illustrating the use of this extension is the purpose of the present paper.

Most of the transversality theorems that we shall prove fit into a larger 
program, which can be abstractly formulated as follows. 
Here one is interested in the moduli space $\Mc$ of a class of geometric 
objects, typically defined by a differential equation. Such a program 
studies the structure of $\Mc$ in the following aspects:

\begin{enumerate}
    \item Transversality.
		
        In the generic case, $\Mc$ is a finite-dimensional smooth manifold. 
        The present paper concerns itself with how one proves such a result.
    
    \item Orientation.
		
        The moduli space $\Mc$ is orientable.
		This seemingly innocent fact could be highly nontrivial. 
        Incidentally, an important tool for proving this is the determinant 
		line bundle on the space of Fredholm operators.
    
    \item Compactness.
		
        Any sequence in $\Mc$ has a subsequence
        ``converging'' in a suitable sense. 
        Adding all possible limits to $\Mc$, 
        we obtain a compactification $\ol{\Mc}$. 
        Of course, such results are necessarily quite analytic in nature.
		
    \item Gluing.
		 
		The compactified moduli space $\ol{\Mc}$ is a smooth manifold 
		with an expected boundary: 
		Taking hint from the previous step, we expect the boundary to be 
		the space of another class of geometric objects. 
		That all such objects arise as boundary points 
		is proved by a gluing procedure.
\end{enumerate}

As an example, $\Mc$ could be the moduli space of flow lines in Morse 
or Floer homology, in which case the boundary (or corner) points are 
broken flow lines.

\section{Transversality in general}



% 建立在上面的一些内容之上,数学家发展了一门叫做非标准分析的学科.非标准实数是实数的扩充,保持了实数大量的性质,并且最有趣的是,很多数学研究上面的结论,例如分析,代数,拓扑,概率,方程等,都有一个更加丰富的非标准分析的版本与之对应.非标准分析的优点在于,首先,非标准的版本的理论会表现得更加直观,并且因此更加容易证明,其次,在原本的理论和非标准版本的理论之间有一种普遍的转移机制,如果证明了二者之一,那与之对应的理论也会成立.

在数学分析中, 函数的极限、连续性、导数等概念都是通过 $\epsilon$-$\delta$ 语言来定义的.
在超实数中, 我们有一个现成的无穷小元素 $\epsilon$,
它可以用来绕过 $\epsilon$-$\delta$ 语言, 而直接定义极限、连续性、导数等概念,
这样定义能与我们的直观更加相符.
这一门学科叫做\textbf{非标准分析} (nonstandard analysis).

\subsection{非标准实数}

% 首先,应用的数并不是所有超实数,那太多了,而是其中$S_{\omega^\omega}$的部分.对于这部分超实数,叫它们非标准实数,非标准实数的全体用$*\mathbb{R}$来表示.对于非标准实数,有不同于超实数定义方式的等价刻画.

在非标准分析中,
我们并不使用所有的超实数, 因为它们太多了, 不构成集合.
非标准分析使用的是\textbf{非标准实数}%
\footnote{也常译作\textbf{超实数}.}
(hyperreal number) $^* \mathbb{R}$,
也就是直到第 $\omega_1$ 天定义出来的超实数,
其中 $\omega_1$ 是最小的不可数序数. 严格地说, 我们定义
\[
    ^* \mathbb{R} =
    \mathbb{S}_{< \omega_1},
\]
可以说明它对四则运算封闭,
从而是一个域.
它包含所有的实数,
它的大小 (势) 和实数集相同.

% 一个数学理论和它的非标准分析的版本之间的对应需要用到抽象的集合论中的概念,这里只是粗浅地描述这件事情.一个转移包含了两方面的内容:

% \begin{enumerate}
%     \item $\mathbb{R}$到$*\mathbb{R}$的数学实体之间的转移.
%     \item 语言$\mathscr{L}_{\mathbb{R}}$中关于$\mathbb{R}$中数学实体的语句$\Phi$到语言$\mathscr{L}_{*\mathbb{R}}$中关于$*\mathbb{R}$中数学实体的语句$*\Phi$的转移.
% \end{enumerate}

% 这些转移是用逻辑学很严密地定义的,但是这里可以简单地理解这件事情,把实数的对象看成超实数的对象,把关于实数的具有一定结构的命题变成关于超实数的命题.

% 对于一个关于实数的性质,可以把它抽象成语言$\mathscr{L}_{\mathbb{R}}$中的语句$\Phi$,然后证明语言$\mathscr{L}_{*\mathbb{R}}$中的语句$*\Phi$,之后用这个转移逆运算,证明想要的结论.

非标准分析的威力实际上不在于非标准实数的构造,
而在于一个叫做\textbf{转移原理} (transfer principle) 的性质:
对任意关于实数 $\mathbb{R}$ 的命题 (当然, 对命题有些限制),
它为真当且仅当对应的关于非标准实数 $^* \mathbb{R}$ 的命题为真.
这意味着, 要证明一个关于 $\mathbb{R}$ 的定理,
只需要在 $^* \mathbb{R}$ 上证明这个定理,
就能直接知道原来的关于 $\mathbb{R}$ 的定理是正确的.

转移原理也意味着,
$\mathbb{R}$ 上的函数可以转移到 $^* \mathbb{R}$ 上,
成为 $^* \mathbb{R}$ 上的函数,
然后在 $^* \mathbb{R}$ 上定义其极限、连续性、导数等概念,
定义出来的概念与 $\mathbb{R}$ 上的定义是相同的.

为了做到这种转移,
在非标准分析中,
我们用另一种等价的方式来定义非标准实数.
我们把非标准实数定义为形如 $\{ a_n \}_{n \in \mathbb{N}}$ 的实数列, 例如:
\begin{align*}
    x &= (x, x, x, x, x, \dotsc) \qquad (x \in \mathbb{R}), \\
    \omega &= (1, 2, 3, 4, 5, \dotsc), \\
    \omega + 1 &= (2, 3, 4, 5, 6, \dotsc), \\
    \omega^2 &= (1, 4, 9, 16, 25, \dotsc), \\
    \epsilon = 1/\omega &= (1, \ 1/2, \ 1/3, \ 1/4, \ \dotsc), \\
    \sqrt{\omega} + \epsilon &= (1 + 1, \ \sqrt{2} + 1/2, \ \sqrt{3} + 1/3, \ \dotsc),
\end{align*}
等等. 严格的定义如下:

% 首先看$*\mathbb{R}$的刻画.取一个$\mathbb{R}^{\mathbb{N}}$的极大真子理想$F$,然后定义$*\mathbb{R}=\mathbb{R}^{\mathbb{N}}/F$.这里的商掉一个极大理想可以视为把一些$\mathbb{R}^{\mathbb{N}}$中的元素$u=(u_n),v=(v_n)\in \mathbb{R}^{\mathbb{N}}$等同起来.数学中,一个“环”商掉一个“极大理想”生成一个“域”.这个过程是依赖于$F$的选取的.如果这里的$F$并不形如$F=\{(u_n):u_i=0\}$,那么这个域扩张的$r\in\mathbb{R}\mapsto (r,r,\cdots)\in \mathbb{R}^{\mathbb{N}}$是严格的域扩张. $*\mathbb{R}$上面的运算都是逐位运算的.

\begin{construction}
    记 $\mathbb{R}^{\mathbb{N}}$ 是所有形如 $\{ a_n \}_{n \in \mathbb{N}}$ 的实数列构成的环,
    其中加法、乘法都按照分量来定义.
    我们注意到, 形如
    \[
        I_k =
        \bigl\{
            \{ a_n \}_{n \in \mathbb{N}} \in \mathbb{R}^{\mathbb{N}} \bigm|
            a_k = 0
        \bigr\}
    \]
    的集合是 $\mathbb{R}^{\mathbb{N}}$ 的极大理想,
    并且 $\mathbb{R}^{\mathbb{N}} / I_k \simeq \mathbb{R}$.
    但是, $\mathbb{R}^{\mathbb{N}}$ 还有别的极大理想:
    例如, 理想
    \[
        \bigl\{
            \{ a_n \}_{n \in \mathbb{N}} \in \mathbb{R}^{\mathbb{N}} \bigm|
            \text{只有有限个 $a_n$ 非零}
        \bigr\}
    \]
    一定包含于某个极大理想, 而这个极大理想不可能是任一个 $I_k$.
    我们就取一个不同于所有 $I_k$ 的极大理想 $I$, 并定义
    \[
        ^* \mathbb{R} = \mathbb{R}^{\mathbb{N}} / I.
    \]
    因为交换环商去极大理想能得到域,
    所以这里 $^* \mathbb{R}$ 是一个域.
    我们就把这个 $^* \mathbb{R}$ 叫做\textbf{非标准实数}.
    可以证明这样得到的 $^* \mathbb{R}$ 和之前的定义相同,
    不过这超出了本文的范围. 读者可参见 \cite{unification}.
\end{construction}

\begin{remark}
    这里, 极大理想 $I$ 的存在性依赖于选择公理.
    也就是说, 我们无法直接构造出一个极大理想, 只能证明其存在性.
    下文的一些构造将会依赖于 $I$ 的选取,
    但无论 $I$ 如何选取, 转移原理都是正确的.
\end{remark}

\begin{remark}
    非标准实数域 $^* \mathbb{R}$ 是一个完备有序域,
    这里完备性是指所有 Cauchy 序列都收敛.
    在数学分析中, 我们知道满足 Archimedes 原理的完备有序域一定同构于 $\mathbb{R}$.
    这里, $^* \mathbb{R}$ 是一个不满足 Archimedes 原理的完备有序域的例子.
\end{remark}

% \begin{remark}
% 可以取$\mathbb{N}^{\mathbb{N}}/F$记为$*\mathbb{N}$,同样还可以有$*\mathbb{Q}$.
% \end{remark}

\begin{remark}
    我们可以定义\textbf{非标准整数} $^* \mathbb{Z}$
    为 $\mathbb{Z}^{\mathbb{N}} \subset \mathbb{R}^{\mathbb{N}}$
    在 $^* \mathbb{R}$ 中的像,
    也可以定义\textbf{非标准有理数} $^* \mathbb{Q}$
    为 $\mathbb{Q}^{\mathbb{N}} \subset \mathbb{R}^{\mathbb{N}}$
    在 $^* \mathbb{R}$ 中的像, 等等.
    例如, $\omega$ 是一个非标准整数;
    $\omega + 1/2$ 不是非标准整数, 但它是非标准有理数.
\end{remark}

% \begin{example}
% $\omega=(1,2,3,\cdots)$是一个不落在$\mathbb{R}$里面的非标准整数.
% \end{example}

% 如果让$U=\{Z(s):s\in F\}$,这里$s$作为一个$\mathbb{N}\to \mathbb{R}$的函数,$Z(s)$代表它的零点集.我们在$*\mathbb{R}$上面可以定义序:$(u_n)\leq (v_n)$当且仅当$\{n:u_n\leq v_n\}\in U$.我们发现这个序是和原来的$\mathbb{R}$在嵌入$*$之下是一致的.
% \begin{remark}
% 这里的$U$实际上是一个超滤子,它和极大理想$F$之间有一个一一对应的关系.
% \end{remark}


% 运用这个$U$,可以定义关系的转移.
% 如果有一个$\mathbb{R}$上的$n$元关系$P$,那么我们定义$*\mathbb{R}$上关于$s^i$的$n$元关系$*P$为$\{j:P<s^1_j,s^2_j,\cdots,s^n_j>\}\in U$,这里$s^i=(s^i_j)_{j\in N}$.因为当$r\in \mathbb{R}$的时候$r=*r\in*\mathbb{R}$,$*P$看做$P$的一个扩张.

% 通过这个定义,我们可以把$\mathbb{R}$上的$n$元函数推广成$*\mathbb{R}$上的$n$元函数.$f(r^1,\cdots,r^n)=r^{n+1}\in \mathbb{R}$作为一个$n+1$元关系$P$,得到一个$n+1$元关系$*P(s^1,\cdots,s^{n+1})$,具体写出来为$\{j:s^{n+1}_j=f(s^1_j,\cdots,s^n_j)\}\in U$.这恰好对应一个非标准实数域上的$n$元函数,并且是由$f$的定义域扩张得到的.
% \begin{example}
% 对于$\mathbb{R}$上的函数$f(x)=\cos(x)$,$(*f)(1,2,3,\cdots)=(\cos {1},\cos{2},\cos{3},\cdots)$.
% \end{example}

通过这种方式, 我们可以把 $\mathbb{R}$ 上的函数转移到 $^* \mathbb{R}$ 上.

\begin{construction}
    设 $f \colon \mathbb{R} \to \mathbb{R}$
    是任何一个函数 (不一定连续).
    它转移到 $^* \mathbb{R}$ 上得到的函数
    $^* f \colon {}^* \mathbb{R} \to {}^* \mathbb{R}$
    定义为
    \[
        (^* f) (a_1, a_2, a_3, \dotsc) =
        \bigl( f (a_1), f (a_2), f (a_3), \dotsc \bigr) .
    \]
    在无歧义时, 也直接把 $^* f$ 记为 $f$.
\end{construction}

\begin{example}
    由这个构造, 我们有
    \[
        \sin \omega =
        \bigl( \sin 1, \sin 2, \sin 3, \dotsc \bigr).
    \]
    我们也可以得出一个奇怪但正确的等式:
    \[
        \sin (\omega \uppi) = (0, 0, 0, 0, \dotsc) = 0.
    \]
    然而,
    \[
        \cos (\omega \uppi) = (-1, 1, -1, 1, \dotsc),
    \]
    注意到它的平方等于 $1$, 而在域 $^* \mathbb{R}$ 中, $1$ 的平方根至多有两个, 从而
    \[
        \cos (\omega \uppi) = \pm 1.
    \]
    那么, 它到底是 $1$ 还是 $-1$ 呢?
    这取决于构造非标准实数时,
    极大理想 $I \subset \mathbb{R}^{\mathbb{N}}$ 的选取,
    两个值都是有可能的.
\end{example}

\begin{example}
    我们也有
    \begin{align*}
        \sin \epsilon
        & = \bigl( \sin 1, \sin (1/2), \sin (1/3), \dotsc \bigr) \\
        & = \sum_{n=0}^\infty \frac{(-1)^n}{(2n + 1)!} \epsilon^{2n + 1},
    \end{align*}
    这是因为, 我们知道
    \[
        \epsilon^n = (1, \ 1 / 2^n, \ 1 / 3^n, \ \dotsc),
    \]
    按照这个公式计算上面的和式, 就能得到
    $\bigl( \sin 1, \sin (1/2), \sin (1/3), \dotsc \bigr)$.
    
    这个例子展示了非标准分析的威力:
    我们还没有引入任何微积分,
    就已经得到了函数的 Taylor 展开!
\end{example}

% 逻辑学中会把一个集合,和上面的一些多元关系,多元函数的全体叫做一个简单系统.先把实数集$\mathscr{R}$作为一个集合,然后加入上面所有的关系和函数,成为一个简单系统.按照之前的论述,可以把这些对象全部延拓到一个非标准实数的简单系统上面去.

% 如果用$mon(0)$表示那些绝对值小于所有正实数的非标准实数,再用$Gal(0)$表示那些绝对值小于某个正实数的非标准实数,前者表示所有的无穷小非标准实数,后者表示所有有界的非标准实数,它们和$*\mathbb{Q}$的交记为${mon}_{\mathbb{Q}}(0),{Gal}_{\mathbb{Q}}(0)$,那么存在唯一的满代数同态$st:{Gal}_{\mathbb{Q}}(0)\to R$满足$s-st(s)\in {mon}_{\mathbb{Q}}(0)$,所以有域同构$\mathbb{R}={Gal}_{\mathbb{Q}}(0)/{mon}_{\mathbb{Q}}(0)$.这是一种类似于$Cauchy$完备化的方式,用有理数域构造实数域.  同样还有$\mathbb{R}=Gal(0)/mon(0)$.

\subsection{非标准微积分}

使用非标准分析的语言, 我们可以重新构建微积分的理论.
作为例子, 我们使用非标准分析来定义函数的极限、导数等概念.

\begin{construction}
    我们回忆, 每个超实数 $x$ 能唯一地写成
    \[
        x = \sum_{n \in \mathbb{S}} a_n \, \omega^n
    \]
    的形式 (定理 \ref{thm-base-omega-expansion}), 其中 $a_n \in \mathbb{R}$.
    在这个展开式中, $\omega^0 = 1$ 这一项的系数 $a_0$
    被称为 $x$ 的\textbf{标准部分} (standard part),
    记为 $\st x$.
    例如, $\st \epsilon = 0$, $\st 1 = 1$, $\st \omega = 0$.
\end{construction}

% 微积分中极限之类的定义用“$\epsilon-\delta$语言”可以照搬到非标准实数域上面,只要把这里面的$\epsilon-\delta$都取非标准实数.

% 因为映射$st$本身就反映了一些逼近和极限的信息,可以证明一些微积分相关的有趣结论.

% \begin{theorem}\label{cts}
% 一个区间$[a,b]$上的实值函数$f$,它是连续的当且仅当对于任意非标准实数$x\in*[a,b]$,有$*f(x)=*f(st(x))$.
% \end{theorem}

% \begin{theorem}\label{dif}
% 一个实值函数$f$在$x\in \mathbb{R}$处可微当且仅当对于任意$h\in mon(0)$,$st(\frac{*f(x+h)-*f(x)}{h})$存在并且独立于$h$的选取,并且此时该值就定义为$f^\prime(x)$.
% \end{theorem}

\begin{definition}
    我们说 $h \in {}^* \mathbb{R}$ 是一个\textbf{无穷小},
    如果它非零, 并且它的绝对值 (作为非标准实数) 小于任何正实数.
\end{definition}

\begin{definition} [极限]
    设 $f \colon \mathbb{R} \to \mathbb{R}$ 是一个函数, 设 $x_0 \in \mathbb{R}$.
    如果存在 $y \in \mathbb{R}$, 使得对任何无穷小 $h \in {}^* \mathbb{R}$, 都有
    \[
        \st f (x_0 + h) = y,
    \]
    就说 $y$ 是 $f (x)$ 在 $x \to x_0$ 时的\textbf{极限}, 记为
    \[
        \lim _{x \to x_0} f (x) = y.
    \]
\end{definition}

在数学分析中, 函数的连续性、函数的导数都是通过极限来定义的.
现在, 我们更换了极限的定义,
连续性和导数的概念可以以直观的方式写下来.

\begin{definition}
    设 $f \colon \mathbb{R} \to \mathbb{R}$ 是一个函数, 设 $x \in \mathbb{R}$.
    我们说 $f$ 在 $x$ 处\textbf{连续}, 如果对任何无穷小 $h \in {}^* \mathbb{R}$, 有
    \[
        \st f (x + h) = f (x).
    \]
\end{definition}

\begin{definition}
    设 $f \colon \mathbb{R} \to \mathbb{R}$ 是一个函数, 设 $x \in \mathbb{R}$.
    我们说 $f$ 在 $x$ 处\textbf{可导},
    如果存在实数 $f' (x) \in \mathbb{R}$, 称为 $f$ 在 $x$ 处的\textbf{导数},
    使得对任何无穷小 $h \in {}^* \mathbb{R}$, 有
    \[
        \st \frac{f (x + h) - f(x)}{h} = f' (x).
    \]
\end{definition}

% 可以定义$f$在点$a$处是微连续的,如果$x\approx a$,则有$f(x)\approx f(a)$,换句话说,$f(x)=f(st(x))$.可以看出,这种微连续的定义中,是只用无穷小的非标准实数去逼近$0$的,并且不像“$\epsilon-\delta$语言”中的那样,需要对任意的$\epsilon$和$\delta$去判断,只需要对所有的无穷小一致地去判断.

% 所以定理\ref{cts}意味着,一个实值函数$f$连续当且仅当$*f$微连续.定理\ref{dif}意味着,实值函数$f$在$x$处可微当且仅当,$g(h):=\frac{*f(x+h)-*f(x)}{h}$在$0$处微连续.

% 这表示,对于一个可导的函数$f$,它在$x$处的导数为$f^\prime(x)=st(\frac{f(x+\epsilon)-f(x)}{\epsilon})$.(这里的$\epsilon$可以用任意的$mon(0)$中元素代替)

% 我们可以用这种观点简单证明链式法则:
% 如果$g(x+\epsilon)\neq g(x)$,
% \[\begin{aligned}
%     &st(\frac{f(g(x+\epsilon))-f(g(x))}{\epsilon})\\
%     =&st(\frac{f(g(x+\epsilon))-f(g(x))}{g(x+\epsilon)-g(x)})\times st(\frac{g(x+\epsilon)-g(x)}{\epsilon})\\
%     =&f^\prime(g(x))\times g^\prime(x).\\
% \end{aligned}\]
% 这是因为$g(x+\epsilon)-g(x)$是无穷小的.并且$st$是个代数同态,和乘法交换.

% \begin{example}
% 计算指数函数$f(x)=e^x$在$x_0=(x_1,x_2,\cdots)$处的导数. 若$h=(h_1,h_2,\cdots)\in {mon}(0)$,则$f^\prime(x_0)=st((\frac{e^{x_1+h_1}-e^{x_1}}{h_1},\cdots))\equiv (e^{x_1},e^{x_2},\cdots)=f(x_0)$. 因此$f^\prime(x)=f(x)$.
% \end{example}

\begin{example}
    我们来计算指数函数 $\exp \colon \mathbb{R} \to \mathbb{R}$ 的导数:
    \[
        \exp' (0)
        = \st \frac{\exp h - 1}{h}
        = \st \left( \sum _{n=0} ^\infty \frac{h^{n - 1}}{n!} - \frac{1}{h} \right)
        = 1.
    \]
\end{example}

这样的定义也能够简化很多命题的证明, 这里我们举一个例子.

\begin{example}
    我们来证明求导的链式法则
    \[
        (f \circ g)' (x) = f' (g (x)) \, g' (x).
    \]
    如果 $g(x + \epsilon) \neq g(x)$, 那么
    \begin{align*}
        \text{左边}
        & =
        \st \frac{f(g(x + \epsilon)) - f(g(x))}{\epsilon} \\
        & =
        \st \frac{f(g(x + \epsilon)) - f(g(x))}{g(x + \epsilon) - g(x)}
        \st \frac{g(x + \epsilon) - g(x)}{\epsilon}
        = \text{右边}.
    \end{align*}
    这是因为对有限大的数而言, $\st$ 与乘法相容.
    如果 $g(x + \epsilon) = g(x)$, 那么上式为 $0$,
    链式法则也成立.
\end{example}

% 积分的理论可以抽象出来.

% 如果$F$是$\mathbb{R}^E$的一个向量子空间,那么一个泛函$I:F\to\mathbb{R}$叫做一个$F$上面的积分,如果$f\in F,f\geq0$时$If\geq0$.$(I,F)$是一个$E$上的积分结构. 它对于一个函数,会给出那个函数值积分之后的数. 对Riemann可测函数和Riemann积分,Lebesgue可测函数和Lebesgue积分都具有这样子的结构.

% 相应地,如果把上面的$\mathbb{R}$都换成$*\mathbb{R}$,相应的$(I,F)$是一个$E$上的非标准积分结构.

% 运用$*$转移,可以得到结论:

% \begin{proposition}
% 如果$(I,F)$是$E$上的积分结构,那么$(*I,*F)$是$E$上的非标准积分结构.
% \end{proposition}





\section{Example: Classical Morse theory}

In this section, we describe the deformation method
which produces irrational varieties.
We show that if we have a good family of varieties,
and if one of them does not have a decomposition of the diagonal,
then a very general one in the family
will not have a decomposition, and hence, will not be retract rational.

This method was developed by C. Voisin \cite{voisin},
and modified by J.\nobreakdash-L. Colliot-Thélène and A. Pirutka \cite{CTP},
to show that a very general
quartic threefold in $\bbP_{\bbC}^4$ is not retract rational.
We will present a proof of this result.


\subsection{Families of cycles}

\begin{definition} [Kollár {\cite[Definition~3.10]{kollar}}] \label{def-3-family}
    Suppose that
    \begin{itemize}
        \item
            $\bbk$ is an algebraically closed field of characteristic $0$.
        \item
            $S$ is a $\bbk$-scheme.
        \item
            $X/S$ is a projective $S$-scheme, with a chosen relatively ample line bundle.
        \item
            $B/S$ is a reduced normal $S$-scheme.
        \item
            $d$ and $d'$ are non-negative integers.
    \end{itemize}
    A \term{well-defined family of $d$-cycles} of $X$ of degree $d'$ parametrised by $B$
    is a cycle 
    \[ \textstyle C = \sum_i m_i [C_i] \quad \text{of} \quad X \times_S B, \]
    such that
    \begin{itemize}
        \item
            Each $C_i$ is an integral closed subscheme of $X \times_S B$.
        \item
            For each $i$, the image of the projection map $g_i \: C_i \to B$
            is an irreducible component of $B$.
            In particular, $g_i$ is flat over a dense open subset of $B$.
        \item
            Each fibre of $g_i$ defines a $d$-cycle of $X$ of degree $d'$.
            This means that the fibre is either empty or of dimension $d$,
            and that $g_i$ is flat over a dense open subset of $B$.
    \end{itemize}
\end{definition}

The deep theorem below shows the existence of
a universal family of cycles,
in that every family of cycles is realised as its pullback.

\begin{theorem}
    Under the assumptions of Definition~\textup{\ref{def-3-family}},
    for an $S$-scheme $Z$, define
    \[ \mathrm{Chow}_{X/S}^{d,d'} (Z) = \biggl\{ \,
        \begin{matrix}
            \text{well-defined families of non-negative $d$-cycles} \\
            \text{of $X$ of degree $d'$ parametrised by $Z$}
        \end{matrix} \,
    \biggr\}. \]
    Then 
    \begin{itemize}
        \item
            $\mathrm{Chow}_{X/S}^{d,d'}$
            is a contravariant functor
            from the semi-normal $S$-schemes to sets.
        \item
            Moreover, this functor is represented by a
            projective semi-normal $S$-scheme $\Chow_{X/S}^{d,d'}$,
            called the \term{Chow scheme},
            so that there exists a universal
            well-defined family of non-negative $d$-cycles
            \[ \Univ_{X/S}^{d,d'} \quad \text{of $X$ parametrised by} \quad \Chow_{X/S}^{d,d'}, \]
            such that every other family of cycles is its pullback.
    \end{itemize}
\end{theorem}

See \cite[Theorem~I.3.21]{kollar}. 

We also recall the existence of Hilbert schemes.

\begin{theorem}
    Let $S$ be a locally noetherian scheme.
    Let $X \to S$ be a projective morphism.
    For an $S$-scheme $Z$, define 
    \[ \mathrm{Hilb}_{X/S}^{d,d'} (Z) = \biggl\{ \,
        \begin{matrix}
            \text{closed subschemes of $X \times_S Z$ flat over $Z$} \\
            \text{of relative dimension $d$ and relative degree $d'$}
        \end{matrix} \,
    \biggr\}. \]
    The functor $\mathrm{Hilb}_{X/S}^{d,d'}$ is represented
    by an $S$-scheme $\Hilb_{X/S}^{d,d'}$, called the \term{Hilbert scheme},
    whose irreducible components are projective over $S$.
    As a result, there exists a universal family of subschemes
    \[ U \quad \subset \quad X \times_S \Hilb_{X/S}^{d,d'}, \]
    such that every other family of subschemes is its pullback.
\end{theorem}

Below, we will write
\[ \Chow_{X/S} = \coprod_{d,d'} \Chow_{X/S}^{d,d'}
    \quad \text{and} \quad 
    \Hilb_{X/S} = \coprod_{d,d'} \Hilb_{X/S}^{d,d'}. \]


\subsection{Locus of rational equivalence}

\begin{situation} \label{sit-3-deformation}
    Suppose
    \begin{itemize}
        \item
            $\bbk$ is an algebraically closed field of characteristic $0$.
        \item
            $B$ is a smooth $\bbk$-scheme.
        \item
            $X \to B$ is a projective morphism.
    \end{itemize}
\end{situation}

\begin{lemma} \label{lem-3-divisor}
    In Situation~\textup{\ref{sit-3-deformation}}, for any non-negative integer $d$, there exists
    \begin{itemize}
        \item
            A countable family of normal, irreducible, quasi-projective $B$-schemes $\{ T_i \}$.
        \item
            For each index $i$, a family of smooth $(d+1)$-dimensional varieties $W_i \to T_i$,
            with two families of divisors $E_{i,1}, E_{i,2} \to T_i$ of~$W_i$,
    \end{itemize}
    such that
    \begin{itemize}
        \item
            For any $b \in B$ and any subvariety $V \subset X_b$ of dimension $d+1$,
            there exists a desingularisation $\widetilde{V}$,
            such that for any two effective divisors $D_1, D_2$ of $\widetilde{V}$,
            such that $D_1 - D_2$ is principal,
            there exists $i$ and $t \in (T_i)_b (\bbk)$, 
            such that the data $(\widetilde{V}, D_1, D_2)$ is identical to
            $((W_i)_t, (E_{i,1})_t, (E_{i,2})_t)$.
    \end{itemize}
\end{lemma}

The reason to consider a desingularisation of $V$, instead of $V$ itself,
is that on a smooth variety, a Weil divisor is the same thing
as a Cartier divisor, and the Weil divisor class group is the same as the Picard group.
The normality of $T_i$ is required in order to (later) satisfy
the definition of a well-defined family of cycles.

\begin{proof}
    \def\WjGj{\widetilde{W}\mkern-6mu_j/\widetilde{G}_j}
    By \cite[Theorem~9.7.7]{EGA4-3}, the set of points in the Hilbert scheme $\Hilb_{X/B}^{d+1}$
    corresponding to the geometrically integral subvarieties is locally constructible.
    Let $G$ be an irreducible component of this set, equipped with the reduced scheme structure.
    Then $G$ is quasi-projective over $S$, as the components of $\Hilb_{X/B}^{d+1}$ are projective. Let
    \[ W \subset G \times_B X \]
    be the universal family of $(d+1)$-dimensional subschemes.
    The morphism $W \to G$ is thus projective, flat, with geometrically integral fibres.

    The generic fibre $W_{\bbk(G)}$ is integral, 
    as its irreducible components correspond to irreducible components of a general fibre.
    By Hironaka's theorem, let $\widetilde{W}_{\bbk(G)} \to W_{\bbk(G)}$ be a desingularisation map. 
    This map extends to a map
    \[ \widetilde{W}_1 \to W_1 \]
    of schemes over an open set $G_1 \subset G$, where $W_1 = W|_{G_1}$.
    Shrinking $G_1$ if necessary, we can assume that for any $t \in G_1(\bbk)$, 
    the map $\widetilde{W}_{1,t} \to W_t$ of fibres over $t$ is a desingularisation map.
    
    By noetherian induction, we can find a decomposition
    \[ \textstyle G = \bigcup_{j=1}^m G_j, \]
    with $G_j$ locally closed in $G$, together with
    maps $\widetilde{W}_j \to W_j$ over $G_j$,
    where $W_j = W|_{G_j}$, such that for all $t \in G_j$,
    the map $\widetilde{W}_{j,t} \to W_t$ is a desingularisation map.

    Let $\widetilde{G}_j \to G_j$ be a desingularisation,
    and we still denote by $\widetilde{W}_j \to \widetilde{G}_j$
    the pullback of the family $\widetilde{W}_j \to G_j$.

    Since $\widetilde{W}_j$ is projective and flat over $\widetilde{G}_j$,
    with geometrically integral fibres, there exist the schemes with a morphism
    \[ \mathrm{Ab} \: \sfname{Div}_{\WjGj} \to \sfname{Pic}_{\WjGj}, \]
    where $\sfname{Div}_{\WjGj}$ is the scheme parametrising the effective Cartier divisors
    \cite[Theorem~9.3.7]{FAG},
    which is quasi-projective over $\widetilde{G_j}$, and hence over $B$,
    and $\sfname{Pic}_{\WjGj}$ is the Picard scheme \cite[Theorem~9.4.8]{FAG}.
    Let
    \[ \Delta_j \subset \sfname{Div}_{\WjGj} \times \sfname{Div}_{\WjGj} \]
    be the inverse image of the diagonal of $\sfname{Pic}_{\WjGj} \times \sfname{Pic}_{\WjGj}$,
    under the map $\mathrm{Ab} \times \mathrm{Ab}$, equipped with the reduced scheme structure.
    Let $T$ be one of its irreducible components, and let $\widetilde{T}$ be the normalisation of $T$.
    Thus $\widetilde{T}$ is quasi-projective over $B$.

    The family of all the schemes $\widetilde{T}$, together with the two universal families of divisors
    given by the $\sfname{Div}$ schemes, satisfies the requirement of the lemma.
\end{proof}

Of course, a rational equivalence of two $d$-cycles may involve
more than one $(d+1)$-dimensional subvariety.
The next lemma deals with this situation.

For simplicity, if $V \subset X_b$ is a subvariety,
we will say ``the desingularisation'' of $V$
when we refer to the variety $\widetilde{V}$ given by the previous lemma,
and we simply add a tilde to indicate this desingularisation.

\begin{lemma} \label{lem-3-divisors}
    In Situation~\textup{\ref{sit-3-deformation}}, for any non-negative integer $d$, there exists
    \begin{itemize}
        \item
            A countable family of normal irreducible $B$-schemes $\{ H_i \}$.
        \item
            For each index $i$, an integer $n_i \geq 1$,
            and $n_i$ triples $(W_{i,j},\ E_{i,j,1},\ E_{i,j,2})_{j=1}^{n_i}$,
            where for each $j$, $W_{i,j} \to H_i$ is a smooth projective family of $(d+1)$-dimensional varieties,
            and $E_{i,j,1},\ E_{i,j,2} \to H_i$ are two families of divisors of~$W_{i,j}$,
    \end{itemize}
    such that
    \begin{itemize}
        \item
            For any $b \in B(\bbk)$, and any data $(V_j,\ D_{j,1},\ D_{j,2})_{j=1}^n$,
            where each $V_j$ is an integral subscheme of $X_b$ of dimension $d+1$,
            and $D_{j,1},\ D_{j,2}$ are two effective Weil divisors on 
            the desingularisation $\widetilde{V}_j$ of $V_j$, 
            such that $D_{j,1} - D_{j,2}$ is a principal divisor on $V_j$, 
            there exists $i$ and $t \in (H_i)_b (\bbk)$, 
            such that the fibre $\bigl( (W_i)_t,\ (E_{i,j,1})_t,\ (E_{i,j,2})_t \bigr)$ 
            is identical to the given data.
    \end{itemize}
\end{lemma}

\begin{proof}
    For each $n$-tuple $(T_1, \dotsc, T_n)$ as given by the previous lemma,
    we consider the normalisation $H$ of the product $T_1 \times \cdots \times T_n$,
    equipped with the data of $n$ triples as given by the previous lemma.
    The collection of all such $H$ satisfies the requirement of this lemma.
\end{proof}

Our effort to parametrise all possibilities for a rational equivalence
allows us to prove the following result.

\begin{lemma} \label{lem-3-rat-equiv}
    In Situation~\textup{\ref{sit-3-deformation}}, let 
    \[ Z_1, Z_2 \in \mathrm{Chow}_{X/B}^{d} (B) \] 
    be two well-defined families of cycles.
    Then there exists
    \begin{itemize}
        \item
            A countable family of quasi-projective $B$-schemes $\{ M_i \}$.
        \item
            For each index $i$, the data $(W_{i,j},\ E_{i,j,1},\ E_{i,j,2})_{j=1}^{n_i}$ 
            as in Lemma~\textup{\ref{lem-3-divisors}}, with $M_i$ in place of $H_i$,
    \end{itemize}
    such that 
    \begin{itemize}
        \item
            The union of the images of $M_i(\bbk)$ in $B(\bbk)$ is exactly the set
            \[ \{ b \in B(\bbk) \mid [Z_{1,b}] = [Z_{2,b}] \text{ in } \CH_\bullet (X_b) \}. \]
        \item
            For any $b \in B(\bbk)$, and any data $(V_i,\ D_{i,1},\ D_{i,2})_{i=1}^n$
            as in Lemma~\textup{\ref{lem-3-divisors}}, such that
            \[ \textstyle Z_{1,b} + \sum_{i=1}^n [D_{i,1}] = Z_{2,b} + \sum_{i=1}^n [D_{i,2}]
                \quad \text{in } \upZ_\bullet (X_b), \]
            there exists $i$ and a point $t \in (M_i)_b (\bbk)$,
            such that the fibre $\bigl( (W_i)_t$, $(E_{i,j,1})_t$, $(E_{i,j,2})_t \bigr)$ 
            is identical to the given data.
    \end{itemize}
\end{lemma}

\begin{proof}
    Let $\{ H_i \}$ be the family in Lemma~\ref{lem-3-divisors}. We define a morphism
    \[ \begin{aligned}
        f \: H_i & \to     \Chow_{X/B} \times \Chow_{X/B}, \\
               t & \mapsto \textstyle \Bigl( Z_1 + \sum_{j=1}^{n_i} (E_{i,j,1})_t, \ Z_2 + \sum_{j=1}^{n_i} (E_{i,j,2})_t \Bigr),
    \end{aligned} \]
    where $(E_{i,j,1 \text{ or } 2})_t$ is regarded as a cycle of $X$
    via the pushforward along the desingularisation map (onto a closed subvariety of $X$).
    Now let $M_i$ be the inverse image of the diagonal along $f$,
    with the reduced scheme structure,
    equipped with the data of $n_i$ triples given by that of $H_i$.
    This proves the second statement.

    For the first statement, write $Z = Z_1 - Z_2$.
    Let $b \in B$ be a point where $Z_b$ is rationally equivalent to zero.
    Then, there exist subvarieties $V_j \subset X_b$, where $j = 1, \dotsc, n$,
    and rational functions $g_j$ on $V_j$, which give rise to
    rational functions $\widetilde{g_j}$ on a desingularisation $\widetilde{V}_j$,
    such that
    \[ \textstyle Z_b = \sum_{j=1}^n (f_j)_* (\operatorname{div} \widetilde{g_j}), \]
    where $f_j$ denotes the map $\widetilde{V}_j \to X_b$.
    Conversely, the existence of this data implies that $Z_b$ is rationally equivalent to zero,
    since $M_j$ is taken to be the inverse image of the diagonal.
    Therefore, the locus where $Z_b$ is equivalent to zero
    is exactly the union of the images of the $M_i$.
\end{proof}

We are now getting close to the main theorem,
which states that the locus where $[Z_{1,b}] = [Z_{2,b}]$
is a countable union of closed sets.
There is one further lemma needed.

\begin{lemma} \label{lem-3-specialise}
    Let $M$ be a smooth $\bbk$-variety of dimension $m$, with $\bbk$ algebraically closed,
    and let $f \: W \to M$ be a flat morphism of relative dimension $r$.
    Let $Z$ be an $n$-cycle on $W$. Suppose that
    \begin{itemize}
        \item
            There is a dense open set $M^\circ \subset M$,
            such that $Z|_{f^{-1} (M^\circ)}$ is rationally equivalent to $0$ in $f^{-1} (M^\circ)$.
    \end{itemize}
    Then
    \begin{itemize}
        \item
            For any $t \in M (\bbk)$,
            the fibre $Z_t$ is rationally equivalent to $0$ in $W_t$.
    \end{itemize}
\end{lemma}

\begin{proof}
    Let $t \in M(\bbk) \setminus M^\circ(\bbk)$ be a point.
    As in the proof of Lemma~\ref{lem-2-moving}, we can find
    a curve $C$ in $M$, passing through $t$, and not contained in $M \setminus M^\circ$.
    Taking the normalisation of this curve,
    we may thus assume that $M$ is a smooth curve.

    Let $D = M \setminus M^\circ$, which is now a finite set.
    There is an exact sequence \cite[\S1.8]{fulton}
    \[ \CH_n (f^{-1} (D)) \overset{i_*}{\longrightarrow} \CH_n (W) 
        \longrightarrow \CH_n (f^{-1} (M^\circ)) \to 0, \]
    so that $Z = i_* (z)$ for some $z \in \CH_n (f^{-1} (D))$,
    where $i \: f^{-1} (D) \to W$ denotes the inclusion.
    But by the projection formula \cite[\S2.3]{fulton}, the intersection of $i_* (z)$
    with the divisor $f^{-1} (D)$ of $W$ is 
    $i_* i^* (f^{-1}(D) \cdot z) = 0$, so that $Z_t$ is rationally equivalent to $0$ for any $t \in D$.
\end{proof}

Now we are ready to prove the main result,
and our proof follows that of \cite[Proposition~2.4]{voisin}.

\begin{theorem} [Voisin] \label{thm-3-locus-equality}
    In Situation~\textup{\ref{sit-3-deformation}}, let 
    \[ Z_1, Z_2 \in \mathrm{Chow}_{X/B}^{d} (B) \] 
    be two well-defined families of cycles.
    Then there exists a countable family $\{ B_i \}$
    of closed subschemes of $B$, such that
    \[ \textstyle
        \bigl\{ \, b \in B (\bbk) \bigm| 
        [Z_{1, b}] = [Z_{2, b}] \text{ \ in \ } {\CH_d (X_b)} \, \bigr\}
        = \bigcup_i B_i (\bbk).
    \]
\end{theorem}

\begin{proof}
    \def\Mibar{\mspace{2mu}\overline{\smash{\mspace{-2mu}M}\vphantom{t}}\mspace{-1mu}\vphantom{t}_i}
    Let $\{ M_i \}$ be as in Lemma~\ref{lem-3-rat-equiv}.
    Replacing each $M_i$ by its desingularisation, we can assume
    that all $M_i$ are smooth.

    Let $B_i \subset B$ be the closure of the image of $M_i$ in $B$,
    as a closed integral subvariety.
    By Lemma~\ref{lem-3-rat-equiv}, the equation $[Z_{1,b}] = [Z_{2,b}]$
    implies $b \in B_i$ for some $i$. 
    Thus, it suffices to show that it holds for all $b \in B_i$.

    Let $B_i^\circ \subset B_i$ be an open subset contained in the image of $M_i$,
    and let $M_i^\circ$ be the inverse image of $B_i^\circ$ in $M_i$.
    Let $X_{M_i} = X \times_B M_i$, and $Z_i$ the pullback of $Z = Z_1 - Z_2$
    along the morphism $X_{M_i} \to X$, 
    which is actually the Chow pullback of families of cycles along the map $M_i \to B$.
    Then $Z_i$ is equal to the universal cycle $\sum_{j=1}^{n_i} (E_{i,j,1} - E_{i,j,2})$ on $M_i$,
    and hence is rationally equivalent to zero.

    By taking the closure in a projective bundle, 
    the morphism $M_i \to B_i$ extends to a projective morphism $\Mibar \to B_i$.
    Again, taking a desingularisation, we may assume $\Mibar$ is smooth.
    Write $X_{\Mibar} = X \times_B \Mibar$.
    Now apply Lemma~\ref{lem-3-specialise} with $M = \Mibar$, $W = X_{\Mibar}$,
    and $M^\circ$ the inverse image of $M_i^\circ$. 
    This shows that for all $b \in B_i$, the cycle $Z_b$ is equivalent to zero.
\end{proof}


\subsection{Locus of decomposability of the diagonal}

\begin{lemma} \label{lem-3-all-cycles}
    Suppose
    \begin{itemize}
        \item
            $\bbk$ is an algebraically closed field of characteristic $0$.
        \item
            $B$ is a smooth $\bbk$-scheme.
        \item
            $X \to B$ is a projective morphism, and write $Y = X \times_B X$.
    \end{itemize}
    Then there exists
    \begin{itemize}
        \item
            A countable family of smooth irreducible $B$-schemes $\{ F_i \}$.
        \item
            For each index $i$, a well-defined family of non-negative
            $d_i$-cycles $C_i$ of $Y$ of degree $d'_i$,
            parametrised by $F_i$,
    \end{itemize}
    such that
    \begin{itemize}
        \item
            For any $b \in B$ and any non-negative $d$-cycle $C$ of $Y_b$ of degree $d'$,
            supported in $Z \times X_b$ for a codimension $1$ subset $Z \subset X_b$,
            there exists $i$ and $x \in (F_i)_b$ such that $C = (C_i)_x$.
        \item
            For any $x \in (F_i)_b$,
            the cycle $C = (C_i)_x$ is supported in $Z \times X_b$
            for a codimension $1$ subset $Z \subset X_b$.
    \end{itemize}
\end{lemma}

The condition ``supported in $Z \times X_b$'' is the main point of this lemma.
In fact, the proof would be a lot easier if we dropped this condition.
This lemma will be used to parametrise all possibilities for
the term $D$ in a decomposition of the diagonal, as in Definition~\ref{def-2-decomp}.

\begin{proof}
    First, we need to parametrise all the subschemes of $X$
    that are codimension $1$ in $X_b$ at each $b \in B$.
    Therefore, we consider an irreducible component 
    \[ H \subset \Hilb_{X/B}, \]
    parametrising the codimension $1$ subschemes.
    Let $U \subset H \times_B X$ be the universal subscheme.
    Thus if we look at the fibre at $b \in B$,
    then $H_b$ parametrises the codimension $1$ subschemes of $X_b$,
    and $U_b \subset H_b \times X_b$ is a subscheme
    whose intersection with $\{c\} \times X_b$ gives the subscheme of $X_b$ corresponding to $c$.

    Next, we want to parametrise all the subschemes of $Y$ which
    have the form $(\text{codim 1 subset}) \times X_b$ when restricted to the fibres.
    This is given by the universal subscheme
    \[ U' = U \times_B X \subset H \times_B X \times_B X, \]
    which, at $b \in B$, 
    when intersected with $\{c\} \times X_b \times X_b$,
    gives the subscheme of $X_b \times X_b$ corresponding to $c$.

    Finally, we parametrise cycles of $Y$ supported in
    a subset of the form of the previous step.
    Thus we consider an irreducible component
    \[ C \subset \Chow_{U'/H}. \]
    Let $V \in \upZ_\bullet (C \times_H U')$ be the universal family. Since
    \[ C \times_H U' \subset C \times_H H \times_B X \times_B X \simeq C \times_B X \times_B X, \]
    we can view $V$ as a family of cycles of $Y$ parametrised by $C$.

    Thus, all choices of $H$ and $C$ will give a countable set of families,
    which together parametrise all the cycles of $Y$ of the given form.

    However, the parametrising schemes need to be smooth.
    We thus apply Hironaka's desingularisation theorem to the schemes $C$.
\end{proof}

\begin{proposition} \label{lem-3-locus-decomp}
    Suppose 
    \begin{itemize}
        \item
            $\bbk$ is an algebraically closed field of characteristic $0$.
        \item
            $B$ is a smooth $\bbk$-scheme.
        \item
            $X \to B$ is a projective morphism.
    \end{itemize}
    Then there exists a countable family $\{ B_i \}$
    of closed subschemes of $B$, such that
    \[ \textstyle
        \{ b \in B (\bbk) \mid X_b \text{ has a decomposition of the diagonal} \}
        = \bigcup_i B_i (\bbk).
    \]
\end{proposition}

\begin{proof}
    Let $F_i, F_{i'}$ be two of the schemes as in Lemma~\ref{lem-3-all-cycles}, 
    with $d_i = d_{i'} = \dim (X/B)$,
    and let $C_i, C_{i'}$ be the universal cycles,
    lying in $X \times_B X \times_B F_{i \text{ or } i'}$. 
    
    Let $G_j, G_{j'}$ be irreducible components of 
    $\Chow_{X/B}^{0,d}$ and $\Chow_{X/B}^{0,d + 1}$,
    respectively, where $d$ is arbitrary,
    and let $D_j, D_{j'}$ be the universal cycles lying in $X \times_B G_{j \text{ or } j'}$.
    
    We define two cycles of $Y = F_i \times_B G_j \times_B X \times_B X \times_B G_{j'} \times_B F_{i'}$ by
    \[ \begin{aligned}
        Z_1 &= \Bigl( [G_j] \times C_i + [F_i] \times [G_j] \times [\Delta_{X/B}] + [F_i] \times [X] \times D_j \Bigr) 
            \times [G_{j'}] \times [F_{i'}], \\[-3pt]
        Z_2 &= [F_i] \times [G_j] \times \Bigl( C_{i'} \times [G_{j'}] + [X] \times D_{j'} \times [F_{i'}] \Bigr),
    \end{aligned} \]
    where $[\Delta_{X/B}]$ is the diagonal class.

    Now apply Theorem~\ref{thm-3-locus-equality}, where we take $X$ to be $Y$, 
    and take $B$ to be 
    \[ F_i \times_B F_{i'} \times_B G_j \times_B G_{j'}. \]
    At the point 
    \[ t = (t_1, t_2, x_1, x_2) \in (F_i)_b \times (F_{i'})_b \times (G_j)_b \times (G_{j'})_b, \]
    the cycle $Z_1$ gives $[\Delta_{X_b}] + z_1 + [X_b] \times x_1$,
    where $z_1$ is a non-negative cycle
    supported in $Z \times X_b$ for $Z \subset X_b$ of codimension $1$,
    and similarly, the cycle $Z_2$ gives $[X_b] \times x_2 + z_2$,
    with $z_2$ likewise.

    Therefore, Theorem~\ref{thm-3-locus-equality} implies that
    the locus where the equation
    \[ [\Delta_{X_b}] + z_1 + [X_b] \times x_1 = [X_b] \times x_2 + z_2 \quad \in \CH_{\dim X_b} (X_b \times X_b) \]
    holds (for non-negative $x_1, x_2, z_1, z_2$)
    is the union of countably many closed subsets.
\end{proof}

This result is restated as follows.

\begin{theorem} \label{thm-3-locus-decomp}
    Suppose 
    \begin{itemize}
        \item
            $\bbk$ is an algebraically closed field of characteristic $0$.
        \item
            $B$ is a smooth $\bbk$-scheme.
        \item
            $X \to B$ is a dominant projective morphism.
        \item
            There exists a $\bbk$-point $0 \in B$,
            such that the fibre $X_0$ does not have a decomposition of the diagonal.
    \end{itemize}
    Then for a ``very general'' $\bbk$-point $b \in B$,
    the fibre $X_b$ will not have a decomposition of the diagonal. \qed
\end{theorem}

By ``\term{very general}'', we mean ``except a countable union of closed sets of codimension $\geq 1$''.

This means that if we can find one example in a family of varieties,
which we can show has non-trivial Brauer group,
and hence does not have a decomposition of the diagonal,
then a very general variety in this family is not retract rational.


\subsection{Stable equivalence}

This subsection gives a variant of the above result,
concerning stable equivalence instead of retract rationality.

\begin{definition}
    Two projective $\bbk$-varieties are \term{stably equivalent}, if
    \[ X \times \bbP^m \quad \text{is birational to} \quad Y \times \bbP^n \]
    for some $m, n \in \bbN$.
\end{definition}

Stable rationality is the same as stable equivalence to a point.

\begin{lemma} \label{lem-3-stable-eq-cycles}
    Let $X,Y$ be two $\bbk$-varieties, such that there exist open sets $U \subset X$,
    $V \subset Y \times \bbP^n$, and two morphisms $p \: U \to V$, $q \: V \to U$,
    such that $q \circ p = \id_U$.
    Then there exist two correspondences
    \[ f \in \operatorname{Corr} (X, Y), \quad g \in \operatorname{Corr} (Y, X), \]
    such that
    for any field extension $\bbK/\bbk$, the induced map
    \[ (g \circ f)_* \: \CH_0 (X_{\bbK}) \to \CH_0 (X_{\bbK}) \]
    is the identity map.
    When $X$ is smooth, we have a decomposition
    \[ [\Delta_X] = D + g \circ f \quad \text{in } \operatorname{Corr} (X, X), \]
    where $D$ is supported in $Z \times X$
    for some closed subvariety $Z \subset X$ of codimension at least $1$.
\end{lemma}

\begin{proof}
    The correspondence $f$ is given by the rational map
    \[ X \overset{\supset}{\mathrel{\rightdasharrow}} U \overset{p}{\to} V \hookrightarrow Y \times \bbP^n \to Y, \]
    and $g$ is given by a rational map
    \[ Y \hookrightarrow Y \times \bbP^n \overset{\supset}{\mathrel{\rightdasharrow}} V \overset{q}{\to} U \hookrightarrow X, \]
    where the inclusion $Y \hookrightarrow Y \times \bbP^n$
    is chosen so that the composition is defined.
    To prove that $g \circ f$ induces the identity map on $\CH_0 (X_{\bbK})$, it suffices to prove that the map
    \[ Y \times \bbP^n \to Y \hookrightarrow Y \times \bbP^n \]
    induces the identity map on $\CH_0$. 
    This is because every closed point of $Y \times \bbP^n$
    is sent to another point that lives in the same slice of $\bbP^n$, and hence,
    is rationally equivalent to it as a $0$-cycle.

    For the second part, we use an argument as in the proof of Theorem~\ref{thm-2-decomp}.
    Namely, we change the base field to $\bbk(X)$, to find that
    \[ g_{\bbk (X)} \circ f_{\bbk (X)} (\beta) = \beta \quad \text{in } \CH_0 (X_{\bbk(X)}), \]
    where $\beta$ is the class of the generic point.
    The rest of the proof is analogous to the proof of Theorem~\ref{thm-2-decomp}, 
    \ref{itm-thm-2-decomp-2} $\Rightarrow$ \ref{itm-thm-2-decomp-3}.
\end{proof}

Note that the assumptions of this lemma is satisfied when $X$ and $Y$ are stably equivalent.

Using an argument as in the proof of Proposition~\ref{lem-3-locus-decomp},
we obtain the following result.

\begin{theorem} \label{thm-3-stable-eq-prototype}
    Suppose 
    \begin{itemize}
        \item
            $\bbk$ is an algebraically closed field of characteristic $0$.
        \item
            $B$ is a smooth $\bbk$-scheme.
        \item
            $X \to B$ and $Y \to B$ are two projective morphisms.
    \end{itemize}
    Then the set of all points $b \in B (\bbk)$ such that
    there exist correspondences
    \[ f \in \operatorname{Corr} ( X_b, Y_b ), \quad g \in \operatorname{Corr} ( Y_b, X_b ), \]
    and $D$ as before, such that 
    \[ [\Delta_{X_b}] = D + g \circ f, \]
    is a countable union of closed sets.
\end{theorem}

\begin{proof}
    We apply Theorem~\ref{thm-3-locus-equality}, 
    where we take $X$ to be
    \[ X \times_B X \times_B F_i \times_B F_{i'} \times_B G_j \times_B G_{j'} \times_B H_k \times_B H_{k'}, \]
    where 
    \begin{itemize}
        \item 
            $F_i$, $F_{i'}$ are given by Lemma~\ref{lem-3-divisor}.
        \item 
            $G_j$, $G_{j'}, H_k, H_{k'}$ are irreducible components of $\Chow_{X \times_B Y / B}$,
            parametrising the correspondences from $X$ to $Y$ for $G_j$ and $G_{j'}$,
            and from $Y$ to $X$ for $H_k$, $H_{k'}$.
    \end{itemize}
    The rest of the proof is analogous to the proof of Proposition~\ref{lem-3-locus-decomp}.
\end{proof}

\begin{corollary} \label{cor-3-stable-eq-prototype}
    Under the assumptions of Theorem~\ref{thm-3-stable-eq-prototype},
    the set of all points $b \in B(\bbk)$ such that $X_b$ is stably equivalent to $Y_b$
    is contained in a countable union of closed sets.
    Moreover, this union does not contain any point $b \in B(\bbk)$
    such that $X_b$ is smooth and has a decomposition of the diagonal,
    and $Y_b$ does not have a decomposition of the diagonal
\end{corollary}

\begin{proof}
    The countable union of closed sets given by Theorem~\ref{thm-3-stable-eq-prototype}
    satisfies this requirement.
    Indeed, for those $b \in B(\bbk)$ such that $X_b$ and $Y_b$ are stably equivalent,
    Lemma~\ref{lem-3-stable-eq-cycles} shows that $b$ is in this union.

    To prove the last statement, let $b \in B(\bbk)$ be such a point.
    We show that such correspondences $f,g$ as in Theorem~\ref{thm-3-stable-eq-prototype} do not exist between $X_b$ and $Y_b$.
    In fact, if they exist, then $g \circ f$ acts on $\CH_0(X)$ by the identity map.
    But since $\id_Y$ sends every $0$-cycle to its degree multiplied by a fixed $0$-cycle of degree~$1$,
    so does the correspondence $g \circ f = g \circ \id_Y \circ f$.
    Moreover, this holds over any field extension of $\bbk$. 
    By Theorem~\ref{thm-2-decomp}, $X$ has a decomposition of the diagonal,
    a contradiction.
\end{proof}

In particular, if we take $X$ to be a constant family which is smooth,
we deduce that every stable equivalence class in a family of varieties is contained
in a countable union of closed sets.

\begin{corollary} \label{thm-3-stable-eq-class}
    Suppose 
    \begin{itemize}
        \item
            $\bbk$ is an uncountable algebraically closed field of characteristic $0$.
        \item
            $B$ is a smooth $\bbk$-scheme.
        \item
            $X \to B$ is a dominant projective morphism, with smooth generic fibre.
        \item
            There exist two $\bbk$-points $b_0, b_1 \in B$,
            such that the fibre $X_{b_0}$ has a decomposition of the diagonal,
            while the fibre $X_{b_1}$ does not have a decomposition of the diagonal.
    \end{itemize}
    Then there are uncountably many stable equivalence classes of varieties
    in this family.
\end{corollary}

\begin{proof}
    For those smooth fibres $X_b$ that do not have a decomposition of the diagonal,
    we apply Corollary~\ref{cor-3-stable-eq-prototype} to the constant family $B \times X_b \to B$ and the family $X \to B$.
    It follows that the set of $b' \in B(\bbk)$ such that $X_b$ is stably equivalent to $X_{b'}$
    is contained in a countable union of closed sets, which can not coincide with the whole space.

    By Theorem~\ref{thm-3-locus-decomp},
    the locus of smooth fibres with no decomposition of the diagonal 
    is the complement of a countable union of closed subsets of $B$.
    Therefore, in order to cover this locus, 
    there must be uncountably many stable equivalence classes of fibres of $X \to B$,
    since each of these classes is contained in a countable union of closed sets.
\end{proof}


\section{Example: Pseudoholomorphic curves}

给定平面上两条射线夹出的角, 仅凭直尺和圆规不一定能将其三等分. 
这是因为三等分角实际上涉及了开立方运算:
$$
\cos(3\theta)=-3\cos(\theta)+4\cos^3(\theta)
$$
其中 $3\theta$ 为已知角. 由于公理 (6) 可以实现开立方运算, 
由定理 3.2 知折纸三等分角是可以办到的. 
下面给出了一种实现方法 \cite{Hul}, 其证明是相似三角形的简单推导. 

\begin{figure}[h]
    \centering
    \includegraphics[scale=0.4]{trisect.png}
    \caption{三等分角}
\end{figure}

\section{Example: Yang--Mills connections}

\subsection{The specialisation map}

The main idea of the specialisation method 
is to build a way to transport properties
between the generic fibre and the special fibre.
We consider the following situation.

\begin{situation} \label{sit-5-specialise}
    Let $A$ be a discrete valuation ring,
    with fraction field $\bbK$ and residue field $\bbk$.
    Let $\mathscr{X}$ be an $A$-scheme. Suppose that
    \begin{itemize}
        \item
            The special fibre $X^{\mathrm{s}} = \mathscr{X} \times_A \bbk$ is a $\bbk$-variety.
        \item
            The generic fibre $X = \mathscr{X} \times_A \bbK$ is a $\bbK$-variety.
    \end{itemize}
\end{situation}

After Colliot-Thélène and Pirutka \cite{CTP},
we introduce the specialisation map on the Chow groups.

\begin{proposition} \label{thm-5-spe-map}
    In Situation \textup{\ref{sit-5-specialise}}, 
    there is a \term{specialisation map}
    \[ \sigma \: \CH_0 (X) \to \CH_0 (X^{\mathrm{s}}), \]
    which preserves the degree of $0$-cycles.
\end{proposition}

\begin{proof}
    By \cite[\S1.8 and \S20.1]{fulton}, there is an exact sequence
    \[ \CH_1 (X^{\mathrm{s}}) \overset{i_*}{\longrightarrow} \CH_1 (\mathscr{X})
        \overset{j^*}{\longrightarrow} \CH_0 (X) \to 0, \]
    where $i, j$ are the obvious inclusions.
    (The last term is $\CH_0$ instead of $\CH_1$,
    since $\Spec \bbK$ is a $1$-dimensional point in $\Spec A$.)

    By \cite[\S2.6 and \S20.1]{fulton},
    there is a Gysin map
    \[ i^! \: \CH_1 (\mathscr{X}) \to \CH_0 (X^{\mathrm{s}}), \]
    given by intersection with the divisor $X^{\mathrm{s}}$ of $\mathscr{X}$.
    By \cite[Proposition~2.6~(c)]{fulton},
    we have $i^! \circ i_* = 0$.
    Thus the map $i^!$ factors through the cokernel of $i_*$, giving the desired map.
\end{proof}

\begin{lemma} \label{lem-5-lifting}
    In Situation \textup{\ref{sit-5-specialise}}, suppose that $A$ is henselian,
    and $\mathscr{X}$ is proper and flat over $A$.
    Let $X^{\mathrm{s}}_{\mathrm{sm}} \subset X^{\mathrm{s}}$ be the open set where $X^{\mathrm{s}}$ is smooth.
    Then every $0$-cycle of $X^{\mathrm{s}}$ supported in $X^{\mathrm{s}}_{\mathrm{sm}}$ 
    can be lifted along the specialisation map
    \[ \sigma \: \CH_0 (X) \to \CH_0 (X^{\mathrm{s}}) \]
    to a $0$-cycle supported in $X_{\mathrm{sm}}$.
\end{lemma}

\begin{proof}
    We follow \cite[\S4]{EKW}.
    It is enough to lift the closed points of $X^{\mathrm{s}}_{\mathrm{sm}}$.
    Let $x \in X^{\mathrm{s}}_{\mathrm{sm}}$ be a closed point,
    and let $a_1, \dotsc, a_n \in \mathscr{O}_{X^{\mathrm{s}},x}$ be a regular sequence generating the maximal ideal.
    Choose liftings $\bar{a}_1, \dotsc, \bar{a}_n \in \mathscr{O}_{\mathscr{X},x}$.
    Since $\mathscr{O}_{X^{\mathrm{s}},x} \simeq \mathscr{O}_{\mathscr{X},x} / \pi \mathscr{O}_{\mathscr{X},x}$,
    where $\pi \in A$ is a uniformiser,
    it follows that $\pi, \bar{a}_1, \dotsc, \bar{a}_n$ is a regular sequence in 
    the $(n+1)$-dimensional local ring $\mathscr{O}_{\mathscr{X},x}$.
    Therefore, the ideal $(\bar{a}_1, \dotsc, \bar{a}_n) \subset \mathscr{O}_{\mathscr{X},x}$
    defines a $1$-dimensional subset of $\Spec \mathscr{O}_{\mathscr{X},x}$,
    whose closure in $\mathscr{X}$ is a $1$-dimensional subscheme $Z \subset \mathscr{X}$.
    Then $Z$ is flat of relative dimension $0$ over $A$,
    and hence quasi-finite over $A$. By properness, it is finite over $A$.
    It follows that $Z \simeq \Spec B$ for a finite $A$-algebra $B$.
    Since $A$ is henselian, $B$ is a product of local rings.
    Therefore, the irreducible component of $Z$ containing $x$
    meets $X^{\mathrm{s}}$ at a single point $x$.
    The corresponding $0$-cycle of $X$ has the desired property.
\end{proof}

\begin{lemma} \label{lem-5-henselian}
    In Situation \textup{\ref{sit-5-specialise}}, suppose that
    \begin{itemize}
        \item
            $A$ is henselian, and $\mathscr{X}$ is proper and flat over $A$.
        \item
            The generic fibre $X$ has a desingularisation $p \: \widetilde{X} \to X$,
            such that $\widetilde{X}$ is universally $\CH_0$-trivial.
    \end{itemize}
    Then every $0$-cycle of $X^{\mathrm{s}}$ of degree $0$,
    supported in the open set $X^{\mathrm{s}}_{\mathrm{sm}} \subset X^{\mathrm{s}}$ where $X^{\mathrm{s}}$ is smooth,
    is zero in $\CH_0 (X^{\mathrm{s}})$.
\end{lemma}

\begin{proof}
    Let $U \subset X$ be a dense open set
    such that $p \: p^{-1} (U) \to U$ is an isomorphism.
    Let $x$ be a $0$-cycle of $X^{\mathrm{s}}_{\mathrm{sm}}$ of degree $0$.
    By Lemma~\ref{lem-5-lifting}, $x$ lifts to a $0$-cycle of $X_{\mathrm{sm}}$ of degree $0$.
    By the moving lemma \ref{lem-2-moving},
    it is equivalent to a $0$-cycle supported in $U$,
    which then lifts to a $0$-cycle of $\widetilde{X}$.
    This $0$-cycle is equivalent to $0$ in $\widetilde{X}$ by hypothesis.
    Therefore, applying the map
    \[ \CH_0 (\widetilde{X}) \overset{p_*}{\longrightarrow} \CH_0 (X) 
        \overset{\sigma}{\longrightarrow} \CH_0 (X^{\mathrm{s}}), \]
    we see that $x = 0$ in $\CH_0 (X^{\mathrm{s}})$.
\end{proof}


\subsection{Rationality and specialisation}

The main result is that the rationality 
(or more precisely, universal $\CH_0$-triviality) 
of the generic fibre 
can be specialised to the special fibre,
so that once we show that the special fibre is irrational,
we know that the generic fibre is also irrational.

First, we make clear what we need
to obtain universal triviality for a desingularisation.

\begin{lemma} \label{lem-5-crit-triv}
    Let $f \: \widetilde{X} \to X$ be a desingularisation map between 
    two projective, geometrically integral $\bbk$-varieties. Suppose that 
    \begin{itemize}
        \item
            $\widetilde{X}$ has a $0$-cycle of degree $1$.
        \item
            $f$ is universally $\CH_0$-trivial.
        \item
            There exists an open set $U \subset X$, with $\widetilde{U} = f^{-1} (U)$,
            such that $f \: \widetilde{U} \to U$ is an isomorphism,
            and for any extension $F/\bbk$,
            every $0$-cycle of degree $0$ supported in $U_F$
            is rationally equivalent to zero in $X_F$.
    \end{itemize}
    Then $\widetilde{X}$ is universally $\CH_0$-trivial.
\end{lemma}

\begin{proof}
    Since the conditions of the lemma is preserved by a base change, 
    we only need to prove that $\deg_{\bbk} \: \CH_0 (\widetilde{X}) \to \bbZ$
    is an isomorphism. By the first hypothesis, this map is surjective.
    Thus, it suffices to show that for every $0$-cycle $x$ of $\widetilde{X}$
    of degree $0$, $x$ is rationally equivalent to $0$.
    But by the moving lemma \ref{lem-2-moving},
    it is equivalent to one supported in $\widetilde{U}$,
    which induces a cycle in $U$, which is equivalent to $0$ in $X$ by hypothesis.
    Since $f$ is universally $\CH_0$-trivial,
    $x$ is equivalent to $0$ in $\widetilde{X}$.
\end{proof}

Before the main theorem,
we mention a convenient result in commutative algebra.

\begin{lemma} \label{lem-5-dvr}
    Let $A$ be a discrete valuation ring with residue field $\bbk$,
    and let $F/\bbk$ be an extension.
    Then there exists a complete discrete valuation ring $B$ with residue field $F$,
    together with a local map $A \to B$ inducing the field map $\bbk \to F$.
\end{lemma}

See \cite[Chapter~IX, Appendix, \S2, Corollary, and Exercise~4]{bourbaki-commutative}.

\begin{theorem} [Colliot-Thélène and Pirutka] \label{thm-5-spe-1}
    In Situation \textup{\ref{sit-5-specialise}}, suppose that
    \begin{itemize}
        \item
            $\mathscr{X}$ is faithfully flat and proper over $A$,
            with geometrically integral fibres.
        \item
            The special fibre $X^{\mathrm{s}}$
            has a desingularisation $f \: \widetilde{X}^{\mathrm{s}} \to X^{\mathrm{s}}$,
            such that $f$ is universally $\CH_0$-trivial,
            and $\widetilde{X}^{\mathrm{s}}$ has a $0$-cycle of degree $1$.
        \item
            The generic fibre $X$ 
            has a desingularisation $\widetilde{X} \to X$.
    \end{itemize}
    Then if $\widetilde{X}$ is universally $\CH_0$-trivial, so is $\widetilde{X}^{\mathrm{s}}$.
\end{theorem}

\begin{proof}
    The proof is done by putting the previous lemmas together.

    \begin{itemize}
        \item    
            By Theorem~\ref{thm-2-decomp}, it suffices to show that
            $\widetilde{X}^{\mathrm{s}}_{\bbk (X^{\mathrm{s}})}$ is $\CH_0$-trivial,
            where we notice that $\bbk (\smash{\widetilde{X}^{\mathrm{s}}}) \simeq \bbk (X^{\mathrm{s}})$.
        \item
            By Lemma~\ref{lem-5-crit-triv}, it suffices to show that
            the open set $U = (X^{\mathrm{s}}_{\bbk (X^{\mathrm{s}})})_{\mathrm{sm}} \subset X^{\mathrm{s}}_{\bbk (X^{\mathrm{s}})}$
            satisfies the third assumption of Lemma~\ref{lem-5-crit-triv}.
        \item
            By Lemma~\ref{lem-5-henselian}, it suffices to show that
            $X^{\mathrm{s}}_{\bbk (X^{\mathrm{s}})}$ can act the rôle of $X^{\mathrm{s}}$ in that lemma.
        \item
            By Lemma~\ref{lem-5-dvr}, we take a complete discrete valuation ring $B$,
            with residue field $\bbk (X^{\mathrm{s}})$, and a local map $A \to B$
            inducing the map of fields $\bbk \to \bbk (X^{\mathrm{s}})$.
            Then $B$ is henselian.
            Doing a base change along the map $A \to B$ for everything will complete the proof.
            \qedhere
    \end{itemize}
\end{proof}

There is a stronger variant of this result,
which considers the geometrical generic fibre over $\bbKbar$,
instead of over $\bbK$.
Before introducing the result, we need a lemma.

\begin{lemma} \label{lem-5-fin-ext}
    Let $X$ be a smooth, integral, projective $\bbk$-variety.
    If $X_{\bbkbar}$ is universally $\CH_0$-trivial,
    then $X_F$ is universally $\CH_0$-trivial for some finite extension $F/\bbk$.
\end{lemma}

\begin{proof}
    By Theorem~\ref{thm-2-decomp}, $X_{\bbkbar}$ has a decomposition of the diagonal
    \[ [\Delta_{X_{\bbkbar}}] = D_{\bbkbar} + [X_{\bbkbar}] \times x_0
        \quad \text{in } \CH_n (X_{\bbkbar} \times_{\bbkbar} X_{\bbkbar}). \]
    By Galois descent, there exists a finite extension $F / \bbk$
    over which everything in the equation is defined, 
    and we have 
    \[ \CH_n (X_{\bbkbar} \times_{\bbkbar} X_{\bbkbar}) =
        \colim_{E/F \text{ finite}} \CH_n (X_E \times_E X_E). \]
    Therefore, there exists a finite extension $E$ such that 
    the equation of the decomposition of the diagonal holds over $E$.
\end{proof}

\begin{theorem} [Colliot-Thélène and Pirutka] \label{thm-5-spe-2}
    In Situation \textup{\ref{sit-5-specialise}}, suppose that
    \begin{itemize}
        \item
            The residue field $\bbk$ is algebraically closed.
        \item
            $\mathscr{X}$ is faithfully flat and proper over $A$,
            with geometrically integral fibres.
        \item
            The special fibre $X^{\mathrm{s}}$ 
            has a desingularisation $f \: \widetilde{X}^{\mathrm{s}} \to X^{\mathrm{s}}$,
            such that $f$ is universally $\CH_0$-trivial.
        \item
            The geometrical generic fibre $\Xbar = \mathscr{X} \times_A \bbKbar$
            is a $\bbKbar$-variety,
            with a desingularisation $\widetilde{X} \to \Xbar$.
    \end{itemize}
    Then if $\widetilde{X}$ is universally $\CH_0$-trivial, so is $\widetilde{X}^{\mathrm{s}}$.
\end{theorem}

\begin{proof}
    Our plan is to find a suitable base change in order to apply Theorem~\ref{thm-5-spe-1}.

    First, we replace $A$ by its completion.

    Let $F$ be a finite extension of $\bbK$, on which $\widetilde{X}$ is defined.
    In other words, there exists a smooth variety $Y$ over $F$,
    such that $Y_{\bbKbar} \simeq \widetilde{X}$,
    and there is a desingularisation map $Y \to X_F$, 
    which coincides with the map $\widetilde{X} \to \Xbar$ over $\bbKbar$.

    By Lemma~\ref{lem-5-fin-ext}, we may replace $F$ by a finite extension of it,
    so we may assume that $Y$ is universally $\CH_0$-trivial.

    Let $B$ be the integral closure of $A$ in $F$.
    By \cite[Proposition~I.3]{serre-local},
    $B$ is also a discrete valuation ring.
    Since $\bbk$ is algebraically closed, the residue field of $B$ is also $\bbk$.
    We can thus do a base change along the map $A \to B$, 
    and apply Theorem~\ref{thm-5-spe-1} to complete the proof.
\end{proof}

There is an even stronger variant of this result,
where instead of $\bbKbar$, we consider any field containing $\bbKbar$.

\begin{lemma} \label{lem-5-fin-ext-rational}
    Let $X$ be a projective $\bbk$-variety.
    If $X_F$ is retract rational for some extension $F/\bbk$,
    then the same is true for a certain finite extension $F/\bbk$.
\end{lemma}

\begin{proof}
    In the definition of the retract rationality of $X_F$, 
    everything is defined over a finitely generated extension of $\bbk$.
    Thus we may assume $F/\bbk$ is finitely generated.

    For the same reason, the lemma is true when $F$ is the algebraic closure of~$\bbk$.
    Therefore, we may assume that $\bbk$ is algebraically closed.

    In this case, there exists a $\bbk$-variety $Y$ such that $F$ is isomorphic to $\bbk (Y)$,
    and there exist non-empty open sets $U \subset X \times_{\bbk} Y$ and $V \subset \bbP_Y^n$,
    such that $U$ is a retract of $V$ as a $Y$-scheme.
    There exists a closed point $y \in Y$ such that the fibres $U_y$ and $V_y$ are non-empty.
    This proves the lemma.
\end{proof}

\begin{theorem} [Colliot-Thélène and Pirutka] \label{thm-5-spe-3}
    Assume that the four assumptions of Theorem~\textup{\ref{thm-5-spe-2}} are satisfied.
    Then, if $\widetilde{X}_F$ is retract rational for a field $F$ containing $\bbKbar$,
    then $\widetilde{X}^{\mathrm{s}}$ is universally $\CH_0$-trivial. \qed
\end{theorem}



\printbibliography

\end{document}