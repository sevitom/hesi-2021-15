折纸实际上是利用已知几何对象构造新的几何对象的过程.
 我们称折纸中构造出的点、线为可构造点、可构造线. 
 我们将折纸的纸面与复平面 $\mathbb{C}$ 等同起来, 
 于是可构造点、可构造线实际上是复平面的一些特殊子集. 

实际操作中, 我们发现主要会使用以下六种构造方式(称之为折纸的六公理): 

\begin{enumerate}[wide,itemindent=2em,label=(\arabic*)]
    \item 连接两个可构造点的直线是可构造的; 
    \item 两条可构造线的交点是可构造点; 
    \item 两个可构造点的中垂线是可构造线; 
    \item 两条可构造线的角平分线是可构造的; 
    \item 给定可构造线 $l$ 和可构造点 $P,Q$, 
    经过 $Q$ 且将 $P$ 反射到 $l$ 上的直线 (如果存在) 是可构造的; 
    \item 给定可构造线 $l,m$ 及可构造点 $P,Q$, 
    将 $P$反射到 $l$, 并将 $Q$ 反射到 $m$ 上的直线 (如果存在) 是可构造的. 
\end{enumerate}

在公理 (5) (6) 中, 满足要求的直线不一定存在. 事实上, 这些直线可由下面的引理刻画:

\begin{lemma}
    给定平面上一点 $P$ 以及直线 $l$, 假设 $P \notin l$, 
    并设 $\Gamma$ 是以 $P$ 为焦点, $l$ 为准线的抛物线. 
    那么直线 $m$ 将 $P$ 反射到 $l$ 上当且仅当 $m$ 是 $\Gamma$ 的一条切线. 
\end{lemma}

请读者自行完成引理的证明. 从引理可以看出, 公理 (5) 实际上是过给定点作抛物线的切线; 
公理 (6) 实际上是作两条抛物线的公切线. 而这样的构造并不总是存在的. 