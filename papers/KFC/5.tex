Firstly we recall the conclusions we have gotten: for initial question,
$(\text{square}, n)$ is good if and only if $n=1$. For any $n$ there exists 
$T$ such that $(T,n)$ is good. For any $T$, when $n$ is large enough 
$(T,n)$ are always bad.
 	
	
A fact is that it is hard to construct another good $(T, n)$ when $n$ 
is more than $2$ and $T$ is not a rectangle (except those in 
theorem 4.1). The difficulty is that we cannot divide a convex polygon 
into centrally symmetric polygons which has more than five edges easily. 
Moreover we need to use area-equivalent polygons: even 
diagonal parallel (polygons with $4n+2$ edges) or every edge has two edges
vertical to it (polygons with $4n$ edges) when a polygon is surrounded by 
other polygons. It is hard to achieve all such things. It can be conjectured 
there are no other good pairs except the cases in theorem 4.1. 
However, on the other hand, as we do not have the condition that $n$ is big enough, 
it is not easy to consider this problem locally. 
An available idea is to prove that a polygon cannot be divided into such polygons. 
We may prove this by considering the lines between origin points
and using the length and angles to calculating areas. 
It is also possible to calculating 
integral in some certain area locally around an origin point to prove 
the existence of $B_i$. However, they all need nontrivial calculation.
	
As a generalization of the question, we can change $T$ into Riemannian 
manifolds with constant curvature, for example, $S^2$ with canonical metric.
It is easy to see that $(S^2, 1)$ and $(S^2, 2)$ are good but $(S^2, 3)$ not.
However, $(S^2,4)$ is good again. The question on Riemannian manifolds 
with constant curvature may be much more complicated. It may be connected 
with isometric group of them.
The condition of constant curvature is necessary here since vertical bisector
is well defined in this case.
	
At the end of this article, we figure out what we have proved that in fact, 
in such a game the second person has advantage most of time. We can 
easily get the corollary that even if the number of points two people 
choose are different, the average of area of the second is almost always 
bigger than the first. It can be seen that in such situations it is always 
important to know the other people's position which will bring advantage 
to you.